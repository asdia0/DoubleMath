\documentclass{echw}

\title{JC1/2023/9649/MYE/EJC}
\author{Eytan Chong}
\date{2024-06-03}

\begin{document}
    \problem{}
        \textbf{Omitted.}

    \problem{}
        A curve $E$ has polar equation $r = \dfrac3{\sqrt{\cos^2\t + (\sin\t - \cos\t)^2}}$ for $0 \leq \t < 2\pi$.

        \begin{enumerate}
            \item Taking the polar axis as the positive $x$-axis, find the Cartesian equation of $E$, leaving your answer in the form $ax^2 + bxy + cy^2 = 9$, where $a$, $b$ and $c$ are constants to be determined.
            \item Hence or otherwise, find the exact Cartesian coordinates of the point(s) of intersection between $E$ and the graph with polar equation $r = \dfrac1{\sin \t - \cos \t}$ for $\dfrac\pi4 < \t < \dfrac{5\pi}4$.
        \end{enumerate}

    \solution
        \part
            \begin{alignat*}{2}
                &&r &= \dfrac3{\sqrt{\cos^2\t + (\sin\t - \cos\t)^2}}\\
                \implies&&r \sqrt{\cos^2\t + (\sin\t - \cos\t)^2} &= 3\\
                \implies&&r^2 \bp{\cos^2\t + (\sin\t - \cos\t)^2} &= 9\\
                \implies&&r^2\bp{2\cos^2\t - 2\cos\t\sin\t + \sin^2\t} &= 9\\
                \implies&&2(r\cos\t)^2 - 2(r\cos\t)(r\sin\t) + (r\sin\t)^2 &= 9\\
                \implies&&2x^2 - 2xy + y^2 &= 9
            \end{alignat*}

            \boxt{$E : 2x^2 - 2xy + y^2 = 9$}

        \part
            \begin{alignat*}{2}
                &&r &= \dfrac1{\sin\t - \cos\t}\\
                \implies&&r(\sin \t - \cos\t) = 1\\
                \implies&&r\sin\t - r\cos\t &= 1\\
                \implies&&y - x &= 1\\
                \implies&&y &= x + 1
            \end{alignat*}

            Substituting $y = x + 1$ into the Cartesian equation of $E$,
            \begin{alignat*}{2}
                &&2x^2 - 2xy + y^2 &= 9\\
                \implies&&2x^2 - 2x(x + 1) + (x + 1)^2 &= 9\\
                \implies&&2x^2 - 2x^2 - 2x + x^2 + 2x + 1 &= 9\\
                \implies&&x^2 &= 8\\
                \implies&&x &= \pm \sqrt{8}\\
                && &= \pm 2\sqrt{2}\\
                \implies&&y &= 1 \pm 2\sqrt{2}
            \end{alignat*}

            \boxt{$\bp{2\sqrt{2}, 1+2\sqrt{2}}$, $\bp{-2\sqrt{2}, 1-2\sqrt{2}}$}

    \problem{}
        The function $f$ is defined by
        \[
            f(x) = 2 - \dfrac1x, \, x \in \R, x \geq 1
        \]

        \begin{enumerate}
            \item Write down expressions for $f^2(x)$, $f^3(x)$ and $f^4(x)$ in the form $\dfrac{ax + b}{cx + d}$, where $a$, $b$, $c$ and $d$ are integers. Hence, make a conjecture for $f^n(x)$ in terms of $n$.
            \item Prove your conjecture for all positive integers $n$.
            \item Let $A$ be the largest subset of the real numbers such that when the domain of $f$ is replaced with $A$, $f^n(x)$ is defined for all positive integers $n$. State $A$.
        \end{enumerate}

    \solution
        \part
            \begin{alignat*}{2}
                &&f^2(x) &= 2 - \dfrac1{f(x)}\\
                && &= 2 - \bp{2 - \frac1x}^2\\
                && &= \dfrac{3x - 2}{2x - 1}\\
                \implies&&f^3(x) &= 2 - \dfrac1{f^2(x)}\\
                && &= 2 - \bp{\frac{3x - 2}{2x - 1}}^{-1}\\
                && &= \dfrac{4x - 3}{3x - 2}\\
                \implies&&f^4(x) &= 2 - \dfrac1{f^3(x)}\\
                && &= 2 - \bp{\frac{4x - 3}{3x - 2}}^{-1}\\
                && &= \dfrac{5x - 4}{4x - 3}
            \end{alignat*}

            \boxt{$f^2(x) = \dfrac{3x - 2}{2x - 1}$, $f^3(x) = \dfrac{4x - 3}{3x - 2}$, $f^4(x) = \dfrac{5x - 4}{4x - 3}$}

            \textbf{Conjecture.} For all $n \in \N$, $f^n(x) = \dfrac{(n+1)x - n}{nx - (n-1)}$.

        \part
            Let $P_n$ be the statement that $f^n(x) = \dfrac{(n+1)x - n}{nx - (n-1)}$ for $n \in \N$.

            \textbf{Base Case.} $n = 1 : f^1(x) = 2 - \dfrac1x = \dfrac{2x - 1}{x - 0}$. Hence, $P_1$ is true.
            
            \textbf{Inductive Hypothesis.} Assume $P_k$ is true for some $k \in \N$.

            \textbf{Inductive Step.} Consider $f^{k+1}(x)$.
            \begin{align*}
                f^{k+1}(x) &= 2 - \dfrac1{f^k(x)}\\
                &= 2 - \bp{\dfrac{(k+1)x - k}{kx - (k-1)}}^{-1}\\
                &= 2 - \dfrac{kx - (k-1)}{(k+1)x - k}\\
                &= \dfrac{2\bs{(k+1)x - k} - \bs{kx - (k-1)}}{(k+1)x - k}\\
                &= \dfrac{(k+2)x - (k+1)}{(k+1) - k}
            \end{align*}

            Hence, $P_k \implies P_{k+1}$. Since $P_1$ is true, by induction, $P_n$ is true for all $n \in \N$.

        \part
            Since $f^n(x) = \dfrac{(n+1)x - n}{nx - (n-1)}$ for all $n \in \N$, we have the restriction $nx - (n-1) \neq 0 \implies x \neq \dfrac{n-1}{n}$. Hence,
            \boxt{$A = \R\setminus\bc{n \in \N : \dfrac{n-1}{n}}$}

    \problem{}
        A curve $T$ has polar equation $r = \sqrt{\dfrac2{\sin 3\t}}$ where $-\pi < \t \leq \pi$.

        \begin{enumerate}
            \item Determine the range of values of $\t$ for which the value of $r$ is undefined. Hence, state the equations of the asymptotes of $T$ in polar form.
            \item Hence, sketch $T$, indicating clearly, in polar form, the equations of the asymptotes and any lines of symmetry, and the polar coordinates of any points where $r$ attains stationary values.
        \end{enumerate}

    \solution
        \part
            For $r$ to be undefined, either $\sin 3\t = 0$ or $\dfrac2{\sin3\t} < 0$. In the latter case, $\sin3\t < 0$. Hence, it suffices to determine the range of values of $\t$ such that $\sin3\t \leq 0$. Note that $- \pi < \t \leq \pi \implies -3\pi < 3\t \leq 3\pi$. Hence,
            \[
                3\t \in (-3\pi, -2\pi] \cup [-\pi, 0] \cup [\pi, 2\pi] \cup \bc{3\pi}\\
            \]

            \boxt{$r$ is undefined for $\t \in \left(-\pi, -\dfrac23\pi\right] \cup \bs{-\dfrac13\pi, 0} \cup \bs{\dfrac13\pi, \dfrac23\pi} \cup \bc{\pi}$.}

            \boxt{Asymptotes: $\t = -\dfrac23 \pi, -\dfrac13\pi, 0, \dfrac13\pi, \dfrac23\pi, \pi$.}

        \part
            \begin{center}
                \begin{tikzpicture}[trim axis left, trim axis right]
                    \begin{axis}[
                        samples = 100,
                        axis y line=middle,
                        axis x line=middle,
                        xtick = \empty,
                        ytick = \empty,
                        xmin=-2,
                        xmax=2,
                        ymin=-2,
                        ymax=2,
                        xlabel = {$\t=0$},
                        ylabel = {$\t = \frac\pi2$},
                        legend cell align={left},
                        legend pos=outer north east,
                        after end axis/.code={
                            \path (axis cs:0,0) 
                                node [anchor=north east] {$O$};
                            }
                        ]
                        \addplot[color=plotRed,data cs=polarrad, domain=0.1:pi/3-0.1] {sqrt(2/sin(3*\x r))};

                        \addplot[color=plotRed,data cs=polarrad, domain=2*pi/3+0.1:pi-0.1] {sqrt(2/sin(3*\x r))};

                        \addplot[color=plotRed,data cs=polarrad, domain=4/3*pi+0.1:5/3*pi-0.1] {sqrt(2/sin(3*\x r))};
                        
                        \addplot[dotted] {sqrt(3) * x};

                        \addplot[dotted] {-sqrt(3) * x};

                        \node[anchor=south west] at (-2, 0) {$\t = \pi$};

                        \node[anchor=north west] at (1, 2) {$\t = \frac13\pi$};

                        \node[anchor=north east] at (-1, 2) {$\t = \frac23\pi$};

                        \node[anchor=south west] at (1, -2) {$\t = -\frac13\pi$};

                        \node[anchor=south east] at (-1, -2) {$\t = -\frac23\pi$};

                        \addlegendentry{$T$};

                        \fill (0, -1.414) circle[radius=2.5pt] node[anchor=south] {$\bp{-\frac\pi2, \sqrt2}$};
                    \end{axis}
                \end{tikzpicture}
            \end{center}

             From the graph, we see that the only stationary point of $T$ occurs when $\t = -\dfrac\pi2$, where $r = \sqrt2$.

    \problem{}
        \textbf{Omitted.}

    \problem{}
        \begin{enumerate}
            \item The sequence $\bc{X_n}$ is given by $X_0 = 6$ and
            \[
                X_n = \dfrac14 X_{n-1} + 2^{1-n}, \, n \geq 1
            \]
            By multiplying the recurrence relation throughout by $2^n$, use a suitable substitution to determine $X_n$ as a function of $n$.
            \item The sequence of real numbers $\bc{u_n}$ is defined by $u_1 = a$, and the recurrence relation
            \[
                u_{n+1} = \dfrac{u_n^2 + 5}{2u_n + 4}, \, n \geq 1
            \]
            \begin{enumerate}
                \item Given that the sequence converges to a limit $l$, find all possible values of $l$.
                \item With the aid of a graphing calculator, determine the long-term behaviour of the sequence when $a = -2.01$ and when $a = -1.99$.
                \item Show that $u_{n+1} > -2$ if $u_n > -2$ and $u_{n+1} < -2$ if $u_n < -2$. Hence, explain the difference in the behaviour of the sequence when $a = -2.01$ and $a = -1.99$.
            \end{enumerate}
        \end{enumerate}

    \solution
        \part
            \begin{alignat*}{2}
                &&X_n &= \dfrac14 X_{n-1} + 2^{1-n}\\
                \implies&&2^nX_n &= \dfrac14 \cdot 2^n X_{n-1} + 2\\
                \implies&&2^n X_n &= \dfrac12 \cdot 2^{n-1} X_{n-1} + 2
            \end{alignat*}
            Let $Y_n = 2^n X_n$. We have
            \[
                Y_n = \dfrac12 Y_{n-1} + 2
            \]
            Let $k$ be a constant such that $Y_n + k = \dfrac12 (Y_{n-1} + k) \implies \dfrac12 k - k = 2 \implies k = -4$.
            \begin{alignat*}{2}
                &&Y_n - 4 &= \dfrac12 (Y_{n-1} - 4)\\
                \implies&&Y_n - 4 &= \dfrac1{2^n} (Y_0 - 4)\\
                \implies&&Y_n &= \dfrac1{2^n} (2^0 X_0 - 4) + 4\\
                && &= \dfrac1{2^n} \cdot 2 + 4\\
                && &= 2^{1-n} + 2^2\\
                &&X_n &= 2^{-n} Y_n\\
                && &= 2^{1-2n} + 2^{2-n}
            \end{alignat*}

            \boxt{$X_n = 2^{1-2n} + 2^{2-n}$}

        \part
            \subpart

                Let $l = \lim_{n \to \infty} u_n$.
                \begin{alignat*}{2}
                    &&l &= \dfrac{l^2 + 5}{2l + 4}\\
                    \implies&&l(2l + 4) &= l^2 + 5\\
                    \implies&&l^2 + 4l - 5 &= 0\\
                    \implies&&(l+5)(l-1) &= 0
                \end{alignat*}
                Hence, $l = -5$ or $l = 1$.

                \boxt{$l = -5 \lor 1$}

            \subpart

                When $a = -2.01$, the sequence is increasing and converges to $-5$. 
            
                When $a = -1.99$, the sequence is decreasing and converges to 1.

            \subpart
                \begin{align*}
                    u_{n+1} &= \dfrac{u_n^2 + 5}{2u_n + 4}\\
                    &= \dfrac12 u_n - 1 + \dfrac9{2u_n + 4}
                \end{align*}
                Suppose $u_n > -2$. Then $2u_n + 4 > 0$. Hence,
                \begin{align*}
                    u_{n+1} &= \dfrac12 u_n - 1 + \dfrac9{2u_n + 4}\\
                    &> \dfrac12 \cdot -2 - 1 + 0\\
                    &> -2
                \end{align*}
                Thus, $u_{n+1} > -2$.

                \medskip

                 Suppose $u_n < -2$. Then $2u_n + 4 < 0$. Hence,
                \begin{align*}
                    u_{n+1} &= \dfrac12 u_n - 1 + \dfrac9{2u_n + 4}\\
                    &< \dfrac12 \cdot -2 - 1 + 0\\
                    &< -2
                \end{align*}
                Thus, $u_{n+1} < -2$.

                \medskip

                 When $a = -2.01 < -2$, all terms in the sequence are less than $-2$. Hence, the sequence converges to $-5$ since the other limiting value (1) is greater than $-2$.

                 When $a = -1.99 > -2$, all terms in the sequence are greater than $-2$. Hence, the sequence converges to 1 since the other limiting value ($-5$) is less than $-2$.

    \problem{}
        After the Omega variant of a new virus emerged in 2023, a group of epidemiologists sought to model the spread of the virus in Singapore. In their model, the increase in cases from week $n-1$ to $n$ is modelled as $k$ times the increase in cases from week $n-2$ to $n-1$, where $k$ is known as the weekly infection growth rate. Let $x_n$ be the total number of Omega variant cases in Singapore within the first $n$ weeks of the outbreak.

        \begin{enumerate}
            \item Show that $x_n$ is defined by the recurrence relation $x_n = ax_{n-1} + bx_{n_2}$, where $a$ and $b$ are constants to be determined in terms of $k$.
        \end{enumerate}

         8 Omega variant cases were reported in the first week of the outbreak, while 15 new cases were reported the week after.

        \begin{enumerate}
            \setcounter{enumi}{1}
            \item Given that $k \neq 0$ and $k \neq 1$, solve the recurrence relation and obtain an expression for $x_n$ of the form $\a + \b\bp{k^{n-1} - 1}$, where $\a$ and $\b$ are constants to be determined in terms of $k$.
            \item Determine the long-term behaviour of $x_n$ for the cases where $k > 1$ and $0 < k < 1$, and hence explain in context what the long-term spread of the Omega variant in Singapore will be for each case.
            \item State a limitation of using this model to predict the spread of the Omega variant in Singapore.
            \item Show that $x_n$ is in arithmetic progression if the weekly infection growth rate is 1, and state the value of its common difference.
        \end{enumerate}

         In one simulation, the weekly infection growth rate is estimated to be 4. After the implementation of a mass vaccination programme in the fifth week of the outbreak, the weekly infection growth rate is estimated to decrease to 1, such that the increase in cases from week 5 to 6 is equal to the increase from week 4 to 5. The weekly infection growth rate then remains unchanged for the rest of the outbreak. Alert Orange is triggered when the 10000th case is reported.

        \begin{enumerate}
            \setcounter{enumi}{5}
            \item Under this simulation, determine the week in which Alert Orange will be triggered.
        \end{enumerate}

    \solution
        \part
            Let $\D x_n$ be the increase in cases from week $n$ to $n+1$. We have the relations
            \[
                x_n = x_{n-1} + \D x_{n-1} \text{ and } \D x_n = k\D x_{n-1}
            \]
            Note that $x_{n-1} = x_{n-2} + \D x_{n-2} \implies \D x_{n-2} = x_{n-1} - x_{n-2}$. Then
            \begin{align*}
                x_n &= x_{n-1} + \D x_{n-1}\\
                &= x_{n-1} + k\D x_{n-2}\\
                &= x_{n-1} + k\bp{x_{n-1} - x_{n-2}}\\
                &= (1+k)x_{n-1} -kx_{n-2}
            \end{align*}

        \part
            Consider the characteristic equation of the recurrence relation.
            \begin{alignat*}{2}
                &&x^2 - (1+k)x + k &= 0\\
                \implies&&(x-1)(x-k) &= 0\\
            \end{alignat*}
            Hence, the roots of the characteristic equation are $1$ and $k$. Since $k \neq 1$, we have
            \begin{align*}
                x_n &= A \cdot 1^n + B \cdot k^n\\
                &= A + B k^n
            \end{align*}
            Consider $n = 1, 2$. Since $x_1 = 8$ and $x_2 = 23$, we have the system
            \[
                \systeme[AB]{A+kB = 8, A+k^2B = 23}
            \]
            We hence have $A = 8-kB = 23-k^2B$. Thus,
            \begin{alignat*}{2}
                &&8-kB &= 23-k^2B\\
                \implies&&Bk^2-Bk &= 15\\
                \implies&&B &= \dfrac{15}{k(k-1)}\\
                \implies&&A &= 8 - \dfrac{15}{k-1}
            \end{alignat*}
            Putting the values of $A$ and $B$ back into our recurrence relation, we obtain
            \begin{align*}
                x_n &= \bp{8 - \dfrac{15}{k-1}} + \bp{\dfrac{15}{k(k-1)}}k^n\\
                &= 8 - \dfrac{15}{k-1} + \dfrac{15}{k-1} k^{n-1}\\
                &= 8 + \dfrac{15}{k-1}\bp{k^{n-1} - 1}
            \end{align*}

            \boxt{$x_n =  8 + \dfrac{15}{k-1}\bp{k^{n-1} - 1}$}

        \part
            Suppose $k > 1$. Then
            \begin{align*}
                \lim_{n \to \infty} x_n &= \lim_{n \to \infty} \bs{8 + \dfrac{15}{k-1}\bp{k^{n-1} - 1}} \to \infty
            \end{align*}
            Hence, everyone in Singapore will get the Omega variant, i.e. the Omega variant will not stop spreading.

            \medskip
            
            Suppose $0 < k < 1$. Then
            \begin{align*}
                \lim_{n \to \infty} x_n &= \lim_{n \to \infty} \bs{8 + \dfrac{15}{k-1}\bp{k^{n-1} - 1}} = 8 - \dfrac{15}{k-1} = 8 + \dfrac{15}{1-k}
            \end{align*}
            Hence, a total of $8 + \dfrac{15}{1-k}$ people in Singapore will get the Omega variant, i.e. the Omega variant will eventually stop spreading.

        \part
            The weekly infection growth rate $k$ is unlikely to be a constant. For instance, as most of the population gets the Omega variant, there are fewer people available for the Omega variant to infect, hence $k$ will start to decrease.

        \part
            Observe that
            \begin{align*}
                x_n - x_{n-1} &= (x_{n-1} + \D x_{n-1}) - x_{n-1}\\
                &= \D x_{n-1}
            \end{align*}
            If the weekly infection growth rate is 1, $\D x_n$ is constant, whence $x_n$ is in arithmetic progression.

            \boxt{The common difference is 15.}

        \part
            When $k = 4$,
            \begin{align*}
                x_4 &= 8 + \dfrac{15}3 \bp{4^3 - 1} = 323\\
                x_5 &= 8 + \dfrac{15}3 \bp{4^4 - 1} = 1283\\
            \end{align*}
            Hence, $\D x_4 = 1283 - 323 = 960$. Let $x_n'$ denote the total number of Omega variant cases in Singapore within the first $n$ weeks of the outbreak in the given simulation. Since $k$ becomes 1 starting from week 5, we have
            \begin{align*}
                x_n' &= x_5 + \D x_5 (n-5), \, n \geq 5\\
                &= 1283 + 960(n-5)\\
                &= 960n - 3517
            \end{align*}
            Consider $x_n' \geq 10000$.
            \begin{alignat*}{2}
                &&x_n' &\geq 10000\\
                \implies&&960n - 3517 &\geq 10000\\
                \implies&&960n &\geq 13517\\
                \implies&&n &\geq \dfrac{13517}{960}\\
                && &= 14.1 \tosf{3}
            \end{alignat*}
            
            \boxt{Alert Orange will be triggered in week 15.}
\end{document}