\documentclass{echw}

\title{JC1/2023/9649/MYE/ASRJC}
\author{Eytan Chong}
\date{2024-06-02}

\begin{document}
    \problem{}
        It is given that $f(r) = \dfrac1{(r+1)(r+2)}$.

        \begin{enumerate}
            \item Show that $f(r-1) - f(r) = \dfrac2{r(r+1)(r+2)}$ and find $\displaystyle\sum\limits_{r = 1}^n \dfrac1{r(r+1)(r+2)}$ in terms of $n$.
            \item \begin{enumerate}
                \item Deduce the exact value of $\displaystyle\sum\limits_{r = 1}^\infty \dfrac1{r(r+1)(r+2)}$.
                \item For $n > 3$, deduce an expression for $\displaystyle\sum\limits_{r = 3}^{N-1} \dfrac1{r(r-1)(r-2)}$ in terms of $N$.
            \end{enumerate}
        \end{enumerate}

    \solution
        \part
            \begin{align*}
                f(r-1) - f(r) &= \dfrac1{(r-1+1)(r-1+2)} - \dfrac1{(r+1)(r+2)}\\
                &= \dfrac1{r(r+1)} - \dfrac1{(r+1)(r+2)}\\
                &= \dfrac{(r+2)-r}{r(r+1)(r+2)}\\
                &= \dfrac2{r(r+1)(r+2)}
            \end{align*}
            \begin{align*}
                \sum_{r=1}^n \dfrac1{r(r+1)(r+2)} &= \dfrac12 \sum_{r=1}^n \dfrac2{r(r+1)(r+2)}\\
                &= \dfrac12 \sum_{r=1}^n \Big(f(r-1) - f(r)\Big)\\
                &= \dfrac12 \bp{\sum_{r=1}^n f(r-1) - \sum_{r=1}^n f(r)}\\
                &= \dfrac12 \bp{\sum_{r=0}^{n-1} f(r) - \sum_{r=1}^n f(r)}\\
                &= \dfrac12 \bp{\left[f(0) + \sum_{r=1}^{n-1} f(r)\right] - \left[\sum_{r=1}^{n-1} f(r) + f(n)\right]}\\
                &= \dfrac12 \Big(f(0) - f(n)\Big)\\
                &= \dfrac12 \bp{\dfrac1{1 \cdot 2} - \dfrac1{(n+1)(n+2)}}\\
                &= \dfrac14 - \dfrac1{2(n+1)(n+2)}
            \end{align*}

            \boxt{$\displaystyle\sum\limits_{r=1}^n \dfrac1{r(r+1)(r+2)} = \dfrac14 - \dfrac1{2(n+1)(n+2)}$}

        \part
            \subpart
                \begin{align*}
                    \sum_{r=1}^\infty \dfrac1{r(r+1)(r+2)} &= \lim_{n \to \infty} \sum_{r=1}^n \dfrac1{r(r+1)(r+2)}\\
                    &= \lim_{n \to \infty} \bp{\dfrac14 - \dfrac1{2(n+1)(n+2)}}\\
                    &= \dfrac14
                \end{align*}

                \boxt{$\displaystyle\sum\limits_{r=1}^\infty \dfrac1{r(r+1)(r+2)} = \dfrac14$}

            \subpart
                \begin{align*}
                    \sum_{r = 3}^{N-1} \dfrac1{r(r-1)(r-2)} &= \sum_{r = 1}^{N-3} \dfrac1{(r+2)(r+1)r}\\
                    &= \dfrac14 - \dfrac1{2(N-3+1)(N-3+2)}\\
                    &= \dfrac14 - \dfrac1{2(N-2)(N-1)} 
                \end{align*}

                \boxt{$\displaystyle\sum\limits_{r = 3}^{N-1} \dfrac1{r(r-1)(r-2)} = \dfrac14 - \dfrac1{2(N-2)(N-1)}$}

    \problem{}
        \begin{enumerate}
            \item A geometric progression $G$ has positive first term $a$, a common ratio $r$ and sum to infinity $S$. The sum to infinity of the even-numbered terms of $G$, i.e. the second, fourth, sixth, $\ldots$ terms, is $-\dfrac12 S$.
            \begin{enumerate}
                \item Find the value of $r$.
                \item In another geometric progression $H$, each term is the modulus of the corresponding term of $G$. Given that the third term of $G$ is 2, show that the sum to infinity of $H$ is 27.
            \end{enumerate}
            \item An arithmetic progression has first term 1000 and common difference $-1.4$. Determine, with clear workings, the value of the first negative term of the sequence and the sum of all the positive terms.
        \end{enumerate}

    \solution
        \part
            \subpart
                 Let the $n$th term of $G$ be $a_n = ar^{n-1}$. Since the sum to infinity of $S$ exists, $\abs{r} < 1$.
                \begin{align*}
                    \sum_{n=1}^\infty a_{2n} &= \sum_{n=1}^\infty ar^{2n-1}\\
                    &= \dfrac{a}{r} \sum_{n=1}^\infty \bp{r^2}^n\\
                    &= \dfrac{a}{r} \cdot \dfrac{r^2}{1-r^2}\\
                    &= \dfrac{ar}{1-r^2}
                \end{align*}
                Note that $S = \dfrac{a}{1-r}$. Thus, we have
                \begin{alignat*}{2}
                    &&-\dfrac12 \cdot \dfrac{a}{1-r} &= \dfrac{ar}{1-r^2}\\
                    \implies&&-\dfrac12 \cdot \dfrac1{1-r} &= \dfrac{r}{1-r^2}\\
                    \implies&&1-r^2 &= -2r(1-r)\\
                    \implies&&3r^2 - 2r - 1 &= 0\\
                    \implies&&(3r+1)(r-1) &= 0
                \end{alignat*}
                Hence, $r = -\dfrac13$. Note that we reject $r = 1$ since $\abs{r} < 1$.

                \boxt{$r = -\dfrac13$}

            \subpart
                Since $a_3 = 2 = a\bp{-\dfrac13}^2$, we have $a = 18$. Let the $n$th term of $H$ be $b_n = \abs{a_n} = \abs{18 \bp{-\dfrac13}^{n-1}} = 18 \bp{\dfrac13}^{n-1}$. Hence, the sum to infinity of $H$ is given by
                \[
                    \sum_{n=1}^\infty b_n = \dfrac{18}{1-1/3} = 27
                \]

        \part
            Let the $n$th term of the arithmetic progression be $a_n = 1000 - 1.4(n-1) = 1001.4 - 1.4n$. Consider $a_n < 0$.
            \begin{alignat*}{2}
                &&a_n &< 0\\
                \implies&&1001.4 - 1.4n &< 0\\
                \implies&&1.4n &> 1001.4\\
                \implies&&n &> \dfrac{1001.4}{1.4}\\
                && &= 715.3
            \end{alignat*}
            Hence, the first negative term of the sequence is achieved when $n = 716$. Thus, the value of the first negative term is $a_{716} = 1001.4 - 1.4\cdot716 = -1$.

            \boxt{The value of the first negative term is $-1$.}

            \begin{align*}
                \sum_{n=1}^{715} a_n &= \sum_{n=1}^{715} \Big(1001.4 - 1.4n\Big)\\
                &= 1001.4 \cdot 715 - 1.4 \cdot \dfrac{715\cdot716}{2}\\
                &= 357643
            \end{align*}

            \boxt{The sum of all the positive terms is 357643.}

    \problem{}
        \textbf{Omitted.}

    \problem{}
        Referring to the pole $O$, the curve $C$ has polar equation $r = \cot \t$, where $\dfrac\pi6 < \t < \dfrac\pi2$.
        
        \begin{enumerate}
            \item Sketch the curve $C$.
            \item Show that $\der{y}{x} = \dfrac1{r(r^2 + 2)}$. Determine the exact range of values of the gradient of $C$.
            \item Obtain a Cartesian equation of $C$ in the form $y = f(x)$.
        \end{enumerate}

    \solution
        \part
            \begin{center}
                \begin{tikzpicture}[trim axis left, trim axis right]
                    \begin{axis}[
                        domain = pi/6:pi/2,
                        samples = 100,
                        axis y line=middle,
                        axis x line=middle,
                        xtick = \empty,
                        ytick = \empty,
                        xmin=-0.1,
                        xmax=2,
                        ymin=-0.1,
                        ymax=1,
                        xlabel = {$\t=0$},
                        ylabel = {$\t = \frac\pi2$},
                        legend cell align={left},
                        legend pos=outer north east,
                        after end axis/.code={
                            \path (axis cs:0,0) 
                                node [anchor=north east] {$O$};
                            }
                        ]
                        \addplot[color=plotRed,data cs=polarrad] {1/tan(\x r)};
            
                        \addlegendentry{$C$};

                        \draw (1.5, 0.866) circle[radius=2.5pt] node[anchor=west] {$\bp{\dfrac\pi6, \sqrt3}$};

                        \draw (0, 0) circle[radius=2.5pt];
                    \end{axis}
                \end{tikzpicture}
            \end{center}

        \part
            Note that $\der{r}{\t} = -\csc^2 \t = -(1 + r^2)$.
            \begin{align*}
                \der{y}{x} &= \dfrac{\der{r}{\t}\sin\t + r\cos\t}{\der{r}{\t}\cos\t - r\sin\t}\\
                &= \dfrac{-(1+r^2)\sin\t + r\cos\t}{-(1+r^2)\cos\t - r\sin\t}\\
                &= \dfrac{-(1+r^2) + r\cot\t}{-(1+r^2)\cot\t - r}\\
                &= \dfrac{-(1+r^2) + r^2}{-(1+r^2)r - r}\\
                &= \dfrac{(1+r^2) - r^2}{(1+r^2)r + r}\\
                &= \dfrac1{r(2+r^2)}
            \end{align*}
            Observe that $r \in (0, \sqrt{3})$. Since $\der{y}{x} = \dfrac1{r(r^2 + 2)}$ is continuous and decreasing on the interval $(0, \sqrt{3})$, we have $\der{y}{x} \in \bp{\dfrac1{\sqrt3(\sqrt{3}^2 + 2)}, \infty} = \bp{\dfrac1{5\sqrt{3}}, \infty}$.

            \boxt{$\der{y}{x} \in \bp{\dfrac1{5\sqrt{3}}, \infty}$}

        \part
            \begin{alignat*}{2}
                    &&r &= \cot\t\\
                    \implies&&r\sin\t &= \cos\t\\
                    \implies&&y &= \cos \arctan \dfrac{y}{x}\\
                    && &= \dfrac{x}{\sqrt{x^2 + y^2}}\\
                    \implies&&y^2 &= \dfrac{x^2}{x^2 + y^2}\\
                    \implies&&y^2 \bp{x^2 + y^2} &= x^2\\
                    \implies&&y^4 + x^2y^2 - x^2 &= 0\\
                    \implies&&y^2 &= \dfrac{-x^2 + \sqrt{x^4 + 4x^2}}2 \tag{$\ast$}\\
                    \implies&&y &= \sqrt{\dfrac{-x^2 + \sqrt{x^4 + 4x^2}}2} \tag{$\ast$}
            \end{alignat*}
            Note that in the steps marked ($\ast$), we reject the negative branch since $y^2 \geq 0$ and $y > 0$ in the given domain. 

            \boxt{$y = \sqrt{\dfrac{-x^2 + \sqrt{x^4 + 4x^2}}2}$}

    \problem{}
        Relative to an origin $O$, an object is placed at point $P$ with coordinates $(-4, c, c)$, where $c$ is a positive real constant, and there is a mirror plane with equation $x + y + z = 1$. It is known that the shortest distance between $P$ and the mirror is $3\sqrt3$.

        \begin{enumerate}
            \item Show that $c = 7$.
        \end{enumerate}
    
         A point $A$ has coordinates $(-15, 17, 5)$.

        \begin{enumerate}
            \setcounter{enumi}{1}
            \item Find the coordinates of $A'$, the point of reflection of $A$ in the mirror.
        \end{enumerate}

         A laser beam is directed from $A$ towards a point on the mirror and is reflected to reach the object at $P$.

        \begin{enumerate}
            \setcounter{enumi}{2}
            \item Find the acute angle that the laser beam makes with the mirror.
        \end{enumerate}
    
    \solution
        \part
            We have that the mirror is defined by the vector equation $\vec r \cdot \cveciii111 = 1$. Note that the point with position vector $\cveciii100$ is on the mirror. Thus,
            \begin{align*}
                \text{Shortest distance between $P$ and mirror} &= \abs{\bs{\cveciii{-4}{c}{c} - \cveciii100} \cdot \cveciii111} \Bigg/ \abs{\cveciii111}\\
                &= \dfrac1{\sqrt3} \abs{\cveciii{-5}{c}{c} \cdot \cveciii111}\\
                &= \dfrac{\sqrt3}{3} \abs{-5+2c}
            \end{align*}
            We are given that the shortest distance between $P$ and the mirror is $3\sqrt3$ units. Hence,
            \begin{alignat*}{2}
                &&\dfrac{\sqrt3}3 \abs{-5+2c} &= 3\sqrt3\\
                \implies&&\abs{-5+2c} &= 9
            \end{alignat*}
            \case{1}{} $-5 + 2c > 0 \implies -5+2c = 9 \implies c = 7$.

            \case{2}{} $-5 + 2c < 0 \implies -5+2c = -9 \implies c = -2$ which cannot be since $c$ is positive. Hence, $c = 7$ as required.

        \part
            Let $F$ be a point on the mirror (i.e. $\oa{OF} \cdot \cveciii111 = 1$) such that $\oa{AF} = \l \cveciii111$ for some $\l \in \R$.
            \begin{alignat*}{2}
                &&\oa{AF} &= \l \cveciii111\\
                \implies&&\oa{OF} - \cveciii{-15}{17}5 &= \l\cveciii111\\
                \implies&&\bs{\oa{OF} - \cveciii{-15}{17}5} \cdot \cveciii111 &= \l\cveciii111 \cdot \cveciii111\\
                \implies&&\oa{OF} \cdot \cveciii111 - \cveciii{-15}{17}5 \cdot \cveciii111 &= \l \cveciii111 \cdot \cveciii111\\
                \implies&&1 - 7 &= 3\l\\
                \implies&&\l &= -2\\
                \implies&&\oa{AF} &= -2\cveciii111
            \end{alignat*}
            Note that since $A'$ is the reflection of $A$ in the mirror, $\oa{AF} = \oa{FA'}$.
            \begin{alignat*}{2}
                &&\oa{AF} &= \oa{FA'}\\
                \implies&&\oa{AA'} &= 2\oa{AF}\\
                \implies&&\oa{OA'} - \oa{OA} &= 2\oa{AF}\\
                \implies&&\oa{OA'} - \cveciii{-15}{17}5 &= 2\cdot -2\cveciii111\\
                \implies&&\oa{OA'} &= \cveciii{-4}{-4}{-4} + \cveciii{-15}{17}5\\
                && &= \cveciii{-19}{13}1
            \end{alignat*}

            \boxt{$A'(-19, 13, 1)$}

        \part
            Let $\t$ be the acute angle the laser beam makes with the mirror. Note that $\oa{A'P} = \cveciii{15}{-6}6 = 3\cveciii5{-2}2$. Hence, the line $A'P$ has direction vector $\cveciii5{-2}2$.
            \begin{alignat*}{2}
                &&\sin\t &= \dfrac{\abs{\cveciii5{-2}2 \cdot \cveciii111}}{\abs{\cveciii5{-2}2}\abs{\cveciii111}}\\
                && &= \dfrac{5}{\sqrt{99}}\\
                \implies&&\t &= 0.527 \tosf{3}
            \end{alignat*}

            \boxt{The laser beams makes an acute angle of $0.527$ with the mirror.}
    \problem{}
        \textbf{Omitted.}

    \problem{}
        A straight street of width 20 metres is bounded on its parallel sides by two vertical walls, one of height 13 metres, the other of height 8 metres. The intensity of light at point $P$ at ground level on the street is proportional to the angle $\t$ radians, where $\t = \angle APB$ as shown in the cross-sectional diagram below.

        \begin{center}
            \begin{tikzpicture}
                \coordinate[label=below:$P$] (P) at (0, 0);
                \coordinate[label=above:$A$] (A) at (-3, 4);
                \coordinate[label=above:$B$] (B) at (7, 13/2);
                \coordinate (AF) at (-3, 0);
                \coordinate (BF) at (7, 0);

                \draw (AF)--(BF);
                \draw (AF)--(A);
                \draw (BF)--(B);
                \draw (P)--(A);
                \draw (P)--(B);

                \draw pic [draw, angle radius=12mm, "$\t$"] {angle = B--P--A};
        
                \node[anchor=east] at ($(A)!0.5!(AF)$) {8 m};
                \node[anchor=west] at ($(B)!0.5!(BF)$) {13 m};
                \node[anchor=north] at ($(AF)!0.5!(P)$) {$x$ m};
            \end{tikzpicture}
        \end{center}

        \begin{enumerate}
            \item Given that the distance of $P$ from the base of the wall of height 8 metres is $x$ metres ($0 \leq x \leq 20$), show that
            \[
                \t = \arctan \dfrac{x}8 + \arctan \dfrac{20-x}{13}
            \]
            \item Find an expression for $\der{\t}{x}$.
            \item Hence, determine the value of $x$ corresponding to the maximum light intensity at $P$. Give your answer to four significant figures. You need not justify that the value of $x$ obtained gives the maximum light intensity at $P$.
            \item Find the minimum value of $\t$ as $x$ varies.
            \item The point $P$ moves across the street from the base of $A$ to the base of $B$ with speed $0.5$ ms$^{-1}$. Determine the rate of change of $\t$ with respect to time when $P$ is at the midpoint of the street.
        \end{enumerate}

    \solution
        \part
            \begin{center}
                \begin{tikzpicture}
                    \coordinate[label=below:$P$] (P) at (0, 0);
                    \coordinate[label=above:$A$] (A) at (-3, 4);
                    \coordinate[label=above:$B$] (B) at (7, 13/2);
                    \coordinate (AF) at (-3, 0);
                    \coordinate (BF) at (7, 0);
                    \coordinate (PA) at (0, 4);
                    \coordinate (PB) at (0, 13/2);

                    \draw (AF)--(BF);
                    \draw (AF)--(A);
                    \draw (BF)--(B);
                    \draw (P)--(A);
                    \draw (P)--(B);
                    \draw[dotted] (PB) -- (P);
                    \draw[dotted] (A) -- (PA);
                    \draw[dotted] (B) -- (PB);

                    \draw pic [draw, angle radius=12mm, "$\f_1$"] {angle = AF--A--P};
                    \draw pic [draw, angle radius=12mm, "$\f_2$"] {angle = P--B--BF};
                    \draw pic [draw, angle radius=12mm, "$\f_1$"] {angle = PA--P--A};
                    \draw pic [draw, angle radius=14mm, "$\f_2$"] {angle = B--P--PB};
            
                    \node[anchor=east] at ($(A)!0.5!(AF)$) {8 m};
                    \node[anchor=west] at ($(B)!0.5!(BF)$) {13 m};
                    \node[anchor=north] at ($(AF)!0.5!(P)$) {$x$ m};
                    \node[anchor=north] at ($(BF)!0.5!(P)$) {$(20-x)$ m};
                \end{tikzpicture}
            \end{center}

             Consider the diagram above. It is clear that $\t = \f_1 + \f_2$. Observe that $\tan \f_1 = \dfrac{x}8$ and $\tan \f_2 = \dfrac{20-x}{13}$. Thus, $\t = \arctan \dfrac{x}8 + \arctan \dfrac{20-x}{13}$.

        \part
            \begin{align*}
                \der{\t}{x} &= \dfrac1{1+\bp{\frac{x}8}^2}\cdot\dfrac18 + \dfrac1{1+\bp{\frac{20-x}{13}}^2} \cdot \bp{-\dfrac1{13}}\\
                &= \dfrac{8^2}{8^2 + x^2} \cdot \dfrac18 + \dfrac{13^2}{13^2 + (20-x)^2} \cdot \bp{-\dfrac1{13}}\\
                &= \dfrac{8}{x^2 + 64} + \dfrac{13}{(20-x)^2 + 169}
            \end{align*}

        \part
            At stationary points, $\der{\t}{x} = 0$. Hence,
            \[
                \dfrac{8}{x^2 + 64} + \dfrac{13}{(20-x)^2 + 169} = 0
            \]
            From G.C., we have $x = 10.05$ and $x = -74.05 \tosf{4}$. Since $0 \leq x \leq 20$, we take $x = 10.05$.

            \boxt{$x = 10.05 \tosf{4}$}

        \part
            Since there is only one stationary point in the interval $[0, 20]$, and it is a maximum, the minimum value of $\t$ occurs either at $x = 0$ or $x = 20$, i.e. the extreme ends of the interval.
            \begin{align*}
                x = 0 &: \t = \arctan \frac{20}{13} = 0.994 \tosf{3}\\
                x = 20 &: \t = \arctan \frac{20}8 = 1.19 \tosf{3}
            \end{align*}

            \boxt{The minimum value of $\t$ is $0.994$.}

        \part
            Let the time elapsed be $t$ s. We have $\der{x}{t} = 0.5$. Also note that when $P$ is at the midpoint of the street, $x = 10$.
            \begin{align*}
                \evalder{\der{\t}{t}}{x=10} &= \evalder{\der{\t}{x} \cdot \der{x}{t}}{x=10}\\
                &= \bp{\dfrac{8}{10^2 + 64} + \dfrac{13}{(20-10)^2 + 169}} \cdot 0.5\\
                &= 0.000227 \tosf{3}
            \end{align*}
            
            \boxt{The rate of change of $\t$ is $0.000227$ rad per second.}
\end{document}