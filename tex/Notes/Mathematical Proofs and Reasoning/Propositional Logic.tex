\chapter{Propositional Logic}

Mathematics is a deductive science, where from a set of basic axioms, we prove more complex results. To do so, we often restate a sentence into \vocab{statements}, which are mathematical expressions. One important axiom that all statements obey is the law of the excluded middle.

\begin{axiom}[Law of the Excluded Middle]
    The \vocab{law of the excluded middle} states that either a statement or its negation is true. Equivalently, a statement cannot be both true and false, nor can it be neither true nor false.
\end{axiom}

\section{Statements}

We call a sentence such as ``$x$ is even'' that depends on the value of $x$ a ``statement about $x$''. We can denote this statement more compactly as $P(x)$. For instance, $P(5)$ is the statement ``5 is even'', while $P(72)$ is the statement ``72 is even'', and so forth. We can also write $P(x)$ more compactly as $P$.

The first type of statement we will examine is the conditional statement.

\begin{definition}
    A \vocab{conditional statement} has the form ``if $P$ then $Q$''. Here, $P$ is the \vocab{hypothesis} and $Q$ is the \vocab{conclusion}, denoted by $P \implies Q$. This statement is defined to have the truth table
    \begin{table}[H]
        \centering
        \begin{tabular}{|c|c|c|}
        \hline
        $P$ & $Q$ & $P \implies Q$ \\ \hline\hline
        T & T & T \\ \hline
        T & F & F \\ \hline
        F & T & T \\ \hline
        F & F & T \\ \hline
        \end{tabular}
    \end{table}
    In words, the statement $P \implies Q$ also reads:
    \begin{itemize}
        \item $P$ \vocab{implies} $Q$.
        \item $P$ is a \vocab{sufficient condition} for $Q$.
        \item $Q$ is a \vocab{necessary condition} for $P$.
        \item $P$ \vocab{only if} $Q$.
    \end{itemize}
\end{definition}

To justify the truth table of $P \implies Q$, consider the following example:
\begin{example}[Conditional Statement]
    Suppose I say \[\text{``If it is raining, then the floor is wet.''}\] We can write this as $P \implies Q$, where $P$ is the statement ``it is raining'' and $Q$ is the statement ``the floor is wet''.

    \begin{itemize}
        \item Suppose both $P$ and $Q$ are true, i.e. it is raining, and the floor is wet. It is reasonable to say that I am telling the truth, whence $P \implies Q$ is true.
        \item Suppose $P$ is true but $Q$ is false, i.e. it is raining, and the floor is not wet. Clearly, I am not telling the truth; the floor would be wet if I was. Hence, $P \implies Q$ is false.
        \item Suppose $P$ is false, i.e. it is not raining. Notice that the hypothesis of my claim is not fulfilled; I did not say anything about the floor when it is not raining. Hence, I am not lying, so $P \implies Q$ is true whenever $P$ is false.
    \end{itemize}
\end{example}

Examples of conditional statements in mathematics include
\begin{itemize}
    \item If $\abs{x - 1} < 4$, then $-3 < x < 5$.
    \item If a function $f$ is differentiable, then $f$ is continuous.
\end{itemize}

We now look at biconditional statements. As the name suggests, a biconditional statement comprises two conditional statements: $P \implies Q$ and $Q \implies P$. The conditional statement is much stronger than the conditional statement.

\begin{definition}
    A \vocab{biconditional statement} has the form ``$P$ if and only if'', denoted $P \iff Q$. This statement is defined to have the truth table
    \begin{table}[H]
        \centering
        \begin{tabular}{|c|c|c|}
        \hline
        $P$ & $Q$ & $P \iff Q$ \\ \hline\hline
        T & T & T \\ \hline
        T & F & F \\ \hline
        F & T & F \\ \hline
        F & F & T \\ \hline
        \end{tabular}
    \end{table}

    When $P \iff Q$ is true, we say that $P$ and $Q$ are \vocab{equivalent}, i.e. $P \equiv Q$.
\end{definition}

An equivalent definition of $P \iff Q$ is the statement \[(P \implies Q) \quad \land \quad (Q \implies P).\] This allows us to easily justify the truth table of $P \iff Q$:
\begin{table}[H]
    \centering
    \begin{tabular}{|c|c|c|c|c|}
    \hline
    $P$ & $Q$ & $P \implies Q$ & $Q \implies P$ & $P \iff Q$ \\ \hline\hline
    T & T & T & T & T \\ \hline
    T & F & F & T & F \\ \hline
    F & T & T & F & F \\ \hline
    F & F & T & T & T \\ \hline
    \end{tabular}
\end{table}

Examples of conditional statements in mathematics include
\begin{itemize}
    \item A triangle $ABC$ is equilateral if and only if its three angles are congruent.
    \item $a$ is a rational number if and only if $2a + 4$ is rational.
\end{itemize}

\section{Quantifiers}

We now introduce two important symbols, namely the universal quantifier ($\forall$) and the existential quantifier ($\exists$)

\begin{definition}
    Let $P(x)$ be a statement about $x$, where $x$ is a member of some set $S$ (i.e. $S$ is the \vocab{domain} of $x$). Then the notation \[\forall x \in S, \, P(x)\] means that $P(x)$ is true for every $x$ in the set $S$. The notation \[\exists x \in S, \, P(x)\] means that there exists at least one element of $x$ of $S$ for which $P(x)$ is true.
\end{definition}

\begin{example}
    Let $P(x)$ be the statement ``$x$ is even''. Clearly, the statement \[\forall x \in \ZZ, \, P(x)\] is not true; not all integers are even. However, the statement \[\exists x \in \ZZ, \, P(x)\] is true, because we can find an integer that is even (e.g. $x = 8$).
\end{example}

Note that a statement $P(x)$ does not necessarily have to mention $x$. For instance, we could define $P(x)$ as the statement ``5 is even''. Compare this with how a function $f(x)$ does not necessarily have to ``mention'' $x$, e.g. we could have $f(x) = 5$.

\section{More Types of Statements}

\begin{definition}
    A \vocab{definition} is a true mathematical statement that gives the precise meaning of a word or phrase that represents some object, property or other concepts.
\end{definition}

\begin{definition}
    An \vocab{axiom} is a mathematical statement that does not require proof.
\end{definition}

\begin{definition}
    A \vocab{theorem} is a true mathematical statement that can be proven mathematically.
\end{definition}

\section{Proofs}

Mathematical proofs are convincing arguments expressed in mathematical language, i.e. a sequence of statements leading logically to the conclusion, where each statement is either an accepted truth, or an assumption, or a statement derived from previous statements. Occasionally there will be the clarifying remark, but this is just for the reader and has no logical bearing on the structure of the proof.

\begin{definition}
    A \vocab{proof} is a deductive argument for a mathematical statement, showing that the stated assumptions logically guarantee the conclusion.
\end{definition}

As an example, we will prove the following statement:

\begin{statement}\label{st:Proof-Eg}
    For all $n \in \ZZ^+$, both $n$ and $n^2$ have the same parity.
\end{statement}

We first define what it means for an integer to be even and odd:

\begin{definition}
    An integer $x$ is even if there exists some integer $y$ such that $x = 2y$.
\end{definition}

\begin{definition}
    An integer $x$ is odd if there exists some integer $y$ such that $x = 2y + 1$.
\end{definition}

We are now ready to prove Statement~\ref{st:Proof-Eg}.

\begin{proof}[Proof of Statement~\ref{st:Proof-Eg}]
    Since $n$ can only be either odd or even, we just need to consider the following cases:

    \case{1} Suppose $n$ is even. By definition, there exists some $k \in \ZZ$ such that $n = 2k$. Then \[n^2 = (2k)^2 = 4k^2 = 2\bp{2k^2} = 2a,\] where $a = 2k^2$. Since $a$ is an integer, it follows from our definition that $n^2$ is even. Hence, $n$ and $n^2$ have the same parity.

    \case{2} Suppose $n$ is odd. By definition, there exists some $h \in \ZZ$ such that $n = 2h + 1$. Then \[n^2 = (2h+1)^2 = 4h^2 + 4h + 1 = 2\bp{2h^2 + 2h} + 1 = 2b + 1,\] where $b = 2h^2 + 2h$. Since $b$ is an integer, it follows from our definition that $n^2$ is odd. Hence, $n$ and $n^2$ have the same parity.
\end{proof}

The above proof is an example of a direct proof.

\begin{definition}
    A \vocab{direct proof} is an approach to prove a conditional statement $P \implies Q$. It is a series of valid arguments that starts with the hypothesis $P$, and ends with the conclusion $Q$.
\end{definition}

In the case where we wish to prove a statement false, we can find a counter-example. In providing a counter-example, it must fulfil the hypothesis, but not the conclusion. That is, to show that $P \implies Q$ is false, we must show that $P$ is true but $Q$ is false.

\begin{example}[Counter-Example]
    Consider the statement $c \mid ab$, then $c \mid a$ or $c \mid b$, where $a, b, c \in \ZZ^+$. We can easily find a counter-example to this statement, e.g. $a = 3 \times 37$, $b = 7 \times 37$, $c = 3 \times 7$.
\end{example}