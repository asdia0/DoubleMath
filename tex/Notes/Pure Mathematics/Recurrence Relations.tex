\chapter{Recurrence Relations}

\paragraph{Prerequisites.} \nameref{sec:Sequences-and-Series}
\paragraph{Exercises.} \nameref{sec:A5}

\begin{definition}
    A \vocab{recurrence relation} is an equation that defines a sequence based on a rule that gives the next term as a function of the previous term(s).
\end{definition}

\section{First Order Linear Recurrence Relation with Constant Coefficients}

\begin{definition}
    A \vocab{first order linear recurrence relation with constant coefficients} is a recurrence relation of the form \[u_n = a u_{n-1} + b,\] where $a$ and $b$ are constants. If $b = 0$, the recurrence relation is said to be \vocab{homogeneous}, otherwise it is \vocab{non-homogeneous}.
\end{definition}

\begin{proposition}
    A first order linear recurrence relation with constant coefficients of the form \[u_n = au_{n-1} + b\] has the solution \[u_n = (u_1 + k) a^{n-1} - k,\] where $k = \frac{b}{a - 1}$.
\end{proposition}
\begin{proof}[Proof (Solving by GP)]
    Let $k$ be the constant such that \[u_n + k = a\bp{u_{n-1} + k}.\] Then we clearly have $k = \frac{b}{a - 1}$. We now define a new sequence $v_n = u_n + k$. This turns our recurrence relation into \[v_n = av_{n-1},\] whence $v_n$ is in geometric progression. Thus, $v_n = v_1 a^{n-1}$. Writing this back in terms of $u_n$, we get \[u_n + k = (u_1 + k) a^{n-1} \implies u_n = (u_1 + k)a^{n-1} - k.\]
\end{proof}

\begin{example}[Solving by GP]
    Consider the recurrence relation \[u_1 = 0, \quad u_n = \frac12 u_{n-1} + 10, \quad n > 1.\] Let $k$ be the constant such that \[u_n + k = \frac12 \bp{u_{n-1} + k}.\] Then \[k = \frac{10}{1/2 - 1} = -20.\] We hence have \[u_n - 20 = \frac12 \bp{u_{n-1} - 20},\] whence the sequence $\bc{u_n - 20}$ is in geometric progression with common ratio $\frac12$. Thus, \[u_n - 20 = \bp{u_1 - 20} \bp{\frac12}^{n-1}.\] Rearranging, we obtain the solution \[u_n = -20 \bp{\frac12}^{n-1} + 20 = -40 \bp{\frac12}^n + 20.\]
\end{example}

\begin{fact}[Solving by Procedure]
    $u_n^{(c)} = Ca^n$ is the general solution of the \vocab{associated homogeneous recurrence relation} $u_n = au_{n-1}$ and is called the \vocab{complementary solution}. $u_n^{(p)} = k$ is known as the \vocab{particular solution} of the recurrence relation. The general solution is given by \[u_n = u_n^{(c)} + u_n^{(p)} = Ca^n + k.\]
\end{fact}

\begin{example}[Solving by Procedure]
    Consider the recurrence relation \[u_1 = 0, \quad u_n = \frac12 u_{n-1} + 10, \quad n > 1.\] Observe that the associated homogeneous recurrence relation is $u_n = \frac12 u_{n-1}$. Hence, the complementary solution is \[u_n^{(c)} = C\bp{\frac12}^n\] for some arbitrary constant $C$. Let the particular solution be $u_n^{(p)} = k$. Then \[k = \frac12 k + 10 \implies k = 20.\] Hence, the general solution is \[u_n = u_n^{(c)} + u_n^{(p)} = C\bp{\frac12}^n + 20.\] Using the initial condition $u_1 = 0$, we have \[0 = C\bp{\frac12}^1 + 20 \implies C = -40.\] Thus, \[u_n =-40\bp{\frac12}^n + 20.\]
\end{example}

\section{Second Order Linear Homogeneous Recurrence Relation with Constant Coefficients}

\begin{definition}
    A \vocab{second order linear homogeneous recurrence relation with constant coefficients} is a recurrence relation of the form \[u_n = a u_{n-1} + b u_{n-2},\] where $a$ and $b$ are constants.
\end{definition}

\begin{method}[Solving by Procedure]
    To solve the recurrence relation \[u_n = a u_{n-1} + b u_{n-2},\]

    \renewcommand{\theenumi}{\arabic{enumi}.}%
    \begin{enumerate}
        \item Form the quadratic equation \[x^2 - ax - b = 0.\] This is called the \vocab{characteristic equation}.
        \item Find the roots $\a$ and $\b$ of this characteristic equation.
        \item Then $u_n$ has the \vocab{general solution} 
        \begin{itemize}
            \item $u_n = A\a^n + B\b^n$, if $\a \neq \b$ (distinct roots, may be real or non-real).
            \item $u_n = (A + Bn) \a^n$, if $\a = \b$ (real and equal roots).
            \item $u_n = Ar^n \cos n\t + Br^n \sin n\t$, if $\a = r\e^{\i \t}$ and $\b = r\e^{-\i \t}$ (non-real roots).
        \end{itemize}
    \end{enumerate}
    \renewcommand{\theenumi}{(\alph{enumi})}
\end{method}
\begin{proof}
    For $u_{n+1} = p u_n + q u_{n-1}$ with given initial conditions $u_1$ and $u_2$, let the constant $k$ be such that \[u_{n+1} - k u_n = (p-k) (u_n - ku_{n-1}). \tag{1}\] Note that this is a GP. Comparing coefficients of $u_{n-1}$, we get \[(p-k)k = -q \implies k^2 - pk - q = 0.\] This is the characteristic equation. Now, using the fact that (1) is in GP, we get \[u_{n+1} - ku_n = (p-k)^{n-1} (u_2 - ku_1). \tag{2}\] Let the roots to the characteristic equation be $k = \a$ and $k = \b$. By Vieta's formulas, \[\a + \b = -\bp{\frac{-p}{1}} = p.\] Substituting $k = \a$ into (2), we obtain \[u_{n+1} - \a u_n = \b^{n-1} \bp{u_2 - \a u_1}. \tag{3a}\] Substituting $k = \b$ into (2), we obtain \[u_{n+1} - \b u_n = \a^{n-1} \bp{u_2 - \b u_1}. \tag{3b}\] We now analyse the case where $\a = \b$ and $\a \neq \b$ separately.

    \case{1}[$\a = \b$] Since the two roots are equal, (3a) and (3b) are equivalent. Taking either, \[u_{n+1} - \a u_n = \a^{n-1} (u_2 - \a u_1) \implies \frac{u_{n+1}}{\a^{n-1}} - \frac{u_n}{\a^{n-2}} = u_2 - \a u_1.\] The sequence $\bc{\frac{u_n}{\a^{n-2}}}$ is hence in AP with common difference $u_2 - \a u_1$. Invoking the closed form for AP, we obtain \[\frac{u_n}{\a^{n-2}} = \frac{u_1}{\a^{-1}} + (n-1)(u_2 - \a u_1) \implies u_n = \a^{n-2} \bp{\frac{u_1}{\a^{-1}} + (n-1)(u_2 - \a u_1)}.\] Simplifying, \[u_n = \bs{\bp{\frac{2u_1}{\a} - \frac{u_2}{\a^2}} + \bp{\frac{u_2}{\a^2} - \frac{u_1}{\a}}n} \a^n = (A + Bn) \a^n.\]

    \case{2}[$\a \neq \b$] Observe that $\frac{\text{(3b)} - \text{(3a)}}{\a - \b}$ yields \[u_n = \frac{\a^{n-1} (u_2 - \b u_1) - \b^{n-1} (u_2 - \a u_1)}{\a - \b}.\] Simplifying, we have \[u_n = \bs{\frac{u_2 - \b u_1}{\a (\a - \b)}} \a^n + \bs{\frac{u_2 - \a u_1}{\b (\b - \a)}} \b^n = A\a^n + B\b^n.\]

    We now consider the case where $\a$ and $\b$ are non-real. By the conjugate root theorem, we can write $\a = r\e^{\i \t}$ and $\b = r \e^{-\i \t}$. Substituting this into the above result, we have \[u_n = A\bp{r\e^{\i \t}}^n + B\bp{r\e^{-\i \t}}^n = r^n \bp{A \e^{\i n \t} + B \e^{-\i n \t}}.\] By Euler's identity, \[u_n = r^n \bs{(A+B) \cos n\t + \i (A-B) \sin n\t} = C r^n \cos n\t + D r^n \sin n\t.\]
\end{proof}