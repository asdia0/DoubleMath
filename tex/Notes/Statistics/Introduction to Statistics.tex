\chapter{Introduction to Statistics}

\section{Samples and Populations}

\begin{definition}
    A \vocab{population} (or universe) is all possible subjects that meet certain criteria. It is the entire group of subjects that we are interested in studying.
\end{definition}

\begin{example}[Population]
    If we were interested in the weight of all 12-year-old kids on Earth, then all the kids who meet the criteria (i.e. 12-year-old kids on Earth) would constitute the population.
\end{example}

We want to know something about a population, but there is a good chance that we can never get a very accurate picture of the population simply because it is constantly changing. Not only are populations often in a constant state of flux, practically speaking, we cannot always have access to an entire population for study. Time and cost often get in the way. As a result, we turn to a sample as a substitute of the entire population.

\begin{definition}
    A \vocab{sample} is a subset of the population. A \vocab{random sample} is a sample that is representative of the population.
\end{definition}

\begin{example}[Sample]
    Realistically, there is no way we can accurately weigh all 12-year-old kids on Earth. Instead, we could weigh a sample of 500 12-year-old kids from all around the globe, which would be representative of the population.
\end{example}

\section{Two Categories of Statistics}

\subsection{Descriptive Statistics}

\begin{definition}
    \vocab{Descriptive statistics} refer to statistics used to summarize or describe data from samples and populations.
\end{definition}

\begin{example}
    Suppose we are interested in the test results of a class of students. We could create a data distribution by listing the test scores of all students in the class and looking at it with the idea of getting some intuitive picture of how they are doing. Alternatively, we could simply calculate the mean of the students' test scores. The calculation of the mean represents the use of descriptive statistics, allowing us to summarize or describe our data.

    Another example of descriptive statistics is the reporting of daily temperatures by weather forecasts. The low and high temperatures are usually reported, i.e. we are given the range - a descriptive statistic that summarizes the temperatures throughout the day.
\end{example}

\subsection{Inferential Statistics}

We first look at the difference between statistics and parameters.

\begin{definition}
    A \vocab{statistic} is a characteristic of a sample, while a \vocab{parameter} is a characteristic of the population.
\end{definition}