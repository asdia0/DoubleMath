\chapter{Permutations and Combinations}\label{chap:Permutations-and-Combinations}

\section{Counting Principles}

\begin{fact}[The Addition Principle]
    Let $E_1$ and $E_2$ be two mutually exclusive events. If $E_1$ and $E_2$ can occur in $n_1$ and $n_2$ different ways respectively, then $E_1$ or $E_2$ can occur in $(n_1 + n_2)$ ways.
\end{fact}

\begin{fact}[The Multiplication Principle]
    Consider a task $S$ that can be broken down into two independent ordered stages $S_1$ and $S_2$. If $S_1$ and $S_2$ can occur in $n_1$ and $n_2$ ways respectively, then $S_1$ and $S_2$ can occur in succession in $n_1 n_2$ ways
\end{fact}

Note that both the Addition and Multiplication Principles can be extended to any finite number of events.

\section{Permutations}

\begin{definition}
    A \vocab{permutation} is an arrangement of a number of objects in which the \textbf{order is important}.
\end{definition}

\begin{example}
    ABC, BAC and CBA are possible permutations of the letters `A', `B' and `C'.
\end{example}

\begin{definition}[Factorial]
    The \vocab{factorial} of a non-negative integer $n$ is given by the recurrence relation \[n! = n(n-1)!, \quad 0! = 1.\] Equivalently, \[n! = n(n-1)(n-2)\dots(3)(2)(1), \quad 0! = 1.\]
\end{definition}

\begin{proposition}[Permutations of Objects Taken from Sets of Distinct Objects]
    The number of permutations of $n$ distinct objects, taken $r$ at a time without replacement, is given by \[\perm{n}{r} = \underbrace{n(n-1)(n-2)\dots(n-r+1)}_{\text{$r$ consecutive integers}} = \frac{n!}{(n-r)!},\] where $0 \leq r \leq n$.
\end{proposition}
\begin{proof}
    Suppose we have $n$ distinct objects that we want to fill up $r$ ordered slots with. This operation can be done in $r$ stages
    \begin{itemize}
        \item \textbf{Stage 1.} The number of ways to fill in the first slot is $n$.
        \item \textbf{Stage 2.} After filling in the first slot, the number of ways to fill in the second slot is $n-1$.
        \item \textbf{Stage 3.} After filling in the first and second slots, the number of ways to fill in the third slot is $n-2$.
    \end{itemize}
    This continues until we reach the last stage:
    \begin{itemize}
        \item \textbf{Stage $r$.} After filling all previous $r-1$ slots, the number of ways to fill in the last slot is $n-(r-1) = n-r+1$.
    \end{itemize}
    Thus, by the Multiplication Principle, the number of ways to fill up the $r$ slots are \[n(n-1)(n-2)\dots(n-r+1) = \frac{n!}{(n-r)!}.\]
\end{proof}

\begin{corollary}[Permutations of Distinct Objects in a Row]
    The number of ways to arrange $n$ distinct objects in a row, taken all at a time without replacement, is given by $n!$.
\end{corollary}
\begin{proof}
    Take $r = n$.
\end{proof}

\begin{proposition}[Permutations of Non-Distinct Objects in a Row]
    The number of permutations of $n$ objects in a row, taken all at a time without replacement, of which $n_1$ are of the 1st type, $n_2$ are of the 2nd type, $\dots$, $n_k$ are of the $k$th type, where $n = n_1 + n_2 + \dots + n_k$, is given by \[\frac{n!}{n_1! \, n_2! \dots n_k!}.\]
\end{proposition}
\begin{proof}
    Let $A_i$ be the set of arrangements where objects in the first $i$ groups are now distinguishable, while objects in the remaining groups remain indistinguishable. For instance, $A_1$ is the set of arrangements of $n$ objects in a row, of which $n_2$ are of the 2nd type, $n_3$ are of the 3rd type, $\dots$, $n_k$ are of the $k$th type, while the objects previously of the 1st type are now distinct. We prove the above result by expressing $\abs{A_0}$ in terms of $\abs{A_k}$.

    Suppose we make objects of the 1st type distinct. For each arrangement in $A_0$, the $n_1$ objects of the 1st type can be permuted among themselves in $n_1!$ ways. Hence, \[\abs{A_1} = n_1! \abs{A_0}.\] Next, suppose we make objects of the 2nd type distinct. For each arrangement in $A_1$, the $n_2$ objects of the 2nd type can be permuted among themselves in $n_2!$ ways. Hence, \[\abs{A_2} = n_2! \abs{A_1}.\] Continuing on, we see that \[\abs{A_k} = n_k! \abs{A_{k-1}} = n_k! \, n_{k-1}! \abs{A_{k-2}} = \cdots = n_k! \, n_{k-1}! \dots n_1! \, \abs{A_0}.\] However, by definition, $A_k$ is the set of arrangements of $n$ distinct objects, which we know to be $n!$. Thus, \[\abs{A_0} = \frac{\abs{A_k}}{n_1! \, n_2! \dots n_k!} = \frac{n!}{n_1! \, n_2! \dots n_k!}.\]
\end{proof}
\begin{remark}
    $\frac{n!}{n_1! n_2! \dots n_k!}$ is known as a \vocab{multinomial coefficient}, which is a generalization of the binomial coefficient and is related to the expansion of $(x_1 + x_2 + \dots + x_k)^n$.
\end{remark}

\begin{sample}
    Find the number of different permutations of the letters in the word ``BEEN''.
\end{sample}
\begin{sampans}
    Note that there is 1 `B', 2`E's and 1 `N' in ``BEEN''. Using the above result, the number of different permutations is given by \[\frac{4!}{1! 2! 1!} = 12.\]
\end{sampans}

\begin{proposition}[Circular Permutations]
    The number of permutations of $n$ distinct objects in a circle is given by $(n-1)!$.
\end{proposition}
\begin{proof}
    Fix one object as the reference point. The remaining $n-1$ objects have $(n-1)!$ possible ways to be arranged in the remaining $n-1$ positions around the circle.
\end{proof}

\begin{proposition}[Permutations of Objects Taken from Sets of Distinct Objects with Replacement]
    The number of permutations of $n$ distinct objects, taken $r$ at a time with replacement, is given by $n^r$, where $0 \leq r \leq n$.
\end{proposition}

\section{Combinations}

\begin{definition}
    A \vocab{combination} is a selection of objects from a given set where the order of selection does not matter.
\end{definition}

\begin{proposition}[Combinations of Objects Taken from Sets of Distinct Objects]
    The number of combinations of $n$ distinct objects, taken $r$ at a time without replacement, is given by \[\comb{n}{r} = \binom{n}{r} = \frac{n!}{r! (n-r)!},\] where $0 \leq r \leq n$.
\end{proposition}
\begin{proof}
    Observe the number of ways to choose $r$ objects from $n$ distinct objects is equivalent to the number of permutations of $n$ objects, where $r$ objects are of the first type (chosen) while $n-r$ objects are of the second type (not chosen). Using the formula derived above, we have \[\comb{n}{r} = \frac{n!}{r! (n-r)!}.\]
\end{proof}
\begin{corollary}
    For integers $r$ and $n$, where $0 \leq r \leq n$, \[\perm{n}{r} = \comb{n}{r} \cdot r!.\]
\end{corollary}
\begin{proof}
    Rearrange the above result.
\end{proof}
\begin{corollary}
    For integers $r$ and $n$, where $0 \leq r \leq n$, \[\comb{n}{r} = \comb{n}{n-r}.\]
\end{corollary}
\begin{proof}
    Observe that \[\frac{n!}{r! (n-r)!}\] is invariant under $r \mapsto n - r$.
\end{proof}

\section{Methods for Solving Combinatorics Problems}

Some problems involving permutations and combinations may involve restrictions. When dealing with such problems, one should consider the restrictions first. There are four basic strategies that can be employed to tackle these restrictions.

\begin{recipe}[Fixing Positions]
    When certain objects must be at certain positions, place those objects first.
\end{recipe}

\begin{sample}
    How many ways are there to arrange the letters of the word ``SOCIETY'' if the arrangements start and end with a vowel?
\end{sample}
\begin{sampans}
    We first address the restriction by placing the vowels at the start and end of the arrangement. Since there are 3 vowels in ``SOCIETY'', there are $3 \cdot 2 = 6$ ways to do so. Next, observe there are $5!$ ways to arrange the remaining 5 letters. Thus, by the Multiplication Principle, there are \[6 \cdot 5! = 720\] arrangements that satsify the given restriction.
\end{sampans}

\begin{recipe}[Grouping Method]
    When certain objects must be placed together, group them together as one unit.
\end{recipe}

\begin{sample}
    Find the number of ways the letters of the word ``COMBINE'' can be arranged if all the consonants are to be together.
\end{sample}
\begin{sampans}
    Consider the consonants `C', `M', `B' and `N' as one unit: \[\boxed{\text{C} \quad \text{M} \quad \text{B} \quad \text{N}} \quad \boxed{\text{O}} \quad \boxed{\text{I}} \quad \boxed{\text{E}}.\]
    \begin{itemize}
        \item \textbf{Stage 1.} There are $4!$ ways to arrange the 4 units.
        \item \textbf{Stage 2.} There are $4!$ ways to arrange `C', `M', `B' and `N' within the group.
    \end{itemize}
    Hence, by the Multiplication Principle, the total number of arrangements is \[4! \cdot 4! = 576.\]
\end{sampans}

\begin{recipe}[Slotting Method]
    When certain objects are to be separated, we first arrange the other objects to form barriers before slotting in those to be separated.
\end{recipe}

\begin{sample}
    Find the number of ways the letters of the word ``COMBINE'' can be arranged if all the consonants are to be separated.
\end{sample}
\begin{sampans}
    We begin by arranging the vowels, of which there are $3!$ ways to do so. \[\uparrow \quad \boxed{\text{O}} \quad \uparrow \quad \boxed{\text{I}} \quad \uparrow \quad \boxed{\text{E}} \quad \uparrow.\] Next, we slot the 4 consonants into the 4 gaps in between the vowels (i.e. where the arrows are). There are $4!$ ways to do so. Thus, by the Multiplication Principle, the total number of arrangements is \[3! \cdot 4! = 144.\]
\end{sampans}

\begin{recipe}[Complementary Method]
    If the direct method is too tedious, it is more efficient to count by taking all possibilities minus the complementary sets. This method can also be used for ``at least/at most'' problems.
\end{recipe}

\begin{sample}
    Find the number of ways the letters of the word ``COMBINE'' can be arranged if all the consonants are to be separated.
\end{sample}
\begin{sampans}
    Note that, without restrictions, there are a total of $7!$ ways to arrange the letters in ``COMBINE''. From the previous example, we saw that the number of arrangements where all consonants are together is $576$. Thus, by the complementary method, the number of arrangement where all consonants are separated is \[\text{total} - \text{complementary} = 7! - 576 = 144,\] which matches the answer given in the above example.
\end{sampans}