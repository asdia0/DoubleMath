\chapter{Polar Coordinates}\label{chap:Polar-Coordinates}

\section{Polar Coordinate System}

\begin{definition}
    Let the \vocab{pole} (or origin) be a point $O$ in the plane. Let the \vocab{initial line} (or polar axis) be a half-line starting at $O$. Let $P$ be any other point in the plane. Then $P$ has polar coordinates $(r, \t)$, where $r$ is the distance from $O$ to $P$ and $\t$ is the angle between the initial line and the line $OP$.

    \begin{figure}[H]\tikzsetnextfilename{333}
        \centering
        \begin{tikzpicture}[trim axis left, trim axis right]
            \begin{axis}[
                domain = 0:2*pi,
                samples = 100,
                axis y line=middle,
                axis x line=middle,
                xtick = \empty,
                ytick = \empty,
                xmin=-2,
                xmax=8,
                ymin=-2,
                ymax=8,
                xlabel = {$\t=0$},
                ylabel = {$\t = \frac\pi2$},
                legend cell align={left},
                legend pos=outer north east,
                after end axis/.code={
                    \path (axis cs:0,0) 
                        node [anchor=north east] {$O$};
                    }
                ]
    
                \coordinate (A) at (4, 0);
                \coordinate (B) at (0, 0);
                \coordinate (C) at (3, 4);
    
                \fill (3, 4) circle[radius=2.5pt];
                \draw (0, 0) -- (3, 4);
                \node[anchor=south east] at (1.5, 2) {$r$};
                \node[anchor=south] at (3, 4) {$P(r, \t)$};
    
                \draw pic [draw, angle radius=10mm, "$\t$"] {angle = A--B--C};
            \end{axis}
        \end{tikzpicture}
        \caption{}
    \end{figure}
\end{definition}

There are some conventions regarding the pole and the initial line.
\begin{itemize}
    \item The initial line is usually drawn horizontally to the right.
    \item The polar angle $\t$ is positive if measured in the anti-clockwise direction from the initial line and negative in the clockwise direction.
    \item If $P = 0$, then $r = 0$, and we may use $(0, \t)$ to represent the pole for any value of $\t$.
\end{itemize}

Recall that in the Cartesian coordinate system, each point has a unique representation. This is not the case in the polar coordinate system. For example, the point $(1, \frac54 \pi)$ could also be written as $(1, \frac{13}{4} \pi)$ or as $(-1, \frac14 \pi)$. In general, because a complete anti-clockwise rotation is given by the angle $2\pi$, the point $(r, \t)$ can also be represented by $(r, \t + 2n\pi)$ and $(-r, (2n + 1)\pi)$, where $n$ is any integer.

To avoid this ambiguity, it is common to restrict to $0 \leq \t < 2\pi$ or $-\pi < \t \leq \pi$ and to take $r \geq 0$.

\section{Relationship between the Polar and Cartesian Coordinate Systems}

Suppose the point $P$ has Cartesian coordinates $(x, y)$ and polar coordinates $(r, \t)$. From the figure above, we have \[\cos \t = \frac{x}{r}, \quad \sin \t = \frac{y}{r}.\] Thus, \[x = r\cos \t, \quad y = r\sin\t.\] Note that while the above were deduced from the case where $r > 0$ and $0 < \t < \frac\pi2$, these equations are valid for all values of $r$ and $\t$.

From the figure, we also have \[r^2 = x^2 + y^2, \quad \tan \t = \frac{y}{x},\] which allows us to find $r$ and $\t$ when $x$ and $y$ are known.

\section{Polar Curves}

\begin{definition}
    The \vocab{graph of a polar equation} $r = f(\t)$ consists of all points $P(r, \t)$ whose coordinates satisfy the equation.
\end{definition}

\begin{fact}[Symmetry of Polar Curves]
    \phantom{.}
    \begin{itemize}
        \item If the equation is invariant under $\t \mapsto -\t$, the curve is symmetric about the polar axis.
        \item If the equation is invariant under $r \mapsto -r$, or when $\t \mapsto \t + \pi$, the curve is symmetric about the pole (i.e. the curve remains unchanged when rotated by $180\deg$ about the origin).
        \item If the equation is invariant when $\t \mapsto \pi - \t$, the curve is symmetric about the vertical line $\t = \frac\pi2$.
        \item If $r$ is a function of $\cos n\t$ only, the curve is symmetric about the horizontal half lines $\t = \frac{k}{n} \pi$, $k \in \ZZ$.
        \item If $r$ is a function of $\sin n\t$ only, the curve is symmetric about the vertical half-lines $\t = \frac{2k+1}{2n} \pi$, $k \in \ZZ$.
        \item If only even powers of $r$ occur in the equation, the curve is symmetric about the pole.
    \end{itemize}
\end{fact}

\begin{proposition}[Tangents to Polar Curves]
    The gradient of the tangent to a polar curve $r = f(\t)$ at any point is \[\der{y}{x} = \frac{r' \sin \t + r \cos \t}{r' \cos \t - r \sin \t}.\]
\end{proposition}
\begin{proof}
    Recall that \[x = r\cos\t, \qquad y = r\sin\t.\] Differentiating with respect to $\t$, \[\der{x}{\t} = r' \cos \t - r\sin\t, \qquad \der{y}{\t} = r' \sin \t + r\cos\t.\] Thus, \[\der{y}{x} = \frac{\derx{y}{\t}}{\derx{x}{\t}} = \frac{r' \sin \t + r \cos \t}{r' \cos \t - r \sin \t}.\]
\end{proof}
\begin{remark}
    To find horizontal tangents (i.e. $\derx{y}{x} = 0$), we can solve $\derx{y}{\t} = 0$ (provided $\derx{x}{\t} \neq 0$). Likewise, to find vertical tangents (i.e. $\derx{y}{x}$ undefined), we can solve $\derx{x}{\t} = 0$ (provided $\derx{y}{\t} \neq 0$). Lastly, if we are looking for tangent lines at the pole, where $r = 0$, the equation simplifies to \[\der{y}{x} = \tan \t,\] provided $\derx{r}{\t} \neq 0$.
\end{remark}