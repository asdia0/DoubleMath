\chapter{Sequences and Series}\label{chap:Sequences-and-Series}

\section{Sequences}

\begin{definition}
    A \vocab{sequence} or \vocab{progression} is a set of numbers $u_1, u_2, u_3, \dots, u_n, \dots$ arranged in a defined order according to a certain rule. In general, $u_n$ is called the \vocab{$n$th term}.
\end{definition}

\begin{remark}
    A sequence can be thought of as a function with domain $\ZZ^+$.
\end{remark}

\begin{definition}
    A sequence is said to be \vocab{finite} if it terminates; otherwise it is an \vocab{infinite sequence}.
\end{definition}

\begin{definition}
    If an infinite sequence $u_n$ approaches a unique value $l$ as $n \to \infty$, then the sequence is said to \vocab{converge} to $l$. We say that $l$ is the \vocab{limit} of $u_n$. A sequence that does not converge is said to \vocab{diverge}.
\end{definition}

When describing sequences, one should identifiy
\begin{itemize}
    \item Trends (increasing/decreasing, constant, alternating)
    \item Long-run behaviour of an infinite sequence (convergent or divergent)
\end{itemize}

\section{Series}

\begin{definition}
    A \vocab{series} is the sum of the terms of a sequence $u_n$. The sum to $n$ terms is denoted by $S_n$, i.e. \[S_n = u_1 + u_2 + \dots + u_{n-1} + u_n.\]
\end{definition}

Similar to sequences, a series can be finite or infinite. If a series is infinite, it can further be categorized as convergent or divergent.

\section{Arithmetic Progression}

\begin{definition}
    An \vocab{arithmetic progression} (AP) is a sequence $u_n$ in which each term differs from the preceding term by a constant called the \vocab{common difference}. The first term of an AP is usually denoted by $a$ and the common difference by $d$. Mathematically, \[u_n = a + (n-1)d.\]
\end{definition}

\begin{definition}
    An \vocab{arithmetic series} is obtained by adding the terms of an arithmetic progression.
\end{definition}

\begin{proposition}
    The $n$th term $S_n$ of an arithmetic series is given by \[S_n = \frac{n(a + l)}{2},\] where $l$ is the last term of the AP, i.e. \[l = u_n = a + (n-1) d.\]
\end{proposition}
\begin{proof}
    Note that for all integers $k \in [1, n]$, \[u_k + u_{n - k + 1} = \bs{a + (k-1)d} + \bs{a + (n-k)d} = a + \bs{a + (n-1)d} = a + l.\] Hence, by pairing the $k$th term with the $(n-k+1)$th term, we get \[2S_n = (u_1 + u_n) + (u_2 + u_{n-1}) + \dots + (u_{n-1} + u_2) + (u_n + u_1) = n\bp{a + l} \implies S_n = \frac{n(a + l)}{2}.\]
\end{proof}

\section{Geometric Progression}

\begin{definition}
    A \vocab{geometric progression} (GP) is a sequence $u_n$ in which each term is obtained form the preceding one by multiplying a non-zero constant, called the \vocab{common ratio}. The first term of a GP is usually denoted by $a$ and the common ratio by $r$. Mathematically, \[u_n = ar^{n-1}.\]
\end{definition}

\begin{remark}
    In the case where $r = 1$, the geometric progression becomes an arithmetic progression.
\end{remark}

\begin{definition}
    A \vocab{geometric series} is the sum of the terms of a geometric progression.
\end{definition}

\begin{proposition}
    The $n$th term $S_n$ of a geometric series is given by \[S_n = \frac{a(1 - r^n)}{1 - r},\] where $r \neq 1$. If the series is infinite, the sum to infinity $S_\infty$ exists only if $\abs{r} < 1$ and is given by \[S_\infty = \frac{a}{1 - r}.\]
\end{proposition}
\begin{proof}
    By the definition of a series, we have \[S_n = a + ar + \dots + ar^{n-2} + ar^{n-1}. \tag{1}\] Multiplying both sides by $r$ yields \[r S_n = ar + ar^2 + \dots + ar^{n-1} + ar^{n}. \tag{2}\] Subtracting (2) from (1), we have \[(1-r)S_n = a - ar^{n} \implies S_n = \frac{a(1 - r^{n})}{1 - r}.\]
    
    Suppose $\abs{r} < 1$. In the limit as $n \to \infty$, we have $r^n \to 0$. Hence \[S_\infty = \frac{a(1 - 0)}{1 - r} = \frac{a}{1-r}.\]
\end{proof}

\section{Sigma Notation}

\begin{definition}
    The series $u_k + u_{k+1} + \dots + u_m$ can be denoted using $\S$ (sigma) notation as \[u_k + u_{k+1} + \dots + u_m = \sum_{r=k}^m u_r.\] Here, $r$ is called the \vocab{index}, and can be replaced with any letter. $k$ is the \vocab{lower limit} of $r$, while $m$ is the \vocab{upper limit} of $r$. There are a total of $m - k + 1$ terms in the sum.
\end{definition}

\begin{fact}[Properties of Sigma Notation]
    \begin{gather*}
        \sum_{r = 1}^n \bp{u_r \pm v_r} = \sum_{r = 1}^n u_r \pm \sum_{r = 1}^n v_r.\\
        \sum_{r = 1}^n cu_r = c\sum_{r = 1}^n u_r.\\
        \sum_{r = m}^n u_r = \sum_{r = 1}^n u_r - \sum_{r = 1}^{m-1} u_r, \quad n > m > 1.
    \end{gather*}
\end{fact}

\begin{fact}[Standard Series]
    The sum of the following standard series can be quoted and applied without proof. Note that $m = q - p + 1$ is the number of terms being summed.

    \begin{itemize}
        \item Series of constants \[\sum_{r = p}^q a = ma.\]
        \item Arithmetic series \[\sum_{r = p}^q r = \frac{m}{2} \bp{p + q}.\]
        \item Geometric series \[\sum_{r = p}^q a^r = \frac{a^p \bp{a^m - 1}}{a - 1}.\]
    \end{itemize}
\end{fact}
