\section{Assignment B13}

\begin{problem}
    Two biological cultures, $X$ and $Y$, react with each other, and their volumes at time $t$ are $x$ and $y$ respectively, in appropriate units. Their rates of growth are modelled by the simultaneous equations
    \begin{align*}
        \der{x}{t} &= (2-x)y,\\
        \der{y}{t} &= \frac{y^2}{x}
    \end{align*}
    When $t = 0$, $x = y = 1$.
    \begin{enumerate}
        \item Show that $x = \frac{2y^2}{1+y^2}$.
        \item Find and simplify expressions for $y$ and $x$ in terms of $t$.
        \item Sketch the graph of $y$ against $x$ for $0 < t < \frac\pi2$.
    \end{enumerate}
\end{problem}
\begin{solution}
    \begin{ppart}
        Note that $x, y > 0$ since they represent volume. Also, for $x \in (0, 2)$, we have $\der{x}{t} = (2-x)y > 0$. When $x = 2$, we have $\der{x}{t} = 0$. Hence, $0 < x \leq 2$. Now observe that \[\der{y}{x} = \frac{\derx{y}{t}}{\derx{x}{t}} = \frac{y^2/x}{(2-x)y} = \frac{y}{x(2-x)} \implies \frac1y \der{y}{x} = \frac1{x(2-x)}.\] Integrating both sides with respect to $x$, we get
        \begin{gather*}
            \implies \int \frac1y \d y = \int \frac1{x(2-x)} \d x = \frac12 \int \bp{\frac1x + \frac1{2-x}} \d x\\
            \implies \ln y = \frac12 \bs{\ln x - \ln
            {2-x}} + C_1 \implies y = C_2 \sqrt{\frac{x}{2-x}}.
        \end{gather*}
        At $t = 0$, $x = y = 1$. Hence, \[1 = C_2 \sqrt{\frac{1}{2-1}} \implies C_2 = 1.\] Thus, \[y = \sqrt{\frac{x}{2-x}} \implies x = \frac{2y^2}{1 + y^2}.\]
    \end{ppart}
    \begin{ppart}
        Observe that \[\der{y}{t} = \frac{y^2}{x} = \frac{y^2}{2y^2/(1+y^2)} = \frac12 (1 + y^2) \implies \frac1{1+y^2} \der{y}{t} = \frac12.\] Integrating both sides with respect to $t$, we get \[\int \frac1{1+y^2} \d y = \int \frac12 \d t \implies \arctan y = \frac{t}2 + C \implies y = \tan{\frac{t}2 + C}.\] At $t = 0$, $y = 1$. Hence, \[1 = \tan C \implies C = \frac\pi4,\] whence \[y = \tan{\frac{t}2 + \frac\pi4} = \frac{1-\cos{t + \pi/2}}{\sin{t+\pi/2}} = \frac{1 + \sin t}{\cos t} = \sec t + \tan t.\]

        Observe that \[\der{x}{t} = (2-x)y = (2-x) \sqrt{\frac{x}{2-x}} = \sqrt{x(2-x)} \implies \frac1{\sqrt{x(2-x)}} \der{x}{t} = 1 .\] Integrating both sides with respect to $t$, we get \[\int \frac1{\sqrt{x(2-x)}} \d x = \int 1 \d t \implies 2 \arcsin{\sqrt{\frac{x}{2}}} = t + C_1 \implies x = 2\sin[2]{\frac{t}2 + C_2}.\] At $t = 0$, $x = 1$. Hence, \[1 = 2\sin[2] C_2 \implies C_2 = \frac\pi4.\] Thus, \[x = 2\sin[2]{\frac12 t + \frac\pi4} = 1 - \cos{t + \frac\pi2} = 1 + \sin t.\]
    \end{ppart}
    \begin{ppart}
        Note that $0 < t < \frac\pi2 \implies 1 < x < 2$.
        \begin{center}
            \begin{tikzpicture}[trim axis left, trim axis right]
                \begin{axis}[
                    domain = 1:2,
                    samples = 101,
                    axis y line=middle,
                    axis x line=middle,
                    xtick = {2},
                    ytick = \empty,
                    xlabel = {$x$},
                    ylabel = {$y$},
                    xmin=0,
                    xmax=2.1,
                    ymax=10,
                    ymin=0,
                    legend cell align={left},
                    legend pos=outer north east,
                    after end axis/.code={
                        \path (axis cs:0,0) 
                            node [anchor=north east] {$O$};
                        }
                    ]
                    \addplot[plotRed] {sqrt(x/(2-x))};
        
                    \addlegendentry{$y = \sqrt{x/(2-x)}$};

                    \draw[dotted] (2, 0) -- (2, 10);

                    \draw (1, 1) circle[radius=2.5pt] node[above] {$\bp{1, 1}$};
                \end{axis}
            \end{tikzpicture}
        \end{center}
    \end{ppart}
\end{solution}

\begin{problem}
    Find the general solution of the differential equation \[x\der{y}{x} + 4y - 10x = 0.\]

    Find the particular solution such that $y \to 0$ as $x \to 0$.

    Show, on a single diagram, a sketch of this particular solution and one typical member of the family, $F$ of solution curves for which $\der{y}{x}$ is positive whenever $x$ is positive.

    Show that there is a straight line which passes through the maximum point of every member of $F$ and find its equation.
\end{problem}
\begin{solution}
    \begin{gather*}
        x\der{y}{x} + 4y - 10x = 0 \implies x^4\der{y}{x} + 4x^3y = \der{}{x} \bp{x^4 y} = 10x^4\\
        \implies x^4 y = \int 10 x^4 \d x = 2x^5 + C \implies y = 2x + Cx^{-4}
    \end{gather*}

    As $x \to 0$, $x^{-4} \to \infty$. Hence, $C$ must be 0, whence the particular solution is $y = 2x$.

    Note that \[\der{y}{x} = 2 - 4Cx^{-5} > 0 \implies C < \frac{x^5}2.\] Since $x > 0$, we hence have the constraint $C \leq 0$ for members of $F$.

    \begin{center}
        \begin{tikzpicture}[trim axis left, trim axis right]
            \begin{axis}[
                domain = -5:3,
                restrict y to domain=-10:5,
                samples = 120,
                axis y line=middle,
                axis x line=middle,
                xtick = \empty,
                ytick = \empty,
                xlabel = {$x$},
                ylabel = {$y$},
                legend cell align={left},
                legend pos=outer north east,
                after end axis/.code={
                    \path (axis cs:0,0) 
                        node [anchor=north east] {$O$};
                    }
                ]
                \addplot[plotRed] {2*x};
    
                \addlegendentry{$C=0$};
    
                \addplot[plotBlue] {2*x - 1/x^4};
    
                \addlegendentry{$C=-1$};
            \end{axis}
        \end{tikzpicture}
    \end{center}

    Consider the stationary points of members of $F$. For stationary points, $\der{y}{x} = 0$. Hence, \[x\der{y}{x} + 4y - 10x = 0 \implies 4y - 10 x = 0 \implies y = \frac52 x.\] Differentiating the original differential equation with respect to $x$, we obtain \[x\der{y}{x} + 4y - 10x = 0 \implies \bp{x \der[2]{y}{x} + \der{y}{x}} + 4\der{y}{x} - 10 = 0 \implies \der[2]{y}{x} = \frac{10}{x}.\]
    Note that for members of $F$, we have that $\der{y}{x} > 0$ for $x > 0$. Hence, there are no stationary points when $x > 0$. That is, any stationary point must occur when $x < 0$ (indeed, there is a stationary point when $x = \sqrt[5]{2C} < 0$). Furthermore, when $x < 0$, $\der[2]{y}{x} < 0$. Hence, all stationary points must be a maximum point. Thus, $y = \frac52 x$ passes through the maximum point of every member of $F$.
\end{solution}

\begin{problem}
    \begin{enumerate}
        \item The variables $x$ and $y$ are related by the differential equation \[x^2 \der{y}{x} - 2xy + y = 0.\]
        \begin{enumerate}
            \item Find the general solution of this differential equation, expressing $y$ in terms of $x$.
            \item Find the particular solution for which $y = -\e$ when $x = 1$. Obtain the coordinates of the turning point of the solution curve of this particular solution and sketch the curve for $x > 0$.
        \end{enumerate}
        \item Find the general solution of the differential equation \[\der{y}{x} + xy = \e^x x^2,\] expressing $y$ in terms of $x$.
    \end{enumerate}
\end{problem}
\begin{solution}
    \begin{ppart}
        \begin{psubpart}
            Note that \[x^2 \der{y}{x} - 2xy + y = 0 \implies \frac1y \der{y}{x} = \frac2{x} - \frac1{x^2}.\] Integrating with respect to $x$ on both sides, we get \[\int \frac1y \d y = \int \bp{\frac2{x} - \frac1{x^2}} \d x \implies \ln \abs{y} = 2 \ln \abs{x} + \frac1x + C_1 \implies y = C_2 x^2 \e^{1/x}.\]
        \end{psubpart}
        \begin{psubpart}
            When $x = 1$, $y = -e$. Hence, \[-e = C_2\bp{1^2}\bp{\e^{1}} \implies C_2 = -1 \implies y = -x^2 e^{1/x}.\]

            For stationary points, $\der{y}{x} = 0$. Hence, $y(2x - 1) = 0$, whence $x = \frac12$. Note that we reject $y = 0$ since $e^{1/x} \neq 0$ and $x \neq 0$ due to the presence of a $\frac1x$ term. Hence, $y$ has a stationary point at $(1/2, -\e^2/4)$.

            Differentiating the original differential equation with respect to $x$, we obtain \[x^2 \der[2]{y}{x} - 2y = 0 \implies \der[2]{y}{x} = \frac{2y}{x^2}.\] Hence, at $(1/2, -\e^2/4)$, we have \[\der[2]{y}{x} = \frac{-e^2/2}{1/4} < 0,\] whence it is a turning point.

            \begin{center}
                \begin{tikzpicture}[trim axis left, trim axis right]
                    \begin{axis}[
                        domain = 0:3,
                        restrict y to domain =-10:5,
                        samples = 101,
                        axis y line=middle,
                        axis x line=middle,
                        xtick = \empty,
                        ytick = \empty,
                        xlabel = {$x$},
                        ylabel = {$y$},
                        legend cell align={left},
                        legend pos=outer north east,
                        after end axis/.code={
                            \path (axis cs:0,0) 
                                node [anchor=north east] {$O$};
                            }
                        ]
                        \addplot[plotRed] {-x^2 * e^(1/x)};
            
                        \addlegendentry{$y = -x^2 \e^{1/x}$};

                        \fill (0.5, -e^2/4) circle[radius=2.5 pt] node[above] {$\bp{\frac12, -\frac{\e^2}{4}}$};
                    \end{axis}
                \end{tikzpicture}
            \end{center}
        \end{psubpart}
    \end{ppart}
    \begin{ppart}
        Observe that \[\der{y}{x} + xy = \e^x x^2 \implies e^{\frac12 x^2}\der{y}{x} + x\e^{\frac12 x^2}y = \der{}{x} \bp{\e^{\frac12 x^2} y} = \e^{\frac12 x^2 + x} x^2.\] Thus, \[\e^{\frac12 x^2} y = \int \e^{\frac12 x^2 + x} x^2 \d x.\]

        Suppose $\int e^{\frac12 x^2 + x} x^2 \d x = P(x) e^{\frac12 x^2 + x} + C$ for some function $P(x)$. Differentiating both sides with respect to $x$, we obtain \[x^2 e^{\frac12 x^2 + x} = e^{\frac12 x^2 + x} \bs{(x+1)P(x) + P'(x)},\] whence \[x^2 = (x+1)P(x) + P'(x).\] Thus, $P(x)$ is a polynomial of degree 1. Let $P(x) = ax + b$. For some constants $a$ and $b$. Then \[x^2 = ax^2 + (a+b)x + (a+b).\] Comparing coefficients of $x^2$, $x$ and constant terms, we have $a = 1$ and $a + b = 0 \implies b = -1$. Thus, \[\int x^2 e^{\frac12 x^2 + x} \d x = (x-1)e^{\frac12 x^2 + x} + C.\] Hence, we have \[y = (x-1)e^x + Ce^{-\frac12 x^2}.\]
    \end{ppart}
\end{solution}