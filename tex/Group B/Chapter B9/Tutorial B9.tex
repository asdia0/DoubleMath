\section{Tutorial B9}

\begin{problem}
    Calculate the exact length of each of the arcs of the following curves.

    \begin{enumerate}
        \item $y^3 = x^2$ for $-1 \leq x \leq 1$.
        \item $x = t^2 - 1, \, y = t^3 + 1$ from $t = 0$ to $t = 1$.
        \item $r = a\cos\t$ from $\t = 0$ to $\t = \pi/2$.
    \end{enumerate}
\end{problem}
\begin{solution}
    \begin{ppart}
        Note that \[y^3 = x^2 \implies y = x^{2/3} \implies \der{y}{x} = \frac23 x^{-1/3}.\] Hence, \[\sqrt{1 + \bp{\der{y}{x}}^2} = \sqrt{1 + \bp{\frac23 x^{-1/3}}^2} = \sqrt{1 + \frac49 x^{-2/3}}.\] Thus,
        \begin{gather*}
            \length = \int_{-1}^1 \sqrt{1 + \bp{\der{y}{x}}^2} \d x = \int_{-1}^1 \sqrt{1 + \frac49 x^{-2/3}} \d x = 2 \int_0^1 \sqrt{1 + \frac49 x^{-2/3}} \d x \\
            = 3 \int_0^1 \frac23 x^{-1/3} \sqrt{x^{2/3} + \frac49} \d x = 3 \evalint{\frac23 \bp{x^{2/3} + \frac49}^{3/2}}01 = \frac2{27} \bp{13\sqrt{13} - 8} \units.
        \end{gather*}
    \end{ppart}
    \begin{ppart}
        Since the arc length of a curve is invariant under translation, it suffices to find the arc length of the curve with parametric equations $x = t^2, y = t^3$, $0 \leq t \leq 1$. The Cartesian equation of this curve is $y = x^{3/2}$, $0 \leq x \leq 1$, which is the inverse of $y = x^{2/3}$, $0 \leq x \leq 1$. From part (a), the required arc length is \[\frac12 \cdot \frac2{27} \bp{13\sqrt{13} - 8} = \frac1{27} \bp{13 \sqrt{13} - 8} \units.\]
    \end{ppart}
    \begin{ppart}
        Since $r = a\cos\t$, $0 \leq \t \leq \pi/2$ describes the top half of a circle with centre $(a/2, 0)$ and diameter $a$, the arc length of the curve is $\pi a /2 \units$.
    \end{ppart}
\end{solution}

\begin{problem}
    Find the exact areas of the surfaces generated by completely rotating the following arcs about the (i) $x$-axis and (ii) $y$-axis.

    \begin{enumerate}
        \item The line $2y = x$ between the origin and the point $(4, 2)$.
        \item The curve $x = t^3 - 3t + 2, \, y = 3\bp{t^2 - 1}, \, t \in \RR$ from $t = 1$ to $t = 2$.
    \end{enumerate}
\end{problem}
\begin{solution}
    \begin{ppart}
        \begin{psubpart}
            When rotated about the $x$-axis, the curve forms a cone with slant height $\sqrt{4^2 + 2^2} = 2\sqrt5$ and radius $2$. Hence, the required surface area is $\pi(2)(2\sqrt5) = 4\sqrt5 \pi$ units$^2$.
        \end{psubpart}
        \begin{psubpart}
            When rotated about the $y$-axis, the curve forms a cone with slant height $\sqrt{4^2 + 2^2} = 2\sqrt5$ and radius $4$. Hence, the required surface area is $\pi(4)(2\sqrt5) = 8\sqrt5 \pi$ units$^2$.
        \end{psubpart}
    \end{ppart}
    \begin{ppart}
        Note that \[\der{x}{t} = 3t^2 - 3, \quad \der{y}{t} = 6t.\] Hence, \[\sqrt{\bp{\der{x}{t}}^2 + \bp{\der{y}{t}}^2} = \sqrt{\bp{3t^2 - 3}^2 + (6t)^2} = \sqrt{\bp{3t^2 + 3}^2} = 3t^2 + 3.\]

        \begin{psubpart}
            \begin{gather*}
                \area = 2\pi \int_1^2 y \, \sqrt{\bp{\der{x}{t}}^2 + \bp{\der{y}{t}}^2} \d t = 2\pi \int_1^2 3\bp{t^2 - 1} \bp{3t^2 + 3} \d t \\
                = 18\pi \int_1^2 \bp{t^4 - 1} \d t = 18\pi \evalint{\frac15 t^5 - t}12 = \frac{468}5 \pi \units[2].
            \end{gather*}
        \end{psubpart}
        \begin{psubpart}
            \begin{gather*}
                \area = 2\pi \int_1^2 x \, \sqrt{\bp{\der{x}{t}}^2 + \bp{\der{y}{t}}^2} \d t = 2\pi \int_1^2 \bp{t^3 - 3t + 2} \bp{3t^2 + 3} \d t \\
                = 6\pi \int_1^2 \bp{t^5 - 2t^3 + 2t^2 - 3t + 2} \d t = 6\pi \evalint{\frac16 t^6 - \frac24 t^4 - \frac23 t^3 - \frac32 t^2 + 2t}12 = 31 \pi \units[2].
            \end{gather*}
        \end{psubpart}
    \end{ppart}
\end{solution}

\begin{problem}
    The section of the curve $y = \e^x$ between $x = 0$ and $x = 1$ is rotated through one revolution about
    \begin{enumerate}
        \item the $x$-axis.
        \item the $y$-axis.
    \end{enumerate}
    Find the numerical values of the areas of the surfaces obtained.
\end{problem}
\begin{solution}
    \begin{ppart}
        \[\area = 2\pi \int_0^1 y \, \sqrt{1 + \bp{\der{y}{x}}^2} \d x = 2\pi \int_0^1 \e^x \sqrt{1 + \e^{2x}} \d x = 22.9 \units[2] \tosf{3}.\]
    \end{ppart}
    \begin{ppart}
        Note that $y = \e^x \implies x = \ln y$ and $\der{y}{x} = \e^x \implies \der{x}{y} = \e^{-x}$. \[\area = 2\pi \int_1^\e x \, \sqrt{1 + \bp{\der{x}{y}}^2} \d y = 2\pi \int_0^1 \ln y \, \sqrt{1 + \e^{-2x}} \d x = 7.05 \units[2] \tosf{3}.\]
    \end{ppart}
\end{solution}

\clearpage
\begin{problem}
    The curve $y^2 = \frac13 x(1-x)^2$ has a loop between $x = 0$ and $x = 1$. Prove that the total length of the loop is $\frac{4\sqrt{3}}{3}$.
\end{problem}
\begin{solution}
    Since the curve is even with respect to $y$, it is symmetric about the $x$-axis. We thus only consider the part of the curve above the $x$-axis, i.e. $y \geq 0$, where $y = (1-x)\sqrt{x/3}$. Differentiating, \[\der{y}{x} = \frac1{\sqrt3} \bp{-\sqrt{x} + \frac{1-x}{2\sqrt{x}}} = \frac{1-3x}{2\sqrt{3x}} \implies 1 + \bp{\der{y}{x}}^2 = 1 + \frac{(1-3x)^2}{12x} = \frac{(1+3x)^2}{12x}.\] Thus, \[\length = 2 \int_0^1 \sqrt{1 + \bp{\der{y}{x}}^2} \d x = 2 \int_0^1 \frac{1+3x}{\sqrt{12x}} \d x = \frac{1}{\sqrt{12}} \evalint{\frac{x^{1/2}}{1/2} + \frac{3x^{3/2}}{3/2}}01 = \frac{4\sqrt3}3 \units.\]
\end{solution}

\begin{problem}
    The tangent at a point $P$ on the curve $x = a\bp{t - \frac13 t^3}, \, y = at^2$ cuts the $x$-axis at $T$. Prove that the distance of the point $T$ from the origin $O$ is half the length of the arc $OP$.
\end{problem}
\begin{solution}
    Let $P$ be the point on the curve with parameter $t = t_P$. Note that \[\der{x}{t} = a\bp{1 - t^2}, \quad \der{y}{t} = 2at.\] Thus, \[\bp{\der{x}{t}}^2 + \bp{\der{y}{t}}^2 = \bs{a\bp{1-t^2}}^2 + (2at)^2 = a^2 \bp{t^2 + 1}^2.\] Thus,
    \begin{gather*}
        \text{Length of arc $OP$} = \int_0^{t_P} \sqrt{\bp{\der{x}{t}}^2 + \bp{\der{y}{t}}^2} \d t = a \int_0^{t_P} \bp{t^2 + 1} \d t \\
        = a \evalint{\frac{t^3}3 + t}0{t_P} = a \bp{\frac{t_P^3}3 + t_P} \units.
    \end{gather*}
    Note that \[\der{y}{x} = \frac{\derx{y}{t}}{\derx{x}{t}} = \frac{2at}{a\bp{1 - t^2}} = \frac{2t}{1-t^2}.\] Hence, the equation of the tangent at $P$ is given by \[y - at_P^2 = \frac{2t_P}{1 - t_P^2} \bs{x - a\bp{t_P - \frac{t_P^3}3}}.\] At $T$, $x = OT$ and $y = 0$. Hence, \[0 - at_P^2 = \frac{2t_P}{1 - t_P^2} \bs{OT - a\bp{t_P - \frac{t_P^3}3}},\] whence
    \begin{gather*}
        OT = \frac{-a t_P^2 \bp{1 - t_P^2}}{2t_P} + a\bp{t_P - \frac{t_P^3}3} = \frac{a}{2} \bs{\bp{-t_P + t_P^3} + \bp{2t_P - \frac{2t_P^3}3}}\\
        = \frac{a}2 \bp{\frac{t_P^3}3 + t_P} = \frac{OP}2.
    \end{gather*}
\end{solution}

\begin{problem}
    Sketch the curve whose parametric equations are $x = a\cos^3 \t, \, y = a\sin^3 \t$, $a > 0$.

    \begin{enumerate}
        \item Find the total length of the curve.
        \item The portion of the curve in the first quadrant is revolved through four right angles about the $x$-axis. Prove that the area of the surface thus formed is $\frac65 \pi a^2$.
    \end{enumerate}
\end{problem}
\begin{solution}
    \begin{center}\tikzsetnextfilename{324}
        \begin{tikzpicture}[trim axis left, trim axis right]
            \begin{axis}[
                domain = 0:10,
                samples = 101,
                axis y line=middle,
                axis x line=middle,
                xtick = {-1, 1},
                ytick = {-1, 1},
                xticklabels = {$-a$, $a$},
                yticklabels = {$-a$, $a$},
                xlabel = {$x$},
                ylabel = {$y$},
                xmin=-1.1,
                xmax=1.1,
                ymin=-1.1,
                ymax=1.1,
                legend cell align={left},
                legend pos=outer north east,
                after end axis/.code={
                    \path (axis cs:0,0) 
                        node [anchor=north east] {$O$};
                    }
                ]
                \addplot[plotRed] ({cos(\x r)^3}, {sin(\x r)^3});
    
                \addlegendentry{$x = a\cos^3 \t, \, y = a\sin^3 \t$};
            \end{axis}
        \end{tikzpicture}
    \end{center}

    \begin{ppart}
        By symmetry, we only consider the length of the curve in the first quadrant. Note that $x = 0 \implies \t = \pi/2$ and $x = a \implies \t = 0$. Also, \[\der{x}{\t} = -3a\cos^2 \t \sin \t, \quad \der{y}{\t} = 3a\sin^2 \t \cos \t.\] Hence,
        \begin{gather*}
            \bp{\der{x}{\t}}^2 + \bp{\der{y}{\t}}^2 = (-3a\cos^2 \t \sin \t)^2 + (3a\sin^2 \t \cos \t)^2 \\= 9a^2\bp{\cos^4 \t \sin^2 \t + \sin^4 \t \cos^2 \t} = \bp{3a\cos \t \sin \t}^2.
        \end{gather*}
        Thus,
        \begin{gather*}
            \length = 4 \int_0^{\pi/2} \sqrt{\bp{\der{x}{\t}}^2 + \bp{\der{y}{\t}}^2} \d \t = 12a \int_0^{\pi/2} \cos\t\sin\t \d \t \\
            = 12a \evalint{\frac{\sin^2 \t}2}0{\pi/2} = 6a \units.
        \end{gather*}
    \end{ppart}
    \begin{ppart}
        \begin{gather*}
            \area = 2\pi \int_0^{\pi/2} x \, \sqrt{\bp{\der{x}{\t}}^2 + \bp{\der{y}{\t}}^2} \d \t = 2\pi \int_0^{\pi/2} a \cos^3 \t \bp{3a \cos \t \sin \t} \d t \\
            = 6\pi a^2 \int_0^{\pi/2} \sin \t \cos^4 \t \d \t = 6\pi a^2 \evalint{-\frac{\cos^5 \t}{5}}0{\pi/2} = \frac65 \pi a^2 \units[2].
        \end{gather*}
    \end{ppart}
\end{solution}

\clearpage
\begin{problem}
    The parametric equations of a curve are given by \[x = \tan t - t, \, y = \ln \sec t, \, t \in \bp{-\frac\pi2, \frac\pi2}.\]

    \begin{enumerate}
        \item Sketch the curve.
        \item Prove that the arc length of the curve measured from the origin to the point $\bp{1 - \frac\pi4, \frac12 \ln 2}$ is $\sqrt2 - 1$.
        \item The arc in (b) is rotated about the $x$-axis through an angle of $360 \deg$. Find the exact surface area formed.
    \end{enumerate}
\end{problem}
\begin{solution}
    \begin{ppart}
        \begin{center}\tikzsetnextfilename{325}
            \begin{tikzpicture}[trim axis left, trim axis right]
                \begin{axis}[
                    domain = -pi/2+0.01:pi/2-0.01,
                    samples = 1000,
                    axis y line=middle,
                    axis x line=middle,
                    xtick = \empty,
                    ytick = \empty,
                    xlabel = {$x$},
                    ylabel = {$y$},
                    legend cell align={left},
                    legend pos=outer north east,
                    after end axis/.code={
                        \path (axis cs:0,0) 
                            node [anchor=north] {$O$};
                        }
                    ]
                    \addplot[plotRed, unbounded coords=jump] ({tan(\x r) - x}, {ln(sec(\x r)});
        
                    \addlegendentry{$x = \tan t - t, \, y = \ln \sec t$};
                \end{axis}
            \end{tikzpicture}
        \end{center}
    \end{ppart}
    \begin{ppart}
        Note that $x = 0 \implies t = 0$ and $x = 1 - \pi/4 \implies = t = \pi/4$. Further, \[\der{x}{t} = \sec^2 t - 1 = \tan^2 t, \quad \der{y}{t} = \tan t.\] Thus, \[\bp{\der{x}{t}}^2 + \bp{\der{y}{t}}^2 = \bp{\tan^2 t}^2 + (\tan t)^2 = \tan^2 t \bp{\tan^2 t + 1} = \tan^2 t \sec^2 t.\] Hence, \[\length = \int_0^{\pi/4} \sqrt{\bp{\der{x}{t}}^2 + \bp{\der{y}{t}}^2} \d t = \int_0^{\pi/4} \tan t \sec t \d t = \evalint{\sec t}{0}{\pi/4} = \sqrt{2} - 1 \units.\]
    \end{ppart}
    \begin{ppart}
        We have \[\area = 2\pi \int_0^{\pi/4} y \, \sqrt{\bp{\der{x}{t}}^2 + \bp{\der{y}{t}}^2} \d t = 2\pi \int_0^{\pi/4} \ln \sec t \cdot \tan t \sec t \d t.\] Integrating by parts, 
        \[\begin{array}{r c @{\hspace*{1.0cm}} c}\toprule
            & D & I \\\cmidrule{1-3}
            + & \ln \sec t & \tan t \sec t \\
            - & \tan t & \sec t \\\bottomrule
        \end{array}\] Thus,
        \begin{gather*}
            \area = 2\pi \bs{\evalint{\sec t \ln \sec t}{0}{\pi/4} - \int_0^{\pi/4} \tan t \sec t \d t} = 2\pi \bs{\sqrt2 \ln \sqrt2 - \bp{\sqrt2 - 1}} \\
            = \sqrt2 \pi \bp{\ln2 - 2 + \sqrt2} \units[2].
        \end{gather*}
    \end{ppart}
\end{solution}

\begin{problem}
    \begin{center}\tikzsetnextfilename{326}
        \begin{tikzpicture}[trim axis left, trim axis right]
            \begin{axis}[
                domain = -2:2,
                samples = 101,
                axis y line=middle,
                axis x line=middle,
                xtick = \empty,
                ytick = \empty,
                xlabel = {$x$},
                ylabel = {$y$},
                ymax=9,
                ymin=-3,
                xmin=-2.1,
                xmax=2.1,
                legend cell align={left},
                legend pos=outer north east,
                after end axis/.code={
                    \path (axis cs:0,0) 
                        node [anchor=north east] {$O$};
                    }
                ]
                \addplot[black] {x^2};

                \draw[<->] (-2, 4) -- (0, 4);
                \draw[<->] (2, 4) -- (0, 4);
                \draw[<->] (-2, 4) -- (-2, 0);
                \draw[<->] (2, 4) -- (2, 0);

                \node[anchor=south] at (-1, 4) {$S$};
                \node[anchor=south] at (1, 4) {$S$};
                \node[anchor=west] at (-2, 2) {$H$};
                \node[anchor=east] at (2, 2) {$H$};
            \end{axis}
        \end{tikzpicture}
    \end{center}

    The diagram shows a cable for a suspension bridge, which has the shape of a parabola with equation $y = kx^2$. The suspension bridge has a total span $2S$ and the height of the cable relative to the lowest point is $H$ at each end. Show that the total length of the cable is $L = 2\int_0^S \sqrt{1 + \frac{4H^2}{S^4} x^2} \d x$.

    \begin{enumerate}
        \item Engineers from country $A$ proposed a suspension bridge across a strait of 8 km wide to country $B$. The plan included suspension towers 380 m high at each end. Find the length of the parabolic cable for this proposed bridge to the nearest metre.
        \item By using the result $\der{}{x} \ln{x + \sqrt{a^2 + x^2}} = \frac1{\sqrt{a^2 + x^2}}$ or otherwise, find $L$ in terms of $S$ and $H$.
    \end{enumerate}
\end{problem}
\begin{solution}
    By symmetry, we only need to consider the length of the curve where $x \geq 0$. Since $(S, H)$ is on the curve, $H = kS^2 \implies k = \frac{H}{S^2}$. Note that \[y = kx^2 \implies \der{y}{x} = 2kx = \frac{2H}{S^2}x.\] Hence, 
    \[L = 2 \int_0^S \sqrt{1 + \bp{\der{y}{x}}^2} \d x = 2 \int_0^S \sqrt{1 + \frac{4H^2}{S^4} x^2} \d x.\]

    \begin{ppart}
        Note that $2S = 8000 \implies S = 4000$ and $H = 380$. Hence, \[L = 2 \int_0^{4000} \sqrt{1 + \frac{4(380)^2}{(4000)^4} x^2} \d x = 8048 \text{ (to the nearest integer)}.\] The bridge is thus 8048 m long.
    \end{ppart}
    \begin{ppart}
        Consider the integral $I = \int \sqrt{1 + (kx)^2} \d x$. Under the substitution $kx = \tan \t$, we get \[I = \int \sqrt{1 + (kx)^2} \d x = \frac1k \int \sqrt{1 + \tan^2 \t} \sec^2 \t \d \t = \frac1k \int \sec^3 \t \d \t.\] Integrating by parts,
        \[\begin{array}{r c @{\hspace*{1.0cm}} c}\toprule
            & D & I \\\cmidrule{1-3}
            + & \sec \t & \sec^2 \t \\
            - & \sec \t \tan \t & \tan t \\\bottomrule
        \end{array}\] Hence,
        \begin{gather*}
            kI = \sec\t \tan \t - \int \sec \t \tan^2 \t \d \t = \sec\t \tan \t - \int \sec \t \bp{\sec^2 \t - 1} \d \t\\
            = \sec\t \tan \t - \int \sec^3 \t \d \t + \int \sec \t \d \t = \sec \t \tan \t - kI + \ln \abs{\sec \t + \tan \t}.
        \end{gather*}
        Thus, \[I = \frac{\sec \t \tan \t + \ln \abs{\sec \t + \tan \t}}{2k} + C = \frac1{2k} \bs{kx\sqrt{(kx)^2 + 1} + \ln \abs{\sqrt{(kx)^2 + 1} + kx}} + C.\] In our case, $k = \frac{2H}{S^2} > 0$. Hence,
        {\allowdisplaybreaks
        \begin{align*}
            L &= 2\evalint{\frac1{2} \bp{\frac{S^2}{2H}} \bs{\bp{\frac{2H}{S^2}x}\sqrt{\bp{\frac{2H}{S^2}x}^2 + 1} + \ln \bp{\sqrt{\bp{\frac{2H}{S^2}x}^2 + 1} + \frac{2H}{S^2}x}}}{0}{S}\\
            &= \frac{S^2}{2H} \bs{\bp{\frac{2H}{S}}\sqrt{\bp{\frac{2H}{S}}^2 + 1} + \ln \bp{\sqrt{\bp{\frac{2H}{S}}^2 + 1} + \frac{2H}{S}}}\\
            &= \sqrt{4H^2 + S^2} + \frac{S^2}{2H} \ln{\frac{\sqrt{4H^2 + S^2} + 2H}{S}}.
        \end{align*}}
    \end{ppart}
\end{solution}

\begin{problem}
    Sketch the semicircle with equation $x^2 + (y-b)^2 = a^2$, $y \geq b$ where $a$ and $b$ are positive constants.

    A solid is formed by rotating the region bounded by the semicircle and its diameter on the line $y = b$ about the $x$-axis through 4 right angles. Find the total surface area of the solid.
\end{problem}
\begin{solution}
    \begin{center}\tikzsetnextfilename{327}
        \begin{tikzpicture}[trim axis left, trim axis right]
            \begin{axis}[
                domain = -10:10,
                samples = 1000,
                axis y line=middle,
                axis x line=middle,
                xtick = {-2, 2},
                ytick = {1},
                xticklabels = {$-a$, $a$},
                yticklabels = {$b$},
                xlabel = {$x$},
                ylabel = {$y$},
                ymin=0,
                ymax=4,
                xmin=-2.5,
                xmax=2.5,
                legend cell align={left},
                legend pos=outer north east,
                after end axis/.code={
                    \path (axis cs:0,0) 
                        node [anchor=north] {$O$};
                    }
                ]
                \addplot[plotRed] {1 + sqrt(4 - x^2)};
    
                \addlegendentry{$x^2 + (y-b)^2 = a^2$};

                \draw[plotRed] (2,1) arc(0:180:2);

                \draw[dotted] (-2, 0) -- (-2, 1);
                \draw[dotted] (2, 0) -- (2, 1);
                \draw[dotted] (-2, 1) -- (2, 1);
            \end{axis}
        \end{tikzpicture}
    \end{center}

    Observe that \[x^2 + (y-b)^2 = a^2 \implies y = b + \sqrt{a^2- x^2},\] whence \[\der{y}{x} = \frac{-2x}{2\sqrt{a^2 - x^2}} = -\frac{x}{\sqrt{a^2 - x^2}}.\] Thus, \[1 + \bp{\der{y}{x}}^2 = 1 + \frac{x^2}{a^2 - x^2} = \frac{a^2}{a^2 - x^2}.\] Hence,
    \begin{gather*}
        \area = 2\pi \int_{-a}^{a} y \sqrt{1 + \bp{\der{y}{x}}^2} \d x + 2\pi(b)(2a) = 4\pi \int_0^{a} y \sqrt{1 + \bp{\der{y}{x}}^2} \d x + 4\pi ab\\
        = 4\pi \int_0^a \bp{b + \sqrt{a^2 - x^2}} \bp{\frac{a}{\sqrt{a^2 - x^2}}} \d x + 4\pi ab = 4\pi a \int_0^a \bp{\frac{b}{\sqrt{{a^2-x^2}}} + 1} \d x + 4\pi ab\\
        = 4\pi a \evalint{b\arcsin \frac{x}{a} + x}{0}{a} + 4\pi ab = \bp{2\pi^2 ab + 4\pi a^2 + 4\pi ab} \units[2]
    \end{gather*}
\end{solution}

\begin{problem}
    Using polar coordinates with pole $O$, the curve $C$ has the equation $r = a\e^{\t/k}$, where $a$ and $k$ are positive constants and $0 \leq \t \leq 2\pi$. The points $A$ and $B$ on the curve corresponds to $\t = 0$ and $\t = \b$ respectively where $0 < \b < \pi$. The length of the arc $AB$ is denoted by $q$ and the area of the sector $OAB$ is denoted by $Q$.

    \begin{enumerate}
        \item Show that $Q = \frac14 k a^2 \bp{\e^{2\b/k} - 1}$.
        \item Show that $q = a(1 + k^2)^{1/2}\bp{\e^{\b/k} - 1}$.
        \item Deduce from the results of parts (a) and (b) that, for large values of $k$, $\frac{Q}{q} \approx \frac12 a$.
        \item Draw a sketch of $C$ for the case where $k$ is large and explain how the result in part (c) can be deduced from the sketch.
    \end{enumerate}
\end{problem}
\begin{solution}
    \begin{ppart}
        \[Q = \frac12 \int_0^\b r^2 \d \t = \frac{a^2}2 \int_0^\b \e^{2\t/k} \d \t = \frac{a^2}2 \evalint{\frac{\e^{2\t/k}}{2/k}}0\b = \frac{a^2k}4 \bp{\e^{2\b/k} - 1}.\]
    \end{ppart}
    \begin{ppart}
        Note that \[r = a\e^{\t/k} \implies \der{r}{\t} = \frac{a\e^{\t/k}}k  = \frac{r}{k}.\] Hence,
        \begin{gather*}
            q = \int_0^\b \sqrt{r^2 + \bp{\der{r}{\t}}^2} \d \t = \int_0^\b \sqrt{r^2 + \frac{r^2}{k^2}} \d \t = \sqrt{1 + k^{-2}} \int_0^\b r \d \t\\
            = \sqrt{1 + k^{-2}} \int_0^\b a\e^{\t/k} \d \t = a\sqrt{1 + k^{-2}} \evalint{\frac{\e^{\t/k}}{1/k}}0\b = a \sqrt{k^2 + 1} \bp{\e^{\b/k} - 1}.
        \end{gather*}
    \end{ppart}
    \begin{ppart}
        \[\lim_{k \to \infty} \frac{Q}{q} = \lim_{k \to \infty} \frac{\frac14 a^2 k \bp{\e^{2\b/k} - 1}}{a \sqrt{k^2 + 1} \bp{\e^{\b/k} - 1}} = \frac{a}4 \lim_{k \to \infty} \bp{\frac{k}{\sqrt{k^2 + 1}}} \lim_{k \to \infty} \bp{\frac{\e^{2\b/k} - 1}{\e^{\b/k} - 1}}.\] Now observe that \[\lim_{k \to \infty} \bp{\frac{k}{\sqrt{k^2 + 1}}} = \lim_{k \to \infty} \frac{1}{1 + k^{-2}} = 1,\] and by the difference of squares identity, \[\lim_{k \to \infty} \bp{\frac{e^{2\b/k} - 1}{e^{\b/k} - 1}} = \lim_{k \to \infty} \bp{\e^{\b/k} + 1} = 2.\] Hence, \[\lim_{k \to \infty} \frac{Q}{q} = \frac{a}{2}.\]
    \end{ppart}
    \begin{ppart}
        Note that \[\lim_{k \to \infty} r = \lim_{k \to \infty} a\e^{\t/k} = a.\]

        \begin{center}\tikzsetnextfilename{328}
            \begin{tikzpicture}[trim axis left, trim axis right]
                \begin{axis}[
                    domain = 0:2*pi,
                    samples = 100,
                    axis y line=middle,
                    axis x line=middle,
                    xtick = \empty,
                    ytick = \empty,
                    xmin=-2,
                    xmax=2,
                    ymin=-1.7,
                    ymax=1.7,
                    xlabel = {$\t=0$},
                    ylabel = {$\t = \frac\pi2$},
                    legend cell align={left},
                    legend pos=outer north east,
                    after end axis/.code={
                        \path (axis cs:0,0) 
                            node [anchor=north east] {$O$};
                        }
                    ]
                    \addplot[color=plotRed,data cs=polarrad] {1};
        
                    \addlegendentry{$r = a$};

                    \coordinate[label=below right:$A$] (A) at (1, 0);
                    \coordinate[label=above left:$B$] (B) at (-1/2, 0.866);
                    \coordinate (O) at (0, 0);

                    \fill (A) circle[radius=2.5 pt];
                    \fill (B) circle[radius=2.5pt];
                    \draw[dotted] (O) -- (B);

                    \draw pic [draw, angle radius=7mm, "$\b$"] {angle = A--O--B};
                \end{axis}
            \end{tikzpicture}
        \end{center}

        As $k \to \infty$, the curve becomes a circle. Hence, $Q$ is the area of a sector with angle $\b$, and $q$ is the arc length of a sector with angle $\b$. Thus, \[\frac{Q}{q} = \bp{\frac{\b}{2\pi} \cdot \pi a^2} \Bigg/ \bp{\frac{\b}{2\pi} \cdot 2\pi a} = \frac{a}2.\]
    \end{ppart}
\end{solution}