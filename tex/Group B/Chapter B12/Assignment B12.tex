\section{Assignment B12}

\begin{problem}
    The curve $y = f(x)$ passes through the origin and has gradient given by \[\der{y}{x} = \frac{3x^2-4x+1}{2y-5}.\]
    \begin{enumerate}
        \item Find $f(x)$.
        \item By considering $\der{y}{x}$, deduce the coordinates of the point on the curve where it is tangent to the $x$-axis.
        \item Determine the interval of validity for the solution.
    \end{enumerate}
\end{problem}
\begin{solution}
    \begin{ppart}
        \begin{gather*}
            \der{y}{x} = \frac{3x^2-4x+1}{2y-5} \implies (2y-5)\der{y}{x} = 3x^2 - 4x + 1\\
            \implies \int (2y-5) \d y = \int \bp{3x^2 - 4x + 1} \d x \implies y^2 - 5y = x^3 - 2x^2 + x + C_1.
        \end{gather*}
        Note that $x^3 - 2x^2 + x = x(x-1)^2$. Hence, \[y^2 - 5y - x(x-1)^2 + C_2 = 0 \implies y = \frac{5 \pm \sqrt{4x(x-1)^2 + C_3}}{2}.\] Since the curve passes through the origin $(0, 0)$, we have \[0 = \frac{5 - \sqrt{C_3}}{2} \implies C_3 = 25.\] Thus, \[f(x) = \frac{5-\sqrt{4x(x-1)^2 + 25}}{2}.\] Note that we reject the positive branch since $f(x) > 0$ in that case.
    \end{ppart}
    \begin{ppart}
        When the curve is tangent to the $x$-axis, we have $\der{y}{x} = 0$ and $y = 0$. Note that \[\der{y}{x} = 0 \implies 3x^2 - 4x + 1 = 0 \implies x = \frac13 \lor 1.\] Also note that \[y = 0 \implies 4x(x-1)^2 = 0 \implies x = 0 \lor 1.\] Hence, the required point is $(1, 0)$.
    \end{ppart}
    \begin{ppart}
        Since the square root function is defined only on the non-negative reals, we require \[4x(x-1)^2 + 25 \geq 0 \implies x \geq -1.24.\] Thus, the interval of validity is $[-1.24, \infty)$.
    \end{ppart}
\end{solution}

\begin{problem}
    \begin{enumerate}
        \item Using the substitution $y = ux$, find the general solution of the differential equation \[\der{y}{x} = \frac{x+y}{x},\] where $x > 0$.
        \item Find the particular solution of the differential equation for which $y = -1$ when $x = 1$.
        \item Without sketching the curve of the solution in (b), determine the number of stationary points the solution curve has.
    \end{enumerate}
\end{problem}
\begin{solution}
    \begin{ppart}
        Note that \[y = ux \implies \der{y}{x} = \der{u}{x} x + u.\] Substituting this into the given differential equation, 
        \begin{gather*}
            \der{y}{x} = \frac{x+y}{x} \implies\der{u}{x} x + u = \frac{x + ux}{x} \implies \der{u}{x} = \frac1x \\
            \implies u = \ln (x) + C \implies y = x\ln x + Cx.
        \end{gather*}
    \end{ppart}
    \begin{ppart}
        Evaluating the solution at $x = 1$ and $y = -1$, we get \[-1 = 1 \ln 1 + C (1) \implies C = -1.\] Thus, \[y = x \ln x - x.\]
    \end{ppart}
    \begin{ppart}
        Note that \[\der{y}{x} = \frac{x+y}{x} = \frac{x + x\ln x - x}{x} = \ln x.\] Since $\ln x$ only has one root (at $x = 1$), the solution curve has only 1 stationary point.
    \end{ppart}
\end{solution}

\begin{problem}
    As a tree grows, the rate of increase of its height, $h$ m, with respect to time, $t$ years after planting, is modelled by the differential equation \[\der{h}{t} = \frac1{10} \sqrt{16 - \frac12 h}.\] The tree is planted as a seedling of negligible height, so that $h = 0$ when $t = 0$.
    \begin{enumerate}
        \item State the maximimum height of the tree, according to this model.
        \item Find an expression for $t$ in terms of $h$, and hence find the time the tree takes to reach half of its maximum height.
    \end{enumerate}
\end{problem}
\begin{solution}
    \begin{ppart}
        Note that $\der{h}{t} \geq 0 \implies h \leq 32$. Thus, the maximum height of the tree is 32 m.
    \end{ppart}
    \begin{ppart}
        \begin{gather*}
            \der{h}{t} = \frac1{10} \sqrt{16 - \frac12 h} \implies 10 \bp{16 - \frac12 h}^{-1/2} \der{h}{t} = 1 \\
            \implies 10 \int \bp{16 - \frac12 h}^{-1/2} \d h = \int 1 \d t \implies -10 \sqrt{16 - \frac12 h} + C = t.
        \end{gather*}
        Since $h = 0$ when $t = 0$, we have \[-10\sqrt{16} + C = 0 \implies C = 40.\] Thus, \[t = 40-10\sqrt{16 - \frac12 h}.\]

        When $h = \frac{32}2 = 16$, we have \[t = 40 - 10\sqrt{16 - \frac12(16)} = 11.7 \tosf{3}.\] Thus, it takes 11.7 years for the tree to reach half its maximimum height.
    \end{ppart}
\end{solution}

\begin{problem}
    \begin{enumerate}
        \item Find $\int\frac1{x(1000-x)} \d x$.
        \item A communicable disease is spreading within a small community with a population of 1000 people. A scientist found out that the rate at which the disease spreads is proportional to the product of the number of people who are infected with the disease and the number of people who are not infected with the disease. It is known that one person in this community is infected initially and five days later, 12\% of the population is infected.

        Given that the infected population is $x$ at time $t$ days after the start of the spread of the disease, show that it takes less than 8 days for half the population to contract the disease.
        \item State an assumption made by the scientist.
    \end{enumerate}
\end{problem}
\begin{solution}
    \begin{ppart}
        \begin{gather*}
            \int \frac1{x(1000-x)} \d x = \int \frac1{1000} \bp{\frac1x + \frac1{1000-x}} \d x\\
            = \frac{\ln \abs{x} - \ln \abs{1000-x}}{1000}+ C = \frac1{1000} \ln \abs{\frac{x}{1000-x}} + C.
        \end{gather*}
    \end{ppart}
    \begin{ppart}
        Note that $\der{x}{t} \propto x(1000-x) \implies \der{x}{t} = kx(1000-x)$ for some $k \in \mathbb{R}^+$.
        \begin{gather*}
            \der{x}{t} = kx(1000-x) \implies \frac1{x(1000-x)} \der{x}{t} = k \\
            \implies \int \frac1{x(1000-x)} \d x = \int k \d t \implies \frac1{1000} \ln{\frac{x}{1000-x}} + C = kt.
        \end{gather*}

        Note that when $t = 0$, we have $x = 1$. Thus, \[\frac1{1000} \ln{\frac{1}{999}} + C = 0 \implies C = \frac{\ln 999}{1000}.\]
        
        When $t = 5$, $x = 120$. Hence, \[\frac1{1000} \ln{\frac{120}{880}} + \frac{\ln 999}{1000} = 5k \implies k = \frac1{5000} \bp{\ln \frac3{22} + \ln 999}.\] Thus, 
        \begin{gather*}
            t = \bs{\frac1{5000} \bp{\ln \frac3{22} + \ln 999}}^{-1} \bs{\frac1{1000} \ln{\frac{x}{1000-x}} + \frac{\ln 999}{1000}} \\
            = \frac{5}{\ln{3/22} + \ln 999} \bs{\ln{\frac{x}{1000-x}} + \ln 999}
        \end{gather*}
        Hence, when half the population is infected, i.e. $x = 500$, we have $t = 7.03 < 8$. Thus, it takes less than 8 days for half the population to contract the disease.
    \end{ppart}
    \begin{ppart}
        The assumption is that there are no measures taken by the population to limit the spread of the disease (e.g. quarantine).
    \end{ppart}
\end{solution}