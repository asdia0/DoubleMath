\section{Assignment B5A}

\begin{problem}
    \begin{enumerate}
        \item Show, algebraically, that the derivative of the function \[\ln{1+x} - \frac{2x}{x+2}\] is never negative.
        \item Hence, show that $\ln{1+x} \geq \frac{2x}{x+2}$ when $x \geq 0$.
    \end{enumerate}
\end{problem}
\begin{solution}
    Let \[f(x) = \ln{1+x} - \frac{2x}{x+2} = \ln{1+x} - 2 + \frac4{x+2}.\]

    \begin{ppart}
        \[f'(x) = \frac1{1+x} - \frac{4}{(x+2)^2} = \frac{x^2}{(1+x)(x+2)^2}.\] Given that $\ln{1+x}$ is defined, it must be that $1 + x > 0$. We also know that $x^2 \geq 0$ and $(x+2)^2 \geq 0$. Hence, $f'(x) \geq 0$ for all $x$ in the domain of $f$ and is thus never negative.
    \end{ppart}
    \begin{ppart}
        Note that $f(0) = 0$. Since $f'(x) \geq 0$ for all $x \geq 0$, \[\ln{1+x} - \frac{2x}{x+2} = f(x) \geq f(0) = 0 \implies \ln{1+x} \geq \frac{2x}{x+2}.\]
    \end{ppart}
\end{solution}

\begin{problem}
    The equation of a curve is $y = ax^2 - 2bx + c$, where $a$, $b$ and $c$ are constants, with $a > 0$.

    \begin{enumerate}
        \item Using differentiation, find the coordinates of the turning point on the curve, in terms of $a$, $b$ and $c$. State whether it is a maximum point or a minimum point.
        \item Given that the turning point of the curve lies on the line $y=x$, find an expression for $c$ in terms of $a$ and $b$. Show that in this case, whatever the value of $b$, $c \geq -1/4a$.
        \item Find the numerical values of $a$, $b$ and $c$ when the curve passes through the point $(0, 6)$ and has a turning point at $(2, 2)$.
    \end{enumerate}
\end{problem}
\begin{solution}
    \begin{ppart}
        For stationary points, $\derx{y}{x} = 0$. Hence, \[\der{y}{x} = 2ax - 2b = 0 \implies x = \frac{b}{a} \implies y = a\bp{\frac{b}{a}}^2 - 2b\bp{\frac{b}{a}} + c = -\frac{b^2}{a} + c.\] Since $a > 0$, the graph of $y$ is concave upwards. Thus, there is a maximum point at $\bp{\frac{b}{a}, -\frac{b^2}a + c}$.
    \end{ppart}
    \begin{ppart}
        Since the turning point $\bp{\frac{b}{a}, -\frac{b^2}a + c}$  lies on the line $y=x$, \[\frac{b}{a} = -\frac{b^2}a + c \implies c = \frac{b + b^2}a = \frac{\bp{b + 1/2}^2 - 1/4}{a}.\] Since $(b + 1/2)^2 \geq 0$, it follows that $c \geq -1/4a$.
    \end{ppart}
    \begin{ppart}
        Since the curve passes through $(0, 6)$, it is obvious to see that $c = 6$. Furthermore, since the curve has a turning point at $(2, 2)$, we know that $\frac{b}{a} = 2$ and $-\frac{b^2}a + c = 2$. Hence, \[-\frac{b^2}{a} = 2 - c = -4 \implies b\bp{\frac{b}{a}} = 4 \implies b = 2 \implies a = 1.\] Thus, $a = 1$, $b = 2$, and $c = 6$.
    \end{ppart}
\end{solution}

\begin{problem}
    The diagram below shows the graph of $y = f(x)$. Sketch the graph of $y = f'(x)$.

    \begin{center}
        \begin{tikzpicture}[trim axis left, trim axis right]
            \begin{axis}[
                domain = -5:5,
                samples = 101,
                axis y line=middle,
                axis x line=middle,
                xtick = {2},
                ytick = \empty,
                xlabel = {$x$},
                ylabel = {$y$},
                ymin=-5,
                ymax=5,
                legend cell align={left},
                legend pos=outer north east,
                after end axis/.code={
                    \path (axis cs:0,0) 
                        node [anchor=north east] {$O$};
                    }
                ]
                \addplot[plotRed, domain=-5:1] {2/(x-1) + 2};

                \addlegendentry{$y = f(x)$};

                \addplot[plotRed, domain=1:3] {6 * (-1/(x-1) + 1)};

                \addplot[plotRed, domain=3:5] {sqrt(2) * (x-2.293) * e^(0.5 - (x-2.293)^2) + 2};

                \fill (3, 3) circle[radius=2.5pt] node[anchor=south] {$(3, 3)$};

                \draw[dotted, thick] (5, 2) -- (-5,2) node[anchor=south west] {$y=2$};

                \draw[dotted, thick] (1, 5) -- (1, -5) node[anchor=south west, fill=white, opacity = 0.6, text opacity=1] {$x=1$};
            \end{axis}
        \end{tikzpicture}
    \end{center}
\end{problem}
\begin{solution}
    \begin{center}
        \begin{tikzpicture}[trim axis left, trim axis right]
            \begin{axis}[
                domain = -5:5,
                samples = 101,
                axis y line=middle,
                axis x line=middle,
                xtick = {3},
                ytick = \empty,
                xlabel = {$x$},
                ylabel = {$y$},
                ymax=5,
                ymin=-5,
                legend cell align={left},
                legend pos=outer north east,
                after end axis/.code={
                    \path (axis cs:0,0) 
                        node [anchor=north east] {$O$};
                    }
                ]
                \addplot[plotRed, domain=-5:0.9] {-2/(x-1)^2};

                \addlegendentry{$y = f'(x)$};

                \addplot[plotRed, domain=1.2:3, unbounded coords=jump] {6/(x-1)-3};

                \addplot[plotRed, domain=3:5] {-(x-3)*e^(-(x-3)^2)};

                \draw[dotted, thick] (1, 5) -- (1, -5) node[anchor=south west] {$x=1$};
            \end{axis}
        \end{tikzpicture}
    \end{center}
\end{solution}

\clearpage
\begin{problem}
    The curve $C$ has equation \[x - y = (x+y)^2.\] It is given that $C$ has only one turning point.

    \begin{enumerate}
        \item Show that $1 + \der{y}{x} = \frac{2}{2x + 2y + 1}$.
        \item Hence, or otherwise, show that $\der[2]{y}{x} = -\bp{1 + \der{y}{x}}^2$.
        \item Hence, state, with a reason, whether the turning point is a maximum or a minimum.
    \end{enumerate}
\end{problem}
\begin{solution}
    \begin{ppart}
        Implicitly differentiating the given equation, \[1 - \der{y}{x} = 2(x+y)\bp{1 + \der{y}{x}} \implies \der{y}{x} = \frac{1 - (2x + 2y)}{2x + 2y + 1} \implies 1 + \der{y}{x} = \frac{2}{2x+2y+1}.\]
    \end{ppart}
    \begin{ppart}
        Implicitly differentiating the above equation, \[\der[2]{y}{x} = -\frac{2\bp{2 + 2 \cdot \der{y}{x}}}{\bp{2x+2y+1}^2} = -\bp{\frac{2}{2x+2y+1}}^2 \bp{1 + \der{y}{x}} = -\bp{1 + \der{y}{x}}^3.\]
    \end{ppart}
    \begin{ppart}
        For turning points, $\derx{y}{x} = 0$. Hence, $\derx[2]{y}{x} = -1 < 0$. Thus, the turning point is a maximum.
    \end{ppart}
\end{solution}