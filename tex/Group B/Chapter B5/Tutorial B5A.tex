\section{Tutorial B5A}

\begin{problem}
    The equation of a curve is $y=2x^3+3x^2+6x+4$. Find $\derx{y}{x}$ and hence show that $y$ is increasing for all real values of $x$.
\end{problem}
\begin{solution}
    \[\der{y}{x} = 6x^2+6x+6 = 6\bp{x + \frac12}^2 + \frac{18}4.\] For all $x \in \RR$, we have $\bp{x + \frac12}^2 \geq 0$. Hence, $\derx{y}{x} > 0$. Thus, $y$ is increasing for all real values of $x$.
\end{solution}

\begin{problem}
    Find, by differentiation, the $x$-coordinates of all the stationary points on the curve $y = \frac{x^3}{(x+1)^2}$ stating, with reasons, the nature of each point.
\end{problem}
\begin{solution}
    \[y = \frac{x^3}{(x+1)^2} \implies (x+1)^2y =x^3 \implies y'(x+1)^2 + 2y(x+1) = 3x^2.\] For stationary points, $y' = 0$. Thus, \[2y(x+1) = \frac{2x^3}{x+1} = 3x^2 \implies 2x^3 - 3x^2 (x+1) = x^2\bp{-x - 3} = 0.\] Hence, $x = 0$ or $x = -3$.

    \begin{table}[H]
        \centering
        \begin{tabular}{|c|c|c|c|c|c|c|c|}
        \hline
        $x$          & $0^-$ & 0 & $0^+$ & \hspace{-1em} & $(-3)^-$ & $-3$ & $(-3)^+$ \\\hline
        $\derx{y}{x}$ & +ve   & 0 & +ve  & \hspace{-1em} & +ve   & 0 & -ve \\\hline
        \end{tabular}
    \end{table}

    By the first derivative test, there is a stationary point of inflexion at $x = 0$ and a maximum point at $x=-3$.
\end{solution}

\begin{problem}
    Differentiate $f(x) = 8\sin{x/2} - \sin x - 4x$ with respect to $x$ and deduce that $f(x) < 0$ for $x>0$.
\end{problem}
\begin{solution}
    \[f'(x) = 4\cos \frac{x}2 - \cos x - 4 = 4\cos \frac{x}2 - \bp{2\cos^2 \frac{x}2 - 1} - 4 = -2\bp{\cos\frac{x}2 - 1}^2 - 1.\]

    Observe that for all $x \in \RR$, $\bp{\cos\frac{x}2 - 1}^2 \geq 0$. Hence, $f'(x) < 0$ for all real values of $x$. Thus, $f(x)$ is strictly decreasing on $\RR$.

    Note that $f(0) = 0$. Since $f(x)$ is strictly decreasing, for all $x > 0$, $f(x) < f(0) = 0$.
\end{solution}

\begin{problem}
    Sketch the graphs of the derivative functions for each of the graphs of the following functions below. In each graph, the point(s) labelled in coordinate form are stationary points.

    \begin{enumerate}
        \item \begin{center}\tikzsetnextfilename{278}
            \begin{tikzpicture}[trim axis left, trim axis right]
                \begin{axis}[
                    domain = -0.25:3.75,
                    samples = 101,
                    axis y line=middle,
                    axis x line=middle,
                    xtick = \empty,
                    ytick = \empty,
                    xlabel = {$x$},
                    ylabel = {$y$},
                    legend cell align={left},
                    legend pos=outer north east,
                    after end axis/.code={
                        \path (axis cs:0,0) 
                            node [anchor=north east] {$O$};
                        }
                    ]
                    \addplot[plotRed] {(x-2)^3 + 2};
                    
                    \fill (2, 2) circle[radius=2.5 pt] node[anchor=south] {$(2, 2)$};
                \end{axis}
            \end{tikzpicture}
        \end{center}
        \item \begin{center}\tikzsetnextfilename{279}
            \begin{tikzpicture}[trim axis left, trim axis right]
                \begin{axis}[
                    domain = -5:5,
                    samples = 101,
                    axis y line=middle,
                    axis x line=middle,
                    xtick = \empty,
                    ytick = \empty,
                    xlabel = {$x$},
                    ylabel = {$y$},
                    ymin=-5,
                    ymax=5,
                    legend cell align={left},
                    legend pos=outer north east,
                    after end axis/.code={
                        \path (axis cs:0,0) 
                            node [anchor=north west] {$O$};
                        }
                    ]
                    \addplot[plotRed, unbounded coords=jump] {x + 1/x};
                    
                    \fill (-1, -2) circle[radius=2.5 pt];
                    
                    \node[anchor=east] at (-1, -1.4) {$(-1, -2)$};

                    \fill (1, 2) circle[radius=2.5 pt];
                    
                    \node[anchor=south] at (1.3, 2) {$(1, 2)$};

                    \addplot[dotted, thick] {x};

                    \node[rotate=45] at (4.5, 4) {$y=x$};
                \end{axis}
            \end{tikzpicture}
        \end{center}
        \item \begin{center}\tikzsetnextfilename{280}
            \begin{tikzpicture}[trim axis left, trim axis right]
                \begin{axis}[
                    domain = -5:5,
                    samples = 101,
                    axis y line=middle,
                    axis x line=middle,
                    xtick = \empty,
                    ytick = \empty,
                    xlabel = {$x$},
                    ylabel = {$y$},
                    ymin=-5,
                    ymax=5,
                    legend cell align={left},
                    legend pos=outer north east,
                    after end axis/.code={
                        \path (axis cs:0,0) 
                            node [anchor=north west] {$O$};
                        }
                    ]
                    \addplot[plotRed, domain=-5:-1] {-1/(x+1) + 2};

                    \addplot[plotRed, domain=-1:2] {4.5 * (-1/(x+1) + 1)};

                    \addplot[plotRed, domain=2:5] {sqrt(2) * (x-1.293) * e^(0.5 - (x-1.293)^2) + 2};

                    \fill (2, 3) circle[radius=2.5pt] node[anchor=south] {$(2, 3)$};

                    \draw[dotted, thick] (5, 2) -- (-5,2) node[anchor=north west] {$y=2$};

                    \draw[dotted, thick] (-1, 5) -- (-1, -5) node[anchor=south east] {$x=-1$};
                \end{axis}
            \end{tikzpicture}
        \end{center}
        \item \begin{center}\tikzsetnextfilename{281}
            \begin{tikzpicture}[trim axis left, trim axis right]
                \begin{axis}[
                    domain = -3:2,
                    samples = 101,
                    axis y line=middle,
                    axis x line=middle,
                    xtick = {-1},
                    xticklabels = {$-a$},
                    ytick = \empty,
                    xlabel = {$x$},
                    ylabel = {$y$},
                    ymin=-2,
                    ymax=2,
                    xmin=-4,
                    xmax=4,
                    legend cell align={left},
                    legend pos=outer north east,
                    after end axis/.code={
                        \path (axis cs:0,0) 
                            node [anchor=north west] {$O$};
                        }
                    ]
                    \addplot[plotRed, domain=1:4] {1/(x-1)};

                    \addplot[plotRed, domain=0:1] {sec(pi * (\x +0.5) r)};

                    \addplot[plotRed, domain=-2:0] {-1/x - 1};

                    \addplot[plotRed, domain=-4:-2] {0.824 * (x+1)*e^(-0.5 * (x+1)^2)};

                    \fill (0.5, -1) circle[radius=2.5pt] node[anchor=west] {$(\frac{a}2, -a)$};

                    \fill (-2, -0.5) circle[radius=2.5pt] node[anchor=north] {$(-2a, -\frac{a}2)$};

                    \draw[dotted, thick] (1, 2) -- (1, -2) node[anchor=south west] {$x=a$};
                \end{axis}
            \end{tikzpicture}
        \end{center}
    \end{enumerate}
\end{problem}
\clearpage
\begin{solution}
    \begin{ppart}
        \begin{center}\tikzsetnextfilename{282}
            \begin{tikzpicture}[trim axis left, trim axis right]
                \begin{axis}[
                    domain = -0.25:3.75,
                    samples = 101,
                    axis y line=middle,
                    axis x line=middle,
                    xtick = {2},
                    ytick = \empty,
                    xlabel = {$x$},
                    ylabel = {$y$},
                    legend cell align={left},
                    legend pos=outer north east,
                    after end axis/.code={
                        \path (axis cs:0,0) 
                            node [anchor=north] {$O$};
                        }
                    ]
                    \addplot[plotRed] {3(x-2)^2};
                \end{axis}
            \end{tikzpicture}
        \end{center}
    \end{ppart}
    \begin{ppart}
        \begin{center}\tikzsetnextfilename{283}
            \begin{tikzpicture}[trim axis left, trim axis right]
                \begin{axis}[
                    domain = -3:3,
                    samples = 101,
                    axis y line=middle,
                    axis x line=middle,
                    xtick = {-1, 1},
                    ytick = \empty,
                    xlabel = {$x$},
                    ylabel = {$y$},
                    ymin=-3,
                    ymax=2,
                    legend cell align={left},
                    legend pos=outer north east,
                    after end axis/.code={
                        \path (axis cs:0,0) 
                            node [anchor=north east] {$O$};
                        }
                    ]
                    \addplot[plotRed, unbounded coords=jump] {1 - 1/x^2};

                    \draw[dotted, thick] (-3, 1) -- (3, 1) node[anchor=south east] {$y=1$};
                \end{axis}
            \end{tikzpicture}
        \end{center}
    \end{ppart}
    \begin{ppart}
        \begin{center}\tikzsetnextfilename{284}
            \begin{tikzpicture}[trim axis left, trim axis right]
                \begin{axis}[
                    domain = -5:5,
                    samples = 101,
                    axis y line=middle,
                    axis x line=middle,
                    xtick = {2},
                    ytick = \empty,
                    xlabel = {$x$},
                    ylabel = {$y$},
                    ymax=3,
                    ymin=-1,
                    legend cell align={left},
                    legend pos=outer north east,
                    after end axis/.code={
                        \path (axis cs:0,0) 
                            node [anchor=north east] {$O$};
                        }
                    ]
                    \addplot[plotRed, domain=-5:-1.2, unbounded coords=jump] {-1/(x+1)};

                    \addplot[plotRed, domain=-0.8:2, unbounded coords=jump] {9/(x+1)-3};

                    \addplot[plotRed, domain=2:5] {-(x-2)*e^(-(x-2)^2)};

                    \draw[dotted, thick] (-1, 5) -- (-1, -1) node[anchor=south east] {$x=-1$};
                \end{axis}
            \end{tikzpicture}
        \end{center}
    \end{ppart}
    \clearpage
    \begin{ppart}
        \begin{center}\tikzsetnextfilename{285}
            \begin{tikzpicture}[trim axis left, trim axis right]
                \begin{axis}[
                    domain = -3:2,
                    samples = 101,
                    axis y line=middle,
                    axis x line=middle,
                    xtick = {-2, 0.5},
                    xticklabels = {$-2a$, $\frac{a}2$},
                    ytick = \empty,
                    xlabel = {$x$},
                    ylabel = {$y$},
                    ymin=-2,
                    ymax=2,
                    xmin=-4,
                    xmax=4,
                    legend cell align={left},
                    legend pos=outer north east,
                    after end axis/.code={
                        \path (axis cs:0,0) 
                            node [anchor=north east] {$O$};
                        }
                    ]
                    \addplot[plotRed, domain=1.2:4] {-1/(x-1)^2};

                    \addplot[plotRed, domain=0.1:0.9] {sec(pi * (\x +0.5) r) * tan(pi * (\x +0.5) r) * pi * x};

                    \addplot[plotRed, domain=-2:-0.1] {1/x^2 -1/4};

                    \addplot[plotRed, domain=-4:-2] {0.25 * (x+2) * e^(-(x+2)^2)};

                    \draw[dotted, thick] (1, 2) -- (1, -2) node[anchor=south west] {$x=a$};
                \end{axis}
            \end{tikzpicture}
        \end{center}
    \end{ppart}
\end{solution}

\begin{problem}
    \begin{enumerate}
        \item Given that $y=ax\sqrt{x+2}$ where $a > 0$, find $\derx{y}{x}$, expressing your answer as a single algebraic fraction. Hence, show that the curve $y = ax\sqrt{x+2}$ has only one turning point, and state its coordinates in exact form.
        \item Sketch the graph of $y = f'(x)$, where $f(x) = ax\sqrt{x+2}$, where $a > 0$.
    \end{enumerate}
\end{problem}
\begin{solution}
    \begin{ppart}
        \[\der{y}{x} = a \bp{\frac{x}{2\sqrt{x+2}} + \frac{2(x+2)}{2\sqrt{x+2}}} = \frac{a(3x+4)}{2\sqrt{x+2}}.\]

        Consider the stationary points of $y=ax\sqrt{x+2}$. For stationary points, $\derx{y}{x} =0$. \[\der{y}{x} = \frac{a(3x+4)}{2\sqrt{x+2}} = 0 \implies a(3x+4) = 0.\] Since $a > 0$, we have $3x+4=0$, whence $x = -4/3$.

        \begin{table}[H]
            \centering
            \begin{tabular}{|c|c|c|c|}
            \hline
            $x$ & $(-4/3)^-$ & $-4/3$ & $(-4/3)^+$ \\\hline
            $\derx{y}{x}$ & -ve & 0 & +ve \\\hline
            \end{tabular}
        \end{table}

        Hence, by the first derivative test, there is a turning point (minimum point) at $x = -4/3$. Thus, $y = ax\sqrt{x+2}$ has only one turning point.

        Substituting $x = -4/3$ into $y = ax\sqrt{x+2}$, we see that the coordinate of the turning point is $(-\frac43, -\frac{4a}3 \sqrt{\frac23})$.
    \end{ppart}
    \clearpage
    \begin{ppart}
        \begin{center}\tikzsetnextfilename{286}
            \begin{tikzpicture}[trim axis left, trim axis right]
                \begin{axis}[
                    domain = -2:10,
                    samples = 180,
                    axis y line=middle,
                    axis x line=middle,
                    ytick = {sqrt(2)},
                    yticklabels = {$\sqrt{2}a$},
                    xtick = {-4/3},
                    xticklabels = {$-\frac43$},
                    xlabel = {$x$},
                    ylabel = {$y$},
                    ymin=-2,
                    xmin=-2,
                    legend cell align={left},
                    legend pos=outer north east,
                    after end axis/.code={
                        \path (axis cs:0,0) 
                            node [anchor=north west] {$O$};
                        }
                    ]
                    \addplot[plotRed] {(3*x + 4)/(2 * sqrt(x+2))};
        
                    \addlegendentry{$y = f'(x)$};

                    \draw[dotted, thick] (-2, -2) -- (-2, 5) node[anchor=north east, rotate=90] {$x=-2$};
                \end{axis}
            \end{tikzpicture}
        \end{center}
    \end{ppart}
\end{solution}

\begin{problem}
    A particle $P$ moves along the $x$-axis. Initially, $P$ is at the origin $O$. At time $t$ s, the velocity is $v$ ms$^{-1}$ and the acceleration is $a$ ms$^{-2}$. Below is the velocity-time graph of the particle for $0 \leq t \leq 25$.

    \begin{center}\tikzsetnextfilename{287}
        \begin{tikzpicture}[trim axis left, trim axis right]
            \begin{axis}[
                domain = -5:5,
                samples = 101,
                axis y line=middle,
                axis x line=middle,
                xtick = {pi, 2*pi},
                xticklabels = {15, 25},
                ytick = \empty,
                xlabel = {$t$},
                ylabel = {$v$},
                ymin=-1.5,
                ymax=1.5,
                xmax=7,
                legend cell align={left},
                legend pos=outer north east,
                after end axis/.code={
                    \path (axis cs:0,0) 
                        node [anchor=east] {$O$};
                    }
                ]
                \addplot[plotRed, domain=0:pi] {sin(\x r)};

                \addplot[plotRed, domain=pi:2*pi] {sin(\x r)};

                \fill (pi/2, 1) circle[radius=2.5pt] node[anchor=south] {$(11, 5)$};

                \fill (3*pi/2, -1) circle[radius=2.5pt] node[anchor=north] {$(21, -4)$};
            \end{axis}
        \end{tikzpicture}
    \end{center}

    \begin{enumerate}
        \item Describe the motion of the particle for $0 \leq t \leq 25$.
        \item Sketch the acceleration-time graph of the particle $P$.
    \end{enumerate}
\end{problem}
\begin{solution}
    \begin{ppart}
        From $t = 0$ to $t = 11$, $P$ speeds up and reaches a top speed of 5 ms$^{-1}$. From $t=11$ to $t=15$, $P$ slows down. At $t=15$, $P$ is instantaneously at rest. From $t=15$ to $t=21$, $P$ speeds up and moves in the opposite direction, reaching a top speed of 4 ms$^{-1}$. From $t=21$ to $t=25$, $P$ slows down. At $t=25$, $P$ is instantaneously at rest.
    \end{ppart}
    \clearpage
    \begin{ppart}
        \begin{center}\tikzsetnextfilename{288}
            \begin{tikzpicture}[trim axis left, trim axis right]
                \begin{axis}[
                    domain = -5:5,
                    samples = 101,
                    axis y line=middle,
                    axis x line=middle,
                    xtick = {pi/2, pi, 3*pi/2, 2*pi},
                    xticklabels = {11, 15, 21, 25},
                    ytick = \empty,
                    xlabel = {$t$},
                    ylabel = {$v$},
                    ymin=-1.5,
                    ymax=1.5,
                    xmax=7,
                    legend cell align={left},
                    legend pos=outer north east,
                    after end axis/.code={
                        \path (axis cs:0,0) 
                            node [anchor=east] {$O$};
                        }
                    ]
                    \addplot[plotRed, domain=0:pi] {cos(\x r)};

                    \addplot[plotRed, domain=pi:2*pi] {cos(\x r)};

                    \draw[dotted, thick] (pi, -0.3) -- (pi, -1);

                    \draw[dotted, thick] (2*pi, 0) -- (2*pi, 1);
                \end{axis}
            \end{tikzpicture}
        \end{center}
    \end{ppart}
\end{solution}

\begin{problem}
    The function $f$ defined by $f(x) = \ln x - 2\bp{x-1/2}$, where $x \in \RR, x > 0$. Find $f'(x)$ and show that the function is decreasing for $x > 1/2$. Hence, show that for $x > 1/2$, $2\bp{x - 1/2} - \ln x > \ln 2$.
\end{problem}
\begin{solution}
    Observe that $f'(x) = 1/x - 2 < 0$ for $x > 1/2$. Thus, $f(x)$ is decreasing for all $x > 1/2$. Since $f(1/2) = -\ln 2$, it follows that \[\bp{\forall x > \frac12}: \quad -\ln 2 = f(1/2) > f(x) = \ln x - 2\bp{x - \frac12} \implies 2\bp{x - \frac12} - \ln x > \ln 2.\]
\end{solution}