\section{Assignment B10}

\begin{problem}
    Given that $y = \e^{-x}\cos x$, show that $\der[2]{y}{x} = -2\bp{y + \der{y}{x}}$. By further differentiation, find the series expansion of $y$, in ascending powers of $x$, up to and including the term in $x^3$. Use the series to obtain an approximate value for $\int_0^{0.2} \frac{\cos x^2}{\e^{x^2}} \d x$, giving your answer correct to 4 decimal places.

    Using the trapezium rule with 4 trapezia of equal width, find another approximation for $\int_0^{0.2}\frac{\cos x^2}{\e^{x^2}} \d x$, giving your answer correct to 4 decimal places.
\end{problem}
\begin{solution}
    Differentiating with respect to $x$, we get \[y' = -\e^{-x}\sin x - \e^{-x} \cos x \implies y' = -\e^{-x} \sin x - y.\] Differentiating once more, \[y'' = -\e^{-x} \cos x + \e^{-x} \sin x - y' = -y + \bp{-y' - y} - y' = -2\bp{y + y'}.\] Further differentiating, we obtain $y''' = -2(y' + y'')$. Evaluating $y$ and its derivatives at $x = 0$, we get \[y(0) = 1, \quad y'(0) = -1, \quad y''(0) = 0, \quad y'''(0) = 2.\] Thus, \[\e^{-x}\cos x = 1 - x + \frac{x^3}3 + \cdots.\]

    Hence, \[\int_0^{0.2} \frac{\cos x^2}{\e^{x^2}} \d x = \int_0^{0.2} \e^{-{x^2}}\cos x^2 \d x \approx \int_0^{0.2} \bp{1 - \bp{x^2} + \frac{\bp{x^2}^3}3} \d x = 0.1973 \todp{4}.\]

    Let $g(x) = \frac{\cos x^2}{\e^{x^2}}$. By the trapezium rule, we have \[\int_0^{0.2} \frac{\cos x^2}{\e^{x^2}} \d x \approx \frac12 \cdot \frac{0.2}{4} \Big[g(0) + 2\big[g(0.05) + g(0.1) + g(0.15)\big] + g(0.2)\Big] = 0.1973 \todp{4}.\]
\end{solution}

\begin{problem}
    The curve $C$ has equation $y^2 = \frac{x}{\sqrt{1 + x^2}}$, $y \geq 0$.

    The finite region $R$ is bounded by $C$, the $x$-axis and the lines $x = 0$ and $x = 2$. $R$ is rotated through $2\pi$ radians about the $x$-axis.

    \begin{enumerate}
        \item Find the exact volume of the solid formed.
    \end{enumerate}

    An estimate for the volume in (a) is found using the trapezium rule with 7 ordinates.

    \begin{enumerate}
        \setcounter{enumi}{1}
        \item Find the percentage error resulting from using this estimate, giving your answer to 3 decimal places.

        Explain, with the help of a sketch, why the estimate given by the trapezium rule is less than the actual value.
    \end{enumerate}
\end{problem}
\clearpage
\begin{solution}
    \begin{ppart}
        \[\volume = \pi \int_0^2 y^2 \d x = \pi \int_0^2 \frac{2x}{2\sqrt{1 + x^2}} \d x = \pi \evalint{\sqrt{1 + x^2}}02 = \pi(\sqrt5 - 1) \units[3].\]
    \end{ppart}
    \begin{ppart}
        Let $f(x) = \frac{x}{\sqrt{1 + x^2}}$. By the trapezium rule, \[\volume = \pi \int_0^2 f(x) \d x \approx \pi \cdot \frac12 \cdot \frac{2-0}{6} \sum_{n=0}^5 \bs{f\of{\frac{n}3} + f\of{\frac{n+1}3}} = 3.8566 \tosf{5}.\] Hence, the percentage error is \[\abs{\frac{\pi(\sqrt5 - 1) - 3.8566}{\pi(\sqrt5 - 1)}} = 0.686\% \todp{3}.\]

        Consider the following graph of $y = f(x)$.

        \begin{center}
            \begin{tikzpicture}[trim axis left, trim axis right]
                \begin{axis}[
                    domain = 0:2,
                    samples = 101,
                    axis y line=middle,
                    axis x line=middle,
                    xtick = \empty,
                    ytick = \empty,
                    xlabel = {$x$},
                    ylabel = {$y$},
                    legend cell align={left},
                    legend pos=outer north east,
                    after end axis/.code={
                        \path (axis cs:0,0) 
                            node [anchor=north east] {$O$};
                        }
                    ]
                    \addplot[plotRed] {x/sqrt(1 + x^2)};
        
                    \addlegendentry{$y = f(x)$};

                    \draw (0.5, 0) -- (0.5, 0.4472);
                    \draw (1, 0) -- (1, 0.7071);
                    \draw (1.5, 0) -- (1.5, 0.8321);
                    \draw (0.5, 0.4472) -- (1, 0.7071);
                    \draw (1, 0.7071) -- (1.5, 0.8321);
                \end{axis}
            \end{tikzpicture}
        \end{center}
        
        From the graph, the curve $y = f(x)$ is clearly concave downwards. Hence, the approximation given by the trapezium rule is an underestimate and is thus less than the actual value.
    \end{ppart}
\end{solution}

\begin{problem}
    Prove that $\int_{-h}^h f(x) \d x = \frac13 h (y_{-1} + 4y_0 + y_1)$, where $y = f(x)$ is the quadratic curve passing through the points $(-h, y_{-1})$, $(0, y_0)$ and $(h, y_1)$.

    Use Simpson's rule with 5 ordinates to find an approximation to \[\int_{-3}^1 \bp{x^4 - 7x^3 + 3x^2 + 6x + 4}^{1/3} \d x\] Find another approximation to the same integral using the trapezium rule with 5 ordinates.

    Which of these approximations would you expect to be more accurate? Justify your answer.
\end{problem}
\begin{solution}
    Let $f(x) = ax^2 + bx + c$ be the quadratic such that the graph $y = f(x)$ passes through the points $(-h, y_{-1})$, $(0, y_0)$ and $(h, y_1)$.

    Note that we have $y_0 = f(0) = c$. We also have \[y_{-1} + y_1 = f(-h) + f(h) = \bs{a(-h)^2 + b(-h) + c} + \bs{ah^2 + bh + c} = 2ah^2 + 2c.\] Hence,
    \begin{gather*}
        \int_{-h}^h f(x) \d x = \int_{-h}^h (ax^2 + bx + c) \d x = \evalint{\frac13 x^3 + \frac12 bx^2 + cx}{-h}{h}= \frac13 h \bp{2h^2 + 6c}\\
        = \frac13 h \bp{2h^2 + 2c + 4c} = \frac13 h \bp{y_{-1} + y_1 + 4y_0} = \frac13 h \bp{y_{-1} + 4y_0 + y_1}.
    \end{gather*}

    Let $f(x) = \bp{x^4 - 7x^3 + 3x^2 + 6x + 4}^{1/3}$. By Simpson's rule,
    \begin{align*}
        \int_{-3}^1 &\bp{x^4 - 7x^3 + 3x^2 + 6x + 4}^{1/3} \d x\\
        &\hspace{0.5em}\approx \frac13 \cdot \frac{1-(-3)}{4} \Big[f(-3) + 4f(-2) + 2f(-1) + 4f(0) + f(1)\Big] = 11.977 \tosf{5}
    \end{align*}

    By the trapezium rule,
    \begin{align*}
        \int_{-3}^1 &\bp{x^4 - 7x^3 + 3x^2 + 6x + 4}^{1/3} \d x\\
        &\hspace{0.5em}\approx \frac12 \cdot \frac{1-(-3)}{4} \Big[f(-3) + 2f(-2) + 2f(-1) + 2f(0) + f(1)\Big] = 12.142 \tosf{5}
    \end{align*}

    The approximation given by Simpson's rule should be more accurate as Simpson's rule accounts for the concavity of the curve $y = f(x)$.
\end{solution}

\begin{problem}
    \begin{enumerate}
        \item Find the exact value of $\int_0^1 \frac1{1 + x^2} \d x$.
        \item The graph of $y = \frac1{1 + x^2}$ is shown in the diagram below. Rectangles, each of width $\frac1n$, are drawn under the curve.
        
        Show that the total area $A$ of all $n$ rectangles is given by \[A = \frac1n \bs{\frac1{1 + \bp{\frac1n}^2} + \frac1{1 + \bp{\frac2n}^2} + \frac1{1 + \bp{\frac3n}^2} + \ldots + \frac12}\] State the limit of $A$ as $n \to \infty$.
    \end{enumerate}

    \begin{center}
        \begin{tikzpicture}[trim axis left, trim axis right]
            \begin{axis}[
                domain = 0:1,
                samples = 101,
                axis y line=middle,
                axis x line=middle,
                xtick = {0.1, 0.2, 0.3, 0.4, 0.8, 0.9, 1},
                xticklabels = {$\frac1n$, $\frac2n$, $\frac3n$, $\frac4n$, $\frac{n-2}n$, $\frac{n-1}n$, 1},
                ytick = \empty,
                xlabel = {$x$},
                ylabel = {$y$},
                ymin=0,
                ymax=1.1,
                xmax=1.02,
                legend cell align={left},
                legend pos=outer north east,
                after end axis/.code={
                    \path (axis cs:0,0) 
                        node [anchor=north] {$O$};
                    }
                ]
                \addplot[plotRed] {1/(1 + x^2)};
    
                \addlegendentry{$y = 1/(1 + x^2)$};

                \draw (0.1, 0) -- (0.1, 100/101);
                \draw (0.0, 100/101) -- (0.1, 100/101);

                \draw (0.2, 0) -- (0.2, 25/26);
                \draw (0.1, 25/26) -- (0.2, 25/26);

                \draw (0.3, 0) -- (0.3, 100/109);
                \draw (0.2, 100/109) -- (0.3, 100/109);
                
                \draw (0.4, 0) -- (0.4, 25/29);
                \draw (0.3, 25/29) -- (0.4, 25/29);

                \draw (0.8, 0) -- (0.8, 25/41);

                \draw (0.9, 0) -- (0.9, 100/181);
                \draw (0.8, 100/181) -- (0.9, 100/181);

                \draw (1, 0) -- (1, 1/2);
                \draw (0.9, 1/2) -- (1, 1/2);
            \end{axis}
        \end{tikzpicture}
    \end{center}

    \begin{enumerate}
        \setcounter{enumi}{2}
        \item It is given that \[B = \frac1n \bs{\frac1{1 + \bp{\frac1n}^4} + \frac1{1 + \bp{\frac2n}^4} + \frac1{1 + \bp{\frac3n}^4} + \ldots + \frac12}\] Find an approximation for the limit of $B$ as $n \to \infty$ by considering an appropriate graph and using the trapezium rule with 5 intervals. Given your answer correct to 2 decimal places.
    \end{enumerate}
\end{problem}
\begin{solution}
    \begin{ppart}
        \[\int_0^1 \frac1{1 + x^2} \d x = \evalint{\arctan x}{0}{1} = \frac\pi4.\]
    \end{ppart}
    \begin{ppart}
        Observe that the $k$th rectangle has height $\frac1{1 + (k/n)^2}$ and width $1/n$. Hence,
        \begin{align*}
            A &= \sum_{k=1}^n \frac1n \cdot \frac1{1 + (k/n)^2} = \frac1n \sum_{k=1}^n \frac1{1 + (k/n)^2}\\
            &= \frac1n \bs{\frac1{1 + \bp{\frac1n}^2} + \frac1{1 + \bp{\frac2n}^2} + \frac1{1 + \bp{\frac3n}^2} + \ldots + \frac1{1 + \bp{\frac{n}{n}}^2}}\\
            &= \frac1n \bs{\frac1{1 + \bp{\frac1n}^2} + \frac1{1 + \bp{\frac2n}^2} + \frac1{1 + \bp{\frac3n}^2} + \ldots + \frac1{2}}
        \end{align*}
        Thus, \[\lim_{n \to \infty} A = \int_0^1 \frac1{1 + x^2} = \frac\pi4.\]
    \end{ppart}
    \begin{ppart}
        Consider the following graph of $y = \frac1{1 + x^4}$.

        \begin{center}
            \begin{tikzpicture}[trim axis left, trim axis right]
                \begin{axis}[
                    domain = 0:1,
                    samples = 101,
                    axis y line=middle,
                    axis x line=middle,
                    xtick = {0.1, 0.2, 0.3, 0.4, 0.8, 0.9, 1},
                    xticklabels = {$\frac1n$, $\frac2n$, $\frac3n$, $\frac4n$, $\frac{n-2}n$, $\frac{n-1}n$, 1},
                    ytick = \empty,
                    xlabel = {$x$},
                    ylabel = {$y$},
                    ymin=0,
                    ymax=1.1,
                    xmax=1.02,
                    legend cell align={left},
                    legend pos=outer north east,
                    after end axis/.code={
                        \path (axis cs:0,0) 
                            node [anchor=north] {$O$};
                        }
                    ]
                    \addplot[plotRed] {1/(1 + x^4)};
        
                    \addlegendentry{$y = 1/(1 + x^4)$};

                    \draw (0.1, 0) -- (0.1, 0.9999);
                    \draw (0.0, 0.9999) -- (0.1, 0.9999);

                    \draw (0.2, 0) -- (0.2, 625/626);
                    \draw (0.1, 625/626) -- (0.2, 625/626);

                    \draw (0.3, 0) -- (0.3, 0.9920);
                    \draw (0.2, 0.9920) -- (0.3, 0.9920);
                    
                    \draw (0.4, 0) -- (0.4, 625/641);
                    \draw (0.3, 625/641) -- (0.4, 625/641);

                    \draw (0.8, 0) -- (0.8, 625/881);

                    \draw (0.9, 0) -- (0.9, 0.60383);
                    \draw (0.8, 0.60383) -- (0.9, 0.60383);

                    \draw (1, 0) -- (1, 1/2);
                    \draw (0.9, 1/2) -- (1, 1/2);
                \end{axis}
            \end{tikzpicture}
        \end{center}

        Using a similar line of logic presented in part (b), we have that $B$ is the total area of the rectangles above. Hence, \[\lim_{n \to \infty} B = \int_{0}^1 \frac1{1 + x^4} \d x.\]
        
        Let $f(x) = \frac1{1 + x^4}$. Using the trapezium rule, \[\lim_{n \to \infty} B \approx \frac12 \cdot \frac{1}{5} \Big[f(0) + 2\big[f(0.2) + f(0.4) + f(0.6) + f(0.8)\big] + f(1)\Big] = 0.86 \todp{2}.\]
    \end{ppart}
\end{solution}