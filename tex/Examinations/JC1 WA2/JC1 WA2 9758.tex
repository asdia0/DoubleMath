\section{JC1 WA2 - H2 Mathematics 9758}

\begin{problem}
    Differentiate $\arccos{\sqrt{1 - 4x}}$ with respect to $x$, simplifying your answer.
\end{problem}
\begin{solution}
    \[\der{}{x} \arccos{\sqrt{1 - 4x}} = -\frac1{\sqrt{1 - (1 - 4x)}} \bp{\frac{-4}{2\sqrt{1 - 4x}}} = \frac2{\sqrt{4x} \sqrt{1 - 4x}} = \frac1{\sqrt{x - 4x^2}}.\]
\end{solution}

\begin{problem}
    It is given that $x$ and $y$ satisfy the equation $xy^2 = \ln{x^2 \e^y} - \frac{2\e}{x}$.
    \begin{enumerate}
        \item Verify that $(\e, 0)$ satisfies the equation.
        \item Hence, show that at $y = 0$, $\der{y}{x} = \frac{k}{\e}$, where $k$ is a constant to be determined.
    \end{enumerate}
\end{problem}
\begin{solution}
    \begin{ppart}
        Substituting $x = \e$ and $y = 0$ into the given equation, \[\text{LHS} = \e \cdot 0^2 = 0, \quad \text{RHS} = \ln{\e^2 \cdot e^0} - \frac{2\e}{\e} = 2 - 2 = 0.\] Since the LHS is equal to the RHS, $(\e, 0)$ satisfies the equation.
    \end{ppart}
    \begin{ppart}
        From the given equation, we have \[xy^2 = 2\ln x + y - \frac{2\e}{x}.\] Implicitly differentiating yields \[x \bp{2y \, \der{y}{x}} + y^2 = \frac2{x} + \der{y}{x} + \frac{2\e}{x^2}.\] Substituting $x = e$ and $y = 0$ into the above equation gives \[0 = \frac2{\e} + \der{y}{x} + \frac{2\e}{\e^2} \implies \der{y}{x} = \frac{-4}{\e}.\] Thus, $k = -4$.
    \end{ppart}
\end{solution}

\begin{problem}
    A toy manufacturer wants to make a toy in the shape of a right circular cone with a cylinder drilled out, as shown in the diagram below. The cylinder is inscribed in the cone. The circumference of the top of the cylinder is in contact with the inner surface of the cone and the base of the cylinder is level with the base of the cone. The base radius of the cylinder is $r$ cm and the base radius of the cone is 6 cm. The height of the cylinder, $BC$, is $h$ cm and the height of the cone, $AC$ is 9 cm.

    \begin{center}\tikzsetnextfilename{17}
        \begin{tikzpicture}[scale=0.75]
            \coordinate[label=above:$A$] (A) at (0, 9);
            \coordinate[label=left:$B$] (B) at (0, 3);
            \coordinate[label=left:$C$] (C) at (0, 0);
            \coordinate (CL) at (-5.97, 0.1);
            \coordinate (CR) at (5.97, 0.1);

            \draw[dotted] (A) -- (C);
            \draw[dashed] (C) ellipse[x radius=4, y radius=0.5];
            \draw[dashed] (B) ellipse[x radius=4, y radius=0.5];
            \draw (C) ellipse[x radius=6, y radius=1];
            \draw (CL) -- (A);
            \draw (CR) -- (A);
            \draw[dashed] (-4, 3) -- (-4, 0);
            \draw[dashed] (4, 3) -- (4, 0);
            \node[anchor=east] at (4, 1.5) {$h$};
            \draw[dotted] (B) -- (4, 3);
            \draw[dotted] (C) -- (4, 0);
            \node at (2, 0) {$r$};
            \draw[<->] (0, -1.3) -- (6, -1.3);
            \node[anchor=north] at (3, -1.3) {6 cm};
            \draw[<->] (6.3, 0) -- (6.3, 9);
            \node[anchor=west] at (6.3, 4.5) {9 cm};
        \end{tikzpicture}
    \end{center}

    Using differentiation, find the minimum volume of the toy, $V$ cm$^3$, in terms of $\pi$.
\end{problem}
\begin{solution}
    \begin{center}\tikzsetnextfilename{18}
        \begin{tikzpicture}[scale=0.75]
            \coordinate[label=above:$A\bp{0, 9}$] (A) at (0, 9);
            \coordinate[label=above left:$B$] (B) at (0, 3);
            \coordinate[label=above left:$C$] (C) at (0, 0);
            \coordinate (CL) at (-6, 0);
            \coordinate (CR) at (6, 0);
            \coordinate (BL) at (-4, 3);
            \coordinate (BR) at (4, 3);

            \draw (CL) -- (A);
            \draw (CR) -- (A);
            \draw (CL) -- (CR);
            \draw[dotted] (A) -- (C);
            \draw (BL) -- (BR);
            \draw (BL) -- (-4, 0);
            \draw (BR) -- (4, 0);

            \node[anchor=south] at (2, 0) {$r$};
            \node[anchor=east] at (4, 1.5) {$h$};

            \draw[<->] (0, -0.3) -- (6, -0.3);
            \node[anchor=north] at (3, -0.3) {6 cm};

            \draw[<->] (6.3, 0) -- (6.3, 9);
            \node[anchor=west] at (6.3, 4.5) {9 cm};

            \node[anchor=west] at (2, 6) {$l_1$};
        \end{tikzpicture}
    \end{center}

    Consider the diagram above. Let $C$ be the origin. Note that $l_1$ has gradient $-\frac96 = -\frac32$. Hence, $l_1$ has equation \[l_1: y = 9 - \frac32 x.\] When $x = r$, we have $y = 9 - \frac32 r$. Thus, the height of the cylinder is $\bp{9 - \frac32 r}$ cm. Let the volume of the cylinder be $V_1$ cm$^3$. \[V_1 = \pi r^2 h = \pi r^2 \bp{9 - \frac32 r} = 9\pi r^2 - \frac32 \pi r^3.\] For stationary points, $\der{V_1}{r} = 0$. \[\der{V_1}{r} = 0 \implies 18\pi r - \frac92 \pi r^2 = 0 \implies \frac92 \pi r (4 - r) = 0.\] Hence, $V_1$ has a stationary point when $r = 4$. Note that we reject $r = 0$ since $r > 0$.
    \[
    \begin{array}{c c c c}\toprule
        r & 4^- & 4 & 4^+ \\\cmidrule{1-4}
        \der{V_1}{r} & +\text{ve} & 0 & -\text{ve}\\\bottomrule
    \end{array}
    \]
    By the first derivative test, $V_1$ attains a maximum when $r = 4$. Hence, \[\min V = \text{Volume of cone} - \max V_1 = \bs{\frac13 \pi \bp{6^2}\bp{9}} - \bs{9\pi \bp{4^2} - \frac32 \pi \bp{4^3}} = 60\pi.\] Thus, the minimum volume of the toy is $60\pi$ cm$^3$.
\end{solution}

\begin{problem}
    A curve $C$ has parametric equations \[x = 2\t + \sin 2\t, \, y = \cos 2\t, \, 0 \leq \t \leq \pi.\]

    \begin{enumerate}
        \item Find $\der{y}{x}$, expressing your answer in terms of only a single trigonometric function.
        \item Hence, find the coordinates of point $Q$, on $C$, whose tangent is parallel to the $y$-axis.
    \end{enumerate}
\end{problem}
\begin{solution}
    \begin{ppart}
        Note that $\der{x}{\t} = 2 + 2\cos 2\t$ while $\der{y}{\t} = -2\sin 2\t$. Hence,
        \begin{gather*}
            \der{y}{x} = \frac{\derx{y}{\t}}{\derx{x}{\t}} = \frac{-2\sin 2\t}{2 + 2\cos2\t} = -\frac{\sin 2\t}{1 + \cos2\t} \\
            = -\frac{2\sin\t\cos\t}{1 + (2\cos^2\t - 1)} = -\frac{2\sin\t\cos\t}{2\cos^2\t} = -\frac{\sin\t}{\cos\t} = -\tan\t.
        \end{gather*}
    \end{ppart}
    \begin{ppart}
        Since the tangent at $Q$ is parallel to the $y$-axis, the derivative $\derx{y}{x} = -\tan\t$ must be undefined there. Hence, $\cos \t = 0 \implies \t = \pi/2$. Substituting $\t = \pi/2$ into the given parametric equations, we obtain $x = \pi$ and $y = -1$, whence $Q(\pi, -1)$.
    \end{ppart}
\end{solution}

\begin{problem}
    \begin{enumerate}
        \item A function is defined as $f(x) = a(2-x)^2 - b$, where $a$ and $b$ are positive constants such that $a < 1$ and $b > 4$.
        
        State a sequence of transformations that will transform the curve with equation $y = x^2$ on to the curve with equation $y = f(x)$.
        \item The diagram shows the graph of $y = g(x)$. The lines $x = 3$ and $y = 2 - x$ are asymptotes to the curve, and the graph passes through the points $(0, 0)$ and $(5, 0)$.
        
        \begin{center}\tikzsetnextfilename{19}
            \begin{tikzpicture}[trim axis left, trim axis right]
                \begin{axis}[
                    domain = -7:10,
                    samples = 171,
                    axis y line=middle,
                    axis x line=middle,
                    xtick = {5},
                    ytick = \empty,
                    xticklabels = {$\bp{5, 0}$},
                    xlabel = {$x$},
                    ylabel = {$y$},
                    ymax = 10,
                    ymin = -10,
                    legend cell align={left},
                    legend pos=outer north east,
                    after end axis/.code={
                        \path (axis cs:0,0) 
                            node [anchor=north east] {$O$};
                        }
                    ]
                    \addplot[plotRed, unbounded coords = jump] {2 - x + (6)/(x-3)};
        
                    \addlegendentry{$y = g(x)$};

                    \addplot[dotted] {2 - x};

                    \draw[dotted] (3, 10) -- (3, -10) node[anchor=south west] {$x = 3$};

                    \node[rotate=-35] at (-5, 8) {$y = 2-x$};
                \end{axis}
            \end{tikzpicture}
        \end{center}

        Sketch the graph of $y = \frac1{g(x)}$, indicating clearly the coordinates of any axial intercepts (where applicable) and the equations of any asymptotes.
        \item Given the graphs of $y = \abs{h(x)}$ and $y = h(\abs{x})$ below, sketch the two possible graphs of $y = h(x)$.
        
        \begin{minipage}{0.45\textwidth}
            \begin{center}\tikzsetnextfilename{20}
                \begin{tikzpicture}[scale=0.95, trim axis left, trim axis right]
                    \begin{axis}[
                        domain = -2.5:2.5,
                        samples = 101,
                        axis y line=middle,
                        axis x line=middle,
                        xtick = \empty,
                        ytick = \empty,
                        xlabel = {$x$},
                        ylabel = {$y$},
                        ymax=10,
                        ymin=-10,
                        legend cell align={left},
                        legend style={at={(0.5, -0.05)},anchor=north},
                        after end axis/.code={
                            \path (axis cs:0,0) 
                                node [anchor=north east] {$O$};
                            }
                        ]
                        \addplot[plotRed, unbounded coords = jump] {abs(-2 - 2/(x-1))};
            
                        \addlegendentry{$y = \abs{h(x)}$};
                        \addplot[dotted] {2};
                        \draw[dotted] (1, 10) -- (1, -10) node[anchor=south east] {$x = 1$};
                        \node[anchor=south west] at (-2.5, 2) {$y = 2$};
                    \end{axis}
                \end{tikzpicture}
            \end{center}
        \end{minipage}
        \begin{minipage}{0.45\textwidth}
            \begin{center}\tikzsetnextfilename{21}
                \begin{tikzpicture}[scale=0.95, trim axis left, trim axis right]
                    \begin{axis}[
                        domain = -2.5:2.5,
                        samples = 101,
                        axis y line=middle,
                        axis x line=middle,
                        xtick = \empty,
                        ytick = \empty,
                        xlabel = {$x$},
                        ylabel = {$y$},
                        ymax=10,
                        ymin=-10,
                        legend cell align={left},
                        legend style={at={(0.5, -0.05)},anchor=north},
                        after end axis/.code={
                            \path (axis cs:0,0) 
                                node [anchor=north east] {$O$};
                            }
                        ]
                        \addplot[plotRed, unbounded coords = jump] {-2 - 2/(abs(x)-1)};
            
                        \addlegendentry{$y = h(\abs{x})$};
                        \addplot[dotted] {-2};
                        \draw[dotted] (1, 10) -- (1, -10) node[anchor=south east] {$x = 1$};
                        \draw[dotted] (-1, 10) -- (-1, -10) node[anchor=south east] {$x = -1$};
                        \node[anchor=north west] at (-2.2, -2) {$y = -2$};
                    \end{axis}
                \end{tikzpicture}
            \end{center}
        \end{minipage}           
    \end{enumerate}
\end{problem}
\begin{solution}
    \begin{ppart}
        \renewcommand{\theenumi}{\arabic{enumi}}%
        \begin{enumerate}
            \item Translate the graph 2 units in the positive $x$-direction.
            \item Scale the graph by a factor of $a$ parallel to the $y$-axis.
            \item Translate the graph $b$ units in the negative $y$-direction.
        \end{enumerate}
        \renewcommand{\theenumi}{(\alph{enumi})}
    \end{ppart}
    \begin{ppart}
        \begin{center}\tikzsetnextfilename{22}
            \begin{tikzpicture}[trim axis left, trim axis right]
                \begin{axis}[
                    domain = -7:10,
                    samples = 171,
                    axis y line=middle,
                    axis x line=middle,
                    xtick = {3},
                    xticklabels = {$\bp{3, 0}$},
                    ytick = \empty,
                    xlabel = {$x$},
                    ylabel = {$y$},
                    ymax = 4,
                    ymin = -4,
                    legend cell align={left},
                    legend pos=outer north east,
                    after end axis/.code={
                        \path (axis cs:0,0) 
                            node [anchor=north east] {$O$};
                        }
                    ]
                    \addplot[plotRed, unbounded coords = jump] {1/(2 - x + (6)/(x-3))};
        
                    \addlegendentry{$y = 1/g(x)$};

                    \draw[dotted] (5, 4) -- (5, -4) node[anchor=south east] {$x = 5$};
                    
                    \node[anchor=north] at (-5, 0) {$y = 0$};

                    \node[anchor=south east] at (0, -4) {$x = 0$};
                \end{axis}
            \end{tikzpicture}
        \end{center}
    \end{ppart}
    \clearpage
    \begin{ppart}
        \begin{center}\tikzsetnextfilename{23}
            \begin{tikzpicture}[trim axis left, trim axis right]
                \begin{axis}[
                    domain = -2.5:2.5,
                    samples = 101,
                    axis y line=middle,
                    axis x line=middle,
                    xtick = \empty,
                    ytick = \empty,
                    xlabel = {$x$},
                    ylabel = {$y$},
                    ymax=10,
                    ymin=-10,
                    legend cell align={left},
                    legend pos=outer north east,
                    after end axis/.code={
                        \path (axis cs:0,0) 
                            node [anchor=north east] {$O$};
                        }
                    ]
                    \addplot[plotRed, unbounded coords = jump] {-2 - 2/(x-1)};
        
                    \addlegendentry{$y = h(x)$};
                    \addplot[dotted] {-2};
                    \draw[dotted] (1, 10) -- (1, -10) node[anchor=south east] {$x = 1$};
                    \node[anchor=north west] at (-2.5, -2) {$y = -2$};
                \end{axis}
            \end{tikzpicture}
        \end{center}

        \begin{center}\tikzsetnextfilename{24}
            \begin{tikzpicture}[trim axis left, trim axis right]
                \begin{axis}[
                    domain = -2.5:2.5,
                    samples = 101,
                    axis y line=middle,
                    axis x line=middle,
                    xtick = \empty,
                    ytick = \empty,
                    xlabel = {$x$},
                    ylabel = {$y$},
                    ymax=10,
                    ymin=-10,
                    legend cell align={left},
                    legend pos=outer north east,
                    after end axis/.code={
                        \path (axis cs:0,0) 
                            node [anchor=north east] {$O$};
                        }
                    ]
                    \addplot[domain=0:2.5, plotRed, unbounded coords = jump] {-2 - 2/(x-1)};
        
                    \addlegendentry{$y = h(x)$};

                    \addplot[domain=-2.5:0, plotRed, unbounded coords = jump] {abs(-2 - 2/(x-1))};

                    \addplot[dotted] {2};
                    \draw[dotted] (1, 10) -- (1, -10) node[anchor=south east] {$x = 1$};
                    \node[anchor=south west] at (-2.5, 2) {$y = 2$};
                \end{axis}
            \end{tikzpicture}
        \end{center}
    \end{ppart}
\end{solution}