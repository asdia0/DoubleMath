\section{Assignment A3}

\begin{problem}
    A university student has a goal of saving at least \$1 000 000 (in Singapore dollars). He begins working at the start of the year 2019. In order to achieve his goal, he saves 40\% of his annual salary at the end of each year. If his annual salary in the year 2019 is \$40800, and it increases by 5\% (of his previous year's salary) every year, find

    \begin{enumerate}
        \item his annual savings in 2027 (to the nearest dollar),
        \item his total savings at the end of $n$ years.
    \end{enumerate}

        What is the minimum number of complete years for which he has to work in order to achieve his goal?
\end{problem}
\begin{solution}
    Let \$$u_n$ be his annual salary in the $n$th year after 2019, with $n \in \NN$. Then $u_{n+1} = 1.05\cdot u_n$, with $u_0 = 40800$. Hence, $u_n = 40800 \cdot 1.05^n$. Let \$$v_n$ be the amount saved in the $n$th year after 2019. Then $v_n = 0.40 \cdot u_n = 16320 \cdot 1.05^n$.

    \begin{ppart}
        In 2027, $n = 8$. Hence, his annual savings in 2027, in dollars, is given by \[v_8 = 16320 \cdot 1.05^8 = 24112 \text{ (to the nearest integer)}.\]
    \end{ppart}
    \begin{ppart}
        His total savings at the end of $n$ years, in dollars, is given by \[16320\bp{1.05^0 + 1.05^1 + \cdots + 1.05^n} = 16320 \bp{\frac{1 - 1.05^n}{1-1.05}} = 326400 \bp{1.05^n - 1}.\]
    \end{ppart}

    Consider $326400 \bp{1.05^n - 1} \geq 1 000 000$. Using G.C., we see that $n \geq 28.7$. Thus, he needs to work for a minimum of 29 complete years to reach his goal.
\end{solution}

\begin{problem}
    \begin{enumerate}
        \item A rope of length $200\pi$ cm is cut into pieces to form as many circles as possible, whose radii follow an arithmetic progression with common difference $0.25$ cm. Given that the smallest circle has an area of $\pi$ cm$^2$, find the area of the largest circle in terms of $\pi$.
        \item The sum of the first $n$ terms of a sequence is given by $S_n = \a^{-n}-1$, where $\a$ is a non-zero constant, $\a \neq 1$.
        \begin{enumerate}
            \item Show that the sequence is a geometric progression and state its common ratio in terms of $\a$.
            \item Find the set of values of $\a$ for which the sum to infinity of the sequence exists.
            \item Find the value of the sum to infinity.
        \end{enumerate}
    \end{enumerate}
\end{problem}
\begin{solution}
    \begin{ppart}
        Let the sequence $r_n$ be the radius of the $n$th smallest circle, in centimetres. Hence, $r_n = \frac14 + r_{n-1}$. Since the smallest circle has area $\pi$ cm$^2$, $r_1 = 1$. Thus, $r_n = 1 + \frac14 (n - 1)$.

        Consider the $n$th partial sum of the circumferences: \[2\pi r_1 + 2\pi r_2 + \cdots + 2\pi r_n = 2\pi \cdot n \bp{\frac{1 + \bs{1 + \frac14 (n - 1)}}{2}} = \frac{\pi(n^2 + 7n)}{4}.\] Since the rope has length $200\pi$ cm, we have the inequality \[\frac{\pi(n^2 + 7n)}{4} \leq 200 \pi \implies n^2 - 7n - 800 \leq 0 \implies (n+32)(n-25) \leq 0.\] Hence, $n \leq 25$. Since the rope is cut to form as many circles as possible, $n = 25$. Thus, the largest circle has area $\pi \cdot r_{25}^2 = 49\pi$ cm$^2$.
    \end{ppart}
    \begin{ppart}
        Let the sequence being summed by $u_1, u_2, \ldots$. Observe that \[u_n = S_n - S_{n-1} = \bp{\a^{-n} - 1} - \bp{\a^{-(n-1)} - 1} = \a^{-n} (1 - \a).\]

        \begin{psubpart}
            Observe that \[\frac{u_{n+1}}{u_n} = \frac{\a^{-(n+1)} (1 - \a)}{\a^{-n} (1 - \a)} = \a^{-1},\] which is a constant. Thus, $u_n$ is in GP with common ratio $\a^{-1}$.
        \end{psubpart}
        \begin{psubpart}
            Consider $S_\infty = \lim_{n \to \infty} S_n = \lim_{n \to \infty} (\a^{-n} - 1)$. For $S_\infty$ to exist, we need $\lim_{n \to \infty} \a^{-n}$ to exist. Hence, $\abs{\a^{-1}} < 1$, whence $\abs{\a} > 1$. Thus, $\a < -1$ or $\a > 1$. The solution set of $\a$ is thus $\bc{ x \in \RR : x < -1 \lor x > 1}$.
        \end{psubpart}
        \begin{psubpart}
            Since $\abs{\a^{-1}} < 1$, we know $\lim_{n \to \infty} \a^{-n} = 0$. Hence, $S_\infty = -1$.
        \end{psubpart}
    \end{ppart}
\end{solution}

\begin{problem}
    A sequence $u_1, u_2, u_3, \ldots$ is such that $u_{n+1} = 2u_n + An$, where $A$ is a constant and $n \geq 1$.

    \begin{enumerate}
        \item Given that $u_1 = 5$ and $u_2 = 15$, find $A$ and $u_3$.
    \end{enumerate}

    It is known that the $n$th term of this sequence is given by \[u_n = a(2^n) + bn +c,\] where $a$, $b$ and $c$ are constants.

    \begin{enumerate}
        \setcounter{enumi}{1}
        \item Find $a$, $b$ and $c$.
    \end{enumerate}
\end{problem}
\begin{solution}
    \begin{ppart}
        Substituting $n = 1$ into the recurrence relation yields $u_2 = 2u_1 + A$. Thus, $A = u_2 - 2u_1 = 5$. Substituting $n = 2$ into the recurrence relation yields $u_3 = 2u_2 + 2A = 40$.
    \end{ppart}
    \begin{ppart}
        Since $u_1=5$, $u_2 = 15$ and $u_3 = 40$, we have the following system \[\systeme{2a + b + c = 5,4a + 2b + c = 15,8a + 3b + c = 40}\] which has the unique solution $a = \frac{15}2$, $b=-5$ and $c=-5$
    \end{ppart}
\end{solution}

\clearpage
\begin{problem}
    The graphs of $y = 2^x / 3$ and $y=x$ intersect at $x = \a$ and $x=\b$ where $\a < \b$. A sequence of real numbers $x_1, x_2, x_3, \ldots$ satisfies the recurrence relation \[x_{n+1} = \frac13 \cdot 2^{x_n}, \qquad n \geq 1.\]

    \begin{enumerate}
        \item Prove algebraically that, if the sequence converges, then it converges to either $\a$ or $\b$.
        \item By using the graphs of $y=\frac13 \cdot 2^x$ and $y=x$, prove that
        \begin{itemize}
            \item if $\a < x_n < \b$, then $\a < x_{n+1} < x_n$
            \item if $x_n < \a$, then $x_n < x_{n+1} < \a$
            \item if $x_n > \b$, then $x_n < x_{n+1}$
        \end{itemize}

        Describe the behaviour of the sequence for the three cases.
    \end{enumerate}
\end{problem}
\begin{solution}
    \begin{ppart}
        Let $L = \lim\limits_{n \to \infty} x_n$. Then $L = \frac13 \cdot 2^L$. Since $y=x$ and $y=\frac13 \cdot 2^x$ intersect only at $x=\a$ and $x = \b$, then $\a$ and $\b$ are the only roots of $x = \frac13 \cdot 2^x$. Since $L$ is also a root of $x = \frac13 \cdot 2^x$, $L$ must be either $\a$ or $\b$.
    \end{ppart}
    \begin{ppart}
        \begin{center}
            \begin{tikzpicture}[trim axis left, trim axis right]
                \begin{axis}[
                    domain = -1:4,
                    samples = 101,
                    axis y line=middle,
                    axis x line=middle,
                    xtick = {0.458, 3.313},
                    xticklabels = {$\a$, $\b$},
                    ytick = \empty,
                    xlabel = {$x$},
                    ylabel = {$y$},
                    legend cell align={left},
                    legend pos=outer north east,
                    after end axis/.code={
                        \path (axis cs:0,0) 
                            node [anchor=north east] {$O$};
                        }
                    ]
                    \addplot[plotRed] {1/3*2^x};
        
                    \addlegendentry{$y = \frac13 \cdot 2^x$};
        
                    \addplot[plotBlue] {x};
        
                    \addlegendentry{$y=x$};

                    \draw[dotted, thick] (-1.5, 0.458) -- (4, 0.458);

                    \draw[dotted, thick] (0.458, 0) -- (0. 458, 0.458);

                    \draw[dotted, thick] (3.313, 0) -- (3.313, 3.313);

                    \draw[dotted, thick] (1.89, 1.89) -- (1.89, 0);

                    \draw[dotted, thick] (-0.7, -0.7) -- (-0.7, 0.205);

                    \draw[dotted, thick] (3.5, 0) -- (3.5, 3.771);

                    \fill (1.89, 1.89) circle[radius=2.5 pt] node[anchor=south east] {$x_n$};

                    \fill (1.89, 1.24) circle[radius=2.5 pt] node[anchor=north west] {$x_{n+1}$};

                    \fill (-0.7, -0.7) circle[radius=2.5 pt] node[anchor=west] {$x_{n}$};

                    \fill (-0.7, 0.205) circle[radius=2.5 pt] node[anchor=south] {$x_{n+1}$};

                    \fill (3.5, 3.5) circle[radius=2.5 pt] node[anchor=north west] {$x_{n}$};

                    \fill (3.5, 3.771) circle[radius=2.5 pt] node[anchor=south east] {$x_{n+1}$};
                \end{axis}
            \end{tikzpicture}
        \end{center}
        
        If $\a < x_n < \b$, then $x_n$ is decreasing and converges to $\a$. If $x_n < \a$, then $x_n$ is increasing and converges to $\a$. If $x_n > \b$, then $x_n$ is increasing and diverges.
    \end{ppart}
\end{solution}