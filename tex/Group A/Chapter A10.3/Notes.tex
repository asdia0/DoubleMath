\section{Notes}

\subsection{Geometrical Effect of Multiplying Complex Numbers}

Let points $P$, $Q$ and $R$ represent the complex numbers $z_1$, $z_2$ and $z_3$ respectively, as illustrated in the Argand diagram below.

\begin{center}\tikzsetnextfilename{344}
    \begin{tikzpicture}[trim axis left, trim axis right]
        \begin{axis}[
            domain = 0:10,
            samples = 101,
            axis y line=middle,
            axis x line=middle,
            xtick = \empty,
            ytick = \empty,
            xmax=3,
            xmin=-6,
            ymin=-2,
            ymax=7,
            xlabel = {$\Re$},
            ylabel = {$\Im$},
            legend cell align={left},
            legend pos=outer north east,
            after end axis/.code={
                \path (axis cs:0,0) 
                    node [anchor=north east] {$O$};
                }
            ]

            \coordinate (O) at (0, 0);
            \coordinate[label=above:$P(z_1)$] (P) at (-1, 3);
            \coordinate[label=above:$Q(z_2)$] (Q) at (2, 1);
            \coordinate[label=above:$R(z_1 z_2)$] (R) at (-5, 5);
            \coordinate (M) at (5, 0);
            \coordinate (N) at (-5, 0);

            \fill (P) circle[radius=2.5pt];
            \fill (Q) circle[radius=2.5pt];
            \fill (R) circle[radius=2.5pt];

            \draw[very thick] (O) -- (R);
            \draw[dotted] (O) -- (P);
            \draw[dotted] (O) -- (Q);

            \node[anchor=west] at ($(O)!0.7!(P)$) {$r_1$};
            \node[anchor=north west] at ($(O)!0.7!(Q)$) {$r_2$};
            \node[anchor=south west] at ($(O)!0.7!(R)$) {$r_1 r_2$};


            \draw pic [draw, angle radius=8mm] {angle = M--O--P};
            \draw pic [draw, angle radius=6mm] {angle = M--O--Q};
            \draw pic [draw, angle radius=6mm] {angle = P--O--R};

            \node[anchor=north] at (0.6, 0) {$\t_2$};
            \node at (-0.8, 1.4) {$\t_2$};
            \node at (0.3, 0.8) {$\t_1$};
        \end{axis}
    \end{tikzpicture}
\end{center}

Geometrically, the point $R(z_1 z_2)$ is obtained by
\begin{enumerate}
    \item scaling by a factor of $r_2$ on $\oa{OP}$ to obtain a new modulus of $r_1 r_2$, followed by
    \item rotating $\oa{OP}$ through an angle $\t_2$ about $O$ in an anti-clockwise direction if $\t_2 > 0$ to obtain a new argument $\t_1 + \t_2$ (or in a clockwise direction if $\t_2 < 0$).
\end{enumerate}

\subsection{De Moivre's Theorem and its Applications}

\begin{theorem}[De Moivre's Theorem]
    For $n \in \QQ$, if $z = r\bp{\cos \t + \i \sin \t} = r \e^{\i \t}$, then \[z^n = r^n \bp{\cos n\t + \i \sin n\t} = r^n \e^{\i n \t}.\]
\end{theorem}
\begin{proof}
    Write $z^n$ in exponential form before converting it into trigonometric form.
\end{proof}

We now discuss some of the applications of De Moivre's theorem.

\begin{method}[Finding $n$th Roots]
    Suppose we want to find the $n$th roots of a complex number $w = r \e^{\i \t}$. We begin by setting up the equation \[z^n = w = r \e^{\i \bp{\t + 2k\pi}},\] where $k \in \ZZ$. Next, we take $n$th roots on both sides, which yields \[z = r^{1/n} \e^{\i \bp{\t + 2k\pi} /n}.\] Lastly, we pick values of $k$ such that $\arg z = \frac{\t + 2k\pi}{n}$ lies in the principal interval $(-\pi, \pi]$.
\end{method}

\begin{definition}
    Let $n \in \ZZ$. The \vocab{$n$th roots of unity} are the $n$ solutions to the equation \[z^n - 1 = 0.\]
\end{definition}

\begin{proposition}[Roots of Unity in Polar Form]
    The $n$th roots of unity are given by \[z = \cos \frac{2k \pi}{n} + \i \sin \frac{2k \pi}{n} = \e^{\i \bp{2k \pi / n}},\] where $k \in \ZZ$.
\end{proposition}
\begin{proof}
    Use De Moivre's theorem.
\end{proof}

\begin{fact}{Geometric Properties of Roots of Unity}
    On an Argand diagram, the $n$th roots of unity
    \begin{itemize}
        \item all lie on a circle of radius 1.
        \item are equally spaced apart.
        \item form a regular $n$-gon.
    \end{itemize}
\end{fact}

De Moivre's theorem can also be used to derive trigonometric identities. The trigonometric identities you will be required to prove typically involve reducing ``powers'' to ``multiple angles'' (e.g. expressing $\sin^3 \t$ in terms of $\sin \t$ and $\sin 3\t$), or vice versa (e.g. expressing $\sin 3\t$ in terms of $\sin \t$ and $\sin^3 \t$).

\begin{proposition}[Power to Multiple Angles]
    Let $z = \cos \t + \i \sin \t = \e^{\i \t}$. Then \[z^n + z^{-n} = 2\cos n\t, \qquad z^n - z^{-n} = 2\i \sin n\t.\]
\end{proposition}

\begin{method}[Multiple Angles to Powers]
    Suppose we want to express $\cos n\t$ and $\sin n\t$ in terms of powers of $\sin \t$ and $\cos \t$. We begin by invoking De Moivre's theorem: \[\cos n\t + \i \sin n\t = \bp{\cos \t + \i \sin \t}^n.\] Next, using the binomial theorem, \[\cos n\t + \i \sin n\t = \sum_{k = 0}^n \binom{n}{k} \cos^k \t \sin^{n-k} \t.\] We then take the real and imaginary parts of both sides to isolate $\cos n\t$ and $\sin n\t$: \[\cos n\t = \Re \sum_{k = 0}^n \binom{n}{k} \cos^k \t \sin^{n-k} \t, \qquad \sin n\t = \Im \sum_{k = 0}^n \binom{n}{k} \cos^k \t \sin^{n-k} \t.\]
\end{method}

\begin{example}
    Suppose we want to write $\sin 2\t$ in terms of $\sin \t$ and $\cos \t$. Using De Moivre's theorem, \[\cos 2\t + \i \sin 2\t = \bp{\cos \t + \i \sin \t}^2 = \cos^2 \t + 2\i \cos \t \sin \t - \sin^2 \t.\] Comparing imaginary parts, we obtain \[\sin 2\t = 2\cos\t\sin\t\] as expected.
\end{example}

Another way to derive new trigonometric identities is to differentiate known identities.

\begin{example}
    Using the ``power to multiple angle'' formula above, one can show that \[\cos^6 \t = \frac1{32} \bp{\cos 6\t + 6\cos4\t + 15\cos2\t + 10}.\] Differentiating, we obtain a new trigonometric identity: \[\sin \t \cos^5 \t = \frac1{32} \bp{\sin 6\t + 4\sin 4\t + 5\sin 2\t}.\]
\end{example}