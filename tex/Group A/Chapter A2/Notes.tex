\section{Notes}

\begin{theorem}[Location of Root Theorem]
    Let $f(x)$ be a continuous function on the interval $[a, b]$. If $f(a)$ and $f(b)$ have opposite signs, i.e. $f(a) f(b) < 0$, then there exists at least one real root in $[a, b]$. Furthermore, if $f(x)$ is strictly monotonic on $[a, b]$, then there is exactly one real root in $[a, b]$.
\end{theorem}
\begin{proof}
    Follows immediately from the Intermediate Value Theorem.
\end{proof}

\subsection{Linear Interpolation}

Linear interpolation is a numerical method based on approximating the curve $y = f(x)$ to a straight line in the vicinity of the root. The approximate root of the equation $f(x) = 0$ is the intersection of this straight line with the $x$-axis.

\begin{center}\tikzsetnextfilename{113}
    \begin{tikzpicture}
        \begin{axis}[
                domain = -4.8:-0.5,
                samples = 101,
                axis y line=none,
                axis x line=middle,
                xtick = {-3.68, -2.90},
                xticklabels = {$\a$, $c$},
                ytick = \empty,
                xlabel = {$x$},
                ylabel = {$y$},
            ]
        \addplot[color=red!50] {0.04 * x^3 + 2};
        
        \coordinate[label=left:$A$] (A) at (-4.5, -1.645);
        \coordinate[label=above:$B$] (B) at (-1, 1.96);
        \draw[dashed] (A) -- (B);
        \fill (A) circle[radius=2.5pt];
        \fill (B) circle[radius=2.5pt];
        \end{axis}
        \end{tikzpicture}
\end{center}

\begin{statement}[Linear Interpolation]
    Suppose $f(x) = 0$ has exactly one root $\a$ in the interval $[a, b]$, where $f(a)$ and $f(b)$ have opposite signs. Then linear interpolation gives \[\a \approx \frac{af(b) - bf(a)}{f(b)-f(a)}.\]
\end{statement}
\begin{proof}
    By the point-slope formula, the line connecting the points $(a, f(a))$ and $(b, f(b))$ is given by \[y - f(a) = \frac{f(b) - f(a)}{b - a} (x - a).\] At the point $(c, 0)$, \[0 - f(a) = \frac{f(b)-f(a)}{b-a} (c -a) \implies c = \frac{af(b) - bf(a)}{f(b)-f(a)}.\]
\end{proof}

Linear interpolation can be repeatedly applied by replacing either the lower or upper bound of the interval with the previously found approximation.

Convergence of the approximations is guaranteed for linear interpolation. However, how good the estimation is depends on how "straight" the graph of $y = f(x)$ is in $[a, b]$, i.e. the rate at which $f'(x)$ is changing in $[a, b]$. This rate also affects the rate of convergence: if $f'(x)$ changes considerably, the rate of convergence is slow; if $f'(x)$ does not change much, the rate of convergence is fast.

\subsection{Fixed Point Iteration}

Suppose $f(x) = 0$ can be rewritten into the form $x = F(x)$. Then if $\a$ is a root to $f(x) = 0$, it must also be a root to $\a = F(\a)$. Due to the recursive nature of this equation, we can keep replacing the $\a$ in the argument with $F(\a)$: \[\a = F(\a) = F(F(\a)) = F(F(F(\a))) = \cdots.\] Let $\{x_{n}\}$ be a sequence such that $x_{n+1} = F(x_n)$ (i.e. $x_n = F^{n} (x)$). Then $\a = \lim_{n \to \infty} x_n$ (assuming it exists). We can thus approximate $\a$ by choosing an initial approximation $x_1$ and repeatedly apply $F$ to each subsequent approximation.

Geometrically, fixed-point iteration can be seen as repeatedly "reflecting" the initial approximation point $(x_1, F(x_1))$ about the line $y = x$, while keeping the resultant point on the curve $y = F(x)$.

\medskip

\begin{minipage}{0.45\textwidth}
    \centering
    \tikzsetnextfilename{331}
    \begin{tikzpicture}[scale=0.8]
        \begin{axis}[
            domain = 0:2,
            samples = 101,
            axis y line=middle,
            axis x line=middle,
            xtick = {1.7, 0.721, 1.112},
            xticklabels = {$x_1$, $x_2$, $x_3$},
            ytick = \empty,
            xlabel = {$x$},
            ylabel = {$y$},
            ymax=1.4,
            xmax=2,
            after end axis/.code={
                \path (axis cs:0,0)
                    node [anchor=north east] {$O$};
                }
            ]
            \addplot[red!50] {cos(\x r) + 0.5 * x};
            \addplot[blue!50] {x};
            \draw[dotted, thick] (1.7, 0) -- (1.7, 0.721);
            \draw[dotted, thick] (0.721, 0.721) -- (0.721, 0);
            \draw[dotted, thick] (1.112, 0.999) -- (1.112, 0);
            \begin{scope}[decoration={
                markings,
                mark=at position 0.5 with {\arrow{>}}}
                ]
                \draw[postaction={decorate}, thick] (1.7, 0.721) -- (0.721, 0.721);
                \draw[postaction={decorate}, thick] (0.721, 0.721) -- (0.721, 1.112);
                \draw[postaction={decorate}, thick] (0.721, 1.112) -- (1.112, 1.112);
                \draw[postaction={decorate}, thick] (1.112, 1.112) -- (1.112, 0.999);
            \end{scope}
        \end{axis}
    \end{tikzpicture}
\end{minipage}
\begin{minipage}{0.45\textwidth}
    \centering
    \tikzsetnextfilename{332}
    \begin{tikzpicture}[scale=0.8]
        \begin{axis}[
            domain = 0:pi/2,
            samples = 101,
            axis y line=middle,
            axis x line=middle,
            xtick = {1.2, 0.725, 1.497},
            ytick = \empty,
            xticklabels = {$x_1$, $x_2$, $x_3$},
            xlabel = {$x$},
            ylabel = {$y$},
            ymax=2.2,
            xmax=1.7,
            after end axis/.code={
                \path (axis cs:0,0)
                    node [anchor=north east] {$O$};
                }
            ]
    
            \addplot[red!50] {2 *cos(\x r)};
            \addplot[blue!50] {x};
            \draw[dotted, thick] (1.2, 0) -- (1.2, 0.725);
            \draw[dotted, thick] (0.725, 0.725) -- (0.725, 0);
            \draw[dotted, thick] (1.497, 0.147) -- (1.497, 0);
            \begin{scope}[decoration={
                markings,
                mark=at position 0.5 with {\arrow{>}}}
                ]
                \draw[postaction={decorate}, thick] (1.2, 0.725) -- (0.725, 0.725);
                \draw[postaction={decorate}, thick] (0.725, 0.725) -- (0.725, 1.497);
                \draw[postaction={decorate}, thick] (0.725, 1.497) -- (1.497, 1.497);
                \draw[postaction={decorate}, thick] (1.497, 1.497) -- (1.497, 0.147);
            \end{scope}
        \end{axis}
    \end{tikzpicture}
\end{minipage}

\medskip

Convergence is not guaranteed (see the above diagram on the right). The rate at which the approximations converge to $\a$ depends on the value of $\abs{F'(x)}$ near $\a$. The smaller $\abs{F'(x)}$ is, the faster the convergence. It should be noted that fixed-point iteration fails (almost surely) if $\abs{F'(x)} > 1$ near $\a$.

\subsection{Newton-Raphson Method}

The Newton-Raphson method is a numerical method that improves on linear interpolation by considering the tangent line at the initial approximation to the root.

\begin{statement}[Newton-Raphson Method]
    Let $\a$ be a root to $f(x) = 0$. The sequence $\{x_n\}$ defined by \[x_{n+1} = x_n - \frac{f(x_n)}{f'(x_n)}\] converges to $\a$, provided certain conditions are met.
\end{statement}
\begin{proof}
    Consider the tangent to $y = f(x)$ at the point where $x = x_1$. In most circumstances, the point $(x_2, 0)$ where this tangent cuts the $x$-axis will be nearer to the point $(\a, 0)$ than $(x_1, 0)$ was. By the point-slope formula, the equation of the tangent to the curve at $x = x_1$ is \[y - f(x_1) = f'(x_1)(x-x_1).\] Since $(x_2, 0)$ lies on the tangent line, we have \[x_2 = x_1 - \frac{f(x_1)}{f'(x_1)}.\] By repeating the Newton-Raphson process, we are able to get better approximations to $\a$. In general, \[x_{n+1} = x_n - \frac{f(x_n)}{f'(x_n)}.\]
\end{proof}

\medskip

\begin{minipage}{0.45\textwidth}
    \centering
    \tikzsetnextfilename{329}
    \begin{tikzpicture}[scale=0.8]
        \begin{axis}[
            domain = 1.5:2.5,
            samples = 50,
            axis y line=middle,
            axis x line=middle,
            xtick = {1.8393, 1.994, 2.4},
            xticklabels = {$\alpha$, $x_2$, $x_1$},
            ytick = \empty,
            xlabel = {$x$},
            ylabel = {$y$},
            ymin = -2,
            legend cell align={left},
            legend pos=outer north east,
            after end axis/.code={
                \path (axis cs:1.5,0) 
                    node [anchor=east] {$O$};
                }
            ]
    
            \addplot[red!50] {x^3-x^2-x-1};
    
            \draw[dotted, thick] (2.4, 0) -- (2.4, 4.644);
    
            \addplot[blue!50] {11.48 * (x-2.4) + 4.644};
        \end{axis}
    \end{tikzpicture}
\end{minipage}\hspace{0.05\textwidth}
\begin{minipage}{0.45\textwidth}
    \centering
    \tikzsetnextfilename{330}
    \begin{tikzpicture}[scale=0.8]
        \begin{axis}[
            domain = -1.3:1.3,
            samples = 201,
            axis y line=middle,
            axis x line=middle,
            xlabel = {$x$},
            ylabel = {$y$},
            xtick={0.864, 0.5, -1.022},
            xticklabels={$\alpha$, $x_1$, $x_2$},
            ytick=\empty,
            ymin=-0.5,
            ymax=1,
            legend cell align={left},
            legend pos=outer north east,
            after end axis/.code={
                \path (axis cs:0,0) 
                    node [anchor=north east] {$O$};
                }
            ]
            
            \addplot[red!50] {sec(\x r)^2 - e^x};
    
            \draw[dotted] (0.5, 0) -- (0.5, -0.35);
    
            \addplot[blue!50] {-0.23 * (x - 0.5) - 0.35};
        \end{axis} 
    \end{tikzpicture}
\end{minipage}

\medskip

The rate of convergence when using the Newton-Raphson method depends on the first approximation used and the shape of the curve in the neighbourhood of the root. In extreme cases, these factors may lead to failure (divergence). The three main cases are:
\begin{itemize}
    \item $\abs{f'(x_1)}$ is too small (extreme case when $f'(x_1) = 0$),
    \item $f'(x)$ increases/decreases too rapidly ($\abs{f''(x)}$ is too large),
    \item $x_1$ is too far away from $\a$.
\end{itemize}