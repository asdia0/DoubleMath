\section{Notes}

\subsection{Equation of a Plane}

\begin{definition}
    Suppose the plane $\pi$ passes through a fixed point $A$ with position vector $\vec a$, and $\pi$ is parallel to two vectors $\vec b_1$ and $\vec b_2$, where $\vec b_1$ and $\vec b_2$ are not parallel to each other. Then the vector equation (in \vocab{parametric form}) of $\pi$ is given by \[\pi : \vec r = \vec a + \l \vec b_1 + \m \vec b_2,\] where $\vec r$ is the position vector of any point $P$ on $\pi$, and $\l$ and $\m$ are real parameters.
\end{definition}

\begin{definition}
    Suppose the plane $\pi$ passes through a fixed point $A$ with position vector $\vec a$, and $\pi$ has normal vector $\vec n$. Let $P$ be an arbitrary point on $\pi$. Then $\oa{AP}$ is perpendicular to the normal vector $\vec n$, i.e. $\oa{AP} \dotp \vec n = 0$. Since $\oa{AP} = \vec r - \vec a$, by the distributivity of the scalar product, one has \[\vec r \dotp \vec n = \vec a \dotp \vec n.\] This is the \vocab{scalar product form} of the vector equation of $\pi$, which is more commonly written as \[\vec r \dotp \vec n = d.\]
\end{definition}

\begin{definition}
    Let the plane $\pi$ have scalar product form \[\pi : \vec r \dotp \vec n = \vec a \dotp \vec n.\] Let $\vec r = \cveciiix{x}{y}{z}$, $\vec a = \cveciiix{a_1}{a_2}{a_3}$ and $\vec n = \cveciiix{n_1}{n_2}{n_3}$. Then \[\pi : n_1 x + n_2 y + n_3 z = a_1 n_1 + a_2 n_2 + a_3 n_3\] is the \vocab{Cartesian equation} of $\pi$, which is more commonly written as \[\pi : n_1 x + n_2 y + n_3 z = d.\]
\end{definition}

\begin{method}[Converting between Forms]
    To convert from parametric form to scalar product form, take $\vec n = \vec b_1 \crossp \vec b_2$. To convert from the Cartesian equation to parametric form, express $x$ in terms of $y$ and $z$, then replace $y$ and $z$ with $\l$ and $\m$ respectively.
\end{method}

\begin{example}[Parametric to Scalar Product Form]
    Let the plane $\pi$ have parametric form $\vec r = \cveciiix123 + \l \cveciiix456 + \m \cveciiix789$. Then the normal vector to $\pi$ is given by \[\vec n = \cveciii456 \crossp \cveciii789 = \cveciii{-3}{6}{-3} \parallel \cveciii{1}{-2}{1}.\] Hence, \[d = \cveciii123 \dotp \cveciii1{-2}1 = 0,\] whence $\pi$ has scalar product form \[\vec r \dotp \cveciii1{-2}1 = 0.\]
\end{example}

\begin{example}[Cartesian to Parametric Form]
    Let the plane $\pi$ have Cartesian equation \[x + y + z = 10.\] Solving for $x$ and replacing $y$ and $z$ with $\l$ and $\m$ respectively, we get \[x = 10 - \l - \m, \quad y = \l, \quad z = \m.\] Hence, $\pi$ has parametric form \[\vec r = \cveciii{x}{y}{z} = \cveciii{10 - \l - \m}{\l}{\m} = \cveciii{10}00 + \l \cveciii{-1}{1}{0} + \m \cveciii{-1}0{1}, \quad \l, \m \in \RR.\]
\end{example}

\subsection{Point and Plane}

\begin{proposition}[Relationship between Point and Plane]
    A point lies on a plane if and only if its position vector (or its equivalent coordinates) satisfies the equation of the plane.
\end{proposition}
\begin{proof}
    Trivial.
\end{proof}

\begin{proposition}[Perpendicular Distance between Point and Plane]
    Let $F$ be the foot of perpendicular from a point $Q$ to the plane $\pi$ with vector equation $\pi : \vec r \dotp \vec n = d$. Let $A$ be a point on $\pi$. Then $QF$, the perpendicular distance from $Q$ to $\pi$, is given by \[QF = \abs{\oa{QA} \dotp \hat{\vec n}} = \frac{\abs{d - \vec q \dotp \vec n}}{\abs{\vec n}}.\]
\end{proposition}
\begin{proof}
    Note that $QF$ is the length of projection of $\oa{QA}$ onto the normal vector $\vec n$. Hence, \[QF = \abs{\oa{QA} \dotp \hat{\vec n}}\] follows directly from the formula for the length of projection. Now, observe that \[\oa{QA} \dotp \vec n = \oa{OA} \dotp \vec n - \oa{OQ} \dotp \vec n = d - \vec q \dotp \vec n.\] Hence, \[QF = \frac{\abs{\oa{QA} \dotp \vec n}}{\abs{\vec n}} = \frac{\abs{d - \vec q \dotp \vec n}}{\abs{\vec n}}.\]
\end{proof}

\begin{corollary}
    $OF$, the perpendicular distance from the plane $\pi$ to the origin $O$, is \[OF = \frac{\abs{d}}{\abs{\vec n}}.\]
\end{corollary}
\begin{proof}
    Take $\vec q = \vec 0$.
\end{proof}

\begin{method}[Foot of Perpendicular from Point to Plane]
    Let $F$ be the foot of perpendicular from a point $Q$ to the plane $\pi$ with vector equation $\pi : \vec r \dotp \vec n = d$. To find the position vector $\oa{OF}$, we use the fact that
    \begin{itemize}
        \item $QF$ is perpendicular to $\pi$, i.e. $\oa{QF} = \l \vec n$ for some $\l \in \RR$, and
        \item $F$ lies on $\pi$, i.e. $\oa{OF} \dotp \vec n = d$.
    \end{itemize}
\end{method}

\begin{example}[Foot of Perpendicular from Point to Plane]
    Let the plane $\pi$ have equation $\pi : \vec r \dotp \cveciiix123 = 10$. Let $Q(4, 5, 6)$. Let $F$ be the foot of perpendicular from $Q$ to $\pi$. We wish to find $\oa{OF}$.

    Since $QF$ is perpendicular to $\pi$, we have \[\oa{QF} = \l \cveciii123, \quad \l \in \RR.\] Hence, \[\oa{OF} = \oa{OQ} + \oa{QF} = \cveciii456 + \l\cveciii123.\] Taking the scalar product on both sides, we get \[10 = \oa{OF} \dotp \cveciii123 = \bs{\cveciii456 + \l\cveciii123} \dotp \cveciii123 = 32 + 14\l.\] Thus, $\l = -11/7$, whence \[\oa{OF} = \cveciii456 - \frac{11}{7}\cveciii123 = \frac17 \cveciii{17}{13}{9}.\]
\end{example}

\subsection{Line and Plane}

\begin{fact}[Relationship between Line and Plane]
    Given a line $l : \vec r = \vec a + \l \vec b$, $\l \in \RR$, and a plane $\pi : \vec r \dotp \vec n = d$, there are three possible cases:
    \begin{itemize}
        \item \textbf{$l$ and $\pi$ do not intersect.} $l$ and $\pi$ are parallel and have no common point.
        \item \textbf{$l$ lies on $\pi$.} $l$ and $\pi$ are parallel and any point on $l$ is also a point on $\pi$.
        \item \textbf{$l$ and $\pi$ intersect once.} $l$ and $\pi$ are not parallel.
    \end{itemize}
\end{fact}

There are two methods to determine the relationship between a line and a plane.

\begin{method}[Using Normal Vector]
    \phantom{.}
    \begin{itemize}
        \item If $l$ and $\pi$ do not intersect, then $\vec b \dotp \vec n = 0$ and $\vec a \dotp vec n \neq d$.
        \item If $l$ lies on $\pi$, then $\vec b \dotp \vec n = 0$ and $\vec a \dotp \vec n =  d$.
        \item If $l$ and $\pi$ intersect once, then $\vec b \dotp \vec n \neq 0$.
    \end{itemize}
\end{method}

\begin{method}[Solving Simultaneous Equations]
    Solve $l : \vec r = \vec a + \l \vec b$, $\l \in \RR$ and $\pi : \vec r \dotp \vec n = d$ simultaneously.
    \begin{itemize}
        \item If there are no solutions, then $l$ and $\pi$ do not intersect.
        \item If there are infinitely many solutions, then $l$ lies on $\pi$.
        \item If there is a unique solution, then $l$ and $\pi$ intersect once.
    \end{itemize}
\end{method}

\begin{proposition}[Acute Angle between Line and Plane]
    Let $\t$ be the acute angle between the line $l : \vec r = \vec a + \l \vec b$, $\l \in \RR$ and the plane $\pi : \vec r \dotp \vec n = d$. Then \[\sin \t = \frac{\abs{\vec b \dotp \vec n}}{\abs{\vec b} \abs{\vec n}}.\]
\end{proposition}
\begin{proof}
    We first find $\f$, the acute angle between $l$ and the normal. Recall that \[\cos \f = \frac{\abs{\vec b \dotp \vec n}}{\abs{\vec b} \abs{\vec n}}.\] Since $\f = \frac\pi2 - \t$, we have \[\cos{\frac\pi2 - \t} = \sin \t = \frac{\abs{\vec b \dotp \vec n}}{\abs{\vec b} \abs{\vec n}}.\]
\end{proof}

\subsection{Two Planes}

\begin{proposition}[Acute Angle between Two Planes]
    The acute angle $\t$ between two planes $\pi_1 : \vec r \dotp \vec n_1 = d_1$ and $\pi_2 : \vec r \dotp \vec n_2 = d_2$ is given by \[\cos \t = \frac{\abs{\vec n_1 \dotp \vec n_2}}{\abs{\vec n_1} \abs{\vec n_2}}.\]
\end{proposition}
\begin{proof}
    Consider the following diagram.

    \begin{center}\tikzsetnextfilename{337}
        \begin{tikzpicture}
            \coordinate (A) at (0, 0);
            \coordinate (B) at (6, 0);
            \coordinate (C) at (4, -1);
            \coordinate (D) at (1, 2);
            \coordinate (E) at (3, 0);

            \coordinate (F) at (2, 1);
            \coordinate (G) at (4, 0);
            \coordinate (H) at (3, 2);
            \coordinate (I) at (4, 1);
            \coordinate (J) at (4, 3);
            \coordinate (K) at (5, 4);
            \coordinate (L) at (4, 4.5);

            \draw[very thick] (A) -- (B);
            \draw[very thick] (C) -- (D);

            \node[anchor=south west] at (A) {$\pi_1$};
            \node[anchor=west] at (C) {$\pi_2$};

            \draw[-Latex] (F) -- (H);
            \draw[-Latex] (G) -- (I);
            \draw[dotted] (H) -- (K);
            \draw[dotted] (I) -- (L);

            \node[anchor=south east] at (H) {$\vec n_2$};
            \node[anchor=west] at (I) {$\vec n_1$};

            \draw pic [draw, angle radius=9mm, "$\t$"] {angle = D--E--A};
            \draw pic [draw, angle radius=9mm, "$\t$"] {angle = K--J--L};

            \draw pic [draw, angle radius=4mm] {right angle = J--F--C};
            \draw pic [draw, angle radius=4mm] {right angle = A--G--L};
        \end{tikzpicture}
    \end{center}

    It is hence clear that the acute angle between the two planes is equal to the acute angle between the two normal vectors. Thus, \[\cos \t = \frac{\abs{\vec n_1 \dotp \vec n_2}}{\abs{\vec n_1} \abs{\vec n_2}}.\]
\end{proof}

\begin{fact}[Relationship between Two Planes]
    Given two distinct planes $\pi_1 : \vec r \dotp \vec n_1 = d_1$ and $\pi_2 : \vec r \dotp \vec n_2 = d_2$, there are two possible cases:
    \begin{itemize}
        \item \textbf{$\pi_1$ and $\pi_2$ do not intersect.} The two planes are parallel ($\vec n_1 \parallel \vec n_2$).
        \item \textbf{$\pi_1$ and $\pi_2$ intersect at a line.} The two planes are not parallel ($\vec n_1 \nparallel \vec n_2$).
    \end{itemize}
\end{fact}

Suppose the two planes are not parallel to each other. There are two methods to obtain the equation of the line of intersection.

\begin{method}[Via Cartesian Form]
    Write the equations of the two planes in Cartesian form and solve the two equations simultaneously.
\end{method}
\begin{method}[Via Normal Vectors]
    Observe that as the line of intersection $l$ lies on both planes, $l$ is perpendicular to both the normal vectors $\vec n_1$ and $\vec n_2$. Hence, $l$ is parallel to their cross product, $\vec n_1 \crossp \vec n_2$. Thus, if we know a point on the line of intersection $l$ (say point $A$ with position vector $\vec a$), then the vector equation of $l$ is given by \[l : \vec r = \vec a + \l \vec b, \quad \l \in \RR,\] where $\vec b$ is any scalar multiple of $\vec n_1 \crossp \vec n_2$.
\end{method}