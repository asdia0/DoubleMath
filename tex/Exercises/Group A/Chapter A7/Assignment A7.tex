\section{Assignment A7}

\begin{problem}
    The points $A$ and $B$ have position vectors relative to the origin $O$, denoted by $\vec a$ and $\vec b$ respectively, where $\vec a$ and $\vec b$ are non-parallel vectors. The point $P$ lies on $AB$ such that $AP : PB = \l : 1$. The point $Q$ lies on $OP$ extended such that $OP = 2PQ$ and $\oa{BQ} = \oa{OA} + \m \oa{OB}$. Find the values of the real constants $\l$ and $\m$.
\end{problem}
\begin{solution}
    By the ratio theorem, \[\oa{OP} = \frac{\vec a + \l \vec b}{1 + \l} \implies \oa{OQ} = \frac32 \oa{OP} = \frac32 \cdot \frac{\vec a + \l \vec b}{1 + \l}.\] However, we also have \[\oa{OQ} = \oa{OB} + \oa{BQ} = \vec a + (1 + \m) \vec b.\] This gives the equality \[\frac32 \cdot \frac{\vec a + \l \vec b}{1 + \l} = \vec a + (1 + \m) \vec b.\] Since $\vec a$ and $\vec b$ are non-parallel, we can compare the $\vec a$- and $\vec b$-components of both vectors separately. This gives us \[\frac32 \cdot \frac1{1 + \l} = 1, \quad \frac32 \cdot \frac\l{1 + \l} = 1 + \m,\] which has the unique solution $\l = 1/2$ and $\m = -1/2$.
\end{solution}

\begin{problem}
    Given that $\vec a = \vec i + \vec j$, $\vec b = 4 \vec i - 2 \vec j + 6 \vec k$ and $\vec p = \l \vec a + (1 - \l) \vec b$ where $\l \in \RR$, find the possible value(s) of $\l$ for which the angle between $\vec p$ and $\vec k$ is $45\deg$.
\end{problem}
\begin{solution}
    Observe that \[\vec p = \l \vec a + (1 - \l)\vec b = \l \cveciii110 + (1 - \l)\cveciii4{-2}6 = \cveciii{4 - 3 \l}{-2 + 3\l}{6 - 6\l}.\] Thus, \[\abs{\vec p}^2 = (4 - 3\l)^2 + (-2 + 3\l)^2 + (6 - 6\l)^2 = 54\l^2 - 108\l + 56.\] Since the angle between $\vec p$ and $\vec k$ is $45\deg$, \[\cos 45\deg = \frac{\vec p \dotp \vec k}{\abs{\vec p} \abs{\vec k}} \implies \frac1{\sqrt2} = \frac{6 - 6\l}{\abs{\vec p}} \implies \frac{\abs{\vec p}^2}{2} = (6 - 6\l)^2.\] We thus obtain the quadratic equation \[\frac{54\l^2 - 108\l + 56}2 = 36\l^2 - 72\l + 36 \implies 9\l^2 - 18\l + 8 = 0,\] which has solutions $\l = 2/3$ and $\l = 4/3$. However, we must reject $\l = 4/3$ since $6 - 6\l = \abs{\vec p}/\sqrt{2} > 0 \implies \l < 1$. Thus, $\l = 2/3$.
\end{solution}

\clearpage
\begin{problem}
    \begin{enumerate}
        \item $\vec a$ and $\vec b$ are non-zero vectors such that $\vec a = (\vec a \dotp \vec b) \vec b$. State the relation between the directions of $\vec a$ and $\vec b$, and find $\abs{\vec b}$.
        \item $\vec a$ is a non-zero vector such that $\vec a = \sqrt3$ and $\vec b$ is a unit vector. Given that $\vec a$ and $\vec b$ are non-parallel and the angle between them is $5\pi/6$, find the exact value of the length of projection of $\vec a$ on $\vec b$. By considering $(2\vec a + \vec b)\dotp(2 \vec a + \vec b)$, or otherwise, find the exact value of $\abs{2\vec a + \vec b}$.
    \end{enumerate}
\end{problem}
\begin{solution}
    \begin{ppart}
        $\vec a$ and $\vec b$ either have the same or opposite direction. Let $\vec b = \l \vec a$ for some $\l \in \RR$. \[\vec a = (\vec a \dotp \vec b) \vec b = (\vec a \dotp \l \vec a) \l \vec a = \l^2 \abs{\vec a}^2 \vec a \implies \l^2\abs{\vec a}^2 = 1 \implies \abs{\vec b} = \abs{\l\abs{\vec a}} = 1.\]
    \end{ppart}
    \begin{ppart}
        Note that $\abs{\vec a \cdot \vec b} = \abs{\vec a}\abs{\vec b} \cos{5\pi/6} = -3/2$. Hence, the length of projection of $\vec a$ on $\vec b$ is $\abs{\vec a \cdot \hat{\vec b}} = 3/2$ units.

        Observe that \[\abs{2\vec a + \vec b}^2 = (2\vec a + \vec b)\dotp (2\vec a + \vec b) = 4 \abs{\vec a}^2 + 4 (\vec a \dotp \vec b) + \abs{\vec b}^2 = 7.\] Thus, $\abs{2\vec a + \vec b} = \sqrt{7}$.
    \end{ppart}
\end{solution}

\begin{problem}
    The points $A$, $B$, $C$, $D$ have position vectors $\vec a$, $\vec b$, $\vec c$, $\vec d$ given by $\vec a = \vec i + 2\vec j + 3\vec k$, $\vec b = \vec i + 2\vec j + 2\vec k$, $\vec c = 3\vec i + 2 \vec j + \vec k$, $\vec d = 4\vec i - \vec j - \vec k$, respectively. The point $P$ lies on $AB$ produced such that $AP = 2AB$, and the point $Q$ is the mid-point of $AC$.

    \begin{enumerate}
        \item Show that $PQ$ is perpendicular to $AQ$.
        \item Find the area of the triangle $APQ$.
        \item Find a vector perpendicular to the plane $ABC$.
        \item Find the cosine of the angle between $\oa{AD}$ and $\oa{BD}$.
    \end{enumerate}
\end{problem}
\begin{solution}
    Note that $\oa{AB} = \cveciiix00{-1}$, $\oa{AC} = \cveciiix20{-2}$ and $\oa{AD} = \cveciiix3{-3}{-4}$.
    \begin{ppart}
        Note that \[\oa{OP} = \oa{OA} + \oa{AP} = \oa{OA} + 2\oa{AB} = \cveciii121\] and \[\oa{OQ} = \oa{OA} + \frac12 \oa{AC} = \cveciii222.\] Thus, \[\oa{PQ} = \oa{OQ} - \oa{OP} = \cveciii101, \quad \oa{AQ} = \oa{OQ} - \oa{OA} = \cveciii10{-1}.\] Since $\oa{PQ} \dotp \oa{AQ} = 0$, the two vectors are perpendicular, whence $PQ \perp AQ$.
    \end{ppart}
    \begin{ppart}
        Note that $\oa{AP} = \cveciiix00{-2}$. Hence, \[[\triangle APQ] = \frac12 \abs{\oa{AP} \crossp \oa{AQ}} = \frac12 \abs{\cveciii00{-2} \crossp \cveciii10{-1}} = 1 \units[2].\]
    \end{ppart}
    \begin{ppart}
        The vector $\oa{AB} \crossp \oa{AC} = \cveciiix0{-2}0$ is perpendicular to the plane $ABC$.
    \end{ppart}
    \begin{ppart}
        Let the angle between $\oa{AD}$ and $\oa{BD}$ be $\t$. Note that $\oa{BD} = -3\cveciiix{-1}11$. Hence, \[\cos\t = \frac{\oa{AD} \dotp \oa{BD}}{\abs{\oa{AD}} \abs{\oa{BD}}} = \frac{30}{\sqrt{34}\cdot3\sqrt{3}} = \frac{10}{\sqrt{102}}.\]
    \end{ppart}
\end{solution}