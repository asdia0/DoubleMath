\section{Self-Practice A6}

\begin{problem}
    A curve $C$ has equation, in polar coordinates, $r = a\sqrt{\bp{4 + \sin^2 \t} \cos \t}$, $-\frac12 \pi \leq \t \leq \frac12 \pi$, where $a$ is a positive constant.

    \begin{enumerate}
        \item Show that $\der{}{\t} \bs{\bp{4 + \sin^2 \t}\cos\t} = -\bp{2 + 3\sin^2 \t} \sin \t$. Hence, state, with a reason, whether $r$ increases or decreases as $\t$ increases, for $0 < \t \leq \frac12 \pi$.
        \item Sketch the curve $C$.
        \item Find the Cartesian equation of $C$ in the form $\bp{x^2 + y^2}^m = a^2 x \bp{bx^2 + cy^2}$, giving the numerical values of $m$, $b$ and $c$.
    \end{enumerate}
\end{problem}
\begin{solution}
    \begin{ppart}
        \begin{align*}
            \der{}{\t} \bs{\bp{4 + \sin^2 \t}\cos\t} &= -\bp{4 + \sin^2 \t}\sin \t + 2\sin \t \cos^2 \t\\
            &= -\sin \t \bp{\sin^2 \t - 2\cos^2 \t + 4}\\
            &= -\sin \t \bs{\sin^2 \t - 2\bp{1 - \sin^2 \t} + 4}\\
            &= -\sin \t \bp{3\sin^2 \t + 2}.
        \end{align*}
    \end{ppart}
    For $t \in (0, \pi/2]$, we have $\sin \t > 0$ and $3\sin^2 \t + 2 > 0$. Hence, $r$ is decreasing.
    \begin{ppart}
        \begin{center}\tikzsetnextfilename{38}
            \begin{tikzpicture}[trim axis left, trim axis right]
                \begin{axis}[
                    domain = -pi/2:pi/2,
                    samples = 100,
                    axis y line=middle,
                    axis x line=middle,
                    xtick = {2},
                    ytick = \empty,
                    xticklabels = {$2a$},
                    xmin=-0.5,
                    xmax=2.5,
                    ymin=-1.5,
                    ymax=1.5,
                    xlabel = {$\t=0$},
                    ylabel = {$\t = \frac\pi2$},
                    legend cell align={left},
                    legend pos=outer north east,
                    after end axis/.code={
                        \path (axis cs:0,0) 
                            node [anchor=north east] {$O$};
                        }
                    ]
                    \addplot[color=plotRed,data cs=polarrad] {sqrt((4 + sin(\x r)^2) * cos(\x r))};
        
                    \addlegendentry{$C$};
                \end{axis}
            \end{tikzpicture}
        \end{center}
    \end{ppart}
    \begin{ppart}
        Squaring, we have \[r^2 = a^2 \bp{4 + \sin^2 \t} \cos \t.\] Recall that $x = r\cos \t$ and $y = r\sin \t$, so \[r^2 = a^2 \bs{4 + \bp{\frac{y}{r}}^2} \bp{\frac{x}{r}} \implies r^5 = a^2 x \bp{4r^2 + y^2}.\] Since $x^2 + y^2 = r^2$, we get \[\bp{x^2 + y^2}^{5/2} = a^2 x \bp{4x^2 + 5y^2},\] whence $m = 5/2$, $b = 4$ and $c = 5$.
    \end{ppart}
\end{solution}

\clearpage
\begin{problem}
    The diagram shows a sketch of the curve $C$ with polar equation $r = a\cos^2 \t$, where $a$ is a positive constant and $-\frac12 \pi \leq \t \leq \frac12 \pi$.

    \begin{center}\tikzsetnextfilename{334}
        \begin{tikzpicture}[trim axis left, trim axis right]
            \begin{axis}[
                domain = -pi/2:pi/2,
                samples = 100,
                axis y line=middle,
                axis x line=middle,
                xtick = \empty,
                ytick = \empty,
                xmin=-0.5,
                xmax=1.5,
                ymin=-1,
                ymax=1,
                xlabel = {$\t=0$},
                ylabel = {$\t = \frac\pi2$},
                legend cell align={left},
                legend pos=outer north east,
                after end axis/.code={
                    \path (axis cs:0,0) 
                        node [anchor=north east] {$O$};
                    }
                ]
                \addplot[color=plotRed,data cs=polarrad] {(cos(\x r))^2};
    
                \addlegendentry{$C$};
            \end{axis}
        \end{tikzpicture}
    \end{center}

    \begin{enumerate}
        \item Explain briefly about how you can tell from this form of the equation that $C$ is symmetrical about the line $\t = 0$ and that the tangent to $C$ at the pole $O$ is perpendicular to the line $\t = 0$.
        \item Show that the equation of $C$ in Cartesian coordinates may be expressed in the form $y^2 = a^{2/3} x^{4/3} - x^2$.
    \end{enumerate}
\end{problem}
\begin{solution}
    \begin{ppart}
        Observe that \[a\cos^2 \t = a\cos[2]{-\t}.\] Hence, $C$ is invariant under the transformation $\t \mapsto -\t$, whence it is symmetrical about the line $\t = 0$.

        For tangents to the pole, we have $r = 0$. Since $a > 0$, we require $\cos \t = 0$, whence $\t = \pm \pi/2$, which are clearly perpendicular to the line $\t = 0$.
    \end{ppart}
    \begin{ppart}
        We have \[r = a\cos^2 \t = a \bp{\frac{x}{r}}^2 \implies r^3 = ax^2.\] Hence, \[x^2 + y^2 = r^2 = \bp{ax^2}^{2/3} \implies y^2 = a^{2/3} x^{4/3} - x^2.\]
    \end{ppart}
\end{solution}

\begin{problem}
    The equation of curve $C$ is given in polar coordinates by $r = 1 + \sin 2\t$, $0 \leq \t \leq 2\pi$. 
    \begin{enumerate}
        \item Prove that $C$ is symmetric about the pole.
        \item Sketch $C$ and any tangents to $C$ at the pole. Label any points of intersection with the axes, and show clearly the symmetries and curvature near the pole.
        \item Determine whether each loop of $C$ is a circle. Justify your answer.
        \item Show that the Cartesian equation of $C$ is $\bp{x^2 + y^2}^3 = (x+y)^4$.
    \end{enumerate}
\end{problem}
\clearpage
\begin{solution}
    \begin{ppart}
        Observe that \[1 + \sin 2\t = 1 + \sin{2\t + 2\pi} = 1 + \sin{2(\t + \pi)}.\] Hence, $C$ is invariant under the transformation $\t \mapsto \t + \pi$, whence $C$ is symmetric about the pole.
    \end{ppart}
    \begin{ppart}
        \begin{center}\tikzsetnextfilename{39}
            \begin{tikzpicture}[trim axis left, trim axis right]
                \begin{axis}[
                    domain = -pi:pi,
                    samples = 100,
                    axis y line=middle,
                    axis x line=middle,
                    xtick = {1, -1},
                    ytick = {1, -1},
                    xticklabels = {1, 1},
                    yticklabels = {1, 1},
                    xmin=-1.8,
                    xmax=1.8,
                    ymin=-1.8,
                    ymax=1.8,
                    xlabel = {$\t=0$},
                    ylabel = {$\t = \frac\pi2$},
                    legend cell align={left},
                    legend pos=outer north east,
                    after end axis/.code={
                        \path (axis cs:0,0) 
                            node [anchor=north east] {$O$};
                        }
                    ]
                    \addplot[color=plotRed,data cs=polarrad] {1 + sin(2 * \x r)};
                    \addlegendentry{$C$};

                    \addplot[dashed] {-x};        
                    \node[anchor=south west] at (1, -1) {$\t = \frac{7\pi}{4}$};
                    \node[anchor=south west] at (-1, 1) {$\t = \frac{3\pi}{4}$};
                \end{axis}
            \end{tikzpicture}
        \end{center}
    \end{ppart}
    \begin{ppart}
        Consider the top-right loop. $r$ attains a maximum of 2 when $\t = \pi/4$. Suppose the loop is a circle (with radius 1). Then the centre should be $(1, \pi/4)$, which is $(1/\sqrt2, 1\sqrt2)$ in Cartesian coordinates. The distance between $(1/\sqrt2, 1/\sqrt2)$ and $(1, 0)$ is given by \[\sqrt{\bp{\frac1{\sqrt2} - 1}^2 + \bp{\frac1{\sqrt2} - 0}^2} = \sqrt{2 - \sqrt{2}} \neq 1.\] Hence, the loop is not a circle.
    \end{ppart}
    \begin{ppart}
        We have \[r = 1 + \sin 2\t = 1 + 2\cos \t \sin \t = 1 + 2\bp{\frac{x}{r}} \bp{\frac{y}{r}}.\] Thus, \[r^3 = r^2 + 2xy \implies \bp{x^2 + y^2}^{3/2} = x^2 + y^2 + 2xy = (x + y)^2.\] Squaring both sides yields the desired equation: \[\bp{x^2 + y^2}^3 = \bp{x + y}^4.\]
    \end{ppart}
\end{solution}

\begin{problem}[\chili]
    Prove that at all points of intersection of the polar curves with equations $r = a(1 + \cos \t)$ and $r = b(1 - \cos\t)$, the tangent lines are perpendicular.
\end{problem}
\begin{solution}
    Consider the gradient of $C_1$. Firstly, we have \[\der{x}{\t} = \der{r}{\t} \cos \t - r\sin \t = -a\sin\t - 2a\sin\t\cos\t = -a\bp{\sin\t + \sin2\t}.\] Next, we have \[\der{y}{\t} = \der{r}{t} \sin \t + r \cos \t = a\cos \t - a \cos^2 \t + a \sin^2 \t = a\bp{\cos \t - \cos 2\t}.\] Thus, \[\der{y}{x} = \frac{\derx{y}{\t}}{\derx{x}{\t}} = -\bp{\frac{\cos \t - \cos 2\t}{\sin\t + \sin2\t}}.\]

    Consider the gradient of $C_2$. Firstly, we have \[\der{x}{\t} = \der{r}{\t} \cos \t - r\sin \t = -b \sin\t + 2b\cos \t \sin \t = b\bp{\sin 2\t - \sin \t}.\] Next, we have \[\der{y}{\t} = \der{r}{t} \sin \t + r \cos \t = b\cos \t - b \cos^2 \t + b \sin^2 \t = b\bp{\cos \t - \cos 2\t}.\] Thus, \[\der{y}{x} = \frac{\derx{y}{\t}}{\derx{x}{\t}} = \frac{\cos \t - \cos 2\t}{\sin 2\t - \sin \t}.\]

    Consider the product of the gradients: \[-\bp{\frac{\cos \t - \cos 2\t}{\sin\t + \sin2\t}}\bp{\frac{\cos \t - \cos 2t}{\sin 2\t - \sin \t}} = -\frac{\cos^2 \t - \cos^2 2\t}{\sin^2 2\t - \sin^2 \t}.\] Observe that \[\cos^2 \t - \cos^2 2\t = \cos^2 \t - \bp{2\cos^2 \t - 1}^2 = -4\cos^4 \t + 5\cos^2 \t - 1.\] Also observe that
    \begin{align*}
        \sin^2 2\t - \sin^2 \t &= 4\sin^2 \t \cos^2 \t - \sin^2 \t\\
        &= 4\bp{1 - \cos^2 \t} \cos^2 \t - \bp{1 - \cos^2 \t}\\
        &= -4\cos^4 \t + 5\cos^2 \t - 1.
    \end{align*}
    Hence, the product of the gradients is \[-\frac{\cos^2 \t - \cos^2 2\t}{\sin^2 2\t - \sin^2 \t} = -\frac{-4\cos^4 \t + 5\cos^2 \t - 1}{-4\cos^4 \t + 5\cos^2 \t - 1} = -1.\] Thus, for any given $\t$, the tangents of $C_1$ and $C_2$ are perpendicular. This immediately implies that the tangent lines at all intersection points of $C_1$ and $C_2$ are perpendicular.
\end{solution}