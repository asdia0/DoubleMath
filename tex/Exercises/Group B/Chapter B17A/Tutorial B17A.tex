\section{Tutorial B17A}

\begin{problem}
    Without the use of G.C., find the following matrix products:
    \begin{enumerate}
        \item $\begin{pmatrix}
            -1 & 4 & 2
        \end{pmatrix}
        \begin{pmatrix}
            5\\
            1\\
            3
        \end{pmatrix}$
        \item $\begin{pmatrix}
            3\\
            9\\
            2
        \end{pmatrix}
        \begin{pmatrix}
            1 & -6 & 3
        \end{pmatrix}$
        \item $\begin{pmatrix}
            4 & -1 & 1\\2&3&0
        \end{pmatrix}
        \begin{pmatrix}
            7 & -3\\
            5 & 4\\
            1 & 2
        \end{pmatrix}$
        \item $\begin{pmatrix}
            2 & 2 & 1\\1 & 0 & 2\\2 & 1 & 2
        \end{pmatrix}
        \begin{pmatrix}
            -2 & -3 & 4\\2 & 2 & -3\\1 & 2 & -2
        \end{pmatrix}$
    \end{enumerate}
\end{problem}
\begin{solution}
    \begin{ppart}
        \[\begin{pmatrix}
            -1 & 4 & 2
        \end{pmatrix}
        \begin{pmatrix}
            5\\
            1\\
            3
        \end{pmatrix} = \begin{pmatrix}
            5
        \end{pmatrix}\]
    \end{ppart}
    \begin{ppart}
        \[\begin{pmatrix}
            3\\
            9\\
            2
        \end{pmatrix}
        \begin{pmatrix}
            1 & -6 & 3
        \end{pmatrix} = \begin{pmatrix}
            3 & -18 & 9\\
            9 & -54 & 27\\
            2 & -12 & 6
        \end{pmatrix}\]
    \end{ppart}
    \begin{ppart}
        \[\begin{pmatrix}
            4 & -1 & 1\\2&3&0
        \end{pmatrix}
        \begin{pmatrix}
            7 & -3\\
            5 & 4\\
            1 & 2
        \end{pmatrix} = \begin{pmatrix}
            24 & -14\\
            29 & 6
        \end{pmatrix}\]
    \end{ppart}
    \begin{ppart}
        \[\begin{pmatrix}
            2 & 2 & 1\\1 & 0 & 2\\2 & 1 & 2
        \end{pmatrix}
        \begin{pmatrix}
            -2 & -3 & 4\\2 & 2 & -3\\1 & 2 & -2
        \end{pmatrix} = \begin{pmatrix}
            1 & 0 & 0\\
            0 & 1 & 0\\
            0 & 0 & 1
        \end{pmatrix}\]
    \end{ppart}
\end{solution}

\begin{problem}
    An orthogonal matrix $\mat M$ has the property \[\mat M \mat M \trp = \mat M \trp \mat M = \mat I,\] where $\mat M \trp$ and $\mat I$ denote the transpose of the matrix $\mat M$ and the identity matrix respectively.

    Given that matrices $\mat A$ and $\mat B$ are orthogonal, are the following true or false?
    \begin{enumerate}
        \item $\mat{AB}$ is orthogonal.
        \item $\mat A + \mat B$ is orthogonal.
    \end{enumerate}
\end{problem}
\begin{solution}
    \begin{ppart}
        The statement is true. Since $\mat A$ and $\mat B$ are both orthogonal, \[\bp{\mat A \mat B}\trp \bp{\mat A \mat B} = \mat B\trp \bp{\mat A \trp \mat A} \mat B = \mat B \trp \mat B = \mat I.\]

        \noindent\textit{Geometric Approach.} A matrix is orthogonal if and only if it is norm-preserving. Since $\mat A$ and $\mat B$ are both orthogonal, \[\norm{\vec v} = \norm{\mat B \vec v} = \norm{\mat A \mat B \vec v},\] thus $\mat A \mat B$ is also orthogonal.
    \end{ppart}
    \begin{ppart}
        The statement is false. Let $\mat M$ be orthogonal. Then $-\mat M$ is also orthogonal (reflection preserves norm). However, their sum, $\mat 0$, is clearly not orthogonal.
    \end{ppart}
\end{solution}

\begin{problem}
    It is given that a matrix $\mat A$ is symmetric if and only if $\mat A \trp = \mat A$. Suppose that $\mat A$ is a symmetric $m \times m$ matrix and that $\mat P$ is any $m \times m$ matrix. Prove that $\mat P \trp \mat A \mat P$ is symmetric.
\end{problem}
\begin{solution}
    \[\bp{\mat P \trp \mat A \mat P}\trp = \mat P \trp \mat A \trp \bp{\mat P \trp}\trp = \mat P \trp \mat A \trp \mat P = \mat P \trp \mat A \mat P.\]
\end{solution}

\begin{problem}
    For what value(s) of the constant $k$ does the following system of linear equations \[\systeme{x - y = 3, 2x-2y =k}\] have
    \begin{enumerate}
        \item no solutions?
        \item exactly one solution?
        \item infinitely many solutions?
    \end{enumerate}
\end{problem}
\begin{solution}
    Note that we have $2x - 2y = 6$ and $2x - 2y = k$. 
    \begin{ppart}
        If $k \neq 6$, there are no solutions.
    \end{ppart}
    \begin{ppart}
        It is impossible for the system to have a unique solution.
    \end{ppart}
    \begin{ppart}
        If $k = 6$, there are infinitely many solutions.
    \end{ppart}
\end{solution}

\begin{problem}
    \begin{enumerate}
        \item Solve the following system of linear equations by using row operations to express the matrix representation of the following system of linear equations in row echelon form. \[\systeme{x_1 + x_2 + x_3 = 8, -x_1 - 2x_2 + 3x_3 = 1, 3x_1 - 7x_2 + 4x_3 = 10}.\]
        \item Solve the following system of linear equations by using row operations to express the matrix representation of the following system of linear equations in reduced row echelon form. \[\systeme{x + y + z = 0, -2x + 5y + 2z = 0, -7x+7y+z = 0}.\]
    \end{enumerate}
\end{problem}
\begin{solution}
    \begin{ppart}
        Note that the given system of linear equations has matrix representation \[\begin{pmatrix}[rrr|r]1 & 1 & 1 & 8\\ -1 & -2 & 3 & 1 \\ 3 & -7 & 4 & 10\end{pmatrix}.\] Performing row operations on this matrix, we get \[\begin{pmatrix}[rrr|r]1 & 1 & 1 & 8\\ -1 & -2 & 3 & 1 \\ 3 & -7 & 4 & 10\end{pmatrix} \rightarrow \begin{matrix}[r]\scriptstyle 2R_1 + R_2 \\ \scriptstyle -R_2 - R_1 \\ \scriptstyle -\frac1{39}(R_3 - 13R_1 - 10 R_2)\end{matrix}\begin{pmatrix}[rrr|r]1 & 0 & 5 & 17\\ 0 & 1 & -4 & -9 \\ 0 & 0 & 1 & \frac83\end{pmatrix}.\] We thus recover the system of linear equations \[\systeme{x_1 + 4x_3 = 17, x_2 - 4x_3 = -9, x_3 = \frac83},\] whence we have $x_1 = \frac{11}3$, $x_2 = \frac53$ and $x_3 = \frac83$.
    \end{ppart}
    \begin{ppart}
        Note that the given system of linear equations has matrix representation \[\begin{pmatrix}[rrr]1 & 1 & 1 \\ -2 & 5 & 2 \\ -7 & 7 & 1\end{pmatrix}.\] Performing row operations on this matrix, we get \[\begin{pmatrix}[rrr]1 & 1 & 1 \\ -2 & 5 & 2 \\ -7 & 7 & 1\end{pmatrix} \rightarrow \begin{matrix}[r] \scriptstyle \frac17(5R_1 - R_2) \\ \scriptstyle \frac17(R_2 + 2R_1) \\ \scriptstyle R_3 + 3R_1 - 2R_2 \end{matrix}\begin{pmatrix}[rrr]1 & 0 & \frac37 \\ 0 & 1 & \frac47 \\ 0 & 0 & 0\end{pmatrix}.\] Since the last row is full of zeroes, we have a free variable. Let $x_3 = t$, where $t \in \RR$. Then we recover the system of linear equations \[\systeme{x_1 + \frac37 x_3 = 0, x_2 + \frac47 x_3 = 0, x_3 = t},\] whence $x_1 = -\frac37 t$, $x_2 = -\frac47 t$ and $x_3 = t$.
    \end{ppart}
\end{solution}

\begin{problem}
    What conditions must the $b$'s satisfy in order for the following system of linear equations to be consistent?
    \begin{enumerate}
        \item $\systeme{x_1-x_2+3x_3=b_1,3x_1-3x_2+4x_3=b_2,-2x_1+2x_2-6x_3=b_3}$
        \item $\systeme{2x_1+3x_2-x_3+x_4=b_1,x_1+5x_2+x_3-2x_4=b_2,-x_1+2x_2+2x_3-3x_4=b_3,3x_1+x_2-3x_3+4x_4=b_4}$
    \end{enumerate}
\end{problem}
\begin{solution}
    \begin{ppart}
        We can represent the given system of linear equations with the matrix \[\begin{pmatrix}[rrr|r]1 & -1 & 3 & b_1 \\ 3 & -3 & 4 & b_2 \\ -2 & 2 & -6 & -b_3\end{pmatrix}.\] Performing row operations, we see that \[\begin{pmatrix}[rrr|r]1 & -1 & 3 & b_1 \\ 3 & -3 & 4 & b_2 \\ -2 & 2 & -6 & -b_3\end{pmatrix} \rightarrow \begin{matrix}[r] \\ \scriptstyle R_2 - 3R_1 \\ \scriptstyle R_3 + 2R_1\end{matrix} \begin{pmatrix}[rrr|r]1 & -1 & 3 & b_1 \\ 0 & 0 & -5 & b_2 - 3b_1 \\ 0 & 0 & 0 & b_3 + 2b_1\end{pmatrix}.\] For the system to be consistent, we require $b_3 + 2b_1 = 0$.
    \end{ppart}
    \begin{ppart}
        We can represent the given system of linear equations with the matrix \[\begin{pmatrix}[rrrr|r]2 & 3 & -1 & 1 & b_1 \\ 1 & 5 & 1 & -2 & b_2 \\ -1 & 2 & 2 & -3 & b_3 \\ 3 & 1 & -3 & 4 & b_4\end{pmatrix}.\] Performing row operations, we see that \[\begin{pmatrix}[rrrr|r]2 & 3 & -1 & 1 & b_1 \\ 1 & 5 & 1 & -2 & b_2 \\ -1 & 2 & 2 & -3 & b_3 \\ 3 & 1 & -3 & 4 & b_4\end{pmatrix} \rightarrow \begin{matrix}[r]\scriptstyle R_1 - 2R_2 \\ \\ \scriptstyle R_3 + R_1 - R_2 \\ \scriptstyle R_4 - 2R_1 + R_2 \end{matrix}\begin{pmatrix}[rrrr|r]0 & -7 & 3 & 5 & b_1 - 2b_2 \\ 1 & 5 & 1 & -2 & b_2 \\ 0 & 0 & 0 & 0 & b_3 + b_1 - b_2 \\ 0 & 0 & 0 & 0 & b_4 - 2b_1 + b_2\end{pmatrix}.\] For the system to be consistent, we require $b_3 + b_1 - b_2 = 0$ and $b_4 - 2b_1 + b_2 = 0$.
    \end{ppart}
\end{solution}

\begin{problem}
    Without the use of a graphing calculator, find $\inv{\mat A}$ for each of the following cases of $\mat A$.
    \begin{enumerate}
        \item $\begin{pmatrix}
            2 & 3 & 1\\3 & 1 & 2\\1 & 2 & 3
        \end{pmatrix}$
        \item $\begin{pmatrix}[rr]
            \cos \a & \sin \a \\ -\sin \a & \cos \a
        \end{pmatrix}$
        \item $\begin{pmatrix}[rrr]
            1 & 0 & 0\\0 & \cos \a & -\sin \a\\0 & \sin \a & \cos \a
        \end{pmatrix}$
    \end{enumerate}
\end{problem}
\begin{solution}
    \begin{ppart}
        Performing row operations on $\begin{pmatrix}[c|c]\mat A & \mat I\end{pmatrix}$, we have \[\begin{pmatrix}[rrr|rrr]2 & 3 & 1 & 1 & 0 & 0 \\ 3 & 1 & 2 & 0 & 1 & 0 \\ 1 & 2 & 3 & 0 & 0 & 1\end{pmatrix} \rightarrow \begin{matrix}[r] \scriptstyle \frac1{18} (R_1 + 7R_2 - 5R_3) \\[0.4em] \scriptstyle \frac1{18} (7R_1 - 5R_2 + R_3) \\[0.4em] \scriptstyle \frac1{18} (-5R_1 + R_2 + 7R_3)\end{matrix} \begin{pmatrix}[rrr|rrr] 1 & 0 & 0 & \frac1{18} & \frac{7}{18} & -\frac{5}{18} \\[0.4em] 0 & 1 & 0 & \frac7{18} & -\frac5{18} & \frac1{18} \\[0.4em] 0 & 0 & 1 & -\frac5{18} & \frac1{18} & \frac7{18}\end{pmatrix}.\] Thus, \[\inv{\mat A} = \frac1{18} \begin{pmatrix}[rrr]1 & 7 & -5 \\ 7 & -5 & 1 \\ -5 & 1 & 7\end{pmatrix}.\]
    \end{ppart}
    \begin{ppart}
        By the formula for the inverse of a $2 \times 2$ matrix, \[\inv{\mat A} = \begin{pmatrix}[rr]\cos \a & -\sin \a \\ \sin \a & \cos \a\end{pmatrix}.\]
    \end{ppart}
    \begin{ppart}
        Observe that $\mat A$ represents a rotation of $\a$ about the $x$-axis. Thus, $\inv{\mat A}$ represents a rotation of $-\a$ about the $x$-axis, i.e. \[\inv{\mat A} = \begin{pmatrix}[rrr]1 & 0 & 0\\0 & \cos \a & \sin \a\\0 & -\sin \a & \cos \a\end{pmatrix}.\]
    \end{ppart}
\end{solution}

\begin{problem}
    Without the use of a graphing calculator, find the determinants of the following matrices:
    \begin{enumerate}
        \item $\mat A = \begin{pmatrix}
            0 & 4 \\ -1 & 2
        \end{pmatrix}$
        \item $\mat B = \begin{pmatrix}
            2 & -1 & 4 \\ 4 & -3 & 1 \\ 1 & 2 & 1
        \end{pmatrix}$
        \item $\mat C = \begin{pmatrix}
            2 & 0 & 0 \\ 4 & -3 & 0 \\ 1 & 2 & 1
        \end{pmatrix}$
        \item $\mat D = \begin{pmatrix}
            2 & -1 & 4 \\ 4 & -3 & 1 \\ 3 & 6 & 3
        \end{pmatrix}$
    \end{enumerate}
\end{problem}
\begin{solution}
    \begin{ppart}
        \[\det \mat A = (0)(2) - (-1)(4) = 4.\]
    \end{ppart}
    \begin{ppart}
        \[\det \mat B = 2 \begin{vmatrix}[rr]-3 & 1 \\ 2 & 1\end{vmatrix} - (-1) \begin{vmatrix}[rr]4 & 1 \\ 1 & 1\end{vmatrix} + 4 \begin{vmatrix}[rr]4 & -3 \\ 1 & 2\end{vmatrix} = 2(-5) - (-1)(3) + 4(11) = 37.\]
    \end{ppart}
    \begin{ppart}
        \[\det \mat C = \det \begin{vmatrix}[rr]2 & 0 \\ 4 & -3\end{vmatrix} = -6.\]
    \end{ppart}
    \begin{ppart}
        Note that $\mat D$ is simply $\mat B$ where the third row has been multiplied by 3. Thus, \[\det \mat D = 3 \det \mat B = 111.\]
    \end{ppart}
\end{solution}

\begin{problem}
    For the case where $\mat A = \begin{pmatrix} 0 & 1 & 0 \\ 1 & 2 & -1 \\ 0 & 1 & 3 \end{pmatrix}$ and $\mat B = \begin{pmatrix} 1 & 0 & 2 \\ 2 & 1 & 0 \\ -1 & 1 & -1 \end{pmatrix}$, verify the results
    \begin{enumerate}
        \item $\det{\mat A \mat B} = \det{\mat A} \det{\mat B}$,
        \item $\det{\inv{\mat A}} = 1/\det{\mat A}$.
    \end{enumerate}
    Determine also if
    \begin{enumerate}
        \setcounter{enumi}{2}
        \item $\det{\mat A + \mat B} = \det{\mat A} + \det{\mat B}$,
        \item $\det{\mat A} = \det{\mat A\trp}$.
    \end{enumerate}
\end{problem}
\begin{solution}
    Note that $\det A = -3$ and $\det B = 5$.
    \begin{ppart}
        Using G.C., \[\mat A \mat B = \begin{pmatrix}[rrr]2 & 1 & 0 \\ 6 & 1 & 3 \\ -1 & 4 & -3\end{pmatrix}.\] Hence, \[\det{\mat A \mat B} = -15 = (-3)(5) = \det{\mat A} \det{\mat B}.\]
    \end{ppart}
    \begin{ppart}
        Using G.C., \[\inv{\mat A} = \frac13 \begin{pmatrix}[rrr]-7 & 3 & 1 \\ 3 & 0 & 0 \\ -1 & 0 & 1\end{pmatrix}.\] Hence, \[\det{\inv{\mat A}} = -\frac13 = \frac1{-3} = \frac1{\det \mat A}.\]
    \end{ppart}
    \begin{ppart}
        Note that \[\mat A + \mat B = \begin{pmatrix}[rrr]1 & 1 & 2 \\ 3 & 3 & -1 \\ -1 & 2 & 2\end{pmatrix}.\] Hence, \[\det{\mat A + \mat B} = 21 \neq -3 + 5 = \det \mat A + \det \mat B.\]
    \end{ppart}
    \begin{ppart}
        Note that \[\mat A \trp = \begin{pmatrix}[rrr] 0 & 1 & 0 \\ 1 & 2 & 1 \\ 0 & -1 & 3\end{pmatrix}.\] Hence, \[\det \mat A\trp = -3 = \det \mat A.\]
    \end{ppart}
\end{solution}

\begin{problem}
    It is given that matrices $\mat A = \begin{pmatrix}2 & 1 & 3  \\ -1 & 0 & 4 \\ 3 & 1 & 0\end{pmatrix}$, $\mat B = \begin{pmatrix} 2 & 1 & 3 \\ -1 & 1 & 12 \\ 3 & 1 & 0 \end{pmatrix}$. Without the use of the G.C., find the inverse of $\mat A$ and $\mat B$ if it exists. For each of (a) and (b) below, solve, if possible, the equation, giving your answers in terms of $a$ (where applicable).

    \begin{enumerate}
        \item $\mat A \vec x = \cveciiix41a$,
        \item $\mat B \vec x = \cveciiix41a$.
    \end{enumerate}

    Hence, determine whether it is possible for $\vec x$ to have a unique solution when \[\mat A \mat B \vec x = \cveciii41a.\]
\end{problem}
\begin{solution}
    Performing row operations on $\begin{pmatrix}[c|c]\mat A & \mat I\end{pmatrix}$, we have \[\begin{pmatrix}[rrr|rrr]2 & 1 & 3 & 1 & 0 & 0 \\ -1 & 0 & 4 & 0 & 1 & 0\\ 3 & 1 & 0 & 0 & 0 & 1\end{pmatrix} \rightarrow \begin{matrix}[r]\scriptstyle -4 R_1 + 3R_2 + 4R_3 \\ \scriptstyle 12R_1 -9 R_2 - 11R_3 \\ \scriptstyle -R_1 + R_2 + R_3 \end{matrix}\begin{pmatrix}[rrr|rrr]1 & 0 & 0 & -4 & 3 & 4 \\ 0 & 1 & 0 & 12 & -9 & -11 \\ 0 & 0 & 1 & -1 & 1 & 1\end{pmatrix}.\] Thus, \[\inv{\mat A} = \begin{pmatrix}[rrr]-4 & 3 & 4 \\ 12 & -9 & -11 \\ -1 & 1 & 1\end{pmatrix}.\]

    Note that \[\det \mat B = 3\begin{vmatrix}[rr]-1 & 1 \\ 3 & 1\end{vmatrix} - 12 \begin{vmatrix}[rr]2 & 1 \\ 3 & 1\end{vmatrix} = 0.\] Hence, $\inv{\mat B}$ does not exist.

    \begin{ppart}
        Since $\mat A$ is invertible, we can pre-multiply $\inv{\mat A}$ on both sides of the equation $\mat A \vec x = \cveciiix41a$, yielding \[\vec x = \begin{pmatrix}[rrr]-4 & 3 & 4 \\ 12 & -9 & -11 \\ -1 & 1 & 1\end{pmatrix} \cveciii41a = \begin{pmatrix}-13 + 4a \\ 39 - 11 a \\ -3 + a\end{pmatrix}.\]
    \end{ppart}
    \begin{ppart}
        Note that the equation $\mat B \vec x = \cveciiix41a$ has matrix representation \[\begin{pmatrix}[rrr|r]2 & 1 & 3 & 4 \\ -1 & 1 & 12 & 1 \\ 3 & 1 & 0 & a\end{pmatrix}.\] Performing Gaussian elimination yields \[\begin{pmatrix}[rrr|r]2 & 1 & 3 & 4 \\ -1 & 1 & 12 & 1 \\ 3 & 1 & 0 & a\end{pmatrix} \rightarrow \begin{matrix}[r]\scriptstyle\frac13 R_1 + \frac23 R_2 \\ \scriptstyle\frac13 R_1 - \frac13 R_2 \\ \scriptstyle-\frac43 R_1 + \frac13 R_2 + R_3\end{matrix} \begin{pmatrix}[rrr|c]0 & 1 & 9 & 2 \\ 1 & 0 & -3 & 1 \\ 0 & 0 & 0 & a-5\end{pmatrix}.\] If $a \neq 5$, the system is inconsistent and there is no solution. If $a = 5$, the system is consistent with one free variable. Let $\vec x = \cveciiix{x_1}{x_2}{x_3}$, with $x_3 = t$, where $t \in \RR$. We have \[\systeme{x_2 + 9x_3 = 2, x_1 - 3x_3 = 1},\] whence $x_1 = 1 + 3t$, $x_2 = 2-9t$ and $x_3 = t$. Thus, \[\vec x = \cveciii120 + t \cveciii3{-9}1, \quad t \in \RR.\]
    \end{ppart}

    Since $\mat A$ is invertible, \[\mat A \mat B \vec x = \cveciii41a \implies \mat B \vec x = \inv{\mat A}\cveciii41a.\] However, because $\mat B$ is not invertible, there cannot be a unique solution $\vec x$ to the above equation (if such a solution exists).
\end{solution}

\begin{problem}
    Let $\mat A \vec x = \vec 0$ be a homogeneous system of $n$ linear equations in $n$ unknowns that has only the trivial solution. Show that if $k$ is any positive integer, then the system $\mat A^k \vec x = \vec 0$ also has only the trivial solution.
\end{problem}
\begin{solution}
    Since $\mat A \vec x = \vec 0$ has only the trivial solution, $\det{\mat A} \neq 0$. Thus, $\det{\mat A^k} = \det{\mat A}^k \neq 0$, whence $\mat A^k \vec x = \vec 0$ has a unique solution, which must clearly be the trivial solution.
\end{solution}

\begin{problem}
    \begin{enumerate}
        \item Let $\mat A$ be a non-zero square matrix such that $\mat A^2 = \mat A$. Determine all possible values of $\det{\mat A}$. Determine if the following statements are true. Justify your answer.
        \begin{enumerate}
            \item $\mat I - \mat A$ is always invertible.
            \item $\mat I + \mat A$ is always invertible.
        \end{enumerate}
        \item Let $\mat B = \begin{pmatrix} a & b & c \\ d & e & f \\ g & h & i \end{pmatrix}$. Given that $\mat B$ is the inverse of a matrix $\mat C$, and $\mat D$ is the matrix obtained from $\mat C$ by adding to the second row of $\mat C$ twice the first row of $\mat C$, find $\inv{\mat D}$ in a similar form to $\mat B$.
    \end{enumerate}
\end{problem}
\begin{solution}
    \begin{ppart}
        Taking determinants on both sides of the given equation, we have \[\det{\mat A^2} = \det{\mat A} \implies \det{\mat A}^2 = \det{\mat A}.\] Thus, the possible values of $\det{\mat A}$ are 0 and 1. An example of $\mat A$ with determinant 0 is \[\mat A = \begin{pmatrix}1 & 0 \\ 0 & 0\end{pmatrix},\] while an example of $\mat A$ with determinant 1 is simply $\mat I$.

        In fact, $\mat I$ is the only such matrix that has determinant 1: if $\mat A$ is invertible, we can pre-multiply its inverse to the equation $\mat A^2 = \mat A$, yielding \[\inv{\mat A} \mat A^2 = \inv{\mat A} \mat A \implies \mat A = \mat I.\]
        
        \noindent\textit{Geometric Approach.} Let $\mat A$ represent a linear transformation of $\RR^n$. $\mat A$ is idempotent if and only if all vectors $\vec v \in \Img A$ are invariant under $\mat A$. If $\mat A$ has non-zero determinant, then its image is $\RR^n$, i.e. $\mat A \vec v = \vec v$ for all $\vec v \in \RR^n$, which is only possible if $\mat A = \mat I$.

        \begin{psubpart}
            The statement is false. Observe that if $\mat A$ is idempotent, then $\mat I - \mat A$ is also idempotent: \[\bp{\mat I - \mat A}^2 = \mat I^2 - \mat I \mat A - \mat A \mat I + \mat A^2 = \mat I - \mat A - \mat A + \mat A = \mat I - \mat A.\] Seeking a contradiction, suppose $\mat I - \mat A$ is invertible. From the above result, it follows that \[\mat I - \mat A = \mat I \implies \mat A = \mat 0,\] which contradicts our assumption that $\mat A$ is non-zero. Hence, $\mat I - \mat A$ is not invertible.

            \noindent\textit{Alternative Approach.} Let $\vec v \in \Img \mat A$. Then \[(\mat I - \mat A) \vec v = \vec v - \mat A \vec v = \vec v - \vec v = \vec 0 \implies \Img{\mat A} \subseteq \Ker \bp{\mat I - \mat A}.\] If $\mat I - \mat A$ is invertible, then its kernel must be trivial. This means that \[\Img{\mat A} = \Ker \bp{\mat I - \mat A} = \bc{\vec 0} \implies \mat A = \mat 0,\] contradicting our assumption that $\mat A$ is non-zero. Hence, $\mat I - \mat A$ is not invertible.
        \end{psubpart}
        \begin{psubpart}
            The statement is true. Let $\mat B = \mat I + \mat A$. Since $\mat A^2 = \mat A$, \[\mat 0 = \mat A^2 - \mat A = \bp{\mat B - \mat I}^2 - \bp{\mat B - \mat I} = \mat B^2 - 3\mat B + 2 \mat I.\] Rearranging, \[\mat I = \frac32 \mat B - \frac12 \mat B^2 = \mat B \bp{\frac32 \mat I - \frac12 \mat B} = \bp{\mat I + \mat A} \bp{\mat I - \frac12 \mat A}.\] Thus, the inverse of $\mat I + \mat A$ exists and is given by $\mat I - \frac12 \mat A$.
            
            \noindent\textit{Alternative Approach.} Let $\vec v \in \Ker \bp{\mat I + \mat A}$. Then \[\bp{\mat I + \mat A} \vec v = \vec v + \mat A \vec v = \vec 0 \implies \mat A \vec v = -\vec v. \tag{1}\] Pre-multiplying both sides by $\mat A$, \[\mat A^2 \vec v = -\mat A \vec v \implies \mat A \vec v = - \mat A \vec v \implies \mat A \vec v = \vec 0.\] Substituting this into (1), we have $\vec v = \vec 0$. Hence, the kernel of $\mat I + \mat A$ is trivial, thus $\mat I + \mat A$ is invertible.
        \end{psubpart}
    \end{ppart}
    \begin{ppart}
        Note that \[\mat D = \begin{pmatrix}1 & 0 & 0 \\ 2 & 1 & 0 \\ 0 & 0 & 1\end{pmatrix} \mat C = \begin{pmatrix}1 & 0 & 0 \\ 2 & 1 & 0 \\ 0 & 0 & 1\end{pmatrix} \inv{\mat B}.\] Thus, our goal matrix $\inv{\mat D}$ is given by
        \begin{gather*}
            \inv{\mat D} = \bs{\begin{pmatrix}1 & 0 & 0 \\ 2 & 1 & 0 \\ 0 & 0 & 1\end{pmatrix} \mat B^{-1}}^{-1} = \mat B \begin{pmatrix}1 & 0 & 0 \\ 2 & 1 & 0 \\ 0 & 0 & 1\end{pmatrix}^{-1} \\
            = \begin{pmatrix} a & b & c \\ d & e & f \\ g & h & i \end{pmatrix}\begin{pmatrix}1 & 0 & 0 \\ -2 & 1 & 0 \\ 0 & 0 & 1\end{pmatrix} = \begin{pmatrix}a-2b & b & c \\ d-2e & e & f \\ g-2h & h & i\end{pmatrix}.
        \end{gather*}
    \end{ppart}
\end{solution}