\section{Self-Practice B5A}

\begin{problem}
    It is given that $f(x) = \frac{x^2 -2 x}{\e^x}$.

    Find the range of values of $x$ for which the curve $y=f(x)$ is concave upward. Hence, sketch the graph of $y=f(x)$, indicating clearly the equations of any asymptotes and the coordinates of any stationary points and any intersections with the axes.
\end{problem}

\begin{problem}
    It is given that $x$ and $y$ satisfy the equation \[y^4 - \ln \frac{y^2}4 = x^4 - 6x^2, \quad y > 0.\]

    \begin{enumerate}
        \item Show that $\der{y}{x} = \frac{2xy\bp{x^2 - 3}}{2y^4 - 1}$.
        \item Hence, obtain the possible exact value(s) of $\der{y}{x}$ when $y=2$.
    \end{enumerate}
\end{problem}

\begin{problem}
    The diagram below shows the graph of $y=g(x)$. The graph has a minimum point at $(0,2)$ and a maximum point at $(3, \frac{1}{2})$. The equations of the asymptotes are $x=1$, $y=0$ and $y=-2x$.

    \begin{figure}[H]\tikzsetnextfilename{400}
        \centering
        \begin{tikzpicture}[trim axis left, trim axis right]
            \begin{axis}[
                restrict y to domain =-2:5,
                samples = 101,
                xmin = -5,
                xmax = 6,
                axis y line=middle,
                axis x line=middle,
                xtick = {2},
                ytick = {2},
                xlabel = {$x$},
                ylabel = {$y$},
                legend cell align={left},
                legend pos=outer north east,
                after end axis/.code={
                    \path (axis cs:0,0) 
                        node [anchor=north east] {$O$};
                    }
                ]
                \addplot[plotRed, domain=-4:1] {-2*x -2/(x-1)};
                \addplot[plotRed, domain=1:6] {2*(x-2)/(x-1)^2};
                \addlegendentry{$y = g(x)$};
                
                \addplot[dashed] {-2*x};
                \node[anchor=north east] at (-2, 4) {$y = -2x$};
                \draw[dashed] (1, -2) -- (1, 5);
                \node[anchor=north west] at (1, 4) {$x = 1$};
                \fill (3, 0.5) circle[radius=2.5pt] node[anchor=south] {$\bp{3, \frac12}$};
            \end{axis}
        \end{tikzpicture}
    \end{figure}

    \begin{enumerate}
        \item State the interval(s) on which $g$ is
        \begin{enumerate}
            \item increasing;
            \item increasing and concave upward.
        \end{enumerate}
        \item Sketch $y=g'(x)$, showing clearly the equations of the asymptotes and the coordinates of the turning points and axial intercepts, where applicable.
    \end{enumerate}
\end{problem}

\begin{problem}
    The diagram below shows the graph of $y=f(x)$. It cuts the axes at the points $(0, 1),$ $(1.5, 0)$ and $(3,0)$. It has a minimum point at $(2.5,-0.5)$. The horizontal, vertical and oblique asymptotes are $y=0$, $x=7a$ and $y=-x+a$ respectively, where $a$ is a positive constant.

    \begin{figure}[H]\tikzsetnextfilename{401}
        \centering
        \begin{tikzpicture}[trim axis left, trim axis right]
            \begin{axis}[
                restrict y to domain =-3:3,
                samples = 101,
                axis y line=middle,
                axis x line=middle,
                xtick = {0.882, 3.118},
                xticklabels = {1.5, 3},
                ytick = {0.786},
                yticklabels = {1},
                xlabel = {$x$},
                ylabel = {$y$},
                xmin=-2,
                xmax=8,
                legend cell align={left},
                legend pos=outer north east,
                after end axis/.code={
                    \path (axis cs:0,0) 
                        node [anchor=north east] {$O$};
                    }
                ]
                \addplot[plotRed, domain=-2:3.5] {-x + 0.5 - 1/(x-3.5)};
                \addplot[plotRed, domain=3.5:8] {1/(3.5 - x)};
                \addlegendentry{$y = f(x)$};
                
                \addplot[dashed] {-x + 0.5};
                \node[anchor=south east] at (3.5, -3) {$y = -x + a$};
                
                \draw[dashed] (3.5, -3) -- (3.5, 3);
                \node[anchor=north west] at (3.5, 2) {$x = 7a$};
                \fill (2.4, -1) circle[radius=2.5pt] node[anchor=north] {$(2.5, -0.5)$};
            \end{axis}
        \end{tikzpicture}
    \end{figure}

    On separate diagrams, sketch the graphs of
    \begin{enumerate}
        \item $y = \frac{1}{f(x)}$,
        \item $y = f'(x)$,
    \end{enumerate}
    showing clearly the axial intercepts, the stationary points and the equations of the asymptotes where applicable.
\end{problem}

\begin{problem}[\chili]
    The graph of $y=\abs{f(x)}$ is shown in the diagram, with a maximum point $(4,a)$, and $x=0$ and $x=6$ are tangents to both graphs.

    \begin{figure}[H]\tikzsetnextfilename{402}
        \centering
        \begin{tikzpicture}[trim axis left, trim axis right]
            \begin{axis}[
                restrict y to domain =0:6,
                samples = 101,
                axis y line=middle,
                axis x line=middle,
                xtick = {6},
                ytick = \empty,
                xlabel = {$x$},
                ylabel = {$y$},
                xmin=-4,
                xmax=10,
                ymin=0,
                legend cell align={left},
                legend pos=outer north east,
                after end axis/.code={
                    \path (axis cs:0,0) 
                        node [anchor=north] {$O$};
                    }
                ]
                \addplot[plotRed, domain=-4:0] {-x + 2 + 1/(x-0.5)};
                \addplot[plotRed, domain=0:6] {x*(-x+6)/(-x+6.4)};
                \addplot[plotRed, domain=6:10] {x - 2 - 1/(x-5.75)};

                \addlegendentry{$y = \abs{f(x)}$};
                
                \addplot[dashed, domain=-4:2] {-x + 2};
                \node[anchor=south east, rotate=-63] at (-1, 3) {$y = -x + 2$};

                \addplot[dashed, domain=2:10] {x - 2};
                \node[anchor=south east, rotate=63] at (7.6, 5.6) {$y = x-2$};
                
                \fill (4.8, 3.6) circle[radius=2.5pt] node[anchor=south east] {$(4, a)$};
            \end{axis}
        \end{tikzpicture}
    \end{figure}

    It is given that the graph of the continuous function $f$ has \textbf{only} one oblique asymptote, and that $f'(1) > 0$ and $f'(7) < 0$.

    Sketch the graph of $y=f'(x)$, showing clearly the stationary point(s), the asymptote(s) and the intercept(s), if any.
\end{problem}