\section{Self-Practice B11}

\begin{problem}
    At what rate is the area of a rectangle changing if its length is 15 units and increasing at 3 units/s while its width is 6 units and increasing at 2 units/s?
\end{problem}
\begin{solution}
    Let $l(t)$ and $w(t)$ be the length and width of the rectangle respectively, where $t$ is the time in seconds. Let $A = lw$ be the area of the rectangle. Note that \[\pder{A}{l} = w \quad \land \quad \pder{A}{w} = l.\] Also, \[\der{l}{t} = 3 \quad \land \quad \der{w}{t} = 2.\] Hence, \[\der{A}{t} = \pder{A}{l} \der{l}{t} + \pder{A}{w} \der{w}{t} = 3w + 2l.\] When $l = 15$ and $w = 6$, we have \[\der{A}{t} = 3(6) + 2(15) = 48.\] Thus, the area of the rectangle is increasing at a rate of 48 units$^2$/s.
\end{solution}

\begin{problem}
    A particle moving along a metal plate in the $xy$-plane has the velocity $\vec v = \vec i - 4 \vec j$ cm/s at the point $(3, 2)$. If the temperature of the plate at points in the $xy$-plane is $T(x, y) = y^2 \ln x$ where $x \geq 1$, in degrees Celsius, find $\derx{T}{t}$ at $(3, 2)$.
\end{problem}

\begin{problem}
    Given that $f(x, y) = x^2 \e^y$, find the maximum value of a directional derivative at $(-2, 0)$ and give a unit vector in the direction in which the maximum value occurs.
\end{problem}

\begin{problem}
    If the electric potential at a point $(x, y)$ in the $xy$-plane is $V(x, y)$, where $V(x, y) = \e^{-2x} \cos 2y$, find the direction where $V$ decreases most rapidly at $(0, \pi/6)$.
\end{problem}

\begin{problem}
    Find all local extrema and saddle points of $f(x, y) = 4xy - x^4 - y^4$.
\end{problem}

\begin{problem}
    Find the absolute extrema of $f(x, y) = x^2 + 2y^2 - x$ such that the domain of this function $f$ is the circular region $x^2 + y^2 \leq 4$.
\end{problem}

\begin{problem}
    A length of sheet metal 27 cm wide is to be made into a water trough by bending up two sides as shown in the figure below. Find the values of $x$ and $\t$ such that the trapezoid-shaped cross-section has a maximum area.

    \begin{center}\tikzsetnextfilename{388}
        \begin{tikzpicture}
            \coordinate (A) at (3, 0);
            \coordinate (B) at (4, 0);
            \coordinate (C) at (4, 2);
            
            \draw (0, 0) -- (A) -- (C) -- (-1, 2) -- (0, 0);
            \node[anchor=north] at (1.5, 0) {$27 - 2x$};
            \node[anchor=east] at (-0.5, 0.9) {$x$};
            \draw[dashed] (A) -- (B);
    
            \draw pic [draw, angle radius=4mm] {angle = B--A--C};
            \node[anchor=south west] at (3.3, 0) {$\t$};
        \end{tikzpicture}
    \end{center}
\end{problem}

\begin{problem}
    A Further Maths student smiled when a question asked for him to find the quadratic approximation for the function of $f(x, y) = xy - 3y - x$ around the point $(2, 3)$. Explain why he is so delighted.
\end{problem}
\begin{solution}
    The function $f(x, y) = xy - 3y - x$ is already a quadratic, hence the quadratic approximation to $f(x, y)$ is simply $f(x, y)$ itself.
\end{solution}

\begin{problem}
    A company produces two products, $A$ and $B$, which require different amounts of two resources, Resource 1 and Resource 2. The profit generated by selling product $A$ is \$10 per unit, and the profit from selling product $B$ is \$15 per unit. Each unit of product $A$ requires 2 units of Resource 1 and 1 unit of Resource 2. Each unit of product $B$ requires 1 unit of Resource 1 and 3 units of Resource 2. The company has a total of 100 units of Resource 1 and 90 units of Resource 2. What should the company produce in order to maximize its profitability?
\end{problem}