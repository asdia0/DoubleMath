\documentclass{echw}

\title{H2 Mathematics\\Weighted Assessment 1}
\author{Eytan Chong}
\date{2024-04-25}

\begin{document}
    \problem{}
        The curve $C$ has parametric equations
        \[
            x = t^2 + t, \, y = 4t - t^2, \, -2 < t < 1
        \]

        \begin{enumerate}
            \item Sketch $C$, indicating the coordinates of the end-points and the axial intercepts (if any) of this curve.
            \item Find the coordinates of the point(s) of intersection between $C$ and the line $8y - 12x = 5$.
        \end{enumerate}

    \solution
        \part
            \begin{center}
                \begin{tikzpicture}[trim axis left, trim axis right]
                    \begin{axis}[
                        domain = -2:1,
                        samples = 101,
                        axis y line=middle,
                        axis x line=middle,
                        xtick = \empty,
                        ytick = {-5},
                        xlabel = {$x$},
                        ylabel = {$y$},
                        xmax=2.5,
                        xmin=-0.5,
                        ymin=-14,
                        ymax=5,
                        legend cell align={left},
                        legend pos=outer north east,
                        after end axis/.code={
                            \path (axis cs:0,0) 
                                node [anchor=north west] {$O$};
                            }
                        ]
                        \addplot[plotRed] ({x^2 + x}, {4*x - x^2});
            
                        \addlegendentry{$C$};

                        \draw (2, -12) circle[radius=2.5pt] node[anchor=north] {$(2, -12)$};

                        \draw (2, 3) circle[radius=2.5pt] node[anchor=south] {$(2, 3)$};
                    \end{axis}
                \end{tikzpicture}
            \end{center}

        \part
            \begin{alignat*}{2}
                &&8y-12x &= 5\\
                \implies&&8(4t-t^2) - 12(t^2 + t) &= 5\\
                \implies&&32t - 8t^2 - 12t^2 - 12t - 5 &= 0\\
                \implies&&-20t^2 + 20t - 5 &= 0\\
                \implies&&t^2 - t + \dfrac14 &= 0\\
                \implies&&\bp{t - \dfrac12}^2 &= 0\\
                \implies&&t &= \dfrac12
            \end{alignat*}
            When $t = \dfrac12$, we have that $x = \dfrac34$ and $y = \dfrac74$. Thus, $C$ and the line $8y-12x=5$ intersect at $\bp{\dfrac34, \dfrac74}$.

            \boxt{$\bp{\dfrac34, \dfrac74}$}

    \problem{}
        \begin{enumerate}
            \item Without using a calculator, solve $\dfrac4{3+2x-x^2} \leq 1$.
            \item Hence, solve $\dfrac4{3+2\abs{x}-x^2} \leq 1$.
        \end{enumerate}

    \solution
        \part
            \begin{alignat*}{2}
                &&\dfrac4{3+2x-x^2} &\leq 1\\
                \implies&&\dfrac4{x^2-2x-3} &\geq -1\\
                \implies&&\dfrac4{(x-3)(x+1)} + 1 &\geq 0\\
                \implies&&\dfrac{4 + (x-3)(x+1)}{(x-3)(x+1)} &\geq 0\\
                \implies&&\dfrac{4 + \bp{x^2 - 2x - 3}}{(x-3)(x+1)} &\geq 0\\
                \implies&&\dfrac{x^2 - 2x + 1}{(x-3)(x+1)} &\geq 0\\
                \implies&&\dfrac{(x-1)^2}{(x-3)(x+1)} &\geq 0
            \end{alignat*}
            We thus have that $x = 1$ is a solution. In the case when $(x-1)^2 > 0$,
            \begin{alignat*}{2}
                &&\dfrac1{(x-3)(x+1)} &\geq 0\\
                \implies&&(x-3)(x+1) &\geq 0
            \end{alignat*}
            whence $x < 1$ or $x > 3$. Putting everything together, we have

            \boxt{$x < -1 \lor x = 1 \lor x > 3$}

        \part
            \begin{alignat*}{2}
                &&\dfrac4{3+2\abs{x}-x^2} &\leq 1\\
                \implies&&\dfrac4{3+2\abs{x}-\abs{x}^2} &\leq 1
            \end{alignat*}
            From part (a), we have that $\abs{x} < -1$, $\abs{x} = 1$ or $\abs{x} > 3$.

            \case{1}{$\abs{x} < -1$.} Since $\abs{x} \geq 0$ this case yields no solutions.
            
            \case{2}{$\abs{x} = 1$.} We have $x = 1$ or $x = -1$.

            \case{3}{$\abs{x} > 3$.} We have $x > 3$ or $x < -3$.

            \boxt{$x < -3 \lor x = -1 \lor x = 1 \lor x > 3$}

    \problem{}
        The curve $C_1$ has equation
        \[
            y = \dfrac{2x^2 + 2x - 2}{x-1}
        \]

        \begin{enumerate}
            \item Sketch the graph of $C_1$, stating the equations of any asymptotes and the coordinates of any axial intercepts and/or turning points.
        \end{enumerate}

        The curve $C_2$ has equation
        \[
            \dfrac{(x-a)^2}{1^2} + \dfrac{(y-6)^2}{b^2} = 1
        \]
        where $b > 0$. It is given that $C_1$ and $C_2$ have no points in common for all $a \in \R$.
        \begin{enumerate}
            \setcounter{enumi}{1}
            \item By adding an appropriate curve in part (a), state the range of values of $b$, explaining your answer.
            \item The function $f$ is defined by
            \[
                f(x) = \dfrac{2x^2+2x-2}{x-1}, \, x < 1
            \]
            \begin{enumerate}
                \item By using the graph in part (a) or otherwise, explain why the inverse function $\inv f$ does not exist.
                \item The domain of $f$ is restricted to $[c, 1)$ such that $c$ is the least value for which the inverse function $\inv f$ exists. State the value of $c$ and define $\inv f$ clearly.
            \end{enumerate}
        \end{enumerate}
    
    \solution
        \part
            \begin{center}
                \begin{tikzpicture}[trim axis left, trim axis right]
                    \begin{axis}[
                        domain = -3:3,
                        samples = 121,
                        axis y line=middle,
                        axis x line=middle,
                        xtick = {-1.62, 0.618},
                        ytick = {2},
                        xlabel = {$x$},
                        ylabel = {$y$},
                        ymax=15,
                        ymin=-3,
                        xmax=3.2,
                        legend cell align={left},
                        legend pos=outer north east,
                        after end axis/.code={
                            \path (axis cs:0,0) 
                                node [anchor=north east] {$O$};
                            }
                        ]
                        \addplot[plotRed, name path=f1, unbounded coords = jump] {(2*x^2 + 2*x - 2)/(x-1)};
            
                        \addlegendentry{$C_1$};
                        
                        \draw[plotBlue] (1, 6) ellipse[x radius=1, y radius=4];

                        \addlegendentry{$C_2$};
            
                        \addplot[dotted] {2*x + 4};

                        \draw[dotted] (1, 15) -- (1, -3) node[anchor=south west] {$x = 1$};

                        \node[rotate=30] at (-1, 3) {$y = 2x + 4$};

                        \fill (0, 2) circle[radius=2.5pt];

                        \fill (2, 10) circle[radius=2.5pt] node[anchor=west] {$(2, 10)$};

                        \addplot[dotted] {2};

                        \addplot[dotted] {10};

                        \addplot[dotted] {6};

                        \fill (1, 6) circle[radius=2.5pt] node[anchor=west] {$(a, 6)$};

                        \draw[->] (1, 6) -- (0, 6);

                        \draw[->] (1, 6) -- (1, 10);

                        \node[anchor=north] at (0.5, 6) {$1$};

                        \node[anchor=west] at (1, 8) {$b$};
                    \end{axis}
                \end{tikzpicture}
            \end{center}

        \part
            Observe that $C_2$ describes an ellipse with vertical radius $b$ and horizontal radius 1. Furthermore, the ellipse is centred at $(a, 6)$. Since $C_1$ and $C_2$ have no points in common for all $a \in \R$, the maximum $y$-value of the ellipse corresponds to the $y$-value of the minimum point $(2, 10)$ of $C_1$. Similarly, the minimum $y$-value of the ellipse corresponds to the $y$-value of the maximum point $(0, 2)$ of $C_1$. Thus, $2 < y < 10$, whence $b < \min \bc{\abs{6-2}, \abs{6-10}} = 4$. Thus,

            \boxt{$0 < b < 4$}

        \part
            \subpart
                Observe that $f(-1.62) = f(0.618) = 0$. Hence, there exist two different values of $x$ in $\dom f$ that have the same image under $f$. Thus, $f$ is not one-one. Hence, $\inv f$ does not exist.
            
            \subpart
                \boxt{$c = 0$}

                \begin{alignat*}{2}
                    &&f(x) &= \dfrac{2x^2 + 2x - 2}{x-1}\\
                    \implies&&(x-1)f(x) &= 2x^2 + 2x-2\\
                    \implies&&xf(x) - x &= 2x^2 + 2x-2\\
                    \implies&&2x^2 + 2x - xf(x) - 2 + f(x) &= 0\\
                    \implies&&2x^2 + \big(2- f(x)\big)x+ \big(f(x) - 2\big) &= 0\\
                    \implies&&x &= \dfrac{-\big(2- f(x)\big) \pm \sqrt{\big(2- f(x)\big)^2 - 4\cdot2\cdot\big(f(x) - 2\big)}}{2\cdot2}\\
                    && &= \dfrac{f(x) - 2 \pm \sqrt{f(x)^2 -12f(x) + 20}}{4}\\
                    \implies&&\inv f(x) &= \dfrac{x - 2 \pm \sqrt{x^2 -12x + 20}}{4}
                \end{alignat*}

                Note that $\dom f = \ran {\inv f} = [0, 1)$. We thus take the positive root. Also note that $\ran f = \dom {\inv f} = (-\infty, 2]$.

                \boxt{$\inv f : x \mapsto \dfrac{x - 2 + \sqrt{x^2 -12x + 20}}{4}, \, x \in \R, \, x \leq 2$}
\end{document}