\documentclass{echw}

\title{Tutorial B4\\Differentiation}
\author{Eytan Chong}
\date{2024-03-25}

\begin{document}
    \problem{}
        Evaluate the following limits.

        \begin{enumerate}
            \item $\lim\limits_{x \to 5} (6x + 7)$
            \item $\lim\limits_{x \to 1} \dfrac{x^3-1}{1-x}$
            \item $\lim\limits_{x \to \infty} \dfrac{3x}{2x^2-5}$
        \end{enumerate}

    \solution
        \part
            \begin{align*}
                \lim\limits_{x \to 5} (6x + 7) &= 6 \cdot 5 + 7\\
                &= 37
            \end{align*}

            \boxt{$\lim\limits_{x \to 5} (6x + 7) = 37$}

        \part
            \begin{align*}
                \lim\limits_{x \to 1} \dfrac{x^3-1}{1-x} &= \lim\limits_{x \to 1} \dfrac{(x-1)\bp{x^2 + x + 1}}{1-x} \\
                &= \lim\limits_{x \to 1} -\bp{x^2 + x + 1} \\
                &= -(1^2 + 1 + 1)\\
                &= -3
            \end{align*}

            \boxt{$\lim\limits_{x \to 1} \dfrac{x^3-1}{1-x} = -3$}

        \part
            \[
                \lim\limits_{x \to \infty} \dfrac{3x}{2x^2-5} = \lim\limits_{x \to \infty} \dfrac{3}{2x-5/x}
            \]

            Note that as $x \to \infty$, $2x - \dfrac5{x} \to \infty - 0$. Hence, $\lim\limits_{x \to \infty} \dfrac{1}{2x-5/x} = 0$.

            \boxt{$\lim\limits_{x \to \infty} \dfrac{3x}{2x^2-5} = 0$}

    \problem{}
        Differentiate the following with respect to $x$ from first principles.
        
        \begin{enumerate}
            \item $3x+4$
            \item $x^3$
        \end{enumerate}

    \solution
        \part
            \begin{align*}
                \der{}{x} (3x+4) &= \lim_{h \to 0}\dfrac{(3(x+h) + 4) - (3x + 4)}{h}\\
                &= \lim_{h \to 0}\dfrac{3h}{h}\\
                &= \lim_{h \to 0} 3\\
                &= 3
            \end{align*}

            \boxt{$\der{}{x} (3x+4) = 3$}

        \part
            \begin{align*}
                \der{}{x} x^3 &= \lim_{h \to 0} \dfrac{(x+h)^3 - x^3}{h} \\
                &= \lim_{h \to 0} \dfrac{x^3 + 3hx^2 + 3h^2x + h^3 - x^3}{h}\\
                &= \lim_{h \to 0} \dfrac{3hx^2 + 3h^2x + h^3}{h}\\
                &= \lim_{h \to 0} \bp{3x^2 + 3hx + h^2}\\
                &= 3x^2
            \end{align*}

            \boxt{$\der{}{x} x^3 = 3x^2$}

    \problem{}
        Differentiate each of the following with respect to $x$, simplifying your answer.

        \begin{enumerate}
            \item $\bp{x^2 + 4}^2\bp{2x^3 -1}$
            \item $\dfrac{x^2}{\sqrt{4-x^2}}$
            \item $\sqrt{1+\sqrt x}$
            \item $\bp{\dfrac{x^3-1}{2x^3+1}}^4$
        \end{enumerate}

    \solution
        \part
            \begin{align*}
                \der{}{x} \bp{x^2 + 4}^2\bp{2x^3 -1} &= \bp{2x^3 - 1} \der{}{x} \bp{x^2 + 4}^2 + \bp{x^2 + 4}^2 \cdot \der{}{x} \bp{2x^3 -1}\\
                &= \bp{2x^3 - 1} \cdot 2\bp{x^2 + 4}\cdot 2x + \bp{x^2 + 4}^2 \cdot 6x^2 \\
                &= 2x\bp{x^2 + 4}\bp{2\bp{2x^3 - 1} + 3x\bp{x^2 + 4}}\\
                &= 2x\bp{x^2 + 4}\bp{7x^3 + 12x - 2}
            \end{align*}

            \boxt{$\der{}{x} \bp{x^2 + 4}^2\bp{2x^3 -1} = 2x\bp{x^2 + 4}\bp{7x^3 + 12x - 2}$}
        
        \part
            \begin{align*}
                \der{}{x} \dfrac{x^2}{\sqrt{4-x^2}} &= \dfrac{\sqrt{4-x^2} \cdot \der{}{x} x^2 - x^2 \cdot \der{}{x} \sqrt{4-x^2}}{4-x^2}\\
                &= \dfrac{\sqrt{4-x^2} \cdot 2x - x^2 \cdot 1/(2\sqrt{4-x^2}) \cdot -2x }{4-x^2}\\
                &= \dfrac{\bp{4-x^2} \cdot 2x - x^2 \cdot 1/2 \cdot -2x }{(4-x^2)^{3/2}}\\
                &= \dfrac{\bp{4-x^2} \cdot 2x + x^3 }{(4-x^2)^{3/2}}\\
                &= \dfrac{x \bp{8-x^2}}{(4-x^2)^{3/2}}
            \end{align*}

            \boxt{$\der{}{x} \dfrac{x^2}{\sqrt{4-x^2}} = \dfrac{x \bp{8-x^2}}{(4-x^2)^{3/2}}$}

        \part
            \begin{align*}
                \der{}{x} \sqrt{1+\sqrt x} &= \dfrac1{2\sqrt{1+\sqrt x}} \cdot \dfrac1{2\sqrt x} \\
                &= \dfrac1{4\sqrt{x(1+\sqrt x)}}
            \end{align*}

            \boxt{$\der{}{x} \sqrt{1+\sqrt x} = \dfrac1{4\sqrt{x(1+\sqrt x)}}$}

        \part
            \begin{align*}
                \der{}{x} \bp{\dfrac{x^3-1}{2x^3+1}}^4 &= 4\bp{\dfrac{x^3-1}{2x^3+1}}^3 \cdot \der{}{x} \dfrac{x^3-1}{2x^3+1}\\
                &= 4\bp{\dfrac{x^3-1}{2x^3+1}}^3 \cdot \dfrac{\bp{2x^3+1} \der{}{x} \bp{x^3 - 1} - \bp{x^3 - 1} \der{}{x} \bp{2x^3+1}}{\bp{2x^3+1}^2}\\
                &= 4\bp{\dfrac{x^3-1}{2x^3+1}}^3 \cdot \dfrac{\bp{2x^3+1} \cdot 3x^2 - \bp{x^3 - 1} \cdot 6x^2}{\bp{2x^3+1}^2}\\
                &= \dfrac{4\bp{x^3-1}^3}{\bp{2x^3+1}^3} \cdot \dfrac{\bp{2x^3+1} \cdot 3x^2 - \bp{x^3 - 1} \cdot 6x^2}{\bp{2x^3+1}^2}\\
                &= \dfrac{4\bp{x^3-1}^3}{\bp{2x^3+1}^5} \cdot \bp{\bp{2x^3+1} \cdot 3x^2 - \bp{x^3 - 1} \cdot 6x^2}\\
                &= \dfrac{12x^2\bp{x^3-1}^3}{\bp{2x^3+1}^5} \cdot \bp{2x^3+1 - \bp{x^3 - 1} \cdot 2}\\
                &= \dfrac{12x^2\bp{x^3-1}^3}{\bp{2x^3+1}^5} \cdot 3\\
                &= \dfrac{36x^2\bp{x^3-1}^3}{\bp{2x^3+1}^5}
            \end{align*}

            \boxt{$\der{}{x} \bp{\dfrac{x^3-1}{2x^3+1}}^4 = \dfrac{36x^2\bp{x^3-1}^3}{\bp{2x^3+1}^5}$}

    \problem{}
        Using a graphing calculator, find the derivative of $\dfrac{e^{2x}}{x^2+1}$ when $x = 1.5$.

    \solution
        \boxt{$\evalder{\der{}{x} \dfrac{e^{2x}}{x^2+1}}{x = 1.5} = 6.66$}

    \problem{}
        Find the derivative with respect to $x$ of
        
        \begin{enumerate}
            \item $\cos x^\circ$
            \item $\cot{1-2x^2}$
            \item $\tan^3 (5x)$
            \item $\dfrac{\sec x}{1 + \tan x}$
        \end{enumerate}

    \solution
        \part
            \begin{align*}
                \der{}{x} \cos x^\circ &= \der{}{x} \cos{\dfrac{\pi}{180} x}\\
                &= \dfrac{\pi}{180} \cdot \bp{-\sin{\dfrac{\pi}{180} x}}\\
                &= -\dfrac{\pi}{180} \sin{\dfrac{\pi}{180} x}
            \end{align*}

            \boxt{$\der{}{x} \cos x^\circ = -\dfrac{\pi}{180} \sin{\dfrac{\pi}{180} x}$}

        \part
            \begin{align*}
                \der{}{x} \cot{1-2x^2} &= -\csc{1 - 2x^2} \cdot (-4x)\\
                &= 4x\csc{1-2x^2}
            \end{align*}

            \boxt{$\der{}{x} \cot{1-2x^2} = 4x\csc{1-2x^2}$}

        \part
            \begin{align*}
                \der{}{x} \tan^3 (5x) &= 3\tan^2(5x) \cdot \sec^2(5x) \cdot 5 \\
                &= 15\tan^2(5x)\sec^2(5x)
            \end{align*}

            \boxt{$\der{}{x} \tan^3 (5x) = 15\tan^2(5x)\sec^2(5x)$}

        \part
            \begin{align*}
                \der{}{x} \dfrac{\sec x}{1 + \tan x} &= \dfrac{(1 + \tan x) \cdot \der{}{x} \sec x - \sec x \cdot \der{}{x} (1 + \tan x)}{(1 + \tan x)^2}\\
                &= \dfrac{(1 + \tan x) \cdot \sec x \tan x - \sec x \cdot \sec^2 x}{(1 + \tan x)^2}\\
                &= \dfrac{\sec x ( \tan x(1 + \tan x) - \sec^2 x)}{(1 + \tan x)^2}\\
                &= \dfrac{\sec x ( \tan x + \tan^2 x) - (\tan^2 x + 1))}{(1 + \tan x)^2}\\
                &= \dfrac{\sec x (\tan x - 1)}{(1 + \tan x)^2}
            \end{align*}

            \boxt{$\der{}{x} \dfrac{\sec x}{1 + \tan x} = \dfrac{\sec x (\tan x - 1)}{(1 + \tan x)^2}$}

    \problem{}
        Find the derivative with respect to $x$ of

        \begin{enumerate}
            \item $y = e^{1 + \sin 3x}$
            \item $y = x^2 e^\frac1{x}$
            \item $y = \ln{\dfrac{1-x}{\sqrt{1+x^2}}}$
            \item $y = \dfrac{\ln{2x}}x$
            \item $y = \log_2 (3x^4 - e^x)$
            \item $y = 3^{\ln \sin x}$
            \item $y = a^{2\log_a x}$
            \item $y = \sqrt[3]{\dfrac{e^x (x+1)}{x^2 +1}}, \, x > 0$
        \end{enumerate}

    \solution
        \part
            \begin{align*}
                \der{y}{x} &= \der{}{x} e^{1 + \sin 3x}\\
                &= e^{1 + \sin 3x} \cdot \der{}{x} (1 + \sin 3x)\\
                &= e^{1 + \sin 3x} \cdot \cos{3x} \cdot 3\\
                &= 3 e^{1 + \sin 3x}\cos{3x} 
            \end{align*}

            \boxt{$\der{y}{x} = 3 e^{1 + \sin 3x}\cos{3x}$}

        \part
            \begin{align*}
                \der{y}{x} &= \der{}{x} x^2 e^\frac1{x}\\
                &= x^2 \cdot \der{}{x} e^{\frac1{x}} + e^{\frac{1}x} \cdot \der{}{x}x^2\\
                &= x^2 \cdot e^{\frac{1}x} \cdot \der{}{x} \dfrac1x + e^{\frac1x}\cdot2x\\
                &= x^2 \cdot e^{\frac{1}x} \cdot -\dfrac1{x^2} + e^{\frac1x}\cdot2x\\
                &= -e^{\frac1x} + 2xe^{\frac1x}\\
                &= e^{\frac1x}(2x-1)
            \end{align*}

            \boxt{$\der{y}{x} = e^{\frac1x}(2x-1)$}

        \part
            \begin{alignat*}{2}
                &&y &= \ln{\dfrac{1-x}{\sqrt{1+x^2}}}\\
                && &= \ln{1+x^2} - \ln{\sqrt{1+x^2}}\\
                && &= \ln{1+x^2} - \dfrac12 \ln{1+x^2}\\
                \implies&&\der{y}{x} &= \dfrac{1}{1-x} \cdot \der{}{x} (1-x) - \dfrac12 \bp{\dfrac{1}{1+x^2} \cdot \der{}{x} (1+x^2)  }\\
                && &= \dfrac{1}{1-x} \cdot -1 - \dfrac12 \bp{\dfrac{1}{1+x^2} \cdot 2x  }\\
                && &= \dfrac{1}{x-1} - \dfrac{x}{1+x^2}\\
                && &= \dfrac{\bp{1+x^2} - x(x-1)}{(x-1)\bp{1 +x^2}}\\
                && &= \dfrac{1+x^2 - x^2 +x}{(x-1)\bp{1 +x^2}}\\
                && &= \dfrac{1+x}{(x-1)\bp{1 +x^2}}
            \end{alignat*}

            \boxt{$\der{y}{x} = \dfrac{1+x}{(x-1)\bp{1 +x^2}}$}

        \part
            \begin{align*}
                \der{y}{x} &= \der{}{x} \dfrac{\ln{2x})}x\\
                &= \dfrac{x \cdot \der{}{x} \ln{2x} - \ln{2x} \cdot \der{}{x} x}{x^2}\\
                &= \dfrac{x \cdot \dfrac1{2x} \cdot \der{}{x} 2x - \ln{2x}}{x^2}\\
                &= \dfrac{x \cdot \dfrac1{2x} \cdot 2 - \ln{2x}}{x^2}\\
                &= \dfrac{1- \ln{2x}}{x^2}
            \end{align*}

            \boxt{$\der{y}{x} = \dfrac{1-\ln{2x}}{x^2}$}

        \part
            \begin{alignat*}{2}
                &&y &= \log_2 (3x^4 - e^x)\\
                \implies&&2^y &= 3x^4 - e^x\\
                \implies&&e^{\ln{2^y}} &= 3x^4 - e^x\\
                \implies&&e^{y\ln{2}} &= 3x^4 - e^x\\
                \implies&&e^{y\ln2} \cdot \der{y}{x}\cdot\ln2 &= 3 \cdot 4x^3 - e^x\\
                \implies&&\der{y}{x} &= \dfrac{12x^3 - e^x}{e^{y\ln2} \ln 2}\\
                && &= \dfrac{12x^3 - e^x}{(3x^4 - e^x) \ln 2}
            \end{alignat*}

            \boxt{$\der{y}{x} = \dfrac{12x^3 - e^x}{(3x^4 - e^x) \ln 2}$}

        \part
            \begin{alignat*}{2}
                &&y &= 3^{\ln \sin x}\\
                \implies&&\log_3 y &= \ln \sin x\\
                \implies&& \dfrac{\ln y}{\ln 3} &= \ln \sin x\\
                \implies&& \ln y &= \ln 3 \cdot\ln \sin x\\
                \implies&& \dfrac{1}{y} \cdot \der{y}{x} &= \ln 3 \cdot \dfrac{1}{\sin{x}} \cdot \der{}{x} \sin x\\
                && &= \ln 3 \cdot \dfrac{1}{\sin{x}} \cdot \cos x\\
                \implies&& \der{y}{x} &= \ln 3 \cdot \cot{x} \cdot y\\
                && &= \ln 3 \cdot \cot{x} \cdot 3^{\ln \sin x}
            \end{alignat*}

            \boxt{$\der{y}{x} = \ln 3 \cdot \cot{x} \cdot 3^{\ln \sin x}$}

        \part
            \begin{alignat*}{2}
                &&y &= a^{2\log_a x}\\
                && &= a^{\log_a{x^2}}\\
                && &= x^2\\
                \implies&&\der{y}{x} &= \der{}{x} x^2\\
                && &= 2x
            \end{alignat*}

            \boxt{$\der{y}{x} = 2x$}
            
        \part
            \begin{alignat*}{2}
                &&y &= \sqrt[3]{\dfrac{e^x (x+1)}{x^2 +1}}\\
                \implies&&y^3 &= \dfrac{e^x (x+1)}{x^2 +1}\\
                \implies&&\bp{x^2+1}y^3 &= e^x(x+1)\\
                \implies&&\bp{x^2 + 1} \cdot \der{}{x} y^3 + y^3 \cdot \der{}{x} \bp{x^2 + 1} &= e^x \cdot \der{}{x} (x+1) + (x+1) \der{}{x} e^x\\
                \implies&&\bp{x^2 + 1} \cdot 3y^2 \cdot \der{y}{x} + y^3 \cdot 2x &= e^x \cdot 1 + (x+1) \cdot e^x\\
                \implies&&\bp{x^2 + 1} \cdot 3y^2 \cdot \der{y}{x} &= e^x (x+2) - 2xy^3\\
                \implies&&\der{y}{x} &= \dfrac{e^x (x+2) - 2xy^3}{\bp{x^2 + 1} \cdot 3y^2}\\
                && &= \dfrac13 \bp{ \dfrac{e^x(x+2)}{\bp{x^2 + 1}y^2} - \dfrac{2xy}{x^2 + 1} }\\
                && &= \dfrac13 \bp{ \dfrac{e^x(x+1)(x+2)}{\bp{x^2 + 1}(x+1)y^2} - \dfrac{2xy}{x^2 + 1} }\\
                && &= \dfrac13 \bp{ \dfrac{y^3(x+2)}{(x+1)y^2} - \dfrac{2xy}{x^2 + 1} }\\
                && &= \dfrac13 \bp{ \dfrac{y(x+2)}{(x+1)} - \dfrac{2xy}{x^2 + 1} }\\
                && &= \dfrac{y}3 \bp{ \dfrac{(x+2)}{(x+1)} - \dfrac{2x}{x^2 + 1} }\\
                && &= \dfrac{y}3 \bp{ 1 + \dfrac{1}{x+1} - \dfrac{2x}{x^2 + 1} }\\
                && &= \dfrac{1}3 \cdot \sqrt[3]{\dfrac{e^x (x+1)}{x^2 +1}} \cdot \bp{ 1 + \dfrac{1}{x+1} - \dfrac{2x}{x^2 + 1} }
            \end{alignat*}

            \boxt{$\der{y}{x} = \dfrac{1}3 \cdot \sqrt[3]{\dfrac{e^x (x+1)}{x^2 +1}} \cdot \bp{ 1 + \dfrac{1}{x+1} - \dfrac{2x}{x^2 + 1} }$}

    \problem{}
        Find the derivative with respect to $x$ of

        \begin{enumerate}
            \item $\arccos \dfrac{x}{10}$
            \item $\arctan \dfrac1{1-x}$
            \item $\arcsin (\tan x)$
        \end{enumerate}

    \solution
        \part
            \begin{align*}
                \der{}{x} \arccos \dfrac{x}{10} &= -\dfrac1{\sqrt{1-(x/10)^2}} \cdot \dfrac{1}{10}\\
                &= -\dfrac{1}{\sqrt{100 - x^2}}
            \end{align*}

            \boxt{$\der{}{x} \arccos \dfrac{x}{10} = -\dfrac{1}{\sqrt{100 - x^2}}$}

        \part
            \begin{align*}
                \der{}{x} \arctan \dfrac1{1-x} &= \dfrac1{1+1/(1-x)^2} \cdot -\dfrac1{(1-x)^2} \cdot -1 \\
                &= \dfrac1{(1-x)^2\bp{1+ 1/(1-x)^2}}\\
                &= \dfrac1{(1-x)^2+1}\\
            \end{align*}

            \boxt{$\der{}{x} \arctan \dfrac1{1-x} = \dfrac1{(1-x)^2+1}$}

        \part
            \[
                \der{}{x} \arcsin (\tan x) = \dfrac1{1-\tan^2 x} \cdot \sec^2x
            \]

            \boxt{$\der{}{x} \arcsin (\tan x) = \dfrac{\sec^2x}{1-\tan^2 x}$}

    \problem{}
        Find an expression for $\der{y}{x}$ in terms of $x$ and $y$.
        
        \begin{enumerate}
            \item $(y-x)^2 = 2a(y+x)$, where $a$ is a constant
            \item $y^2 = e^{2x}y + xe^x$
            \item $y = x^y$
            \item $\sin x \cos y = \dfrac12$
        \end{enumerate}

    \solution
        \part
            \begin{alignat*}{2}
                && (y-x)^2 &= 2a(y+x)\\
                \implies&& 2(y-x) \bp{\der{y}{x} - 1} &= 2a\bp{\der{y}{x} + 1}\\
                \implies&& (y-x) \bp{\der{y}{x} - 1} &= a\bp{\der{y}{x} + 1}\\
                \implies&& (y-x)\cdot\der{y}{x} - (y-x) &= a\cdot\der{y}{x} + a\\
                \implies&& (y-x-a)\cdot\der{y}{x} &= a + y-x\\
                \implies&& \der{y}{x} &= \dfrac{a + y-x}{y-x-a}
            \end{alignat*}

            \boxt{$\der{y}{x} = \dfrac{a + y-x}{y-x-a}$}

        \part
            \begin{alignat*}{2}
                &&y^2 &= e^{2x}y + xe^x\\
                \implies&&2y\cdot\der{y}{x} &= e^{2x} \cdot \der{y}{x} + y \cdot \der{}{x} e^{2x} + x \cdot \der{}{x} e^x + e^x \cdot \der{}{x} x\\
                \implies&&2y\cdot\der{y}{x} &= e^{2x} \cdot \der{y}{x} + y \cdot e^{2x} \cdot 2 + xe^x + e^x\\
                \implies&&\bp{2y - e^{2x}}\cdot\der{y}{x} &= y \cdot e^{2x} \cdot 2 + xe^x + e^x\\
                \implies&&\bp{2y - e^{2x}}\cdot\der{y}{x} &= e^x \bp{2e^{x}y + x + 1}\\
                \implies&&\der{y}{x} &= \dfrac{e^x \bp{2e^{x}y + x + 1}}{2y - e^{2x}}
            \end{alignat*}

            \boxt{$\der{y}{x} = \dfrac{e^x \bp{2e^{x}y + x + 1}}{2y - e^{2x}}$}

        \part
            \begin{alignat*}{2}
                &&y &= x^y\\
                \implies&&\ln y &= y \ln x\\
                \implies&& \dfrac1{y} \cdot \der{y}{x} &= y \cdot\der{}{x} \ln x + \ln x \cdot \der{y}{x}\\
                && &= y \cdot \dfrac1x + \ln x \cdot \der{y}{x}\\
                \implies&&\bp{\dfrac1y - \ln x} \cdot \der{y}{x} &= \dfrac{y}{x}\\
                \implies&&\der{y}{x} &= \dfrac{\tfrac{y}{x}}{\tfrac1y - \ln x}\\
                && &= \dfrac{y^2}{x - xy\ln x}
            \end{alignat*}

            \boxt{$\der{y}{x} = \dfrac{y^2}{x - xy\ln x}$}

        \part
            \begin{alignat*}{2}
                &&\sin x \cos y &= \dfrac12\\
                \implies&&\sin x \cdot \der{}{x} \cos y + \cos y \cdot \der{}{x} \sin x &= 0\\
                \implies&&\sin x \cdot -\sin y \cdot \der{y}{x} + \cos y \cdot \cos x &= 0\\
                \implies&&-\sin x \sin y \cdot \der{y}{x} &= -\cos x \cos y\\
                \implies&&\der{y}{x} &= \dfrac{\cos x \cos y}{\sin x \sin y}\\
                && &= \cot x \cot y
            \end{alignat*}

            \boxt{$\der{y}{x} = \cot x \cot y$}

    \problem{}
        It is given that $x$ and $y$ satisfy the equation        
        \begin{equation}\label{P9}
            \arctan x + \arctan y  + \arctan{xy} = \dfrac7{12} \pi
        \end{equation}

        \begin{enumerate}
            \item Find the exact value of $y$ when $x = 1$.
            \item Express $\der{}{x} \arctan{xy}$ in terms of $x$, $y$ and $y'$.
            \item Show that, when $x = 1$, $y' = -\dfrac13 - \dfrac1{2\sqrt3}$.
        \end{enumerate}
            
    \solution
        \part
            Substituting $x = 1$ into Equation~\ref{P9},
            \begin{alignat*}{2}
                &&\arctan1 + \arctan y + \arctan y &= \dfrac{7}{12} \pi\\
                \implies&&\dfrac14 \pi + 2\arctan y &= \dfrac{7}{12} \pi\\
                \implies&& \arctan y &= \dfrac16 \pi\\
                \implies&& y &= \dfrac1{\sqrt3}
            \end{alignat*}
            
            \boxt{$y = \dfrac1{\sqrt3}$}
        
        \part
            \begin{align*}
                \der{}{x} \arctan{xy} &= \dfrac1{1+(xy)^2} \cdot \der{}{x} (xy) \\
                &= \dfrac1{1+(xy)^2} \cdot (xy' + y)
            \end{align*}

            \boxt{$\der{}{x} \arctan{xy} = \dfrac1{1+(xy)^2} \cdot (xy' + y)$}

        \part
            Differentiating Equation~\ref{P9} with respect to $x$,
            \[
                \dfrac{1}{1+x^2} + \dfrac{y'}{1+y^2} + \dfrac1{1+(xy)^2} \cdot (xy' + y) = 0
            \]

            Substituting $x=1$,
            \begin{alignat*}{2}
                &&\dfrac12 + \dfrac34 y' + \dfrac34 (y' + y) &= 0\\
                \implies&&y'\bp{\dfrac34 + \dfrac34} &= \dfrac34 \bp{-\dfrac1{\sqrt3}} - \dfrac12\\
                \implies&&\dfrac32 y' &= \dfrac{-3-2\sqrt3}{4\sqrt3}\\
                \implies&&y' &= \dfrac23 \cdot \dfrac{-3-2\sqrt3}{4\sqrt3}\\
                && &= -\dfrac{1}{2\sqrt3} - \dfrac13
            \end{alignat*}

    \problem{}
        Find $\der{y}{x}$ for

        \begin{enumerate}
            \item $x = \dfrac1{1+t^2}, \, y = \dfrac{t}{1+t^2}$
            \item $x = \dfrac12 (e^t - e^{-t}), \, y = \dfrac12 (e^t + e^{-t})$
            \item $x = a\sec\t, \, y = a\tan\t$
            \item $x = e^{3\t}\cos{3\t}, \, y = e^{3\t}\sin{3\t}$
        \end{enumerate}

    \solution
        \part
            Observe that $y = xt$.
            \begin{align*}
                \der{y}{x} &= x \cdot \der{t}{x} + t\\
                &= x \bp{\der{x}{t}}^{-1} + t\\
                &= \dfrac1{1+t^2} \bp{ -\dfrac{1}{\bp{1+t^2}^2} \cdot 2t }^{-1} + t\\
                &= \dfrac1{1+t^2} \cdot \bp{ -\dfrac{\bp{1+t^2}^2}{2t} } + t\\
                &= -\dfrac{1+t^2}{2t} + \dfrac{2t^2}{2t}\\
                &= \dfrac{t^2-1}{2t}
            \end{align*}

            \boxt{$\der{y}{x} = \dfrac{t^2-1}{2t}$}

        \part
            \begin{align*}
                \der{y}{x} &= \der{y}{t} \cdot \der{t}{x}\\
                &= \der{y}{t} \cdot \bp{\der{x}{t}}^{-1}\\
                &= \dfrac{\tfrac12(e^t - e^{-t})}{\tfrac12(e^t + e^{-t})}\\
                &= \dfrac{e^t - e^{-t}}{e^t + e^{-t}}
            \end{align*}

            \boxt{$\der{y}{x} = \dfrac{e^t - e^{-t}}{e^t + e^{-t}}$}

        \part
            Recall that $\tan^2 \t + 1 = \sec^2 \t$. Hence, $y^2 + a^2 = x^2$. Implicitly differentiating, we have
            \begin{alignat*}{2}
                &&2y\cdot\der{y}{x} &= 2x\\
                \implies&&\der{y}{x} &= \dfrac{x}{y}\\
                && &= \dfrac{a\sec \t}{a \tan \t}\\
                && &= \csc \t
            \end{alignat*}

            \boxt{$\der{y}{x} = \csc\t$}

        \part
            Recall that $\sin^2 \t + \cos^2 \t = 1$. Hence, $x^2 + y^2 = e^{6\t}$. Implicitly differentiating, we have
            \begin{alignat}{2}
                &&2x + 2y\der{y}{x} &= e^{6\t} \cdot 6\der{\t}{x}\nonumber\\
                \implies&&x + y\der{y}{x} &= 3e^{6\t}\cdot\der{\t}{x}\label{P10D-1}
            \end{alignat}

            Observe that $\dfrac{y}{x} = \tan{3\t}$. Implicitly differentiating,
            \begin{alignat}{2}
                &&\dfrac{x\der{y}{x} - y}{x^2} &= \sec^2{3\t} \cdot 3\der{\t}{x}\nonumber\\
                \implies&&x\der{y}{x} - y &= 3 x^2 \sec^2{3\t} \cdot \der{\t}{x}\nonumber\\
                && &= 3 (e^{3\t}\cos{3\t})^2 \cdot \dfrac1{\cos^2{3\t}} \cdot \der{\t}{x}\nonumber\\
                && &= 3e^{6\t}\cdot\der{\t}{x}\label{P10D-2}
            \end{alignat}

            Subtracting Equation~\ref{P10D-1} from Equation~\ref{P10D-2}, 
            \begin{alignat*}{2}
                &&x\der{y}{x} - y \der{y}{x} - y - x &= 0\\
                \implies&&(x-y)\der{y}{x} &= x+y\\
                \implies&&\der{y}{x} &= \dfrac{x+y}{x-y}\\
                && &= \dfrac{e^{3\t}(\cos 3\t + \sin 3\t)}{e^{3\t}(\cos 3\t - \sin 3\t)}\\
                && &= \dfrac{\cos 3\t + \sin 3\t}{\cos 3\t - \sin 3\t}\\
                && &= \dfrac{1 + \tan 3\t}{1 - \tan 3\t}\\
                && &= \tan{3\t + \dfrac{\pi}4}
            \end{alignat*}

            \boxt{$\der{y}{x} = \tan{3\t + \dfrac{\pi}4}$}

    \problem{}
        A curve is defined by the parametric equation
        \[
            x = 120t - 4t^2, \, y = 60t - 6t^2
        \]

        Find the value of $\der{y}{x}$ at each of the points where the curve cross the $x$-axis.

    \solution
        The curve crosses the $x$-axis when $y=0$.
        \begin{alignat*}{2}
            &&y &= 0\\
            \implies&&60t-6t^2 &= 0\\
            \implies&&10t-t^2 &= 0\\
            \implies&&t(10 - t) &= 0
        \end{alignat*}

        Hence, $t = 0$ or $t = 10$. Now, consider the derivative with respect to $x$ of the curve.
        \begin{align*}
            \der{y}{x} &= \der{y}{t} \cdot \der{t}{x}\\
            &= \der{y}{t} \cdot \bp{\der{x}{t}}^{-1}\\
            &= \dfrac{60 - 12t}{120 - 8t}
        \end{align*}

        \case{1}{$t =0$}
        \begin{align*}
            \evalder{\der{y}{x}}{t = 0} &= \dfrac{60 - 12\cdot 0}{120 - 8\cdot0}\\
            &= \dfrac12
        \end{align*}

        \case{2}{$t =10$}
        \begin{align*}
            \evalder{\der{y}{x}}{t = 10} &= \dfrac{60 - 12\cdot 10}{120 - 8\cdot10}\\
            &= -\dfrac32
        \end{align*}

        \boxt{$\der{y}{x} = \dfrac12 \lor -\dfrac32$}

    \problem{}
        A curve has parametric equations $x = 2t-\ln{2t}, \, y = t^2 - \ln t^2$, where $t > 0$. Find the value of $t$ at the point on the curve at which the gradient is 2.

    \solution
        \begin{align*}
            \der{y}{x} &= \der{y}{t} \cdot \der{t}{x}\\
            &= \der{y}{t} \cdot \bp{\der{x}{t}}^{-1}\\
            &= \dfrac{2t - \tfrac2t}{2 - \tfrac1t}\\
            &= \dfrac{2t^2 - 2}{2t - 1}
        \end{align*}

        Consider $\der{y}{x} = 2$.
        \begin{alignat*}{2}
            &&\der{y}{x} &= 2\\
            \implies&&\dfrac{2t^2 - 2}{2t - 1} &= 2\\
            \implies&&\dfrac{t^2 - 1}{2t - 1} &= 1\\
            \implies&&t^2 - 1 &= 2t -1\\
            \implies&&t^2 - 2t &= 0\\
            \implies&&t(t-2) &= 0
        \end{alignat*}

        Hence, $t = 0$ or $t = 2$. Since $t > 0$, we reject $t = 0$. Thus, $t = 2$.

        \boxt{$t = 2$}

    \problem{}
        If $y = \ln{\sin^3 2x}$, find $\der{y}{x}$ and prove that $3\der[2]{y}{x} + \bp{\der{y}{x}}^2 + 36 = 0$.

    \solution
        \begin{alignat*}{2}
            &&y &= \ln{\sin^3 2x}\\
            \implies&&e^y &= \sin^3 2x\\
            \implies&&e^y \cdot \der{y}{x} &= 3\sin^2 2x \cdot \cos 2x \cdot 2\\
            && &= 6\sin^2 2x \cos 2x\\
            \implies&& \der{y}{x} &= \dfrac{6\sin^2 2x \cos 2x}{e^y}\\
            && &= \dfrac{6\sin^2 2x \cos 2x}{\sin^3 2x}\\
            && &= \dfrac{6 \cos 2x}{\sin 2x}\\
            && &= 6 \cot 2x
        \end{alignat*}
        
        \boxt{$\der{y}{x} = 6\cot 2x$}

        \begin{alignat*}{2}
            &&\der{y}{x} &= 6\cot 2x\\
            \implies&&\der[2]{y}{x} &= 6 \cdot -\csc^2 2x \cdot 2\\
            && &= -12 \csc^2 2x\\
            && &= -12 \bp{1 + \cot^2 2x}\\
            && &= -12 - 12\cot^2 2x\\
            && &= -12 - \dfrac13 \bp{\der{y}{x}}^2\\
            \implies&&3\der[2]{y}{x} &= -36 - \bp{\der{y}{x}}^2\\
            \implies&& 3\der[2]{y}{x} + \bp{\der{y}{x}}^2 + 36 &= 0
        \end{alignat*}

    \problem{}
        Given that $y = e^{\arcsin{2x}}$, show that $\bp{1-4x^2}\der[2]{y}{x} - 4x\der{y}{x} = 4y$. Differentiate this result further to obtain a differential equation for $\der[3]{y}{x}$.

    \solution
        \begin{alignat*}{2}
            &&y &= e^{\arcsin{2x}}\\
            \implies&& \ln y &= \arcsin{2x}\\
            \implies&& \dfrac1y \cdot \der{y}{x} &= \dfrac{1}{\sqrt{1 - (2x)^2}} \cdot 2\\
            \implies&& \der{y}{x} &= \dfrac{2y}{\sqrt{1 - 4x^2}}\\
            \implies&& \der[2]{y}{x} &= \dfrac{\sqrt{1 - 4x^2} \cdot 2\der{y}{x} - 2y \cdot \dfrac{1}{2\sqrt{1-4x^2}} \cdot -8x}{1-4x^2}\\
            \implies&& \bp{1-4x^2}\der[2]{y}{x} &= 2\sqrt{1 - 4x^2} \der{y}{x} + 4x \dfrac{2y}{\sqrt{1-4x^2}}\\
            && &= 2\sqrt{1 - 4x^2} \cdot \dfrac{2y}{\sqrt{1 - 4x^2}} + 4x \der{y}{x} \\
            && &= 4y + 4x \der{y}{x} \\
            \implies&& \bp{1-4x^2}\der[2]{y}{x} -4x \der{y}{x} &= 4y
        \end{alignat*}
        \begin{alignat*}{2}
            && \bp{1-4x^2}\der[2]{y}{x} -4x \der{y}{x} &= 4y\\
            \implies&&\bp{1 - 4x^2} \der[3]{y}{x} + \der[2]{y}{x} \cdot (-8x) - 4\bp{x  \der[2]{y}{x} + \der{y}{x} } &= 4\der{y}{x}\\
            \implies&&\bp{1 - 4x^2} \der[3]{y}{x} -8x \der[2]{y}{x} - 4x  \der[2]{y}{x} - 4\der{y}{x} &= 4\der{y}{x}\\
            \implies&&\bp{1 - 4x^2} \der[3]{y}{x} -12x \der[2]{y}{x} - 8\der{y}{x} &= 0
        \end{alignat*}

        \boxt{$\bp{1 - 4x^2} \der[3]{y}{x} -12x \der[2]{y}{x} - 8\der{y}{x} = 0$}

\end{document}