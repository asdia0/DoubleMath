\documentclass{echw}

\title{Tutorial B6\\Maclaurin Series}
\author{Eytan Chong}
\date{2024-04-15}

\begin{document}
    \problem{}
        \begin{enumerate}
            \item Given that $f(x) = e^{\cos x}$, find $f(0)$, $f'(0)$ and $f''(0)$. Hence write down the first two non-zero terms in the Maclaurin series for $f(x)$. Give the coefficients in terms of $e$.
            \item Given that $g(x) = \tan\left(2x + \dfrac14 \pi \right)$, find $g(0)$, $g'(0)$ and $g''(0)$. Hence find the first three terms in the Maclaurin series of $g(x)$.
        \end{enumerate}

    \solution
        \part
            \begin{alignat*}{2}
                &&f(x) &= e^{\cos x}\\
                \implies&&f'(x) &= e^{\cos x} \cdot -\sin x\\
                && &= -\sin x \cdot f(x)\\
                \implies&&f''(x) &=-\cos x \cdot f(x) - \sin x \cdot f'(x)
            \end{alignat*}

            Evaluating the above derivatives at $x = 0$,

            \begin{align*}
                f(0) &= e\\
                f'(0) &= 0\\
                f''(0) &= -e
            \end{align*}

            Hence,

            \begin{align*}
                f(x) &=  \sum_{n = 0}^{\infty} \dfrac{f^{(n)}(0)}{n!} x^n\\
                &= \dfrac{e}{0!}x^0 + \dfrac{0}{1!}x^1 + \dfrac{-e}{2!}x^2 + \ldots\\
                &= e - \dfrac{e}{2}x^2 + \ldots
            \end{align*}

            \boxt{
                $f(x) = e - \dfrac{e}{2}x^2 + \ldots$
            }

        \part
            \begin{alignat*}{2}
                &&g(x) &= \tan\left(2x + \dfrac14 \pi \right)\\
                \implies&&g'(x) &= \sec^2 \left(2x + \dfrac14 \pi \right) \cdot 2\\
                && &= 2\left(1 + \tan^2\left(2x + \dfrac14 \pi \right)\right)\\
                && &= 2 + 2g^2(x)\\
                \implies&&g''(x) &= 2 \cdot 2g(x) \cdot g'(x)\\
                && &= 4g(x)g'(x)
            \end{alignat*}

            Evaluating the above derivatives at $x = 0$,

            \begin{align*}
                g(x) &= 1\\
                g'(x) &= 4\\
                g''(x) &= 16
            \end{align*}

            Hence,

            \begin{align*}
                g(x) &= \sum_{n = 0}^{\infty} \dfrac{g^{(n)}(0)}{n!} x^n\\
                &= \dfrac{1}{0!}x^0 + \dfrac{4}{1!}x^1 + \dfrac{16}{2!}x^2 + \ldots\\
                &= 1 + 4x + 8x^2 + \ldots
            \end{align*}

            \boxt{
                $g(x) = 1 + 4x + 8x^2 + \ldots$
            }

    \problem{}
        Find the first three non-zero terms of the Maclaurin series for the following in ascending powers of $x$. In each case, find the range of values of $x$ for which the series is valid.

        \begin{enumerate}
            \item $\dfrac{(1+3x)^4}{\sqrt{1+2x}}$
            \item $\dfrac{\sin 2x}{2 + 3x}$
        \end{enumerate}

    \solution
        \part
            \begin{alignat}{2}
                &&y &= \dfrac{(1+3x)^4}{\sqrt{1 + 2x}}\label{P2-1-1}\\
                \implies&&y^2 &= \dfrac{(1+3x)^8}{1 + 2x}\nonumber\\
                \implies&&(1 + 2x)\cdot y^2 &= (1+3x)^8\label{P2-1-2}\\\nonumber
            \end{alignat}

            Implicitly differentiating Equation~\ref{P2-1-2},
            \begin{alignat}{2}
                &&(1+2x)\cdot 2y \cdot y' + y^2\cdot2 &= 8(1+3x)^7 \cdot 3\nonumber\\
                \implies&&(1+2x)\cdot y \cdot y' + y^2 &= 12(1+3x)^7\nonumber\\
                \implies&&y\left((1+2x)\cdot y' + y\right) &= 12(1+3x)^7\label{P2-1-3}\\\nonumber
            \end{alignat}

            Implicitly differentiating Equation~\ref{P2-1-3},
            \begin{alignat}{2}
                && y' \big((1+2x)\cdot y' + y\big) + y\big((1+2x)\cdot y'' + y' \cdot 2 + y'\big) &= 12 \cdot 7(1+3x)^6 \cdot 3\nonumber\\
                \implies&& (1+2x)\big(y'\big)^2 + (1+2x)y\cdot y'' + 4y \cdot y' &= 252(1+3x)^6 \label{P2-1-4}\\\nonumber
            \end{alignat}

            Evaluating Equations~\ref{P2-1-1},~\ref{P2-1-3} and~\ref{P2-1-4} at $x = 0$,
            \begin{align*}
                y(0) &= 1\\
                y'(0) &= 11\\
                y''(0) &= 87\\
            \end{align*}

            Hence,
            \begin{align*}
                \dfrac{(1+3x)^4}{\sqrt{1+2x}} &= \sum_{n=0}^\infty \dfrac{y^{(n)}(0)}{n!}x^n\\
                &= \dfrac{1}{0!}x^0 + \dfrac{11}{1!}x^1 + \dfrac{87}{2!}x^2 + \ldots\\
                &= 1 + 11x + \dfrac{87}2 x^2 + \ldots
            \end{align*}

            \boxt{
                $\dfrac{(1+3x)^4}{\sqrt{1+2x}} = 1 + 11x + \dfrac{87}2 x^2 + \ldots$
            }

            Note that the series is valid only when $\abs{2x} < 1 \implies -\dfrac12 < x < \dfrac12$.

            \boxt{
                $-\dfrac12 < x < \dfrac12$
            }

        \part
            \begin{alignat}{2}
                &&y &= \dfrac{\sin 2x}{2 + 3x}\label{P2-2-1}\\
                \implies&& (2+3x)y &= \sin 2x\label{P2-2-2}\\\nonumber
            \end{alignat}

            Implicitly differentiating Equation~\ref{P2-2-2},
            \begin{alignat}{2}
                &&(2+3x)y' + y\cdot 3 &= \cos 2x \cdot 2\nonumber\\
                \implies&&(2+3x)y' + 3y &= 2\cos 2x\label{P2-2-3}\\\nonumber
            \end{alignat}

            Implicitly differentiating Equation~\ref{P2-2-3},
            \begin{alignat}{2}
                &&(2+3x)y'' + y' \cdot 3 + 3y' &= 2\cdot-\sin 2x \cdot 2\nonumber\\
                \implies&& (2+3x)y'' + 6y' &= -4\sin 2x\label{P2-2-4}\\\nonumber
            \end{alignat}

            Implicitly differentiating Equation~\ref{P2-2-4},
            \begin{alignat}{2}
                && (2+3x)y''' + y'' \cdot 3 + 6y'' &= -4\cdot \cos 2x \cdot 2\nonumber\\
                \implies&& (2+3x)y''' + 9y'' &= -8\cos 2x\label{P2-2-5}\\\nonumber
            \end{alignat}

            Evaluating Equations~\ref{P2-2-1},~\ref{P2-2-3},~\ref{P2-2-4} and~\ref{P2-2-5} at $x = 0$,
            \begin{align*}
                y(0) &= 0\\
                y'(0) &= 1\\
                y''(0) &= -3\\
                y'''(0) &= \dfrac{19}2
            \end{align*}

            Hence,
            \begin{align*}
                \dfrac{\sin 2x}{2 + 3x} &= \sum_{n=0}^\infty \dfrac{y^{(n)}(0)}{n!}x^n\\
                &= \dfrac{0}{0!}x^0 + \dfrac{1}{1!}x^1 + \dfrac{-3}{2!}x^2 + \dfrac{\frac{19}2}{3!}x^3 + \ldots\\
                &= x - \dfrac32 x^2 + \dfrac{19}{12}x^3+\ldots
            \end{align*}

            \boxt{
                $\dfrac{\sin 2x}{2 + 3x} = x - \dfrac32 x^2 + \dfrac{19}{12}x^3+\ldots$
            }

            Note that the denominator can be rewritten as $2\left(1 + \dfrac32 x \right)$. Hence, the series is only valid when $\abs{\dfrac32 x} < 1 \implies -\dfrac23 < x < \dfrac23$.


    \problem{}
        Find the Maclaurin series of $\ln (1 + \cos x)$, up to and including the term in $x^4$.

    \solution
        Recall that 

        \begin{equation*}
            \ln (1 + x) = \sum_{n=0}^\infty (-1)^n \dfrac{x^{n+1}}{n+1}
        \end{equation*}

        and 

        \begin{equation*}
            \cos x = \sum_{n=0}^\infty \dfrac{(-1)^n}{(2n)!} x^{2n}
        \end{equation*}

        Hence,

        \begin{align*}
            \ln (1 + \cos x) &= \sum_{n=0}^\infty (-1)^n \dfrac{\cos^{n+1} x}{n+1}\\
            &= \sum_{n=0}^\infty \dfrac{(-1)^n}{n+1} \left(\sum_{k=0}^\infty \dfrac{(-1)^k}{(2k)!} x^{2k} \right)^{n+1}
        \end{align*}

        Consider $\left(\sum_{k=0}^\infty \dfrac{(-1)^k}{(2k)!} x^{2k} \right)^{n+1}$, which is equivalent to

        \begin{equation*}
            \underbrace{\left(1 - \dfrac{x^2}{2!} + \dfrac{x^4}{4!} + \ldots \right)\left(1 - \dfrac{x^2}{2!} + \dfrac{x^4}{4!} + \ldots \right)\ldots\left(1 - \dfrac{x^2}{2!} + \dfrac{x^4}{4!} + \ldots \right)}_\text{$(n + 1)$ copies}
        \end{equation*}

        The constant term is clearly $1$. Now consider the coefficient of the $x^2$ term. The only to obtain a $x^2$ term is to select a constant term ($1$) from $n$ copies, and a $x^2$ term ($-\dfrac{x^2}{2!}$) from the remaining copy. There are $\displaystyle\binom{n+1}{1} = n+1$ ways to do this. Hence, the coefficient of the $x^2$ term is $(n+1) \cdot 1 \cdot -\dfrac{1}{2!} = -\dfrac{n+1}2$.

        Now consider the coefficient of the $x^4$ term. The are two ways to obtain a $x^4$ term. The first way is to select a constant term ($1$) from $n$ copies, and a $x^4$ term ($\dfrac{x^4}{4!}$) from the remaining copy. There are $\displaystyle\binom{n+1}1 = n+1$ ways to do this, which contributes $(n+1) \cdot 1 \cdot \dfrac{1}{4!} = \dfrac{n+1}{24}$ to the coefficient of $x^4$.

        The second way to obtain a $x^4$ term is to select a $x^2$ term ($-\dfrac{x^2}{2!}$) from 2 copies and a constant term ($1$) from the remaining copies. There are $\displaystyle \binom{n+1}{2} = \dfrac{(n+1)n}2$ ways to do this, which further contributes $\dfrac{(n+1)n}2 \cdot 1 \cdot \left(-\dfrac{1}{2!}\right)^2 = \dfrac{n(n+1)}8$ to the coefficient of $x^4$. Hence, the coefficient of $x^4$ is given by $\dfrac{n+1}{24} + \dfrac{n(n+1)}8 = \dfrac{(n+1)(3n + 1)}{24}$. 

        Thus, up to and including the term in $x^4$,
        \begin{align*}
            \ln (1 + \cos x) &= \sum_{n=0}^\infty \dfrac{(-1)^n}{n+1} \left(1 - \dfrac{n+1}2 x^2 + \dfrac{(n+1)(3n+1)}{24} + \ldots\right)\\
            &= \sum_{n=0}^\infty (-1)^n \left(\dfrac1{n+1} - \dfrac12 x^2 + \dfrac{3n+1}{24}x^4 + \ldots\right)\\
            &= \sum_{n=0}^\infty \dfrac{(-1)^n}{n+1} - \dfrac12 x^2 \sum_{n=0}^\infty (-1)^n + \dfrac{3}{24} x^4 \sum_{n=0}^\infty n(-1)^n + \dfrac1{24} x^4 \sum_{n=0}^\infty (-1)^n + \ldots
        \end{align*}

        Observe that
        \begin{align*}
            \sum_{n=0}^\infty \dfrac{(-1)^n}{n+1} &= \sum_{n=0}^\infty (-1)^n \dfrac{1^{n+1}}{n+1}\\
            &= \ln(1 + 1)\\
            &= \ln 2
        \end{align*}

        Now consider the Abel regularization of $\sum\limits_{n=0}^\infty (-1)^n$.
        \begin{align*}
            \sum_{n=0}^\infty (-1)^n &= \lim_{x \to 1^-} \sum_{n=0}^\infty (-1)^n x^n\\
            &= \lim_{x \to 1^-} \sum_{n=0}^\infty (-x)^n\\
            &= \lim_{x \to 1^-} \dfrac{1}{1-(-x)}\\
            &= \dfrac12
        \end{align*}

        Now observe that $\sum\limits_{n=0}^\infty x^n$ is absolutely convergent for $\abs{x} < 1$. Hence,

        \begin{align*}
            \sum_{n=0}^\infty nx^{n-1} &= \sum_{n=0}^\infty \der{}{x} x^n\\
            &= \der{}{x} \sum_{n=0}^\infty x^n\\
            &= \der{}{x} \dfrac1{1-x}\\
            &= \dfrac{1}{(1-x)^2}
        \end{align*}

        Multiplying by $x$ on both sides gives

        \begin{equation*}
            \sum_{n=0}^\infty nx^n = \dfrac{x}{(1-x)^2}
        \end{equation*}

        Hence, the Abel regularization of $\sum\limits_{n=0}^\infty n(-1)^n$ is given by
        \begin{align*}
            \sum_{n=0}^\infty n(-1)^n &= \lim_{x \to 1^-} \sum_{n=0}^\infty n(-1)^n x^n\\
            &= \lim_{x \to 1^-} \sum_{n=0}^\infty n(-x)^n\\
            &= \lim_{x \to 1^-} \dfrac{-x}{(1-(-x))^2}\\
            &= -\dfrac14
        \end{align*}

        Finally,
        \begin{align*}
            \ln (1 + \cos x) &= \ln 2 - \dfrac12 x^2 \cdot \dfrac12 + \dfrac{3}{24} x^4 \cdot -\dfrac14 + \dfrac1{24} x^4 \cdot\dfrac12 + \ldots\\
            &= \ln 2 - \dfrac14 x^2 - \dfrac1{96}x^4 + \ldots
        \end{align*}

        \boxt{
            $\ln (1 + \cos x) = \ln 2 - \dfrac14 x^2 - \dfrac1{96}x^4 + \ldots$
        }
     
    \problem{}
        \begin{enumerate}
            \item Find the first three terms of the Maclaurin series for $e^x (1 + \sin 2x)$.
            \item It is given that the first two terms of this series are equal to the first two terms in the series expansion, in ascending powers of $x$, of $\left(1 + \dfrac43 x\right)^n$. Find $n$ and show that the third terms in each of these series are equal.
        \end{enumerate}
        
    \solution
        \part
            \begin{alignat*}{2}
                && f(x) &= e^x (1 + \sin 2x)\\
                && &= e^x + e^x \sin 2x\\
                && &= e^x + e^x \Im{e^{i2x}}\\
                && &= e^x + \Im{e^x e^{i2x}}\\
                && &= e^x + \Im{e^{x(1+2i)}}\\
                \implies && f^{(n)}(x) &= e^x + \Im{\derm{}{x}{n} e^{x(1+2i)}}\\
                && &= e^x + \Im{(1+2i)^n e^{x(1+2i)}}\\
                && &= e^x + \Im{\left(\sqrt 5 e^{i \arctan 2}\right)^n e^{x(1+2i)}}\\
                && &= e^x + \Im{5^{\frac{n}2} e^{in \arctan 2} e^{x(1+2i)}}\\
                && &= e^x + 5^{\frac{n}2} e^{x} \Im{ e^{i(n \arctan 2 + 2x)} }\\
                && &= e^x + 5^{\frac{n}2} e^{x} \sin{\left(n \arctan 2 + 2x\right)}\\
                \implies && f^{(n)}(0) &= 1 + 5^{\frac{n}2} e^{x} \sin{(n \arctan 2)}
            \end{alignat*}

            Hence,
            \begin{alignat*}{2}
                &&f^{(0)}(0) &= 1 + 5^{\frac{0}2} e^{x} \sin{(0 \arctan 2)}\\
                && &= 1\\
                &&f^{(1)}(0) &= 1 + 5^{\frac{1}2} e^{x} \sin{(1 \arctan 2)}\\
                && &= 1 + \sqrt5 \cdot \dfrac{2}{\sqrt5}\\
                && &= 3\\
                &&f^{(2)}(0) &= 1 + 5^{\frac{2}2} e^{x} \sin{(2 \arctan 2)}\\
                && &= 1 + 5 \cdot 2\sin\left(\arctan 2\right) \cos\left(\arctan 2\right)\\
                && &= 1 + 5 \cdot 2 \cdot \dfrac{2}{\sqrt5} \cdot \dfrac1{\sqrt5}\\
                && &= 5
            \end{alignat*}

            Thus,
            \begin{align*}
                e^x(1 + \sin 2x) &= \sum_{n=0}^\infty \dfrac{f^{(n)}(0)}{n!} x^n\\
                &= \dfrac{1}{0!}x^0 + \dfrac{3}{1!}x^1 + \dfrac{5}{2!}x^2 + \ldots\\
                &= 1 + 3x + \dfrac52 x^2 + \ldots
            \end{align*}

            \boxt{
                $e^x(1 + \sin 2x) = 1 + 3x + \dfrac52 x^2 + \ldots$
            }

        \part
            By the Binomial Theorem,
            \begin{align*}
                \left(1 + \dfrac43x\right)^n &= \sum_{k=0}^n \binom{n}{k} \left(\dfrac43 x\right)^k  1^{n-k}\\
                &= \sum_{k=0}^n \binom{n}{k} \left(\dfrac43 \right)^k x^k\\
                &= \binom{n}{0} \left(\dfrac43 \right)^0 x^0 + \binom{n}{1} \left(\dfrac43 \right)^1 x^1 + \ldots\\
                &= 1 + \dfrac43 nx + \ldots
            \end{align*}

            Comparing the coefficient of $x$ terms, we have $3 = \dfrac43 n$, whence $n = \dfrac94$. Hence, the third term is in the expansion of $\left(1 + \dfrac43x\right)^n$ is given by

            \begin{align*}
                \binom{\frac94}{2} \left(\dfrac43 \right)^2 x^2 &= \dfrac{\frac94(\frac94-1)}{2}  \left(\dfrac43\right)^2 x^2\\
                &= \dfrac52 x^2
            \end{align*}

            Hence, the third terms in each of these series are equal.

    \problem{}
        \begin{enumerate}
            \item Show that the first three non-zero terms in the expansion of $\left(\dfrac8{x^3} - 1\right)^{\tfrac13}$ in ascending powers of $x$ are in the form $\dfrac{a}x + bx^2 + cx^5$, where $a$, $b$ and $c$ are constants to be determined.
            \item By putting $x = \dfrac23$ in your result, obtain an approximation for $\sqrt[3]{26}$ in the form of a fraction in its lowest terms.

            A student put $x = 6$ into the expansion to obtain an approximation of $\sqrt[3]{26}$. Comment on the suitability of this choice of $x$ for the approximation of $\sqrt[3]{26}$.
        \end{enumerate}

    \solution
        \part
            \begin{align*}
                \left(\dfrac8{x^3} - 1\right)^{\tfrac13} &= \dfrac2x \left(1 - \dfrac{x^3}8\right)^{\tfrac13}\\ 
                &= \dfrac2x \sum_{k=0}^\infty \binom{\frac13}{k} \left(- \dfrac{x^3}8\right)^k\\
                &= \dfrac2x \left(\binom{\frac13}{0} \left(- \dfrac{x^3}8\right)^0 + \binom{\frac13}{1} \left(- \dfrac{x^3}8\right)^1 + \binom{\frac13}{2} \left(- \dfrac{x^3}8\right)^2 + \ldots \right)\\
                &= \dfrac2x \left(1 + \dfrac13 \cdot - \dfrac{x^3}8 + \dfrac{\frac13 (\frac13 - 1)}{2} \cdot \dfrac{x^6}{64} + \ldots \right)\\
                &= \dfrac2x - \dfrac{x^2}{12} - \dfrac{x^5}{288} + \ldots
            \end{align*}

        \part
            Evaluating the above equation at $x = \dfrac23$,
            \begin{alignat*}{2}
                && \left(\dfrac8{\left(\frac23\right)^3} - 1\right)^{\tfrac13} &= \dfrac2{\tfrac23} - \dfrac{\left(\frac23\right)^2}{12} - \dfrac{\left(\frac23\right)^5}{288} + \ldots\\
                \implies && \sqrt[3]{26} &= 3 - \dfrac1{27} - \dfrac1{2187}\\
                && &= \dfrac{6479}{2187}
            \end{alignat*}

            \boxt{
                $\sqrt[3]{26} = \dfrac{6479}{2187}$
            }

            Since $\abs{6} > 1$, the binomial expansion of $ \left(\dfrac8{x^3} - 1\right)^{\tfrac13}$ does not hold. Hence, $x = 6$ is not an appropriate choice.

    \problem{}
        Let $f(x) = e^x \sin x$.

        \begin{enumerate}
            \item Sketch the graph of $y = f(x)$ for $-3 \leq x \leq 3$.
            \item Find the series expansion of $f(x)$ in ascending powers of $x$, up to and including the term in $x^3$.
        \end{enumerate}

        \noindent Denote the answer to part (b) by $g(x)$.

        \begin{enumerate}
            \setcounter{enumi}{2}
            \item On the same diagram, sketch the graph of $y = f(x)$ and $y = g(x)$. Label the two graphs clearly.
            \item Find, for $-3 \leq x \leq 3$, the set of values of $x$ for which the value of $g(x)$ is within $\pm 0.5$ of the value of $f(x)$.
        \end{enumerate}

    \solution
        \part
            \begin{center}
                \begin{tikzpicture}[trim axis left, trim axis right]
                    \begin{axis}[
                        domain = -3:3,
                        samples = 101,
                        axis y line=middle,
                        axis x line=middle,
                        xtick = \empty,
                        ytick = \empty,
                        xlabel = {$x$},
                        ylabel = {$y$},
                        legend cell align={left},
                        legend pos=outer north east,
                        after end axis/.code={
                            \path (axis cs:0,0) 
                                node [anchor=north west] {$O$};
                            }
                        ]
                        \addplot[plotRed] {e^x * sin(\x r)};
            
                        \addlegendentry{$y = e^x \sin x$};
                    \end{axis}
                \end{tikzpicture}
            \end{center}

        \part
            \begin{alignat*}{2}
                && f(x) &= e^x \sin x\\
                && &= e^x \Im{e^{ix}}\\
                && &= \Im{e^x e^{ix}}\\
                && &= \Im{e^{x(1 + i)}}\\
                \implies&&f^{(n)}(x) &= \Im{\derm{}{x}n e^{x(1 + i)}}\\
                && &= \Im{(1+i)^n e^{x(1 + i)}}\\
                && &= \Im{\left(\sqrt2 e^{i\tfrac{\pi}4}\right)^n e^{x(1 + i)}}\\
                && &= \Im{2^{\tfrac{n}2} e^x e^{i\tfrac{\pi}4 n} e^{ix}}\\
                && &= 2^{\tfrac{n}2} e^x \Im{e^{i\left(\tfrac{\pi}4 n + x\right)}}\\
                && &= 2^{\tfrac{n}2} e^x \sin{\left(\tfrac{\pi}4 n + x\right)}
            \end{alignat*}

            Evaluating $f^{(n)}(x)$ at $x = 0$,
            \begin{equation*}
                f^{(n)}(x) = 2^{\tfrac{n}2}\sin{\left(\tfrac{\pi}4 n\right)}
            \end{equation*}

            Hence,
            \begin{align*}
                f(0) &=  2^{\tfrac{0}2}\sin{\left(\tfrac{\pi}4 \cdot 0\right)} = 0\\
                f'(0) &=  2^{\tfrac{1}2}\sin{\left(\tfrac{\pi}4 \cdot 1\right)} = 1\\
                f''(0) &=  2^{\tfrac{2}2}\sin{\left(\tfrac{\pi}4 \cdot 2\right)} = 2\\
                f^{(3)}(0) &=  2^{\tfrac{3}2}\sin{\left(\tfrac{\pi}4 \cdot 2\right)} = 2
            \end{align*}

            Thus,
            \begin{align*}
                f(x) &= \sum_{n=0}^\infty \dfrac{f^{(n)}(0)}{n!} x^n \\
                &= \dfrac{f^{(0)}(0)}{0!} x^0 + \dfrac{f'(0)}{1!} x^1 + \dfrac{f''(0)}{2!} x^2 + \dfrac{f^{(3)}(0)}{3!} x^3 + \ldots\\
                &= x + x^2 + \dfrac13 x^3 + \ldots
            \end{align*}

            \boxt{
                $f(x) = x + x^2 + \dfrac13 x^3 + \ldots$
            }

        \part
            \begin{center}
                \begin{tikzpicture}[trim axis left, trim axis right]
                    \begin{axis}[
                        domain = -3:3,
                        samples = 101,
                        axis y line=middle,
                        axis x line=middle,
                        xtick = \empty,
                        ytick = \empty,
                        xlabel = {$x$},
                        ylabel = {$y$},
                        legend cell align={left},
                        legend pos=outer north east,
                        after end axis/.code={
                            \path (axis cs:0,0) 
                                node [anchor=north west] {$O$};
                            }
                        ]
                        \addplot[plotRed] {e^x * sin(\x r)};
            
                        \addlegendentry{$y = f(x)$};

                        \addplot[plotBlue] {x + x^2 + 1/3 * x^3};
            
                        \addlegendentry{$y = g(x)$};
                    \end{axis}
                \end{tikzpicture}
            \end{center}

        \part
            Consider $\abs{f(x) - g(x)} \leq 0.5$ for $-3 \leq x \leq 3$, where $g(x) = x + x^2 + \dfrac13 x^3$.

            \smallskip

            \case{1}{$f(x) - g(x) \leq 0.5$}
            \begin{alignat*}{2}
                &&f(x) - g(x) &\leq 0.5\\
                \implies&&e^x \sin x - (x + x^2 + \dfrac13 x^3) &\leq 0.5\\
                \implies&&x &\geq -1.96
            \end{alignat*}

            \case{2}{$-\left(f(x) - g(x)\right) \leq 0.5$}
            \begin{alignat*}{2}
                &&-\left(f(x) - g(x)\right) &\leq 0.5\\
                \implies&&g(x) - f(x) &\leq 0.5\\
                \implies&&x + x^2 + \dfrac13 x^3 - e^x \sin x &\leq 0.5\\
                \implies&&x &\leq 1.56
            \end{alignat*}

            Putting both inequalities together, we have

            \boxt{
                $-1.96 \leq x \leq 1.56$
            }

    \problem{}
        It is given that $y = \dfrac1{1 + \sin 2x}$. Show that, when $x = 0$, $\derm{y}{x}2 = 8$. Find the first three terms of the Maclaurin series for $y$.

        \begin{enumerate}
            \item Use the series to obtain an approximate value for $\int\limits_{-0.1}^{0.1} y \d x$, leaving your answer as a fraction in its lowest terms.
            \item Find the first two terms of the Maclaurin series for $\der{y}x$.
            \item Write down the equation of the tangent at the point where $x = 0$ on the curve $y = \dfrac1{1 + \sin 2x}$.
        \end{enumerate}

    \solution
        \begin{alignat}{2}
            && y &= \dfrac1{1 + \sin 2x} \label{P7-1} \\
            \implies&&y' &= -\dfrac1{(1+\sin 2x)^2} \cdot (\cos 2x \cdot 2)\nonumber\\
            && &= -2 y^2\cos 2x \label{P7-2}\\
            \implies&&y'' &= -2 \left(\cos 2x \cdot 2y\cdot y' + y^2 \cdot -\sin 2x \cdot 2\right)\nonumber\\
            && &= -4\left(y\cdot y' \cos 2x - y^2\sin 2x  \right)\label{P7-3}
        \end{alignat}
        
        From Equations~\ref{P7-1},~\ref{P7-2} and~\ref{P7-3},
        \begin{align*}
            y(0) &= 1\\
            y'(0) &= -2\\
            y''(0) &= 8
        \end{align*}

        Hence,
        \begin{align*}
            \dfrac1{1 + \sin 2x} &= \sum_{n = 0}^\infty \dfrac{y^{(n)}(0)}{n!} x^n\\
            &= \dfrac{y(0)}{0!}x^0 + \dfrac{y'(0)}{1!}x^1 + \dfrac{y''(0)}{2!}x^2 + \ldots\\
            &= 1 - 2x + 4x^2 + \ldots
        \end{align*}

        \part
            \begin{align*}
                \int_{-0.1}^{0.1} y \d x &\approx \int_{-0.1}^{0.1} \left(1 - 2x + 4x^2 \right) \d x\\
                &= \eval{x - 2 \cdot \dfrac12 x^2 + 4 \cdot \dfrac13 x^3}{-0.1}{0.1}\\
                &= \dfrac{76}{275}
            \end{align*}

            \boxt{
                $\int_{-0.1}^{0.1} y \d x \approx \dfrac{76}{275}$
            }

        \part
            \begin{align*}
                y' &= \der{}{x} y\\
                &= \der{}{x} \left(1 - 2x + 4x^2 + \ldots\right)\\
                &= -2 + 8x + \ldots
            \end{align*}

            \boxt{
                $y' = -2 + 8x + \ldots$
            }

        \part
            Using the point-slope formula,            
            \begin{alignat*}{2}
                &&y-1&=-2(x-0)\\
                \implies&&y &= -2x + 1
            \end{alignat*}

            \boxt{
                $y = -2x + 1$
            }

    \problem{}
        It is given that $y = e^{\arcsin 2x}$.
        
        
        \begin{enumerate}
            \item Show that $(1-4x^2)\derm{y}{x}2 - 4x\der{y}x=4y$.
            \item By further differentiating this result, find the Maclaurin series for $y$ in ascending powers of $x$, up to an including the term in $x^3$.
            \item Hence, find an approximation value of $e^{\tfrac{\pi}2}$, by substituting a suitable value of $x$ in the Maclaurin series for $y$.
            \item Suggest one way to improve the accuracy of the approximated value obtained.
        \end{enumerate}

    \solution
        \part
            \begin{alignat}{2}
                && y &= e^{\arcsin 2x}\label{P8-1}\\
                \implies&&\ln y &= \arcsin 2x\label{P8-2}
            \end{alignat}

            Implicitly differentiating Equation~\ref{P8-2},
            \begin{alignat}{2}
                &&\dfrac{y'}{y} &= \dfrac1{\sqrt{1 - (2x)^2}} \cdot 2\nonumber\\
                && &= \dfrac2{\sqrt{1 - 4x^2}}\nonumber\\
                \implies&&y'\sqrt{1-4x^2} &= 2y\label{P8-3}
            \end{alignat}

            Implicitly differentiating Equation~\ref{P8-3},
            \begin{alignat}{2}
                &&y''\sqrt{1-4x^2} + y'\dfrac{1}{2\sqrt{1 - 4x^2}} \cdot -8x &= 2y'\nonumber\\
                \implies&& \left(1 -4x^2\right)y''  - 4xy' &= 2y' \sqrt{1-4x^2}\nonumber\\
                && &= 2 \left(\dfrac{2y}{\sqrt{1-4x^2}}\right) 
                \sqrt{1-4x^2}\nonumber\\
                && &= 4y\label{P8-4}
            \end{alignat}

        \part
            Implicitly differentiating Equation~\ref{P8-4},
            \begin{alignat}{2}
                && y^{(3)}(1 - 4x^2) + y'' \cdot -8x - 4(xy'' + y') &= 4y'\nonumber\\
                \implies&&y^{(3)}(1-4x^2) -12xy'' -8y' &= 0\label{P8-5}
            \end{alignat}
            
            From Equations~\ref{P8-1},~\ref{P8-3},~\ref{P8-4} and~\ref{P8-5},
            \begin{align*}
                y(0) &= 1\\
                y'(0) &= 2\\
                y''(0) &= 4\\
                y^{(3)}(0) &= 16\\
            \end{align*}

            Hence,
            \begin{align*}
                y &= \sum_{n = 0}^\infty \dfrac{y^{(n)}(0)}{n!}x^n\\
                &= \dfrac{y(0)}{1!}x^0 + \dfrac{y'(0)}{1!}x^1 + \dfrac{y''(0)}{2!}x^2 + \dfrac{y^{(3)}(0)}{3!}x^3 + \ldots\\
                &= 1 + 2x + 2x^2 + \dfrac83 x^3 + \ldots
            \end{align*}

            \boxt{
                $y = 1 + 2x + 2x^2 + \dfrac83 x^3 + \ldots$
            }

        \part
            Consider $y = e^{\tfrac{\pi}2} \implies \arcsin 2x = \dfrac{\pi}2 \implies x = \dfrac12\cdot \sin \dfrac{\pi}2 = \dfrac12$. Hence, substituting $x = \dfrac12$ in the Maclaurin series for $y$,
            \begin{align*}
                e^{\tfrac{\pi}2} &\approx 1 + 2\cdot\dfrac12 + 2\left(\dfrac12\right)^2 + \dfrac83 \left(\dfrac12\right)^3\\
                &= \dfrac{17}6
            \end{align*}

            \boxt{
                $e^{\tfrac{\pi}2} \approx \dfrac{17}6$
            }
            
        \part
            More terms of the Maclaurin series of $y$ could be considered.
        
    \problem{}
        The curve $y = f(x)$ passes through the point $(0, 1)$ and satisfies the equation $\der{y}{x} = \dfrac{6-2y}{\cos 2x}$.
        
        \begin{enumerate}
            \item Find the Maclaurin series of $f(x)$, up to and including the term in $x^3$.
            \item Using standard results given in the List of Formulae (MF27), express $\dfrac{1-\sin x}{\cos x}$ as a power series of $x$, up to and including the term in $x^3$.
            \item Using the two power series you have found, show to this degree of approximation, that $f(x)$ can be expressed as $a(\tan 2x - \sec 2x) + b$, where $a$ and $b$ are constants to be determined.
        \end{enumerate}

    \solution
        \part
            \begin{alignat}{2}
                &&y' &= \dfrac{6-2y}{\cos 2x}\nonumber\\
                \implies&&y' \cos 2x &= 6-2y\label{P9-1}
            \end{alignat}

            Implicitly differentiating Equation~\ref{P9-1},
            \begin{alignat}{2}
                &&-\sin 2x \cdot 2 \cdot y' + y'' \cos 2x &= -2 y'\nonumber\\
                \implies&&-2y'\sin 2x + y'' \cos 2x &= -2 y'\label{P9-2}
            \end{alignat}

            Implicitly differentiating Equation~\ref{P9-2},
            \begin{alignat}{2}
                &&-2\left(y''\sin 2x + y'\cos 2x \cdot 2 \right) + \left(y'' \cdot -\sin 2x \cdot 2 + y^{(3)} \cos 2x\right) &= -2y''\nonumber\\
                \implies&&-4 y' \cos 2x -3y''\sin 2x  + y^{(3)}\cos 2x &= -2 y''\label{P9-3}
            \end{alignat}

            Given that $y$ passes through the point $(0, 1)$, and from Equations~\ref{P9-1},~\ref{P9-2} and~\ref{P9-3},
            \begin{align*}
                y(0) &= 1\\
                y'(0) &= 4\\
                y''(0) &= -8\\
                y^{(3)}(0) &= 32
            \end{align*}

            Thus,
            \begin{align*}
                f(x) &= \sum_{n=0}^\infty \dfrac{y^{(n)}(0)}{n!} x^n\\
                &= \dfrac{y(0)}{0!} x^0 + \dfrac{y'(0)}{1!} x^1 + \dfrac{y''(0)}{2!} x^2 + \dfrac{y^{(3)}(0)}{3!} x^3 + \ldots \\
                &= 1 + 4x -4x^2 + \dfrac{16}3 x^3 + \ldots
            \end{align*}

            \boxt{
                $f(x) = 1 + 4x -4x^2 + \dfrac{16}3 x^3 + \ldots$
            }

        \part
            Observe that
            \begin{equation*}
                \dfrac{1-\sin x}{\cos x} = \sec x - \tan x
            \end{equation*}

            Since $\sec x$ is even, $\sec x$ only contributes even powers of $x$ to the power series expansion of $\dfrac{1-\sin x}{\cos x}$. Likewise, since $\tan x$ is odd, $\tan x$ only contributes odd powers of $x$ to the power series expansion of $\dfrac{1-\sin x}{\cos x}$.

            Let $f(x) = \sec x$ and $g(x) = \tan x$.
            
            \begin{alignat*}{2}
                &&f(x) &= \sec x\\
                \implies&&f'(x) &= \ln (\sec x + \tan x)\\
                && &= \ln (f(x) + g(x))\\
                \implies&&f''(x) &= \dfrac{f'(x) + g'(x)}{f(x) + g(x)}
            \end{alignat*}

            \begin{alignat*}{2}
                &&g(x) &= \tan x\\
                \implies&&g'(x) &= \sec^2(x)\\
                && &= f^2(x)\\
                \implies&&g''(x) &= 2f(x)f'(x)\\
                \implies&&g^{(3)}(x) &= 2f(x)f''(x) + 2\left(f'(x)\right)^2
            \end{alignat*}

            Evaluating the above derivatives at $x = 0$, we have
            \begin{alignat*}{3}
                f(0) &= 1, \qquad &g(0) &&= 0\\
                f'(0) &= 0, \qquad &g'(0) &&= 1\\
                f''(0) &= 1, \qquad &g''(0) &&= 0\\
                & \qquad &g^{(3)}(0) &&= 2
            \end{alignat*}

            Thus,
            \begin{align*}
                \dfrac{1 - \sin x}{\cos x} &= \sec x - \tan x\\
                &= \sum_{n = 0}^\infty \dfrac{f^{(n)}(0)}{n!}x^n - \sum_{n=0}^\infty \dfrac{g^{(n)}(0)}{n!} x^n\\
                &= \left(1 + \dfrac12 x^2 + \ldots \right) - \left(x + \dfrac13 x^3 + \ldots\right)\\
                &= 1 - x + \dfrac12 x^2 - \dfrac13 x^3 + \ldots
            \end{align*}

            \boxt{
                $\dfrac{1 - \sin x}{\cos x} = 1 - x + \dfrac12 x^2 - \dfrac13 x^3 + \ldots$
            }

        \part
            \begin{align*}
                a(\tan 2x - \sec 2x) + b &= -a(\sec 2x - \tan 2x) + b\\
                &= -a \left( 1 - 2x + \dfrac12 (2x)^2 - \dfrac13 (2x)^3 + \ldots\right) + b\\
                &\approx -a \left( 1 - 2x + \dfrac12 (2x)^2 - \dfrac13 (2x)^3\right) + b\\
                &= -a \left( 1 - 2x + 2x^2 - \dfrac83 x^3\right) + b\\
                &= a \left(-1 + 2x - 2x^2 + \dfrac83 x^3\right) + b\\
                &= a \left(-1 + \dfrac12 \left(f(x) - 1\right)\right) + b\\
                &= -\dfrac32 a + b + \dfrac{a}2 f(x)
            \end{align*}

            Hence,
            \begin{equation*}
                \dfrac{a}2 f(x) - \dfrac32 a + b \approx a(\tan 2x - \sec 2x) + b
            \end{equation*}
        
            In order to obtain an approximation for $f(x)$, we need $\dfrac{a}2 = 1$ and $-\dfrac32 a + b = 0$, whence $a = 2$ and $b = 3$.

            \boxt{
                $a = 2$, $b = 3$
            }

    \problem{}
        Given that $x$ is sufficiently small for $x^3$ and higher powers of $x$ to be neglected, and that $13 - 59\sin x = 10(2 - \cos 2x)$, find a quadratic equation for $x$ and hence solve for $x$.

    \solution
        \begin{alignat*}{2}
            &&13-59\sin x &= 10\left(2 - \cos 2x\right)\\
            && &= 10\left(2 - \left(1-2\sin^2 x\right)\right)\\
            && &= 10 \left(1 + 2\sin^2 x\right)\\
            && &= 10 + 20\sin^2 x \\
            \implies&&20\sin^2 x + 59 \sin x - 3 &= 0\\
            \implies&&(20 \sin x - 1)(\sin x + 3) &= 0
        \end{alignat*}

        Hence, $\sin x = \dfrac1{20}$. Note that we reject $\sin x = -3$ since $\abs{\sin x} \leq 1$. Since $x$ is sufficiently small for $x^3$ and higher powers of $x$ to be neglected, $\sin x \approx x$. Thus, $x \approx \dfrac1{20}$.
        
        \boxt{
            $x \approx \dfrac1{20}$
        }
    \problem{}
        In triangle $ABC$, angle $A = \dfrac{\pi}3$ radians, angle $B = \left(\dfrac{\pi}3 + x\right)$ radians and angle $C = \left(\dfrac{\pi}3 - x\right)$ radians, where $x$ is small. The lengths of the sides $BC$, $CA$ and $AB$ are denoted by $a$, $b$ and $c$ respectively. Show that $b-c \approx \dfrac{2ax}{\sqrt{3}}$.

    \solution
        By the sine rule,
        \begin{equation*}
            \dfrac{a}{\sin A} = \dfrac{b}{\sin B} = \dfrac{c}{\sin C}
        \end{equation*}

        Hence,
        \begin{align*}
            b &= a \cdot \dfrac{\sin B}{\sin A} = a \cdot \dfrac{\sin B}{\sqrt3 / 2} = \dfrac{2a}{\sqrt3}\sin B\\
            c &= a \cdot \dfrac{\sin C}{\sin A} = a \cdot \dfrac{\sin C}{\sqrt3 / 2} = \dfrac{2a}{\sqrt3}\sin C\\
        \end{align*}

        This gives
        \begin{align*}
            b-c &= \dfrac{2a}{\sqrt3} \left(\sin B - \sin C\right)\\
            &= \dfrac{2a}{\sqrt3} \left(\sin\left(\dfrac{\pi}3 + x\right) - \sin\left(\dfrac{\pi}3 - x\right)\right)\\
            &= \dfrac{2a}{\sqrt3} \left(\left(\sin\dfrac{\pi}3\cos x + \cos\dfrac{\pi}3 \sin x \right)- \left(\sin\dfrac{\pi}3\cos x - \cos \dfrac{\pi}3 \sin x \right)\right)\\
            &= \dfrac{2a}{\sqrt3} \cdot 2\cos\dfrac{\pi}3 \sin x\\
            &= \dfrac{2a}{\sqrt3} \cdot 2\cdot\dfrac12 \sin x\\
            &= \dfrac{2a}{\sqrt3} \sin x
        \end{align*}

        Since $x$ is small, $\sin x \approx x$. Hence, $b-c \approx \dfrac{2ax}{\sqrt{3}}$.

    \problem{}
        D'Alembert's ratio test states that a series of the form $\sum\limits_{r = 0}^\infty a_r$ converges when $\lim\limits_{n \to \infty} \abs{\dfrac{a_{n+1}}{a_n}} < 1$, and diverges when $\lim\limits_{n \to \infty} \abs{\dfrac{a_{n+1}}{a_n}} > 1$. When $\lim\limits_{n \to \infty} \abs{\dfrac{a_{n+1}}{a_n}} = 1$, the test is inconclusive. Using the test, explain why the series $\sum\limits_{r=0}^\infty \dfrac{x^r}{r!}$ converges for all real values of $x$ and state the sum to infinity of this series, in terms of $x$.

    \solution
        Let $a_n = \dfrac{x^n}{n!}$ and consider $\lim\limits_{n \to \infty} \abs{\dfrac{a_{n+1}}{n}}$.

        \begin{align*}
            \lim_{n \to \infty} \abs{\dfrac{a_{n+1}}{n}} &= \lim_{n \to \infty} \abs{\dfrac{x^{n+1}}{(n+1)!} \Big/ \dfrac{x^n}{n!}}\\
            &= \lim_{n \to \infty} \abs{\dfrac{x^{n+1}}{x^n} \cdot \dfrac{n!}{(n+1)!}}\\
            &= \lim_{n \to \infty} \abs{\dfrac{x}{n+1}}\\
            &= 0
        \end{align*}

        Since $\lim\limits_{n \to \infty} \abs{\dfrac{a_{n+1}}{n}} < 1$ for all $x \in \mathbb{R}$, it follows by D'Alembert's ratio test that $\sum\limits_{r=0}^\infty \dfrac{x^r}{r!}$ converges for all real values of $x$. The sum to infinity of the series in question is $e^x$.
\end{document}