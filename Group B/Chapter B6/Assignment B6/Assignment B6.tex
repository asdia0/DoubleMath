\documentclass{jhwhw}

\title{Assignment B6\\Maclaurin Series}
\author{Eytan Chong}
\date{2024-04-29}

\begin{document}
    \problem{}
        Expand $(1 + 2x)^{-\tfrac13}$, where $\abs{x} < \dfrac12$, as a series of ascending powers of $x$, up to an including the term in $x^2$, simplifying the coefficients.

        By choosing $x = \dfrac1{14}$, find an approximate value of $\sqrt[3]{7}$ in the form $\dfrac{p}{q}$, where $p$ and $q$ are to be determined.

        Using your calculator, calculate the numerical value of $\sqrt[3]{7}$. Compare this value to the approximate value found, and with reference to the value of $x$ chosen, comment on the accuracy of your approximation.

    \solution
        \begin{align*}
            (1 + 2x)^{-\tfrac13} &= 1 + -\dfrac13 \cdot 2x + \dfrac{-\tfrac13 (-\tfrac13 - 1)}2 \cdot (2x)^2 + \ldots\\
            &= 1 - \dfrac23 x + \dfrac89 x^2 + \ldots
        \end{align*}

        Substituting $x = \dfrac1{14}$,
        \begin{alignat*}{2}
            &&(1 + 2 \cdot \dfrac1{14})^{-\tfrac13} &= 1 - \dfrac23 \cdot \dfrac1{14} + \dfrac89 \left(\dfrac1{14}\right)^2 + \ldots\\
            \implies&& \left(\dfrac87\right)^{-\tfrac13} &\approx \dfrac{422}{441}\\
            \implies&& \left(\dfrac78\right)^{\tfrac13} &\approx \dfrac{422}{441}\\
            \implies&& \left(7\right)^{\tfrac13} \cdot \dfrac12 &\approx \dfrac{422}{441}\\
            \implies&& \sqrt[3]7 &\approx \dfrac{844}{441}\\
            && &= 1.9138 \tosf{5}
        \end{alignat*}

        Since $\sqrt[3]7 = 1.9129 \tosf{5}$, the approximation is accurate.

    \problem{}
        In the triangle $ABC$, $AB = 1$, $BC = 3$ and angle $ABC = \theta$ radians. Given that $\theta$ is a sufficiently small angle, show that
        \begin{equation*}
            AC \approx (4 + 3\theta^2)^{\tfrac12} \approx a + b\theta^2
        \end{equation*}
        \noindent for constants $a$ and $b$ to be determined.

    \solution
        By cosine rule,
        \begin{alignat*}{2}
            &&AC^2 &= AB^2 + BC^2 - 2\cdot AB \cdot BC \cdot \cos ABC\\
            \implies&&AC^2 &= 1^2 + 3^2 - 2 \cdot 1 \cdot 3 \cdot \cos \theta\\
            && &= 10 - 6 \cos\theta
        \end{alignat*}
        Since $\theta$ is sufficiently small, $\cos \theta \approx 1 - \dfrac{\theta^2}{2}$. Hence,
        \begin{alignat*}{2}
            &&AC^2 &\approx 10 - 6\left(1 - \dfrac{\theta^2}{2}\right)\\
            && &= 4 + 3\theta^2\\
            \implies&&AC &\approx (4+3\theta^2)^{\tfrac12}\\
            && &= 2\left(1 + \dfrac34 \theta^2\right)^{\tfrac12}\\
            && &\approx 2 \left(1 + \dfrac12 \cdot \dfrac34 \theta^2\right)\\
            && &= 2 + \dfrac34 \theta^2
        \end{alignat*}

    \problem{}
        Given that $y = \ln \sec x$, show that

        \begin{enumerate}
            \item $\derm{y}{x}{3} = 2 \, \derm{y}{x}{2} \, \der{y}{x}$
            \item the value of $\derm{y}{x}{4}$ when $x = 0$ is 2.
        \end{enumerate}

        \noindent Write down the Maclaurin series for $\ln \sec x$ up to and including the term in $x^4$. By substituting $x = \dfrac\pi4$, show that $\ln 2 \approx \dfrac{\pi^2}{16} + \dfrac{\pi^4}{1536}$.

        \solution
            \part
                Note that $y = \ln \sec x = -\ln \cos x$. Hence,

                \begin{equation}
                    e^{-y} = \cos x \label{P3-1}
                \end{equation}

                Implicitly differentiating Equation~\ref{P3-1},
                \begin{alignat}{2}
                    &&e^{-y}\cdot (-y^\prime) &= -\sin x\notag\\
                    \implies&&y^\prime e^{-y} &= \sin x\label{P3-2}
                \end{alignat}

                Implicitly differentiating Equation~\ref{P3-2},
                \begin{alignat}{2}
                    &&y^\prime e^{-y}\cdot (-y^\prime) + e^{-y} \cdot y^{\prime\prime} &= \cos x\notag\\
                    \implies&&y^\prime e^{-y}\cdot (-y^\prime) + e^{-y} \cdot y^{\prime\prime} &= e^{-y}\notag\\
                    \implies&&y^{\prime\prime} - (y^\prime)^2 &= 0\label{P3-3}
                \end{alignat}

                Implicitly differentiating Equation~\ref{P3-3},
                \begin{alignat}{2}
                    &&y^{(3)} - 2\cdot y^\prime \cdot y^{\prime\prime} &= 0\notag\\
                    \implies&&y^{(3)} &= 2 y^{\prime\prime} y^\prime\label{P3-4}
                \end{alignat}

            \part
                Implicitly differentiating Equation~\ref{P3-4},
                \begin{equation}
                    y^{(4)} = 2\left(y^{(3)}y^\prime + (y^{\prime\prime})^2\right)\label{P3-5}
                \end{equation}

                Substituting $x = 0$ into Equations~\ref{P3-1},~\ref{P3-2},~\ref{P3-3},~\ref{P3-4} and~\ref{P3-5}, we see that
                \begin{align*}
                    y &= 0\\
                    y^\prime &= 0\\
                    y^{\prime\prime} &= 1\\
                    y^{(3)} &= 0\\
                    y^{(4)} &= 2
                \end{align*}
                Thus, $\left.\derm{y}{x}{4}\right|_{x=0} = 2$.
                \begin{align*}
                    \ln \sec x &= \sum_{n=0}^\infty \dfrac{y^{(n)}}{n!}x^n\\
                    &= \dfrac12 x^2 + \dfrac1{12} x^4 + \ldots 
                \end{align*}

                Substituting $x = \dfrac\pi4$,
                \begin{alignat*}{2}
                    &&\ln \sec \dfrac\pi4 &= \dfrac12 \left(\dfrac\pi4\right)^2 + \dfrac1{12} \left(\dfrac\pi4\right)^4 + \ldots\\
                    \implies&&\ln \sqrt2 &= \dfrac{\pi^2}{32} + \dfrac{\pi^2}{3072} + \ldots\\
                    \implies&&\dfrac12 \ln 2 &= \dfrac{\pi^2}{32} + \dfrac{\pi^2}{3072} + \ldots\\
                    \implies&&\ln 2 &= \dfrac{\pi^2}{16} + \dfrac{\pi^2}{1536} + \ldots\\
                    \implies&&\ln 2 &\approx \dfrac{\pi^2}{16} + \dfrac{\pi^2}{1536}
                \end{alignat*}
\end{document}