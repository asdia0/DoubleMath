\documentclass{echw}

\title{Assignment B6\\MacLaurin Series}
\author{Eytan Chong}
\date{2024-04-29}

\begin{document}
    \problem{}
        Expand $(1 + 2x)^{-\tfrac13}$, where $\abs{x} < \dfrac12$, as a series of ascending powers of $x$, up to an including the term in $x^2$, simplifying the coefficients.

        By choosing $x = \dfrac1{14}$, find an approximate value of $\sqrt[3]{7}$ in the form $\dfrac{p}{q}$, where $p$ and $q$ are to be determined.

        Using your calculator, calculate the numerical value of $\sqrt[3]{7}$. Compare this value to the approximate value found, and with reference to the value of $x$ chosen, comment on the accuracy of your approximation.

    \solution
        \begin{align*}
            (1 + 2x)^{-1/3} &= 1 + -\dfrac13 \cdot 2x + \dfrac{1/3 \cdot (-1/3 - 1)}2 \cdot (2x)^2 + \ldots\\
            &= 1 - \dfrac23 x + \dfrac89 x^2 + \ldots
        \end{align*}
        Substituting $x = \dfrac1{14}$,
        \begin{alignat*}{2}
            &&(1 + 2 \cdot \dfrac1{14})^{-1/3} &= 1 - \dfrac23 \cdot \dfrac1{14} + \dfrac89 \bp{\dfrac1{14}}^2 + \ldots\\
            \implies&& \bp{\dfrac87}^{-1/3} &\approx \dfrac{422}{441}\\
            \implies&& \bp{\dfrac78}^{1/3} &\approx \dfrac{422}{441}\\
            \implies&& \bp{7}^{1/3} \cdot \dfrac12 &\approx \dfrac{422}{441}\\
            \implies&& \sqrt[3]7 &\approx \dfrac{844}{441}\\
            && &= 1.9138 \tosf{5}
        \end{alignat*}
        Since $\sqrt[3]7 = 1.9129 \tosf{5}$, the approximation is accurate.

    \problem{}
        In the triangle $ABC$, $AB = 1$, $BC = 3$ and angle $ABC = \t$ radians. Given that $\t$ is a sufficiently small angle, show that
        \[
            AC \approx (4 + 3\t^2)^{\tfrac12} \approx a + b\t^2
        \]
        for constants $a$ and $b$ to be determined.

    \solution
        By cosine rule,
        \begin{alignat*}{2}
            &&AC^2 &= AB^2 + BC^2 - 2\cdot AB \cdot BC \cdot \cos ABC\\
            \implies&&AC^2 &= 1^2 + 3^2 - 2 \cdot 1 \cdot 3 \cdot \cos \t\\
            && &= 10 - 6 \cos\t
        \end{alignat*}
        Since $\t$ is sufficiently small, $\cos \t \approx 1 - \dfrac{\t^2}{2}$. Hence,
        \begin{alignat*}{2}
            &&AC^2 &\approx 10 - 6\bp{1 - \dfrac{\t^2}{2}}\\
            && &= 4 + 3\t^2\\
            \implies&&AC &\approx (4+3\t^2)^{1/2}\\
            && &= 2\bp{1 + \dfrac34 \t^2}^{1/2}\\
            && &\approx 2 \bp{1 + \dfrac12 \cdot \dfrac34 \t^2}\\
            && &= 2 + \dfrac34 \t^2
        \end{alignat*}

    \problem{}
        Given that $y = \ln \sec x$, show that

        \begin{enumerate}
            \item $\der[3]{y}{x} = 2 \, \der[2]{y}{x} \, \der{y}{x}$
            \item the value of $\der[4]{y}{x}$ when $x = 0$ is 2.
        \end{enumerate}

         Write down the MacLaurin series for $\ln \sec x$ up to and including the term in $x^4$. By substituting $x = \dfrac\pi4$, show that $\ln 2 \approx \dfrac{\pi^2}{16} + \dfrac{\pi^4}{1536}$.

    \solution
        \part
            Note that $y = \ln \sec x = -\ln \cos x$. Hence,
            \begin{equation}\label{P3-1}
                e^{-y} = \cos x 
            \end{equation}
            Implicitly differentiating Equation~\ref{P3-1},
            \begin{alignat}{2}
                &&e^{-y}\cdot (-y') &= -\sin x\notag\\
                \implies&&y' e^{-y} &= \sin x\label{P3-2}
            \end{alignat}
            Implicitly differentiating Equation~\ref{P3-2},
            \begin{alignat}{2}
                &&y' e^{-y}\cdot (-y') + e^{-y} \cdot y'' &= \cos x\notag\\
                \implies&&y' e^{-y}\cdot (-y') + e^{-y} \cdot y'' &= e^{-y}\notag\\
                \implies&&y'' - (y')^2 &= 0\label{P3-3}
            \end{alignat}
            Implicitly differentiating Equation~\ref{P3-3},
            \begin{alignat}{2}
                &&y^{(3)} - 2\cdot y' \cdot y'' &= 0\notag\\
                \implies&&y^{(3)} &= 2 y'' y'\label{P3-4}
            \end{alignat}

        \part
            Implicitly differentiating Equation~\ref{P3-4},
            \begin{equation}
                y^{(4)} = 2\bp{y^{(3)}y' + (y'')^2}\label{P3-5}
            \end{equation}
            Substituting $x = 0$ into Equations~\ref{P3-1},~\ref{P3-2},~\ref{P3-3},~\ref{P3-4} and~\ref{P3-5}, we see that
            \begin{align*}
                y &= 0\\
                y' &= 0\\
                y'' &= 1\\
                y^{(3)} &= 0\\
                y^{(4)} &= 2
            \end{align*}
            Thus, $\evalder{\der[4]{y}{x}}{x=0} = 2$.
            \begin{align*}
                \ln \sec x &= \sum_{n=0}^\infty \dfrac{y^{(n)}}{n!}x^n\\
                &= \dfrac12 x^2 + \dfrac1{12} x^4 + \ldots 
            \end{align*}
            Substituting $x = \dfrac\pi4$,
            \begin{alignat*}{2}
                &&\ln \sec \dfrac\pi4 &= \dfrac12 \bp{\dfrac\pi4}^2 + \dfrac1{12} \bp{\dfrac\pi4}^4 + \ldots\\
                \implies&&\ln \sqrt2 &= \dfrac{\pi^2}{32} + \dfrac{\pi^2}{3072} + \ldots\\
                \implies&&\dfrac12 \ln 2 &= \dfrac{\pi^2}{32} + \dfrac{\pi^2}{3072} + \ldots\\
                \implies&&\ln 2 &= \dfrac{\pi^2}{16} + \dfrac{\pi^2}{1536} + \ldots\\
                \implies&&\ln 2 &\approx \dfrac{\pi^2}{16} + \dfrac{\pi^2}{1536}
            \end{alignat*}
\end{document}