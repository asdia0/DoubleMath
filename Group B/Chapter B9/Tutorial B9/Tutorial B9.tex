\documentclass{echw}

\title{Tutorial B9\\Applications of Integration II}
\author{Eytan Chong}
\date{2024-06-28}

\begin{document}
    \problem{}
        Calculate the exact length of each of the arcs of the following curves.

        \begin{enumerate}
            \item $y^3 = x^2$ for $-1 \leq x \leq 1$.
            \item $x = t^2 - 1, \, y = t^3 + 1$ from $t = 0$ to $t = 1$.
            \item $r = a\cos\t$ from $\t = 0$ to $\t = \pi/2$.
        \end{enumerate}

    \solution
        \part
            Note that $y^3 = x^2 \implies y = x^{2/3} \implies \der{y}{x} = \dfrac23 x^{-1/3}$.
            \begin{align*}
                \length &= \int_{-1}^1 \sqrt{1 + \bp{\der{y}{x}}^2} \d x\\
                &= \int_{-1}^1 \sqrt{1 + \bp{\dfrac23 x^{-1/3}}^2} \d x\\
                &= \int_{-1}^1 \sqrt{1 + \dfrac49 x^{-2/3}} \d x\\
                &= 2 \int_0^1 \sqrt{1 + \dfrac49 x^{-2/3}} \d x\\
                &= 2 \int_0^1 \sqrt{x^{-2/3} \bp{x^{2/3} + \dfrac49}} \d x\\
                &= 2 \int_0^1 x^{-1/3} \sqrt{x^{2/3} + \dfrac49} \d x\usub{u &= x^{2/3} + \tfrac49\\\d u &= \tfrac23 x^{-1/3} \d x}\\
                &= 2 \int_{4/9}^{13/9} \sqrt{u} \cdot \dfrac32 \d u\\
                &= 3 \evalint{\dfrac23 u^{3/2}}{4/9}{13/9}\\
                &= \dfrac2{27} \bp{13\sqrt{13} - 8}
            \end{align*}

            \boxt{The arc length of the curve is $\dfrac2{27} \bp{13\sqrt{13} - 8}$ units.}

        \part
            Since the arc length of a curve is invariant under translation, it suffices to find the arc length of the curve with parametric equations $x = t^2, y = t^3$, $0 \leq t \leq 1$. The Cartesian equation of this curve is $y = x^{3/2}$, $0 \leq x \leq 1$, which is the inverse of $y = x^{2/3}$, $0 \leq x \leq 1$. From part (a), the required arc length is $\dfrac12 \cdot \dfrac2{27} \bp{13\sqrt{13} - 8} = \dfrac1{27} \bp{13 \sqrt{13} - 8}$.

            \boxt{The arc length of the curve is $\dfrac1{27} \bp{13 \sqrt{13} - 8}$ units.}

        \part
            Since $r = a\cos\t$, $0 \leq \t \leq \dfrac\pi2$ describes the top half of a circle with centre $\bp{\dfrac{a}2, 0}$ and diameter $a$, the arc length of the curve is $\dfrac12 \cdot \pi a = \dfrac\pi2 a$.

            \boxt{The arc length of the curve is $\dfrac\pi2 a$ units.}

    \problem{}
        Find the exact areas of the surfaces generated by completely rotating the following arcs about the (i) $x$-axis and (ii) $y$-axis.

        \begin{enumerate}
            \item The line $2y = x$ between the origin and the point $(4, 2)$.
            \item The curve $x = t^3 - 3t + 2, \, y = 3\bp{t^2 - 1}, \, t \in \R$ from $t = 1$ to $t = 2$.
        \end{enumerate}

    \solution
        \part
            \subpart

                When rotated about the $x$-axis, the curve forms a cone with slant height $\sqrt{4^2 + 2^2} = 2\sqrt5$ and radius $2$. Hence, the required surface area is $\pi \cdot 2 \cdot 2\sqrt5 = 4\sqrt5 \pi$.

                \boxt{The surface area is $4\sqrt5 \pi$ units$^2$.}

            \subpart

                When rotated about the $y$-axis, the curve forms a cone with slant height $\sqrt{4^2 + 2^2} = 2\sqrt5$ and radius $4$. Hence, the required surface area is $\pi \cdot 4 \cdot 2\sqrt5 = 8\sqrt5 \pi$.

                \boxt{The surface area is $8\sqrt5 \pi$ units$^2$.}

        \part
            Note that $x = t^3 -3t + 2 \implies \der{x}{t} = 3t^2 - 3$ and $y = 3\bp{t^2 - 1} \implies \der{y}{t} = 6t$.

            \subpart
                \begin{align*}
                    \area &= 2\pi \int_1^2 y \, \sqrt{\bp{\der{x}{t}}^2 + \bp{\der{y}{t}}^2} \d t\\
                    &= 2\pi \int_1^2 3\bp{t^2 - 1} \sqrt{\bp{3t^2 - 3}^2 + (6t)^2} \d t\\
                    &= 6\pi \int_1^2 \bp{t^2 - 1} \sqrt{9t^4 + 18t^2 + 9} \d t\\
                    &= 6\pi \int_1^2 \bp{t^2 - 1} \sqrt{\bp{3t^2 + 3}^2} \d t\\
                    &= 18\pi \int_1^2 \bp{t^2-1}\bp{t^2 + 1} \d t\\
                    &= 18\pi \int_1^2 \bp{t^4 - 1} \d t\\
                    &= 18\pi \evalint{\dfrac15 t^5 - t}12\\
                    &= \dfrac{468}5 \pi
                \end{align*}

                \boxt{The surface area is $\dfrac{468}5 \pi$ units$^2$.}
                

            \subpart
                \begin{align*}
                    \area &= 2\pi \int_1^2 x \, \sqrt{\bp{\der{x}{t}}^2 + \bp{\der{y}{t}}^2} \d t\\
                    &= 2\pi \int_1^2 \bp{t^3 - 3t + 2} \sqrt{\bp{3t^2 - 3}^2 + (6t)^2} \d t\\
                    &= 6\pi \int_1^2 \bp{t^3 - 3t + 2} \bp{t^2 + 1} \d t\\
                    &= 6\pi \int_1^2 \bp{t^5 - 2t^3 + 2t^2 - 3t + 2} \d t\\
                    &= 6\pi \evalint{\dfrac16 t^6 - \dfrac24 t^4 - \dfrac23 t^3 - \dfrac32 t^2 + 2t}12\\
                    &= 31 \pi
                \end{align*}

                \boxt{The surface area is $31\pi$ units$^2$.}

    \problem{}
        The section of the curve $y = e^x$ between $x = 0$ and $x = 1$ is rotated through one revolution about
        \begin{enumerate}
            \item the $x$-axis.
            \item the $y$-axis.
        \end{enumerate}
        Find the numerical values of the areas of the surfaces obtained.
    
    \solution
        \part
            \begin{align*}
                \area &= 2\pi \int_0^1 y \, \sqrt{1 + \bp{\der{y}{x}}^2} \d x\\
                &= 2\pi \int_0^1 e^x \sqrt{1 + e^{2x}} \d x\\
                &= 22.9 \tosf{3}
            \end{align*}

            \boxt{The surface area is $22.9$ units.}

            \part
                Note that $y = e^x \implies x = \ln y$ and $\der{y}{x} = e^x \implies \der{x}{y} = e^{-x}$.
                \begin{align*}
                    \area &= 2\pi \int_1^e x \, \sqrt{1 + \bp{\der{x}{y}}^2} \d y\\
                    &= 2\pi \int_0^1 \ln y \, \sqrt{1 + e^{-2x}} \d x\\
                    &= 7.05 \tosf{3}
                \end{align*}

                \boxt{The surface area is $7.05$ units.}

    \problem{}
        The curve $y^2 = \dfrac13 x(1-x)^2$ has a loop between $x = 0$ and $x = 1$. Prove that the total length of the loop is $\dfrac{4\sqrt{3}}{3}$.

    \solution
        Since the curve is even in $y$, it is symmetric about the $x$-axis. We thus only consider the part of the curve above the $x$-axis, i.e. $y \geq 0$, where $y = \sqrt{\dfrac13 x(1-x)^2}$. Differentiating,
        \begin{align*}
            \der{y}{x} &= \dfrac12 \bs{\dfrac13 x(1-x)^2}^{-1/2} \cdot \dfrac13 \bs{(1-x)^2 + x \cdot 2(1-x) \cdot -1}\\
            &= \dfrac16 \bs{\dfrac13 x(1-x)^2}^{-1/2} \bp{3x^2 - 4x + 1}\\
            &= \dfrac16 \bs{\dfrac13 x(1-x)^2}^{-1/2} (3x-1)(x-1)\\
            &= \dfrac{\sqrt{3}}6  x^{-1/2} (x-1)^{-1} (3x-1)(x-1)\\
            &= \dfrac{3x-1}{2\sqrt{3x}}
        \end{align*}
        Hence,
        \begin{align*}
            \length &= 2 \int_0^1 \sqrt{1 + \bp{\der{y}{x}}^2} \d x\\
            &= 2 \int_0^1 \sqrt{1 + \bp{\dfrac{3x-1}{2\sqrt{3x}}}^2} \d x\\
            &= 2 \int_0^1 \sqrt{1 + \dfrac{(3x-1)^2}{12x} } \d x\\
            &= \dfrac2{\sqrt{12}} \int_0^1 \sqrt{\dfrac{12 x + 9x^2 - 6x + 1}{x}} \d x\\
            &= \dfrac{\sqrt3}{3} \int_0^1 \sqrt{\dfrac{9x^2 + 6x + 1}{x}} \d x\\
            &= \dfrac{\sqrt3}{3} \int_0^1 \sqrt{\dfrac{(3x+1)^2}{x}} \d x\\
            &= \dfrac{\sqrt3}{3} \int_0^1 \dfrac{3x+1}{\sqrt{x}} \d x\\
            &= \dfrac{\sqrt3}{3} \evalint{\dfrac3{3/2} x^{3/2} + \dfrac1{1/2} x^{1/2}}01\\
            &= \dfrac{4\sqrt3}3
        \end{align*}

    \problem{}
        The tangent at a point $P$ on the curve $x = a\bp{t - \dfrac13 t^3}, \, y = at^2$ cuts the $x$-axis at $T$. Prove that the distance of the point $T$ from the origin $O$ is half the length of the arc $OP$.

    \solution
        Let $P$ be the point on the curve where $t = t_P$. Note that $x = a\bp{t - \dfrac13 t^3} \implies \der{x}{t} = a\bp{1 - t^2}$ and $y = at^2 \implies \der{y}{t} = 2at$.
        \begin{align*}
            \text{Length of arc } OP &= \int_0^{t_P} \sqrt{\bp{\der{x}{t}}^2 + \bp{\der{y}{t}}^2} \d t\\
            &= \int_0^{t_P} \sqrt{\bs{a\bp{1-t^2}}^2 + (2at)^2} \d t\\
            &= a \int_0^{t_P} \sqrt{t^4 + 2t^2 + 1} \d t\\
            &= a \int_0^{t_P} \sqrt{\bp{t^2 + 1}^2} \d t\\
            &= a \int_0^{t_P} \bp{t^2 + 1} \d t\\
            &= a \evalint{\dfrac13 t + t}0{t_P}\\
            &= a \bp{\dfrac13 t_P^3 + t_P}
        \end{align*}
        Note that $\der{y}{x} = \der{y}{t} \cdot \der{t}{x} = \dfrac{2at}{a\bp{1 - t^2}} = \dfrac{2t}{1-t^2}$. Hence, the equation of the tangent at $P$ is given by
        \[
            y - at_P^2 = \dfrac{2t_P}{1 - t_P^2} \bs{x - a\bp{t_P - \dfrac13 t_P^3}}
        \]
        At $T$, $x = OT$ and $y = 0$. Hence,
        \begin{alignat*}{2}
            && - at_P^2 &= \dfrac{2t_P}{1 - t_P^2} \bs{OT - a\bp{t_P - \dfrac13 t_P^3}}\\
            \implies && - at_P &= \dfrac{2}{1 - t_P^2} \bs{OT - a\bp{t_P - \dfrac13 t_P^3}}\\
            \implies && OT &= \dfrac{- at_P \bp{1 - t_P^2}}2 + a\bp{t_P - \dfrac13 t_P^3}\\
            && &= \dfrac12 a \bp{t_P^3 - t_P + 2t_P - \dfrac23 t_P^3}\\
            && &= \dfrac12 a \bp{\dfrac13 t_P^3 + t_P}\\
            && &= \dfrac12 \cdot \text{ length of arc } OP
        \end{alignat*}


    \problem{}
        Sketch the curve whose parametric equations are $x = a\cos^3 \t, \, y = a\sin^3 \t$, $a > 0$.

        \begin{enumerate}
            \item Find the total length of the curve.
            \item The portion of the curve in the first quadrant is revolved through four right angles about the $x$-axis. Prove that the area of the surface thus formed is $\dfrac65 \pi a^2$.
        \end{enumerate}

    \solution
        \begin{center}
            \begin{tikzpicture}[trim axis left, trim axis right]
                \begin{axis}[
                    domain = 0:10,
                    samples = 101,
                    axis y line=middle,
                    axis x line=middle,
                    xtick = {-1, 1},
                    ytick = {-1, 1},
                    xticklabels = {$-a$, $a$},
                    yticklabels = {$-a$, $a$},
                    xlabel = {$x$},
                    ylabel = {$y$},
                    xmin=-1.1,
                    xmax=1.1,
                    ymin=-1.1,
                    ymax=1.1,
                    legend cell align={left},
                    legend pos=outer north east,
                    after end axis/.code={
                        \path (axis cs:0,0) 
                            node [anchor=north east] {$O$};
                        }
                    ]
                    \addplot[plotRed] ({cos(\x r)^3}, {sin(\x r)^3});
        
                    \addlegendentry{$x = a\cos^3 \t, \, y = a\sin^3 \t$};
                \end{axis}
            \end{tikzpicture}
        \end{center}

        \part
            By symmetry, we only consider the length of the curve in the first quadrant. Note that $x = 0 \implies \t = \dfrac\pi2$ and $x = a \implies \t = 0$. Also, $x = a\cos^3 \t \implies \der{x}{\t} = -3a\cos^2 \t \sin \t$ and $y = a\sin^3 \t \implies \der{y}{\t} = 3a\sin^2 \t \cos \t$.
            \begin{align*}
                \length &= 4 \int_0^{\pi/2} \sqrt{\bp{\der{x}{\t}}^2 + \bp{\der{y}{\t}}^2} \d \t\\
                &= 4 \int_0^{\pi/2} \sqrt{(-3a\cos^2 \t \sin \t)^2 + (3a\sin^2 \t \cos \t)^2} \d \t\\
                &= 12a \int_0^{\pi/2} \sqrt{\cos^4 \t \sin^2 \t + \sin^4 \t \cos^2 \t} \d \t\\
                &= 12a \int_0^{\pi/2} \sqrt{\cos^2\t\sin^2\t(\cos^2 \t + \sin^2 \t)} \d \t\\
                &= 12a \int_0^{\pi/2} \cos\t\sin\t \d \t\\
                &= 12a \evalint{\dfrac{\sin^2 \t}2}0{\pi/2}\\
                &= 6a
            \end{align*}
            \boxt{The total length of the curve is $6a$ units.}

        \part
            \begin{align*}
                \area &= 2\pi \int_0^{\pi/2} x \, \sqrt{\bp{\der{x}{\t}}^2 + \bp{\der{y}{\t}}^2} \d \t\\
                &= 2\pi \int_0^{\pi/2} a \cos^3 \t \cdot 3a \cos \t \sin \t \d t\\
                &= 3\pi a^2 \cdot 2\int_0^{\pi/2} \sin \t \cos^4 \t \d \t\\
                &= 3\pi a^2 B(1, 5/2)\\
                &= 3\pi a^2 \cdot \dfrac{\G(1)\G(5/2)}{\G(1 + 5/2)}\\
                &= 3\pi a^2 \cdot \dfrac{\G(5/2)}{\G(5/2) \cdot 5/2}\\
                &= 3\pi a^2 \cdot \dfrac25\\
                &= \dfrac65 \pi a^2
            \end{align*}

    \problem{}
        The parametric equations of a curve are given by
        \[
            x = \tan t - t, \, y = \ln \sec t, \, t \in \bp{-\dfrac\pi2, \dfrac\pi2}
        \]

        \begin{enumerate}
            \item Sketch the curve.
            \item Prove that the arc length of the curve measured from the origin to the point $\bp{1 - \dfrac\pi4, \dfrac12 \ln 2}$ is $\sqrt2 - 1$.
            \item The arc in (b) is rotated about the $x$-axis through an angle of $360 \deg$. Find the exact surface area formed.
        \end{enumerate}

    \solution
        \part
            \begin{center}
                \begin{tikzpicture}[trim axis left, trim axis right]
                    \begin{axis}[
                        domain = -pi/2+0.01:pi/2-0.01,
                        samples = 1000,
                        axis y line=middle,
                        axis x line=middle,
                        xtick = \empty,
                        ytick = \empty,
                        xlabel = {$x$},
                        ylabel = {$y$},
                        legend cell align={left},
                        legend pos=outer north east,
                        after end axis/.code={
                            \path (axis cs:0,0) 
                                node [anchor=north] {$O$};
                            }
                        ]
                        \addplot[plotRed, unbounded coords=jump] ({tan(\x r) - x}, {ln(sec(\x r)});
            
                        \addlegendentry{$x = \tan t - t, \, y = \ln \sec t$};
                    \end{axis}
                \end{tikzpicture}
            \end{center}

        \part
            Note that $x = 0 \implies t = 0$ and $x = 1 - \dfrac\pi4 \implies = t = \dfrac\pi4$. Further, $x = \tan t - t \implies \der{x}{t} = \sec^2 t - 1 = \tan^2 t$ and $y = \ln \sec t \implies \der{y}{t} = \tan t$.
            {\allowdisplaybreaks
            \begin{align*}
                \length &= \int_0^{\pi/4} \sqrt{\bp{\der{x}{t}}^2 + \bp{\der{y}{t}}^2} \d t\\
                &= \int_0^{\pi/4} \sqrt{\bp{\tan^2 t}^2 + (\tan t)^2} \d t\\
                &= \int_0^{\pi/4} \tan t \sqrt{\tan^2 t + 1} \d t\\
                &= \int_0^{\pi/4} \tan t \sec t \d t\\
                &= \evalint{\sec t}{0}{\pi/4}\\
                &= \sqrt{2} - 1 
            \end{align*}}

        \part
            \begin{align*}
                \area &= 2\pi \int_0^{\pi/4} y \, \sqrt{\bp{\der{x}{t}}^2 + \bp{\der{y}{t}}^2} \d t\\
                &= 2\pi \int_0^{\pi/4} \ln \sec t \cdot \tan t \sec t \d t\\\\
                & \begin{array}{r c @{\hspace*{1.0cm}} c}\toprule
                    & D & I \\\cmidrule{1-3}
                    + & \ln \sec t & \tan t \sec t \\
                    - & \tan t & \sec t \\\bottomrule
                \end{array}\\\\
                &= 2\pi \bs{\evalint{\sec t \ln \sec t}{0}{\pi/4} - \int_0^{\pi/4} \tan t \sec t \d t}\\
                &= 2\pi \bs{\sqrt2 \cdot \dfrac12 \ln 2 - (\sqrt2 - 1)}\\
                &= \sqrt2 \pi \bp{\ln2 - 2 + \sqrt2}
            \end{align*}

            \boxt{The surface area is $\sqrt2 \pi \bp{\ln2 - 2 + \sqrt2}$ units$^2$.}

    \problem{}
        \begin{center}
            \begin{tikzpicture}[trim axis left, trim axis right]
                \begin{axis}[
                    domain = -2:2,
                    samples = 101,
                    axis y line=middle,
                    axis x line=middle,
                    xtick = \empty,
                    ytick = \empty,
                    xlabel = {$x$},
                    ylabel = {$y$},
                    ymax=9,
                    ymin=-3,
                    xmin=-2.1,
                    xmax=2.1,
                    legend cell align={left},
                    legend pos=outer north east,
                    after end axis/.code={
                        \path (axis cs:0,0) 
                            node [anchor=north east] {$O$};
                        }
                    ]
                    \addplot[black] {x^2};

                    \draw[<->] (-2, 4) -- (0, 4);
                    \draw[<->] (2, 4) -- (0, 4);
                    \draw[<->] (-2, 4) -- (-2, 0);
                    \draw[<->] (2, 4) -- (2, 0);

                    \node[anchor=south] at (-1, 4) {$S$};
                    \node[anchor=south] at (1, 4) {$S$};
                    \node[anchor=west] at (-2, 2) {$H$};
                    \node[anchor=east] at (2, 2) {$H$};
                \end{axis}
            \end{tikzpicture}
        \end{center}

        The diagram shows a cable for a suspension bridge, which has the shape of a parabola with equation $y = kx^2$. The suspension bridge has a total span $2S$ and the height of the cable relative to the lowest point is $H$ at each end. Show that the total length of the cable is $L = 2\displaystyle\int_0^S \sqrt{1 + \dfrac{4H^2}{S^4} x^2} \d x$.

        \begin{enumerate}
            \item Engineers from country $A$ proposed a suspension bridge across a strait of 8 km wide to country $B$. The plan included suspension towers 380 m high at each end. Find the length of the parabolic cable for this proposed bridge to the nearest metre.
            \item By using the result $\der{}{x} \ln{x + \sqrt{a^2 + x^2}} = \dfrac1{\sqrt{a^2 + x^2}}$ or otherwise, find $L$ in terms of $S$ and $H$.
        \end{enumerate}

    \solution
        By symmetry, we only need to consider the length of the curve where $x \geq 0$. Since $(S, H)$ is on the curve, $H = kS^2 \implies k = \dfrac{H}{S^2}$. Note that $y = kx^2 \implies \der{y}{x} = 2kx = \dfrac{2H}{S^2}x$. Hence,
        \begin{align*}
            L &= 2 \int_0^S \sqrt{1 + \bp{\der{y}{x}}^2} \d x\\
            &= 2 \int_0^S \sqrt{1 + \dfrac{4H^2}{S^4} x^2} \d x
        \end{align*}

        \part
            Note that $2S = 8000 \implies S = 4000$ and $H = 380$. Hence,
            \begin{align*}
                L &= 2 \int_0^{4000} \sqrt{1 + \dfrac{4(380)^2}{(4000)^4} x^2} \d x\\
                &= 8048 \text{ (to nearest integer)}
            \end{align*}
            \boxt{The cable is $8048$ m long.}
            
        \part
            Consider the integral $I = \displaystyle\int \sqrt{1 + (kx)^2} \d x$.
            \begin{alignat*}{2}
                && I &= \int \sqrt{1 + (kx)^2} \d x \usub{\tan \t &= kx\\\sec^2 \t \d \t &= k \d x}\\
                && &= \dfrac1k \int \sqrt{1 + \tan^2 \t} \sec^2 \t \d \t\\
                && &= \dfrac1k \int \sec^3 \t \d \t\\\\
                && &\begin{array}{r c @{\hspace*{1.0cm}} c}\toprule
                    & D & I \\\cmidrule{1-3}
                    + & \sec \t & \sec^2 \t \\
                    - & \sec \t \tan \t & \tan t \\\bottomrule
                \end{array}\\\\
                && &= \dfrac1k \bp{\sec\t \tan \t - \int \sec \t \tan^2 \t \d \t}\\
                && &= \dfrac1k \bp{\sec\t \tan \t - \int \sec \t (\sec^2 \t - 1) \d \t}\\
                &&  &= \dfrac1k \bp{\sec\t \tan \t - \int \sec^3 \t \d \t + \int \sec \t \d \t}\\
                &&  &= \dfrac1k \bp{\sec\t \tan \t - I + \ln \abs{\sec \t + \tan t}}\\
                \implies && 2I &= \dfrac1k \bp{\sec \t \tan \t + \ln \abs{\sec \t + \tan \t}} + C\\
                \implies && I &= \dfrac1{2k} \bp{\sec \t \tan \t + \ln \abs{\sec \t + \tan \t}} + C \usub{\sec \t &= \sqrt{\tan^2 \t + 1}\\&= \sqrt{(kx)^2 + 1}}\\
                && &= \dfrac1{2k} \bs{\sqrt{(kx)^2 + 1} \cdot kx + \ln \abs{\sqrt{(kx)^2 + 1} + kx}} + C
            \end{alignat*}
            In our case, $k = \dfrac{2H}{S^2} > 0$.
            {\allowdisplaybreaks
            \begin{align*}
                L &= 2\evalint{\dfrac1{2} \cdot \dfrac{S^2}{2H} \bp{\sqrt{\bp{\dfrac{2H}{S^2}x}^2 + 1} \cdot \dfrac{2H}{S^2}x + \ln \bp{\sqrt{\bp{\dfrac{2H}{S^2}x}^2 + 1} + \dfrac{2H}{S^2}x}}}{0}{S}\\
                &= \dfrac{S^2}{2H} \bp{\sqrt{\bp{\dfrac{2H}{S^2}S}^2 + 1} \cdot \dfrac{2H}{S^2}S + \ln \bp{\sqrt{\bp{\dfrac{2H}{S^2}S}^2 + 1} + \dfrac{2H}{S^2}S}}\\
                &= \sqrt{\bp{\dfrac{2H}{S^2}S}^2 + 1} \cdot S + \dfrac{S^2}{2H} \ln \bp{\sqrt{\bp{\dfrac{2H}{S^2}S}^2 + 1} + \dfrac{2H}{S^2}S}\\
                &= \sqrt{\bp{\dfrac{2H}{S}}^2 + 1} \cdot S + \dfrac{S^2}{2H} \ln \bp{\sqrt{\bp{\dfrac{2H}{S}}^2 + 1} + \dfrac{2H}{S}}\\
                &= \sqrt{\dfrac{4H^2}{S^2} + 1} \cdot S + \dfrac{S^2}{2H} \ln \bp{\sqrt{\dfrac{4H^2}{S^2} + 1} + \dfrac{2H}{S}}\\
                &= \sqrt{4H^2 + S^2} + \dfrac{S^2}{2H} \ln \bp{\dfrac{\sqrt{4H^2 + S^2} + 2H}{S}}
            \end{align*}}
            \boxt{$L = \sqrt{4H^2 + S^2} + \dfrac{S^2}{2H} \ln \bp{\dfrac{\sqrt{4H^2 + S^2} + 2H}{S}}$}

    \problem{}
        Sketch the semicircle with equation $x^2 + (y-b)^2 = a^2$, $y \geq b$ where $a$ and $b$ are positive constants.

        A solid is formed by rotating the region bounded by the semicircle and its diameter on the line $y = b$ about the $x$-axis through 4 right angles. Find the total surface area of the solid.

    \solution
        \begin{center}
            \begin{tikzpicture}[trim axis left, trim axis right]
                \begin{axis}[
                    domain = -10:10,
                    samples = 1000,
                    axis y line=middle,
                    axis x line=middle,
                    xtick = {-2, 2},
                    ytick = {1},
                    xticklabels = {$-a$, $a$},
                    yticklabels = {$b$},
                    xlabel = {$x$},
                    ylabel = {$y$},
                    ymin=0,
                    ymax=4,
                    xmin=-2.5,
                    xmax=2.5,
                    legend cell align={left},
                    legend pos=outer north east,
                    after end axis/.code={
                        \path (axis cs:0,0) 
                            node [anchor=north] {$O$};
                        }
                    ]
                    \addplot[plotRed] {1 + sqrt(4 - x^2)};
        
                    \addlegendentry{$x^2 + (y-b)^2 = a^2$};

                    \draw[plotRed] (2,1) arc(0:180:2);

                    \draw[dotted] (-2, 0) -- (-2, 1);
                    \draw[dotted] (2, 0) -- (2, 1);
                    \draw[dotted] (-2, 1) -- (2, 1);
                \end{axis}
            \end{tikzpicture}
        \end{center}

        Note that $x^2 + (y-b)^2 = a^2 \implies y = b + \sqrt{a^2- x^2}$ since $y \geq b \implies y - b \geq 0$. Hence, $\der{y}{x} = \dfrac1{2\sqrt{a^2 - x^2}} \cdot -2x = -\dfrac{x}{\sqrt{a^2 - x^2}}$.
        \begin{align*}
            \area &= 2\pi \int_{-a}^{a} y \sqrt{1 + \bp{\der{y}{x}}^2} \d x + 2\pi \cdot b \cdot 2a\\
            &= 2\pi \int_{-a}^a (b + \sqrt{a^2 - x^2}) \sqrt{1 + \bp{-\dfrac{x}{\sqrt{a^2-x^2}}}^2} \d x + 4\pi ab\\
            &= 4\pi \int_0^a (b + \sqrt{a^2 - x^2}) \sqrt{\dfrac{\bp{a^2 - x^2} + x^2}{a^2-x^2}} \d x + 4\pi ab\\
            &= 4\pi \int_0^a (b + \sqrt{a^2 - x^2}) \sqrt{\dfrac{a^2}{a^2-x^2}} \d x + 4\pi ab\\
            &= 4\pi a \int_0^a (b + \sqrt{a^2 - x^2}) \dfrac1{\sqrt{{a^2-x^2}}} \d x + 4\pi ab\\
            &= 4\pi a \bp{b \int_0^a \dfrac1{\sqrt{{a^2-x^2}}} \d x + \int_0^a \d x} + 4\pi ab\\
            &= 4\pi a \bp{b \evalint{\arcsin \dfrac{x}{a}}{0}a + \evalint{x}{0}{a}} + 4\pi ab\\
            &= 4\pi a \bp{b \cdot \dfrac\pi2 + a} + 4\pi ab\\
            &= 2\pi^2 ab + 4\pi a^2 + 4\pi ab
        \end{align*}

        \boxt{The total surface area is $2\pi^2 ab + 4\pi a^2 + 4\pi ab$ units$^2$.}

    \problem{}
        Using polar coordinates with pole $O$, the curve $C$ has the equation $r = ae^{\t/k}$, where $a$ and $k$ are positive constants and $0 \leq \t \leq 2\pi$. The points $A$ and $B$ on the curve corresponds to $\t = 0$ and $\t = \b$ respectively where $0 < \b < \pi$. The length of the arc $AB$ is denoted by $q$ and the area of the sector $OAB$ is denoted by $Q$.

        \begin{enumerate}
            \item Show that $Q = \dfrac14 k a^2 \bp{e^{2\b/k} - 1}$.
            \item Show that $q = a(1 + k^2)^{1/2}\bp{e^{\b/k} - 1}$.
            \item Deduce from the results of parts (a) and (b) that, for large values of $k$, $\dfrac{Q}{q} \approx \dfrac12 a$.
            \item Draw a sketch of $C$ for the case where $k$ is large and explain how the result in part (c) can be deduced from the sketch.
        \end{enumerate}

    \solution
        \part
            \begin{align*}
                Q &= \dfrac12 \int_0^\b r^2 \d \t\\
                &= \dfrac12 \int_0^\b \bp{ae^{\t/k}}^2 \d \t\\
                &= \dfrac12 a^2 \int_0^\b e^{2\t/k} \d \t\\
                &= \dfrac12 a^2 \evalint{\dfrac{e^{2\t/k}}{2/k}}0\b\\
                &= \dfrac14 a^2 k \bp{e^{2\b/k} - 1}
            \end{align*}

        \part
            Note that $r = ae^{\t/k} \implies \der{r}{\t} = \dfrac{ae^{\t/k}}k  = \dfrac{r}{k}$.
            {\allowdisplaybreaks
            \begin{align*}
                q &= \int_0^\b \sqrt{r^2 + \bp{\der{r}{\t}}^2} \d \t\\
                &= \int_0^\b \sqrt{r^2 + \dfrac{r^2}{k^2}} \d \t\\
                &= \sqrt{1 + k^{-2}} \int_0^\b r \d \t\\
                &= \sqrt{1 + k^{-2}} \int_0^\b ae^{\t/k} \d \t\\
                &= a\sqrt{1 + k^{-2}} \evalint{\dfrac{e^{\t/k}}{1/k}}0\b\\
                &= ak \sqrt{1 + k^{-2}} \bp{e^{\b/k} - 1}\\
                &= a \sqrt{k^2 + 1} \bp{e^{\b/k} - 1}
            \end{align*}}

        \part
            \begin{align*}
                \lim_{k \to \infty} \dfrac{Q}{q} &= \lim_{k \to \infty} \dfrac{\frac14 a^2 k \bp{e^{2\b/k} - 1}}{a \sqrt{k^2 + 1} \bp{e^{\b/k} - 1}}\\
                &= \dfrac14 a \lim_{k \to \infty} \bp{\dfrac{k}{\sqrt{k^2 + 1}} \cdot \dfrac{e^{2\b/k} - 1}{e^{\b/k} - 1}}\\
                &= \dfrac14 a \lim_{k \to \infty} \bp{\dfrac{k}{\sqrt{k^2 + 1}}} \lim_{k \to \infty} \bp{\dfrac{e^{2\b/k} - 1}{e^{\b/k} - 1}}\\
                &= \dfrac14 a \cdot 1 \cdot \lim_{k \to \infty} \bp{\dfrac{2 \cdot e^{\b/k} \cdot \derx{}{k} \bp{e^{\b/k}}}{\derx{}{k} \bp{e^{\b/k}}}}\\
                &= \dfrac12 a \cdot \lim_{k \to \infty} e^{\b/k}\\
                &= \dfrac12 a
            \end{align*}

        \part
            Note that $\displaystyle\lim_{k \to \infty} r = \lim_{k \to \infty} ae^{\t/k} = a$.

            \begin{center}
                \begin{tikzpicture}[trim axis left, trim axis right]
                    \begin{axis}[
                        domain = 0:2*pi,
                        samples = 100,
                        axis y line=middle,
                        axis x line=middle,
                        xtick = \empty,
                        ytick = \empty,
                        xmin=-2,
                        xmax=2,
                        ymin=-1.7,
                        ymax=1.7,
                        xlabel = {$\t=0$},
                        ylabel = {$\t = \frac\pi2$},
                        legend cell align={left},
                        legend pos=outer north east,
                        after end axis/.code={
                            \path (axis cs:0,0) 
                                node [anchor=north east] {$O$};
                            }
                        ]
                        \addplot[color=plotRed,data cs=polarrad] {1};
            
                        \addlegendentry{$r = a$};

                        \coordinate[label=below right:$A$] (A) at (1, 0);
                        \coordinate[label=above left:$B$] (B) at (-1/2, 0.866);
                        \coordinate (O) at (0, 0);

                        \fill (A) circle[radius=2.5 pt];
                        \fill (B) circle[radius=2.5pt];
                        \draw[dotted] (O) -- (B);

                        \draw pic [draw, angle radius=7mm, "$\b$"] {angle = A--O--B};
                    \end{axis}
                \end{tikzpicture}
            \end{center}

            As $k \to \infty$, the curve becomes a circle. Hence, $Q$ is the area of a sector with angle $\b$, and $q$ is the arc length of a sector with angle $\b$. Thus, $\dfrac{Q}{q} = \bp{\dfrac{\b}{2\pi} \cdot \pi a^2} \Big/ \bp{\dfrac{\b}{2\pi} \cdot 2\pi a} = \dfrac12 a$.
\end{document}