\documentclass{echw}

\title{Assignment B13\\Linear First Order DE}
\author{Eytan Chong}
\date{2024-08-21}

\begin{document}
    \problem{}
        Two biological cultures, $X$ and $Y$, react with each other, and their volumes at time $t$ are $x$ and $y$ respectively, in appropriate units. Their rates of growth are modelled by the simultaneous equations
        \begin{align*}
            \der{x}{t} &= (2-x)y,\\
            \der{y}{t} &= \dfrac{y^2}{x}
        \end{align*}
        When $t = 0$, $x = y = 1$.
        \begin{enumerate}
            \item Show that $x = \dfrac{2y^2}{1+y^2}$.
            \item Find and simplify expressions for $y$ and $x$ in terms of $t$.
            \item Sketch the graph of $y$ against $x$ for $0 < t < \dfrac\pi2$.
        \end{enumerate}

    \solution
        \part
            Note that $x, y > 0$ since they represent volume. Also, for $x \in (0, 2)$, we have $\der{x}{t} = (2-x)y > 0$. When $x = 2$, we have $\der{x}{t} = 0$. Hence, $0 < x \leq 2$.
            \begin{alignat*}{2}
                &&\der{y}{x} &= \dfrac{\derx{y}{t}}{\derx{x}{t}}\\
                && &= \dfrac{y^2/x}{(2-x)y}\\
                && &= \dfrac{y}{x(2-x)}\\
                \implies&& \dfrac1y \der{y}{x} &= \dfrac1{x(2-x)}\\
                \implies&& \int \dfrac1y \der{y}{x} \d x &= \int \dfrac1{x(2-x)} \d x\\
                \implies&& \int \dfrac1y \d y &= \int \dfrac1{x(2-x)} \d x\\
                && &= \dfrac12 \int \bp{\dfrac1x + \dfrac1{2-x}} \d x\\
                \implies&& \ln y &= \dfrac12 \bs{\ln x - \ln
                {2-x}} + C_1\\
                && &= \ln \sqrt{\dfrac{x}{2-x}} + C_1\\
                \implies&& y &= C_2 \sqrt{\dfrac{x}{2-x}}
            \end{alignat*}
            At $t = 0$, $x = y = 1$. Hence, $1 = C_2 \sqrt{1}{2-1} \implies C_2 = 1$.
            \begin{alignat*}{2}
                &&y &= \sqrt{\dfrac{x}{2-x}}\\
                \implies&& y^2 &= \dfrac{x}{2-x}\\
                \implies&& (2-x)y^2 &= x\\
                \implies&& 2y^2 - xy^2 &= x\\
                \implies&& x + xy^2 &= 2y^2\\
                \implies&& x(1 + y^2) &= 2y^2\\
                \implies&& x &= \dfrac{2y^2}{1+y^2}
            \end{alignat*}

        \part
            \begin{alignat*}{2}
                && \der{y}{t} &= \dfrac{y^2}{x}\\
                && &= \dfrac{y^2}{2y^2/(1+y^2)}\\
                && &= \dfrac12 (1 + y^2)\\
                \implies&& \dfrac1{1+y^2} \der{y}{t} &= \dfrac12\\
                \implies&& \int \dfrac1{1+y^2} \der{y}{t} \d t&= \int \dfrac12 \d t\\
                \implies&& \int \dfrac1{1+y^2} \d y &= \int \dfrac12 \d t\\
                \implies&& \arctan y &= \dfrac12 t + C\\
                \implies&& y &= \tan{\dfrac12 t + C}
            \end{alignat*}
            At $t = 0$, $y = 1$. Hence, $1 = \tan C \implies C = \dfrac\pi4$.
            \begin{align*}
                y &= \tan{\dfrac12 t + \dfrac\pi4}\\
                &= \dfrac{1-\cos{t + \pi/2}}{\sin{t+\pi/2}}\\
                &= \dfrac{1 + \sin t}{\cos t}\\
                &= \sec t + \tan t
            \end{align*}
            \boxt{$y = \sec t + \tan t$}

            \begin{alignat*}{2}
                && \der{x}{t} &= (2-x)y\\
                && &= (2-x) \sqrt{\dfrac{x}{2-x}}\\
                && &= \sqrt{x(2-x)}\\
                \implies&& \dfrac1{\sqrt{x(2-x)}} \der{x}{t} &= 1 \\
                \implies&& \int \dfrac1{\sqrt{x(2-x)}} \der{x}{t} \d t&= \int \d t\\
                \implies&& \int \dfrac1{\sqrt{x(2-x)}} \d x &= \int \d t\\
                \implies&& \int \dfrac1{\sqrt{x} \sqrt{2-x}} \d x &= t + C_1 \usub{u &= \sqrt{x}\\\d x &= 2u \d u}\\
                \implies&& \int \dfrac{2u}{u \sqrt{2 - u^2}} \d u &= t + C_1\\
                \implies&& 2 \int \dfrac{1}{\sqrt{2 - u^2}} \d u &= t + C_1\\
                \implies&& 2 \arcsin{\dfrac{u}{\sqrt2}} &= t + C_1\\
                \implies&& 2 \arcsin{\sqrt{\dfrac{x}{2}}} &= t + C_1\\
                \implies&& \arcsin{\sqrt{\dfrac{x}{2}}} &= \dfrac12 t + C_2\\
                \implies&& \sqrt{\dfrac{x}{2}} &= \sin{\dfrac12 t + C_2}\\
                \implies&& \dfrac{x}{2} &= \sin[2]{\dfrac12 t + C_2}\\
                \implies&& x &= 2\sin[2]{\dfrac12 t + C_2}
            \end{alignat*}
            At $t = 0$, $x = 1$. Hence, $1 = 2\sin[2] C_2$, whence $C_2 = \dfrac\pi4$.
            \begin{align*}
                x &= 2\sin[2]{\dfrac12 t + \dfrac\pi4}\\
                &= 1 - \cos{t + \dfrac\pi2}\\
                &= 1 + \sin t
            \end{align*}
            \boxt{$x = 1 + \sin t$}

        \part
            Note that $0 < t < \dfrac\pi2 \implies 1 < x < 2$.
            \begin{center}
                \begin{tikzpicture}[trim axis left, trim axis right]
                    \begin{axis}[
                        domain = 1:2,
                        samples = 101,
                        axis y line=middle,
                        axis x line=middle,
                        xtick = {2},
                        ytick = \empty,
                        xlabel = {$x$},
                        ylabel = {$y$},
                        xmin=0,
                        xmax=2.1,
                        ymax=10,
                        ymin=0,
                        legend cell align={left},
                        legend pos=outer north east,
                        after end axis/.code={
                            \path (axis cs:0,0) 
                                node [anchor=north east] {$O$};
                            }
                        ]
                        \addplot[plotRed] {sqrt(x/(2-x))};
            
                        \addlegendentry{$y = \sqrt{\frac{x}{2-x}}$};

                        \draw[dotted] (2, 0) -- (2, 10);

                        \draw (1, 1) circle[radius=2.5pt] node[above] {$\bp{1, 1}$};
                    \end{axis}
                \end{tikzpicture}
            \end{center}

    \problem{}
        Find the general solution of the differential equation \[x\der{y}{x} + 4y - 10x = 0,\]

        Find the particular solution such that $y \to 0$ as $x \to 0$.

        Show, on a single diagram, a sketch of this particular solution and one typical member of the family, $F$ of solution curves for which $\der{y}{x}$ is positive whenever $x$ is positive.

        Show that there is a straight line which passes through the maximum point of every member of $F$ and find its equation.

    \solution
        \begin{alignat*}{2}
            &&x\der{y}{x} + 4y - 10x &= 0\\
            \implies&&x^4\der{y}{x} + 4x^3y &= 10x^4\\
            \implies&& \der{}{x} \bp{x^4 y} &= 10x^4\\
            \implies&& x^4 y &= \int 10 x^4 \d x\\
            \implies&& x^4 y &= 2x^5 + C\\
            \implies&& y &= 2x + Cx^{-4}
        \end{alignat*}

        \boxt{$y = 2x + Cx^{-4}$}

        As $x \to 0$, $x^{-4} \to \infty$. Hence, $C$ must be 0.
        \boxt{$y = 2x$}

        Note that $\der{y}{x} = 2 - 4Cx^{-5} > 0 \implies C < \dfrac12 x^5$. Since $x > 0$, we hence have the constraint $C \leq 0$ for members of $F$.

        \begin{center}
            \begin{tikzpicture}[trim axis left, trim axis right]
                \begin{axis}[
                    domain = -5:3,
                    restrict y to domain=-10:5,
                    samples = 120,
                    axis y line=middle,
                    axis x line=middle,
                    xtick = \empty,
                    ytick = \empty,
                    xlabel = {$x$},
                    ylabel = {$y$},
                    legend cell align={left},
                    legend pos=outer north east,
                    after end axis/.code={
                        \path (axis cs:0,0) 
                            node [anchor=north east] {$O$};
                        }
                    ]
                    \addplot[plotRed] {2*x};
        
                    \addlegendentry{$C=0$};
        
                    \addplot[plotBlue] {2*x - 1/x^4};
        
                    \addlegendentry{$C=-1$};
                \end{axis}
            \end{tikzpicture}
        \end{center}

        Consider the stationary points of members of $F$. For stationary points, $\der{y}{x} = 0$. Hence,
        \begin{alignat*}{2}
            && x\der{y}{x} + 4y - 10x &= 0\\
            \implies&& 4y - 10 x &= 0\\
            \implies&& y &= \dfrac52 x
        \end{alignat*}
        Differentiating the original differential equation with respect to $x$, we obtain
        \begin{alignat*}{2}
            x\der{y}{x} + 4y - 10x &= 0\\
            \implies&& x \der[2]{y}{x} + \der{y}{x} + 4\der{y}{x} - 10 &= 0\\
            \implies&& x \der[2]{y}{x} &= 10\\
            \implies&& \der[2]{y}{x} &= \dfrac{10}{x}
        \end{alignat*}
        Note that for members of $F$, we have that $\der{y}{x} > 0$ for $x > 0$. Hence, there are no stationary points when $x > 0$. That is, any stationary point must occur when $x < 0$. (Indeed, there is a stationary point when $x = \sqrt[5]{2C} < 0$.) Furthermore, when $x < 0$, $\der[2]{y}{x} < 0$. Hence, all stationary points must be a maximum point. Thus, $\boxed{y = \dfrac52 x}$ passes through the maximum point of every member of $F$.

    \problem{}
        \begin{enumerate}
            \item The variables $x$ and $y$ are related by the differential equation \[x^2 \der{y}{x} - 2xy + y = 0.\]
            \begin{enumerate}
                \item Find the general solution of this differential equation, expressing $y$ in terms of $x$.
                \item Find the particular solution for which $y = -e$ when $x = 1$. Obtain the coordinates of the turning point of the solution curve of this particular solution and sketch the curve for $x > 0$.
            \end{enumerate}
            \item Find the general solution of the differential equation \[\der{y}{x} + xy = e^x x^2,\] expressing $y$ in terms of $x$.
        \end{enumerate}

    \solution
        \part
            \subpart
                \begin{alignat*}{2}
                    && x^2 \der{y}{x} - 2xy + y &= 0\\
                    \implies&& x^2\der{y}{x}  &= y(2x-1)\\
                    \implies&& \dfrac1y \der{y}{x} &= \dfrac{2x-1}{x^2}\\
                    && &= \dfrac2{x} - \dfrac1{x^2}\\
                    \implies&& \int \dfrac1y \der{y}{x} \d x &= \int \bp{\dfrac2{x} - \dfrac1{x^2}} \d x\\
                    \implies&& \int \dfrac1y \d y &= \int \bp{\dfrac2{x} - \dfrac1{x^2}} \d x\\
                    \implies&& \ln \abs{y} &= 2 \ln{x} + \dfrac1x + C_1\\
                    \implies&& y &= C_2 x^2 e^{1/x}
                \end{alignat*}
                \boxt{$y = C_2 x^2 e^{1/x}$}

            \subpart

                When $x = 1$, $y = -e$. Hence, $-e = C_2 \cdot 1^2 \cdot e^{1} \implies C_2 = -1$.
                \boxt{$y = -x^2 e^{1/x}$}

                For stationary point, $\der{y}{x} = 0$. Hence, $y(2x - 1) = 0$, whence $x = \dfrac12$. Note that we reject $y = 0$ since $e^{1/x} \neq 0$ and $x \neq 0$ due to the presence of a $\dfrac1x$ term. Hence, $y$ has a stationary point at $\bp{\dfrac12, -\dfrac{e^2}{4}}$.

                Differentiating the original differential equation with respect to $x$, we obtain
                \begin{alignat*}{2}
                    &&x^2 \der{y}{x} - 2xy + y &= 0\\
                    \implies&& x^2 \der[2]{y}{x} + 2x \der{y}{x} - 2x\der{y}{x} - 2y + \der{y}{x} &= 0\\
                    \implies&& x^2 \der[2]{y}{x} - 2y &= 0\\
                    \implies&& \der[2]{y}{x} &= \dfrac{2y}{x^2}
                \end{alignat*}
                Hence, at $\bp{\dfrac12, -\dfrac{e^2}{4}}$, we have $\der[2]{y}{x} = \dfrac{-e^2/2}{1/4} < 0$. Thus, $\boxed{\bp{\dfrac12, -\dfrac{e^2}{4}}}$ is a maximum point and is thus a turning point.

                \begin{center}
                    \begin{tikzpicture}[trim axis left, trim axis right]
                        \begin{axis}[
                            domain = 0:3,
                            restrict y to domain =-10:5,
                            samples = 101,
                            axis y line=middle,
                            axis x line=middle,
                            xtick = \empty,
                            ytick = \empty,
                            xlabel = {$x$},
                            ylabel = {$y$},
                            legend cell align={left},
                            legend pos=outer north east,
                            after end axis/.code={
                                \path (axis cs:0,0) 
                                    node [anchor=north east] {$O$};
                                }
                            ]
                            \addplot[plotRed] {-x^2 * e^(1/x)};
                
                            \addlegendentry{$y = -x^2e^{1/x}$};

                            \fill (0.5, -e^2/4) circle[radius=2.5 pt] node[above] {$\bp{\frac12, -\frac{e^2}{4}}$};
                        \end{axis}
                    \end{tikzpicture}
                \end{center}

        \part
            \begin{alignat*}{2}
                && \der{y}{x} + xy &= e^x x^2\\
                \implies&& e^{\frac12 x^2}\der{y}{x} + xe^{\frac12 x^2}y &= e^{\frac12 x^2 + x} x^2\\
                \implies&& \der{}{x} \bp{e^{\frac12 x^2} y} &= e^{\frac12 x^2 + x} x^2\\
                \implies&& e^{\frac12 x^2} y &= \int e^{\frac12 x^2 + x} x^2 \d x
            \end{alignat*}
            Suppose $\displaystyle\int e^{\frac12 x^2 + x} x^2 \d x = P(x) e^{\frac12 x^2 + x} + C$ for some function $P(x)$. Differentiating both sides with respect to $x$, we obtain \[x^2 e^{\frac12 x^2 + x} = e^{\frac12 x^2 + x} \bs{(x+1)P(x) + P'(x)},\] whence \[x^2 = (x+1)P(x) + P'(x).\] Thus, $P(x)$ is a polynomial of degree 1. Let $P(x) = ax + b$. For some constants $a$ and $b$. Then \[x^2 = ax^2 + (a+b)x + (a+b).\] Comparing coefficients of $x^2$, $x$ and constant terms, we have $a = 1$ and $a + b = 0 \implies b = -1$. Thus, \[\int x^2 e^{\frac12 x^2 + x} \d x = (x-1)e^{\frac12 x^2 + x} + C.\] Thus,
            \boxt{$y = (x-1)e^x + Ce^{-\frac12 x^2}$}


\end{document}