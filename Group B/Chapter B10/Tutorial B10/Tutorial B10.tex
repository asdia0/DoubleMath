\documentclass{echw}

\title{Tutorial B10\\Applications of Integration III}
\author{Eytan Chong}
\date{2024-07-09}

\begin{document}
    \problem{}
        Estimate, using the trapezium rule, the values of the following definite integrals, taking the number ordinates given in each case.

        \begin{enumerate}
            \item $\displaystyle\int_{-\pi/2}^0 \dfrac1{1 + \cos\t} \d \t$, 3 ordinates
            \item $\displaystyle\int_{-0.4}^{0.2} \dfrac{x^2 - 4x + 1}{4x - 4}$, 4 ordinates
        \end{enumerate}

    \solution
        \part
            Let $f(\t) = \dfrac1{1+\cos\t}$.
            \[
              \int_{-\pi/2}^0 \dfrac1{1 + \cos\t} \d \t \approx \dfrac12 \cdot \dfrac{0 - (-\pi/2)}{3-1} \cdot \bs{f\of{-\dfrac\pi2} + 2f\of{-\dfrac\pi4} + f(0)} = 1.05
            \]

            \boxt{$\displaystyle\int_{-\pi/2}^0 \dfrac1{1 + \cos\t} \d \t \approx 1.05$}

        \part
            Let $f(x) = \dfrac{x^2 - 4x + 1}{4x - 4}$.
            \[
                \int_{-0.4}^{0.2} \dfrac{x^2 - 4x + 1}{4x - 4} \d x \approx \dfrac12 \cdot \dfrac{0.2 - (-0.4)}{4 - 1} \cdot \Big[f(-0.4) + 2\big(f(-0.2) + f(0)\big) + f(0.2)\Big] = -0.183
            \]

            \boxt{$\displaystyle\int_{-0.4}^{0.2} \dfrac{x^2 - 4x + 1}{4x - 4} \d x \approx -0.183$}

    \problem{}
        Use the trapezium rule with intervals of width 0.5 to obtain an approximation to $\displaystyle\int_2^{3.5} \ln \dfrac1x \d x$, giving your answer to 2 decimal places.

    \solution
        \begin{align*}
            \int_2^{3.5} \ln \dfrac1x \d x &= -\int_2^{3.5} \ln x \d x\\
            &\approx - \bp{\dfrac12 \cdot \dfrac{3.5 - 2}{4 - 1} \cdot \bs{\ln 2 + 2(\ln 2.5 + \ln 3) + \ln 3.5}}\\
            &= -1.49 \todp{2}
        \end{align*}

        \boxt{$\displaystyle\int_2^{3.5} \ln \dfrac1x \d x = -1.49$}

    \problem{}
        Estimate, using Simpson's rule, the values of the following definite integrals, taking the number of ordinates given in each case.

        \begin{enumerate}
            \item $\displaystyle\int_{-\pi/2}^0 \dfrac1{1 + \cos\t} \d \t$, 3 ordinates
            \item $\displaystyle\int_0^{0.4} \sqrt{1 - x^2} \d x$, 5 ordinates
        \end{enumerate}

    \solution
        \part
            Let $f(\t) = \dfrac1{1+\cos\t}$.
            \[
              \int_{-\pi/2}^0 \dfrac1{1 + \cos\t} \d \t \approx \dfrac13 \cdot \dfrac{0 - (-\pi/2)}{3 - 1} \cdot \bs{f\of{-\dfrac\pi2} + 4f\of{-\dfrac\pi4} + f(0)} = 1.01
            \]

            \boxt{$\displaystyle\int_{-\pi/2}^0 \dfrac1{1 + \cos\t} \d \t \approx 1.01$}

        \part
            Let $f(x) = \sqrt{1 - x^2}$.
            \[
                \int_0^{0.4} \sqrt{1 - x^2} \d x \approx \dfrac13 \cdot \dfrac{0.4 - 0}{5 - 1} \cdot \Big[f(0) + 4f(0.1) + 2f(0.2) + 4f(0.3) + f(0.4)\Big] = 0.389
            \]

            \boxt{$\displaystyle\int_0^{0.4} \sqrt{1 - x^2} \d x \approx 0.389$}

    \problem{}
        Show, by means of substitution $u = \sqrt{x}$, that
        \[
            \int_0^{0.25} \dfrac1{\sqrt{x}} e^{-x} \d x = \int_0^{0.5} 2e^{-u^2} \d u
        \]
        Use the trapezium rule, with ordinates at $u = 0$, $u = 0.1$, $u = 0.2$, $u = 0.3$, $u = 0.4$ and $u = 0.5$, to estimate the value of $I = \displaystyle\int_0^{0.5} 2e^{-u^2} \d u$, giving three decimal places in your answer.

        Explain briefly why the trapezium rule cannot be used directly to estimate the value of $\displaystyle\int_0^{0.25} \dfrac1{\sqrt{x}} e^{-x} \d x$.

        By using the first four terms of the expansion of $e^{-x}$, obtain an estimate for the integral $\displaystyle\int_0^{0.25} \dfrac1{\sqrt{x}}e^{-x} \d x$, giving three decimal places in your answer.

    \solution
        Note that $u = \sqrt{x} \implies u^2 = x \implies 2u \d u = \d x$. Furthermore, $x = 0 \implies u = 0$ and $x = 0.25 \implies u = 0.5$.
        \begin{align*}
            \int_0^{0.25} \dfrac1{\sqrt{x}} e^{-x} \d x &= \int_{0}^{0.5} \dfrac1u e^{-u^2} \cdot 2u \d u\\
            &= \int_0^{0.5} 2e^{-u^2} \d u
        \end{align*}

        Let $f(u) = 2e^{-u^2}$. Using the trapezium rule,
        \begin{align*}
            \int_0^{0.5} 2e^{-u^2} \d u &\approx \dfrac12 \cdot \dfrac{0.5 - 0}{5} \cdot \Big[f(0) + 2\big(f(0.1) + f(0.2) + f(0.3) + f(0.4)\big) + f(0.5) \Big]\\
            &= 0.921 \todp{3}
        \end{align*}

        \boxt{$I \approx 0.921$}

        At $x = 0$, $\dfrac1{\sqrt{x}}e^{-x}$ is undefined. Hence, the trapezium rule cannot be used.

        Recall that $e^{-x} = 1 - x + \dfrac12 x^2 - \dfrac16 x^3 + \ldots$.
        \[
            \int_0^{0.25} \dfrac1{\sqrt{x}} e^{-x} \d x \approx \int_0^{0.25} \dfrac1{\sqrt x}\bp{1 - x + \dfrac12 x^2 - \dfrac16 x^3} \d x = 0.923 \todp{3}
        \]

        \boxt{$\displaystyle\int_0^{0.25} \dfrac1{\sqrt{x}} e^{-x} \d x \approx 0.923$}

    \problem{}
        \begin{center}
            \begin{tikzpicture}[trim axis left, trim axis right]
                \begin{axis}[
                    domain = 0:1,
                    samples = 101,
                    axis y line=middle,
                    axis x line=middle,
                    xtick = {1},
                    ytick = \empty,
                    xlabel = {$x$},
                    ylabel = {$y$},
                    ymax=1.1,
                    ymin=0,
                    xmin=0,
                    xmax=1.1,
                    legend cell align={left},
                    legend pos=outer north east,
                    after end axis/.code={
                        \path (axis cs:0,0) 
                            node [anchor=north east] {$O$};
                        }
                    ]
                    \addplot[plotRed] {e^(-1.5 * ln(x^2 + 1))};
        
                    \addlegendentry{$y = (x^2 + 1)^{-3/2}$};

                    \draw[dotted] (1, 0) -- (1, 0.35355);

                    \node at (0.5, 0.3577) {$R$};
                \end{axis}
            \end{tikzpicture}
        \end{center}

        The diagram (not to scale) show the region $R$ bounded by the axes, the curve $y = (x^2 + 1)^{-3/2}$ and the line $x = 1$. The integral $\displaystyle\int_0^1 (x^2 + 1)^{-3/2}$ is denoted by $I$.

        \begin{enumerate}
            \item Use the trapezium rule and Simpson's rule, with ordinates at $x = 0$, $x = 0.5$ and $x = 1$, to estimate the value of $I$ correct to 4 significant figures.
            \item Use the substitution $x = \tan \t$ to show that $I = \dfrac12 \sqrt{2}$.
            Comment on the approximations using the 2 rules and give a reason why one gives a better approximation than the other.
            \item By using the trapezium rule, with the same ordinates as in part (a), or otherwise, estimate the volume of the solid formed when $R$ is rotated completely about the $x$-axis, giving your answer to 2 significant figures.
        \end{enumerate}

    \solution
        \part
            Let $f(x) = (x^2 + 1)^{-3/2}$. Using the trapezium rule,
            \[
                I \approx \dfrac12 \cdot \dfrac{1 - 0}{3 -1} \cdot \Big[f(0) + 2f(0.5) + f(1)\Big] = 0.6962 \tosf{4}
            \]
            \boxt{$I \approx 0.6962$}

            Using Simpson's rule,
            \[
                I \approx \dfrac13 \cdot \dfrac{1 - 0}{3 -1} \cdot \Big[f(0) + 4f(0.5) + f(1)\Big] = 0.7026 \tosf{4}
            \]
            \boxt{$I \approx 0.7026$}

        \part
            \begin{align*}
                \int_0^1 (x^2 + 1)^{-3/2} &= \int_0^{\pi/4} \bp{\tan^2 \t + 1}^{-3/2} \sec^2 \t \d \t \usub{x &= \tan \t\\\d x &= \sec^2 \t \d \t}\\
                &= \int_0^{\pi/4} (\sec^2 \t)^{-3/2} \sec^2 \t \d \t\\
                &= \int_0^{\pi/4} \dfrac1{\sec \t} \d \t\\
                &= \int_0^{\pi/4} \cos \t \d \t\\
                &= \evalint{\sin \t}0{\pi/4}\\
                &= \dfrac12 \sqrt2
            \end{align*}

            The curve $y = (x^2 + 1)^{-3/2}$ resembles a quadratic more than a line on the interval $[0, 1]$. Hence, Simpson's rule, which uses quadratic curves to approximate $y$, gives a better approximation for $I$ than the trapezium rule, which uses lines to approximate $y$.

        \part
            Let $g(x) = (x^2 + 1)^{-3}$.
            \begin{align*}
                \volume &= \pi \int_0^1 y^2 \d x\\
                &= \pi \int_0^1 (x^2 + 1)^{-3} \d x\\
                &\approx \pi \bp{\dfrac12 \cdot \dfrac{1- 0}{3 - 1} \cdot \Big[g(0) + 2g(0.5) + g(1)\Big]}\\
                &= 1.7 \tosf{2}
            \end{align*}

            \boxt{The volume of the solid formed is $1.7$ units$^3$.}
    
    \problem{}
        It is given that $f(x) = \dfrac1{\sqrt{1 + \sqrt{x}}}$, and the integral $\displaystyle\int_0^1 f(x) \d x$ is denoted by $I$.

        \begin{enumerate}
            \item Using the trapezium rule, with four trapezia of equal width, obtain an approximation $I_1$ to the value of $I$, giving 3 decimal places in your answer.
            \item Explain, with the aid of a sketch, why $I < I_1$.
            \item Evaluate $I_2$, where $I_2 = \dfrac13 \displaystyle\sum_{r = 1}^3 f\of{\dfrac13 r}$, giving 3 decimal places in your answer, and use the sketch in (b) to justify the inequality $I > I_2$.
            \item By means of a substitution $\sqrt{x} = u -1$, show that the value of $I$ is $\dfrac43(2 - \sqrt2)$.
        \end{enumerate}

    \solution
        \part
            \[
                I_1 = \dfrac12 \cdot \dfrac{1 - 0}{4} \cdot \Big[ f(0) + 2\big(f(0.25) + f(0.5) + f(0.75)\big) + f(1)\Big] = 0.792 \todp{3}
            \]

            \boxt{$I_1 = 0.792$}

        \part
            \begin{center}
                \begin{tikzpicture}[trim axis left, trim axis right]
                    \begin{axis}[
                        domain = 0:1,
                        samples = 101,
                        axis y line=middle,
                        axis x line=middle,
                        xtick = {0.25, 0.5, 0.75, 1},
                        ytick = \empty,
                        xlabel = {$x$},
                        ylabel = {$y$},
                        ymin=0,
                        xmax=1.1,
                        legend cell align={left},
                        legend pos=outer north east,
                        after end axis/.code={
                            \path (axis cs:0,0) 
                                node [anchor=north east] {$O$};
                            }
                        ]
                        \addplot[plotRed] {1/sqrt(1 + sqrt(x)) - 0.5};
                        \addlegendentry{$y = f(x)$};
                        
                        \draw[dashed] (0.25, 0) -- (0.25, 0.31650);
                        \draw[dashed] (0.5, 0) -- (0.5, 0.26537);
                        \draw[dashed] (0.75, 0) -- (0.75, 0.23205);
                        \draw[dashed] (1, 0) -- (1, 0.20711);
                        \draw[dashed] (0.25, 0.31650) -- (0, 0.5);
                        \draw[dashed] (0.25, 0.31650) -- (0.5, 0.26537);
                        \draw[dashed] (0.5, 0.26537) -- (0.75, 0.23205);
                        \draw[dashed] (0.75, 0.23205) -- (1, 0.20711);
                    \end{axis}
                \end{tikzpicture}
            \end{center}

            $I$ is the area under the curve $y = f(x)$, while $I_1$ is the sum of the areas of the trapeziums. Hence, from the sketch, $I_1 > I$.

        \part
            \begin{align*}
                I_2 &= \dfrac13 \sum_{r=1}^3 f\of{\dfrac13 r}\\
                &= \dfrac13 \bs{f\of{\dfrac13} + f\of{\dfrac23} + f(1)}\\
                &= 0.748 \todp{3}
            \end{align*}

            \boxt{$I_2 = 0.748$}

            \begin{center}
                \begin{tikzpicture}[trim axis left, trim axis right]
                    \begin{axis}[
                        domain = 0:1,
                        samples = 101,
                        axis y line=middle,
                        axis x line=middle,
                        xtick = {1/3, 2/3, 1},
                        xticklabels = {$1/3$, $2/3$, $1$},
                        ytick = \empty,
                        xlabel = {$x$},
                        ylabel = {$y$},
                        ymin=0,
                        xmax=1.1,
                        legend cell align={left},
                        legend pos=outer north east,
                        after end axis/.code={
                            \path (axis cs:0,0) 
                                node [anchor=north east] {$O$};
                            }
                        ]
                        \addplot[plotRed] {1/sqrt(1 + sqrt(x)) - 0.5};
                        \addlegendentry{$y = f(x)$};
                        
                        \draw[dashed] (1/3, 0) -- (1/3, 0.29623);
                        \draw[dashed] (2/3, 0) -- (2/3, 0.24196);
                        \draw[dashed] (1, 0) -- (1, 0.20711);
                        \draw[dashed] (0, 0.29623) -- (1/3, 0.29623);
                        \draw[dashed] (1/3, 0.24196) -- (2/3, 0.24196);
                        \draw[dashed] (2/3, 0.20711) -- (1, 0.20711);
                    \end{axis}
                \end{tikzpicture}
            \end{center}

            $I$ is the area under the curve $y = f(x)$, while $I_2$ is the sum of the areas of the rectangles. Hence, from the sketch, $I_2 < I$.

        \part
            Note $\sqrt{x} = u -1 \implies x = u^2 - 2u + 1 \implies \d x = (2u - 2) \d u$. Furthermore, $x=0 \implies u = 1$ and $x = 1 \implies u = 2$.
            \begin{align*}
                \int_0^1 \dfrac1{\sqrt{1 + \sqrt{x}}} \d x &= \int_1^2 \dfrac1{\sqrt{1 + (u-1)}} \cdot (2u - 2) \d u\\
                &= 2\int_1^2 \dfrac{u-1}{\sqrt{u}} \d u\\
                &= 2\evalint{\dfrac23 u^{3/2} - 2u^{1/2}}12\\
                &= \dfrac43 (2 - \sqrt2)
            \end{align*}

    \problem{}
        For $0 < x < \pi$, the curve $C$ has the equation $y = \ln \sin x$. The region of the plane bounded by $C$, the $x$-axis and the lines $x = \dfrac\pi4$ and $x = \dfrac\pi2$ is rotated through $2\pi$ radians about the $x$-axis.

        Show that the surface area of the solid generated in this way is given by $S$, where
        \[
            S = 2\pi \int_{\pi/4}^{\pi/2} \abs{\dfrac{\ln \sin x}{\sin x}} \d x
        \]
        Use Simpson's rule with 5 ordinates to find an approximate value of $S$, giving your answer to 3 decimal places.

    \solution
        Note that $\der{y}{x} = \dfrac1{\sin x} \cdot \cos x = \cot x \implies 1 + \bp{\der{y}{x}}^2 = 1 + \cot^2 x = \csc^2 x$.
        \begin{align*}
            S &= 2\pi \int_{\pi/4}^{\pi/2} \abs{y} \sqrt{1 + \bp{\der{y}{x}}^2} \d x\\
            &= 2\pi \int_{\pi/4}^{\pi/2} \abs{\ln \sin x} \sqrt{\csc^2 x} \d x\\
            &= 2\pi \int_{\pi/4}^{\pi/2} \abs{\ln \sin x} \abs{\csc x} \d x\\
            &= 2\pi \int_{\pi/4}^{\pi/2} \abs{\ln \sin x} \abs{\dfrac1{\sin x}} \d x\\
            &= 2\pi \int_{\pi/4}^{\pi/2} \abs{\dfrac{\ln \sin x}{\sin x}} \d x
        \end{align*}
        Let $f(x) = \abs{\dfrac{\ln \sin x}{\sin x}}$.
        \begin{align*}
            S &\approx 2\pi \bp{\dfrac13 \cdot \dfrac{\pi/2 - \pi/4}{5 - 1} \cdot \bs{f\of{\dfrac4{16}\pi} + 4f\of{\dfrac5{16}\pi} + 2f\of{\dfrac6{16}\pi} + 4f\of{\dfrac7{16}\pi} + f\of{\dfrac8{16}\pi}}}\\
            &= 0.670 \todp{3}
        \end{align*}
        \boxt{$S \approx 0.670$}

    \problem{}
        The value of the integral $\displaystyle\int_{0.2}^{0.4} f(x) \d x$ is to be estimated from information in the table below.
        \[
            \begin{array}{c c c c}\toprule
                x & 0.2 & 0.3 & 0.4 \\\cmidrule{1-4}
                f(x) & 1.2030 & 1.2441 & 1.2777\\\bottomrule
            \end{array}
        \]

        \begin{enumerate}
            \item Find the best possible estimate for the integral using the trapezium rule.
            \item Using the table of values above, find an approximate value for $f''(0.3)$ and use your answer to explain why the estimate found in part (a) is likely to be smaller than the actual value.
            \item Estimate the integral using Simpson's rule and determine the equation of the curve used in this method.
        \end{enumerate}

    \solution
        \part
            \[
                \int_{0.2}^{0.4} f(x) \d x \approx \dfrac12 \cdot \dfrac{0.4 - 0.2}{3 - 1} \cdot \Big[f(0.2) + 2f(0.3) + f(0.4)\Big] = 0.248
            \]

            \boxt{$\displaystyle\int_{0.2}^{0.4} f(x) \d x \approx 0.248$}

        \part
            Note that $f'(0.25) \approx \dfrac{f(0.3) - f(0.2)}{0.3 - 0.2} = 0.411$ and $f'(0.35) \approx \dfrac{f(0.4) - f(0.3)}{0.4 - 0.3} = 0.336$. Hence,
            \begin{align*}
                f''(0.30) \approx \dfrac{f'(0.35) - f'(0.25)}{0.35 - 0.25} = -0.75
            \end{align*}
            \boxt{$f''(0.30) \approx -0.75$}

            Since $f''(0.3) < 0$, $f(x)$ is concave downwards around $x = 0.3$. Hence, the estimate is likely to be smaller than the actual value.

        \part
            \[
                \int_{0.2}^{0.4} f(x) \d x \approx = \dfrac13 \cdot \dfrac{0.4-0.2}{3 - 1} \cdot \Big[f(0.2) + 4f(0.3) + f(0.4)\Big] = 0.249
            \]

            \boxt{$\displaystyle\int_{0.2}^{0.4} f(x) \d x \approx 0.249$}

            Let the equation of the quadratic used be $P(x) = ax^2 + bx + c$, where $a, b, c \in \R$. Since $P(0.2) = f(0.2)$, $P(0.3) = f(0.3)$ and $P(0.4) = f(0.4)$, we obtain the system
            \[
                \systeme{(0.2)^2a + 0.2 b + c = 1.2030, (0.3)^2a + 0.3 b + c = 1.2441, (0.4)^2a + 0.4 b + c = 1.2777}
            \]
            which has the unique solution $a = -0.375$, $b = 0.5985$, $c = 1.0983$. Thus, the required equation is $y = -0.375x^2 + 0.5985x + 1.0983$.

            \boxt{$y = -0.375x^2 + 0.5985x + 1.0983$}

    \problem{}
        The curve $C$ is given by $y = \dfrac1x$, where $x > 0$.

        \begin{enumerate}
            \item Apply the trapezium rule with ordinates at unit intervals to the function $f : x \mapsto \dfrac1x$, $x \in \R^+$, to show that $\ln n < \dfrac12 + \dfrac1{2n} + \displaystyle\sum_{r=2}^{n-1} \dfrac1r$ where $n \geq 3$.
            \item Obtain the area of the trapezium bounded by the axis, the lines $x = r \pm \dfrac12$, and the tangent to the curve $y = \dfrac1x$ at the point $\bp{r, \dfrac1r}$.

            Hence, show that $\displaystyle\sum_{r=2}^{n-1} \dfrac1r < \ln \bp{\dfrac{2n-1}3}$, where $n \geq 3$.
            \item From these results, obtain numerical values between which the value of $\displaystyle\sum_{r=2}^{99} \dfrac1r$ lies, and show that $4.110 < \dfrac12 + \dfrac13 + \ldots + \dfrac1{100} < 4.205$.
        \end{enumerate}

    \solution
        \part
            \begin{align*}
                \int_1^n \dfrac1x \d x &\approx \dfrac12 \cdot 1 \cdot \bs{\dfrac11 + 2\bp{\dfrac12 + \dfrac13 + \ldots + \dfrac1{n-1}} + \dfrac1n}\\
                &= \dfrac12 \bp{1 + \dfrac1n + 2\sum_{r=2}^{n-1} \dfrac1r}\\
                &= \dfrac12 + \dfrac1{2n} + \sum_{r=2}^{n-1} \dfrac1r
            \end{align*}

            Note that $\der[2]{}{x} \dfrac1x = \dfrac2{x^3} > 0$ for $x > 0$. Hence, $y = \dfrac1x$ is concave upwards. Thus,
            \[
                \dfrac12 + \dfrac1{2n} + \sum_{r=2}^{n-1} \dfrac1r > \int_1^n \dfrac1x \d x = \ln n
            \]

        \part
            Since $\der{y}{x} = -\dfrac1{x^2}$, the equation of the tangent at $x = r$ is given by
            \[
                y - \dfrac1r = -\dfrac1{r^2} (x - r) \implies y = -\dfrac1{r^2} x + \dfrac2r
            \]
            \begin{align*}
                &\text{Area of trapezium centred at $r$} \\
                &=\int_{r-1/2}^{r + 1/2} \bp{-\dfrac1{r^2} x + \dfrac2r} \d x\\
                &= \evalint{-\dfrac1{r^2} \cdot \dfrac12 x^2 + \dfrac2r x}{r-1/2}{r + 1/2}\\
                &= \bs{-\dfrac1{2r^2} \bp{r + \dfrac12}^2 + \dfrac2r \bp{r + \dfrac12}} - \bs{-\dfrac1{r^2} \cdot \dfrac12 \bp{r - \dfrac12}^2 + \dfrac2r \bp{r - \dfrac12}}\\
                &= \dfrac1{2r^2} \bs{\bp{r-\dfrac12}^2 - \bp{r + \dfrac12}^2} + \dfrac2r \bs{\bp{r + \dfrac12}- \bp{r - \dfrac12}}\\
                &= \dfrac1{2r^2} \cdot {2r \cdot -1} + \dfrac2r \cdot 1\\
                &= \dfrac{-1}r + \dfrac2r\\
                &= \dfrac1r
            \end{align*}

            \boxt{The area of the trapezium is $\dfrac1r$ units$^2$.}

            Observe that the area of the trapezium centred at $r$ is less than the area under the curve $y = \dfrac1x$ from $r - \dfrac12$ to $r + \dfrac12$. That is,
            \[
                \dfrac1r < \int_{r - 1/2}^{r + 1/2} \dfrac1x \d x = \ln\of{r + \dfrac12} - \ln\of{r - \dfrac12}
            \]

            Summing from $r = 2$ to $n-1$,
            \begin{align*}
                \sum_{r = 2}^{n-1} \dfrac1r &< \sum_{r=2}^{n-1} \bs{\ln\of{r + \dfrac12} - \ln\of{r - \dfrac12}}\\
                &= \sum_{r=2}^{n-1} \ln\of{r + \dfrac12} - \sum_{r=2}^{n-1} \ln\of{r - \dfrac12}\\
                &= \sum_{r=3}^{n} \ln\of{r - \dfrac12} - \sum_{r=2}^{n-1} \ln\of{r - \dfrac12}\\
                &= \ln\of{n - \dfrac12} - \ln\of{2 - \dfrac12}\\
                &= \ln\of{\dfrac{n - 1/2}{3/2}}\\
                &= \ln\of{\dfrac{2n - 1}{3}}
            \end{align*}

        \part
            Taking $n = 100$, we have
            \[
                \dfrac12 + \dfrac1{2 \cdot 100} + \sum_{r=2}^{100 - 1} \dfrac1{r} > \ln 100 \implies \sum_{r=2}^{99} \dfrac1{r} > \ln 100 - \dfrac12 - \dfrac1{200} = 4.100
            \]
            We also have
            \[
                \sum_{r = 2}^{100 - 1} \dfrac1r <\ln\of{\dfrac{2 \cdot 100 - 1}{3}} \implies \sum_{r = 2}^{100 - 1} \dfrac1r < \ln{\dfrac{199}3} = 4.195
            \]
            Putting both inequalities together, we obtain
            \boxt{$4.100 < \displaystyle\sum_{r = 2}^{99} \dfrac1r < 4.195$}

            Adding $\dfrac1{100} = 0.01$ to all sides of the inequality, we see that
            \[
                4.110 < \displaystyle\sum_{r = 2}^{100} \dfrac1r < 4.205
            \]
\end{document}