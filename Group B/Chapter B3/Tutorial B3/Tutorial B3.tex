\documentclass{echw}

\title{Tutorial B3\\Functions}
\author{Eytan Chong}
\date{2024-03-18}

\begin{document}
    \problem{}
        Sketch the following graphs and determine whether each graph represents a function for the given domain.

        \begin{enumerate}
            \item $y = \sqrt{9-x^2}, \, -3 \leq x \leq 3$
            \item $x = (y-4)^2, \, y \in \R$
        \end{enumerate}

    \solution
        \part
            \begin{center}
                \begin{tikzpicture}[trim axis left, trim axis right]
                    \begin{axis}[
                        domain = -3:3,
                        samples = 101,
                        axis y line=middle,
                        axis x line=middle,
                        xtick = {-3, 3},
                        ytick = {3},
                        xlabel = {$x$},
                        ylabel = {$y$},
                        xmin=-4,
                        xmax=4,
                        ymax=4,
                        ymin=-1,
                        legend cell align={left},
                        legend pos=outer north east,
                        after end axis/.code={
                            \path (axis cs:0,0) 
                                node [anchor=north east] {$O$};
                            }
                        ]
                        \addplot[plotRed] {sqrt(9-x^2)};
            
                        \addlegendentry{$y = \sqrt{9-x^2}$};
                    \end{axis}
                \end{tikzpicture}
            \end{center}

            $y = \sqrt{9-x^2}$ passes the vertical line test for $-3 \leq x \leq 3$ and is hence a function.

            \boxt{$y = \sqrt{9-x^2}, \, -3 \leq x \leq 3$ is a function.}

        \part
            \begin{center}
                \begin{tikzpicture}[trim axis left, trim axis right]
                    \begin{axis}[
                        domain = 0:30,
                        samples = 101,
                        axis y line=middle,
                        axis x line=middle,
                        xtick = {16},
                        ytick = {4},
                        xlabel = {$x$},
                        ylabel = {$y$},
                        xmin=-3,
                        legend cell align={left},
                        legend pos=outer north east,
                        after end axis/.code={
                            \path (axis cs:0,0) 
                                node [anchor=north east] {$O$};
                            }
                        ]

                        \addplot[plotRed] {sqrt(x) + 4};

                        \addplot[plotRed] {-sqrt(x) + 4};
            
                        \addlegendentry{$x = (y-4)^2$};
                    \end{axis}
                \end{tikzpicture}
            \end{center}

            $x = (y-4)^2$ does not pass the vertical line test for $y \in \R$ and is hence not a function.

            \boxt{$x = (y-4)^2, \, y \in \R$ is not a function.}

    \problem{}
        Sketch the graph and find the range for each the following functions.

        \begin{enumerate}
            \item $g \colon x \mapsto x^2 - 4x + 2, \, 1 < x \leq 5$
            \item $h \colon x \mapsto \abs{2x-3}, \, x < 3$
        \end{enumerate}

    \solution
        \part
            \begin{center}
                \begin{tikzpicture}[trim axis left, trim axis right]
                    \begin{axis}[
                        domain = 1:5,
                        samples = 101,
                        axis y line=middle,
                        axis x line=middle,
                        ytick = \empty,
                        xtick = {3.414},
                        xticklabels = {$2 + \sqrt2$},
                        xlabel = {$x$},
                        ylabel = {$y$},
                        xmin=0,
                        xmax=6,
                        ymax=9,
                        ymin=-4,
                        legend cell align={left},
                        legend pos=outer north east,
                        after end axis/.code={
                            \path (axis cs:0,0) 
                                node [anchor=east] {$O$};
                            }
                        ]
                        \addplot[plotRed] {x^2 - 4*x + 2};
            
                        \addlegendentry{$y = g(x)$};

                        \draw (1, -1) circle[radius=2.5 pt];
                        \node[anchor=north] at (0.7, -1.1) {$(1, -1)$};

                        \fill (5, 7) circle[radius=2.5pt] node[anchor=east] {$(5, 7)$};

                        \fill (2, -2) circle[radius=2.5 pt] node[anchor=north] {$(2, -2)$};
                    \end{axis}
                \end{tikzpicture}
            \end{center}

            \boxt{$\ran{g} = [-2, 7)$}

        \part
            \begin{center}
                \begin{tikzpicture}[trim axis left, trim axis right]
                    \begin{axis}[
                        domain = -2:3,
                        samples = 101,
                        axis y line=middle,
                        axis x line=middle,
                        ytick = {3},
                        xtick = {1.5},
                        xticklabels = {$\tfrac32$},
                        xlabel = {$x$},
                        ylabel = {$y$},
                        xmax=4,
                        legend cell align={left},
                        legend pos=outer north east,
                        after end axis/.code={
                            \path (axis cs:0,0) 
                                node [anchor=north] {$O$};
                            }
                        ]
                        \addplot[plotRed] {abs(2 * x - 3)};
            
                        \addlegendentry{$y = h(x)$};

                        \draw (3, 3) circle[radius=2.5pt] node[anchor=south] {$(3, 3)$};
                    \end{axis}
                \end{tikzpicture}
            \end{center}

            \boxt{$\ran{h} = [\,0, \infty)$}

    \problem{}
        For each of the following functions, sketch its graph and determine if the function is one-one. If it is, find its inverse in a similar form.

        \begin{enumerate}
            \item $g \colon x \mapsto \abs{x} - 2, \, x \in \R$
            \item $h \colon x \mapsto x^2 + 2x + 5, \, x \leq -2$
        \end{enumerate}

    \solution
        \part
            \begin{center}
                \begin{tikzpicture}[trim axis left, trim axis right]
                    \begin{axis}[
                        domain = -4:4,
                        samples = 101,
                        axis y line=middle,
                        axis x line=middle,
                        ytick = {-2},
                        xtick = {-2, 2},
                        xlabel = {$x$},
                        ylabel = {$y$},
                        ymin=-2.5,
                        legend cell align={left},
                        legend pos=outer north east,
                        after end axis/.code={
                            \path (axis cs:0,0) 
                                node [anchor=north east] {$O$};
                            }
                        ]
                        \addplot[plotRed] {abs(x) - 2};
            
                        \addlegendentry{$y = g(x)$};
                    \end{axis}
                \end{tikzpicture}
            \end{center}

            $y=g(x)$ does not pass the horizontal line test. Hence, $g$ is not one-one.

            \boxt{$g$ is not one-one.}

        \part
            \begin{center}
                \begin{tikzpicture}[trim axis left, trim axis right]
                    \begin{axis}[
                        domain = -7:-2,
                        samples = 101,
                        axis y line=middle,
                        axis x line=middle,
                        ytick = \empty,
                        xtick = \empty,
                        xlabel = {$x$},
                        ylabel = {$y$},
                        ymin=0,
                        xmax=1,
                        legend cell align={left},
                        legend pos=outer north east,
                        after end axis/.code={
                            \path (axis cs:0,0) 
                                node [anchor=north] {$O$};
                            }
                        ]
                        \addplot[plotRed] {x^2 + 2*x + 5};
            
                        \addlegendentry{$y = h(x)$};

                        \fill (-2, 5) circle[radius=2.5 pt] node[anchor=west] {$(-2, 5)$};
                    \end{axis}
                \end{tikzpicture}
            \end{center}

            $y=h(x)$ passes the horizontal line test. Hence, $h$ is one-one.
            
            \boxt{$h$ is not one-one.}

            Note that $y = h(x) \implies x = \inv h(y)$. Now consider $y = h(x)$.
            \begin{alignat*}{2}
                &&y &= h(x)\\
                \implies&&y &= x^2 + 2x + 5\\
                \implies&&y &= x^2 + 2x + 1 + 4\\
                \implies&&y &= (x+1)^2 + 4\\
                \implies&&(x+1)^2 &= y - 4\\
                \implies&& x+1 &= \pm\sqrt{y-4}
            \end{alignat*}
            Since $x \leq -2$, we have $x+1 \leq -1$. Hence, we reject $x + 1 = \sqrt{y-4}$ since $\sqrt{y-4} \geq 0$.
            \begin{alignat*}{2}
                \implies&&x + 1 &= -\sqrt{y-4}\\
                \implies&& x &= -1-\sqrt{y-4}
            \end{alignat*}
            Hence, $\inv h(x) = -1 -\sqrt{x-4}$. Note that $\dom{\inv h} = \ran h = [5, \infty)$. Hence,

            \boxt{$\inv h \colon x \mapsto -1-\sqrt{x-4}, \, x \geq 5$}

    \problem{}
        The function $f$ is defined by
        \[
            f \colon x \mapsto x + \dfrac1{x}, \, x \neq 0
        \]

        \begin{enumerate}
            \item Sketch the graph of $f$ and explain why $\inv{f}$ does not exist.
            \item The function $h$ is defined by $h \colon x \mapsto f(x), \, x \in \R, \, x \geq \a$, where $\a \in \R^+$. Find the smallest value of $\a$ such that the inverse function of $h$ exists.
        \end{enumerate}

        Using this value of $\a$,
        
        \begin{enumerate}
            \setcounter{enumi}{2}
            \item State the range of $h$.
            \item Express $\inv{h}$ in a similar form and sketch on a single diagram, the graphs of $h$ and $\inv{h}$, showing clearly their geometrical relationship.
        \end{enumerate}

    \solution
        \part
            \begin{center}
                \begin{tikzpicture}[trim axis left, trim axis right]
                    \begin{axis}[
                        domain = -4:4,
                        samples = 81,
                        axis y line=middle,
                        axis x line=middle,
                        ytick = \empty,
                        xtick = \empty,
                        xlabel = {$x$},
                        ylabel = {$y$},
                        legend cell align={left},
                        legend pos=outer north east,
                        after end axis/.code={
                            \path (axis cs:0,0) 
                                node [anchor=south east] {$O$};
                            }
                        ]
                        \addplot[plotRed, unbounded coords=jump] {x + 1/x};
            
                        \addlegendentry{$y = f(x)$};

                        \addplot[dotted, thick] {x};

                        \node[anchor=south west, rotate=16] at (-4, -4) {$y=x$};

                        \fill (-1, -2) circle[radius=2.5 pt];
                        
                        \node[anchor=north, fill=white, opacity = 0.6, text opacity=1] at (-1.2, -2.2) {$(-1, -2)$};

                        \fill (1, 2) circle[radius=2.5 pt] node[anchor=south] {$(1, 2)$};
                    \end{axis}
                \end{tikzpicture}
            \end{center}

            $y=f(x)$ does not pass the horizontal line test. Hence, $f$ is not one-one. Hence, $\inv f$ does not exist.

        \part
            Consider $f'(x) = 0$ for $x > 0$.
            \begin{alignat*}{2}
                &&f'(x) &= 0\\
                \implies&&1 - \dfrac1{x^2} &= 0\\
                \implies&&x^2 &= 1\\
                \implies&&x &= 1 \reject{x = -1 \because x > 0}
            \end{alignat*}

            Looking at the graph of $y=f(x)$, we see that $f(x)$ achieves a minimum at $x = 1$. Hence, $f$ is increasing for all $x \geq 1$. Thus, the smallest value of $\a$ is 1.

            \boxt{$\min \a = 1$}

        \part
            Note $f(1) = 2$. Hence, from the graph,

            \boxt{$\ran h = [2, \infty)$}

        \part
            Note that $y = h(x) \implies x = \inv h(y)$. Now consider $y = h(x)$.
            \begin{alignat*}{2}
                &&y &= h(x)\\
                \implies&&y &= x + \dfrac1{x}\\
                \implies&&xy &= x^2 + 1\\
                \implies&&x^2 - yx + 1 &= 0\\
                \implies&&x &= \dfrac12 (y \pm \sqrt{y^2 - 4})
            \end{alignat*}
            Note that $f(2) = \dfrac52$. Since $2 = \dfrac12 \bp{\tfrac52 + \sqrt{\bp{\tfrac52}^2 - 4}}$ and $2 \neq \dfrac12 \bp{\tfrac52 - \sqrt{\bp{\tfrac52}^2 - 4}}$, we reject $x = \dfrac12 (y - \sqrt{y^2 - 4})$. Hence, $\inv h(x) = \dfrac12 (x + \sqrt{x^2 - 4})$. Note that $\dom{\inv f} = \ran f = [2, \infty)$. Thus,

            \boxt{$\inv h \colon x \mapsto \dfrac12\bp{x + \sqrt{x^2-4} }, \, x \geq 2$}
    
    \problem{}
        The function $f$ is defined as follows:
        \[
            f \colon x \mapsto x^3 + x - 7, \, x \in \R
        \]

        \begin{enumerate}
            \item By using a graphical method or otherwise, show that the inverse of $f$ exists.
            \item Solve exactly the equation $\inv f(x) = 0$. Sketch the graph of $\inv f$ together with the graph of $f$ on the same diagram.
            \item Find, in exact form, the coordinates of the intersection point(s) of the graphs of $f$ and $\inv f$.
            \item Given that the gradient of the tangent to the curve with equation $y = \inv f(x)$ is $\dfrac14$ at the point with $x = p$, find the possible values of $p$.
        \end{enumerate}

    \solution
        \part
            \begin{center}
                \begin{tikzpicture}[trim axis left, trim axis right]
                    \begin{axis}[
                        domain = -2.5:2.5,
                        samples = 161,
                        axis y line=middle,
                        axis x line=middle,
                        ytick = {-7},
                        xtick = \empty,
                        xlabel = {$x$},
                        ylabel = {$y$},
                        legend cell align={left},
                        legend pos=outer north east,
                        after end axis/.code={
                            \path (axis cs:0,0) 
                                node [anchor=north east] {$O$};
                            }
                        ]
                        \addplot[plotRed] {x^3 + x - 7};

                        \addlegendentry{$y = f(x)$};
                    \end{axis}
                \end{tikzpicture}
            \end{center}

            $y = f(x)$ passes the horizontal line test. Hence, $f$ is one-one. Thus, $\inv f$ exists.

        \part
            \begin{alignat*}{2}
                &&\inv f(x) &= 0\\
                \implies&&x &= f(0)\\
                \implies&&x &= 0^3 + 0 - 7\\
                && &= -7
            \end{alignat*}

            \boxt{$x = -7$}

            \begin{center}
                \begin{tikzpicture}[trim axis left, trim axis right]
                    \begin{axis}[
                        domain = -2.5:2.5,
                        samples = 161,
                        axis y line=middle,
                        axis x line=middle,
                        ytick = {-7},
                        xtick = {-7},
                        xlabel = {$x$},
                        ylabel = {$y$},
                        ymin=-10,
                        ymax=8,
                        xmin=-10,
                        xmax=8,
                        legend cell align={left},
                        legend pos=outer north east,
                        after end axis/.code={
                            \path (axis cs:0,0) 
                                node [anchor=south east] {$O$};
                            }
                        ]
                        \addplot[plotRed] {x^3 + x - 7};

                        \addlegendentry{$y = f(x)$};

                        \plot[plotBlue, samples=161] (\x^3 + \x -7, \x);

                        \addlegendentry{$y = \inv f(x)$};

                        \addplot[dotted, thick, domain=-10:10] {x};

                        \node[rotate=45, anchor=south] at (-7, -7) {$y=x$};
                    \end{axis}
                \end{tikzpicture}
            \end{center}

        \part
            Let ($\a$, $\b$) be the coordinates of the intersection between $f(x)$ and $\inv f$. From the graph, we see that $\a=\b$, hence $f(\a) = \a$.
            \begin{alignat*}{2}
                &&f(\a) &= \a\\
                \implies&&\a^3 + \a - 7 &= \a\\
                \implies&&\a^3 &= 7\\
                \implies&&\a &= \sqrt[3]{7}
            \end{alignat*}

            \boxt{$(\sqrt[3]{7}, \sqrt[3]{7})$}
            
        \part
            \begin{alignat*}{2}
                &&[\inv f(x)]' &= \dfrac1{f'(\inv f(x))}\\
                \implies&&\evalder{[\inv f(x)]'}{x = p} &= \evalder{\dfrac1{f'(\inv f(x))}}{x=p}\\
                \implies&&\dfrac14 &= \evalder{\dfrac1{f'(\inv f(x))}}{x=p}\\
                \implies&&\evalder{f'(\inv f(x))}{x=p} &= 4
            \end{alignat*}
            Note that $f'(x) = 3x^2 + 1$.
            \begin{alignat*}{2}
                \implies&&\evalder{\bp{3\inv f(x)^2 + 1}}{x=p} &= 4\\
                \implies&&3\inv f(p)^2 + 1 &= 4\\
                \implies&&\inv f(p)^2 &= 1\\
                \implies&&\inv f(p) &= \pm 1
            \end{alignat*}

            \case{1}{$\inv f(p) = 1$}
            \begin{alignat*}{2}
                &&\inv f(p) &= 1\\
                \implies&&p &= f(1)\\
                \implies&&p &= 1^3 + 1 - 7\\
                && &= -5
            \end{alignat*}

            \case{2}{$\inv f(p) = -1$}
            \begin{alignat*}{2}
                &&\inv f(p) &= -1\\
                \implies&&p &= f(-1)\\
                \implies&&p &= (-1)^3 - 1 - 7\\
                && &= -9
            \end{alignat*}

            \boxt{$p = -9 \lor -5$}


    \problem{}
        The functions $g$ and $h$ are defined as follows:
        \begin{alignat*}{2}
            g & \colon x \mapsto \ln{x+2}, &&\qquad x \in (-1, 1)\\
            h & \colon x \mapsto x^2 - 2x - 1, &&\qquad x \in \R^+
        \end{alignat*}

        \begin{enumerate}
            \item Sketch, on separate diagrams, the graphs of $g$ and $h$.
            \item Determine whether the composite function $gh$ exists.
            \item Give the rule and domain of the composite function $hg$ and find its range.
            \item The image of $a$ under the composite function $hg$ is -1.5. Find the value of $a$.
        \end{enumerate}

    \solution
        \part
            \begin{center}
                \begin{tikzpicture}[trim axis left, trim axis right]
                    \begin{axis}[
                        domain = -1:1,
                        samples = 161,
                        axis y line=middle,
                        axis x line=middle,
                        ytick = {ln(2)},
                        yticklabels = {$\ln 2$},
                        xtick = {-1},
                        xlabel = {$x$},
                        ylabel = {$y$},
                        xmin=-1.5,
                        xmax=1.5,
                        ymax=1.5,
                        ymin=-0.5,
                        legend cell align={left},
                        legend pos=outer north east,
                        after end axis/.code={
                            \path (axis cs:0,0) 
                                node [anchor=north east] {$O$};
                            }
                        ]
                        \addplot[plotRed] {ln(x+2)};

                        \addlegendentry{$y = g(x)$};
                        
                        \draw (-1, 0) circle[radius=2.5pt];

                        \draw (1, 1.0986) circle[radius=2.5 pt] node[anchor=south east] {$(1, \ln 3)$};
                    \end{axis}
                \end{tikzpicture}
            \end{center}

            \begin{center}
                \begin{tikzpicture}[trim axis left, trim axis right]
                    \begin{axis}[
                        domain = 0:4,
                        samples = 161,
                        axis y line=middle,
                        axis x line=middle,
                        xtick = {1 + sqrt(2)},
                        xticklabels = {$1 + \sqrt2$},
                        ytick = {-1},
                        xlabel = {$x$},
                        ylabel = {$y$},
                        xmin=-1,
                        ymin=-3,
                        legend cell align={left},
                        legend pos=outer north east,
                        after end axis/.code={
                            \path (axis cs:0,0) 
                                node [anchor=south east] {$O$};
                            }
                        ]
                        \addplot[plotRed] {x^2 - 2*x - 1};

                        \addlegendentry{$y = h(x)$};
                        
                        \draw (0, -1) circle[radius=2.5pt];

                        \fill (1, -2) circle[radius=2.5 pt] node[anchor=north] {$(1, -2)$};
                    \end{axis}
                \end{tikzpicture}
            \end{center}

        \part
            Observe that $\ran h = [-2, \infty)$ and $\dom g = (-1, 1)$. Hence, $\ran h \nsubseteq \dom g$. Thus, $gh$ does not exist.

            \boxt{$gh$ does not exist.}

        \part
            \begin{align*}
                hg(x) &= h(\ln{x+2})\\
                &= \ln{x+2}^2 - 2\ln{x+2} - 1
            \end{align*}
            Note that $\dom{hg} = \dom{g} = (-1, 1)$.

            \boxt{$hg \colon x \mapsto \ln{x+2}^2 - 2\ln{x+2} - 1, \, x \in (-1, 1)$}

            Observe that $h$ is decreasing on the interval $(0, 1]$ and increasing on the interval $[1, \infty)$. Note that $\ran g = (0, \ln 3)$. Hence,
            \begin{align*}
                \ran{hg} &= [-2, \max \bc{h(0), h(\ln 3)})\\
                &= [-2, -1)
            \end{align*}

        \part
            Note that $h(x) = (x-1)^2 -2$. Hence, $\inv h(x) = 1 + \sqrt{x+2}$ (we reject $\inv h(x) = 1 - \sqrt{x+2}$ since $\ran{\inv h} = \dom h = \R^+$). Also note that $\inv g = e^x - 2$.
            \begin{alignat*}{2}
                &&hg(a) &= -1.5\\
                \implies&&g(a) &= \inv{h}(-1.5)\\
                && &= 1 + \sqrt{-1.5 + 2}\\
                && &= 1 + \dfrac1{\sqrt{2}}\\
                \implies&&a &= \inv{g}\bp{1 + \dfrac1{\sqrt{2}}}\\
                && &= e^{1 + \tfrac1{\sqrt{2}}} - 2
            \end{alignat*}

            \boxt{$a = e^{1 + \tfrac1{\sqrt{2}}} - 2$}

    \problem{}
        The functions $f$ and $g$ are defined as follows:
        \begin{alignat*}{2}
            f & \colon x \mapsto 3 - x, &&\qquad x \in \R\\
            g & \colon x \mapsto \dfrac4{x}, &&\qquad x \in \R, \, x \neq 0
        \end{alignat*}

        \begin{enumerate}
            \item Show that the composite function $fg$ exists and express the definition of $fg$ in a similar form. Find the range of $fg$.
            \item Find, in similar form, $g^2$ and $g^3$, and deduce $g^{2017}$.
            \item Find the set of values of $x$ for which $g(x) = \inv g(x)$.
        \end{enumerate}

    \solution
        \part
            Note that $\ran g = \R \setminus \bc{0}$ and $\dom g = \R$. Hence, $\ran g \subseteq \dom g$. Thus, $fg$ exists.
            \begin{align*}
                fg(x) &= f\bp{\dfrac4x}\\
                &= 3 - \dfrac4x
            \end{align*}
            Observe that $\dom{fg} = \dom{g} = \R\setminus \bc{0}$.

            \boxt{$fg \colon x \mapsto 3 - \dfrac4x, \, x \in \R \setminus \bc{0}$}

            Since $\dfrac4x$ can take on any value except 0, then $fg(x) = 3 - \dfrac4x$ can take on any value except 3.

            \boxt{$\ran{fg} = \R \setminus \bc{3}$}

        \part
            \begin{align*}
                g^2(x) &= g\bp{\dfrac4x}\\
                &= \dfrac4{\tfrac4x}\\
                &= x
            \end{align*}

            \boxt{$g^2 \colon x \mapsto x, \, x \in \R \setminus \bc{0}$}

            \begin{align*}
                g^3(x) &= g(g^2(x))\\
                &= g(x)\\
                &= \dfrac4x
            \end{align*}

            \boxt{$g^3 \colon x \mapsto \dfrac4x, \, x \in \R \setminus \bc{0}$}

            \begin{align*}
                g^{2017} &= g^{2016}(g(x))\\
                &= \bp{g^2}^{1008}(g(x))\\
                &= g(x)\\
                &= \dfrac4x
            \end{align*}

            \boxt{$g^{2017} \colon x \mapsto \dfrac4x, \, x \in \R \setminus \bc{0}$}

        \part
            \begin{alignat*}{2}
                &&g(x) &= \inv g(x)\\
                \implies&&g^2(x) &= x
            \end{alignat*}

            From the definition of $g^2(x)$, we know that $g^2(x) = x$ for all $x$ in $\dom{g^2}$.

            \boxt{$\R \setminus \bc{0}$}

    \problem{}
        The function $f$ is defined by
        \[
            f(x) =
            \begin{cases}
                2x+1, & 0 \leq x < 2\\
                (x-4)^2+1, & 2 \leq x < 4
            \end{cases}
        \]

         It is further given that $f(x) = f(x+4)$ for all real values of $x$.

        \begin{enumerate}
            \item Find the values of $f(1)$ and $f(5)$ and hence explain why $f$ is not one-one.
            \item Sketch the graph of $y = f(x)$ for $-4 \leq x < 8$.
            \item Find the range of $f$ for $-4 \leq x < 8$.
        \end{enumerate}

    \solution
        \part
            \begin{alignat*}{2}
                &&f(1) &= 2(1) + 1\\
                && &= 3\\
                &&f(5) &= f(1 + 4)\\
                && &= f(1)\\
                && &= 3
            \end{alignat*}

            \boxt{$f(1) = 3$, $f(5) = 3$}

            Since $f(1) = f(5)$, but $1 \neq 5$, $f$ is not one-one.

        \part
            \begin{center}
                \begin{tikzpicture}[trim axis left, trim axis right]
                    \begin{axis}[
                        samples = 161,
                        axis y line=middle,
                        axis x line=middle,
                        xtick = {-4, -2, 2, 4, 6, 8},
                        ytick = {1},
                        xlabel = {$x$},
                        ylabel = {$y$},
                        xmin=-5,
                        xmax=9,
                        ymin=-1,
                        ymax=6,
                        legend cell align={left},
                        legend pos=outer north east,
                        after end axis/.code={
                            \path (axis cs:0,0) 
                                node [anchor=north east] {$O$};
                            }
                        ]

                        \addplot[plotRed, domain=-4:-2] {2*(x+4)+1};

                        \addplot[plotRed, domain=-2:0] {(x)^2 + 1};

                        \addplot[plotRed, domain=0:2] {2*x+1};

                        \addplot[plotRed, domain=2:4] {(x-4)^2 + 1};

                        \addplot[plotRed, domain=4:6] {2*(x-4)+1};

                        \addplot[plotRed, domain=6:8] {(x-8)^2 + 1};

                        \addlegendentry{$y = f(x)$};

                        \fill (-4, 1) circle[radius=2.5 pt] node[anchor=north west] {$(-4, 1)$};

                        \draw (8, 1) circle[radius=2.5 pt] node[anchor=north east] {$(8, 1)$};
                        
                        \draw[dotted, thick] (-4, 5) -- (8, 5);

                        \node[anchor=south] at (4, 5) {$y = 5$};
                    \end{axis}
                \end{tikzpicture}
            \end{center}

        \part
            \boxt{$\ran f = [1, 5]$}

    \problem{}
        \begin{enumerate}
            \item The function $f$ is given by $f \colon x \mapsto 1 + \sqrt{x}$ for $x \in \R^+$. \begin{enumerate}
                \item Find $\inv f(x)$ and state the domain of $\inv f$.
                \item Find $f^2(x)$ and the range of $f^2$.
                \item Show that if $f^2(x) = x$ then $x^3 - 4x^2 + 4x - 1 = 0$. Hence, find the value of $x$ for which $f^2(x) = x$. Explain why this value of $x$ satisfies the equation $f(x) = \inv f(x)$.
            \end{enumerate}
            \item The function $g$, with domain the set of non-negative integers, is given by
            \[
                g(n) =
                \begin{cases}
                    1, & n = 0\\
                    2 + g\bp{\dfrac12 n}, & n \text{ even}\\
                    1 + g(n-1), & n \text{ odd}
                \end{cases}
            \]
            \begin{enumerate}
                \item Find $g(4)$, $g(7)$ and $g(12)$.
                \item Does $g$ have an inverse? Justify your answer.
            \end{enumerate}
        \end{enumerate}

    \solution
        \part
            \subpart
                Let $y = f(x)$. Then $x = \inv f(y)$.
                \begin{alignat*}{2}
                    &&y &= f(x)\\
                    \implies&&y &= 1 + \sqrt{x}\\
                    \implies&&\sqrt{x} &= y -1\\
                    \implies&&x &= (y-1)^2
                \end{alignat*}

                \boxt{$\inv f(x) = (x-1)^2$}

                Observe that $\dom{\inv f} = \ran f = (1, \infty)$.

                \boxt{$\dom{\inv f} = (1, \infty)$}

            \subpart
                \begin{align*}
                    f^2(x) &= f(1+\sqrt{x})\\
                    &= 1 + \sqrt{1 + \sqrt{x}}
                \end{align*}

                \boxt{$f^2(x) = 1 + \sqrt{1 + \sqrt{x}}$}

                Observe that $\sqrt{1 + \sqrt{x}} > 1$. Hence, $1 + \sqrt{1 + \sqrt{x}} > 1 + 1 = 2$.

                \boxt{$\ran{f^2} = (2, \infty)$}

            \subpart
                \begin{alignat*}{2}
                    &&f^2(x) &= x\\
                    \implies&& 1 + \sqrt{1 + \sqrt{x}} &= x\\
                    \implies&& \sqrt{1 + \sqrt{x}} &= x - 1\\
                    \implies&& 1 + \sqrt{x} &= (x - 1)^2\\
                    \implies&& \sqrt{x} &= (x - 1)^2 - 1\\
                    \implies&& &= x(x-2)\\
                    \implies&& x &= (x(x-2))^2\\
                    \implies&& x(x-2)^2 &= 1 \qquad (\because x \neq 0)\\
                    \implies&& x\bp{x^2 -4x + 4 } &= 1 \\
                    \implies&& x^3 -4x^2 + 4x  &= 1 \\
                    \implies&& x^3 -4x^2 + 4x -1 &= 0
                \end{alignat*}
                Hence, if $f^2(x) = x$, then $x^3 -4x^2 + 4x -1 = 0$.
                \begin{alignat*}{2}
                    &&f^2(x) &= x\\
                    \implies&&x^3 -4x^2 + 4x -1 &= 0 \\
                    \implies&&(x-1)\bp{x^2-3x+1} &= 0
                \end{alignat*}
                Hence, $x = 1$ or $\bp{x^2-3x+1} = 0$. However, since $x \geq 2$, $x$ cannot be 1. We thus consider $\bp{x^2-3x+1} = 0$.
                \begin{alignat*}{2}
                    &&\bp{x^2-3x+1} &= 0\\
                    \implies&&x &= \dfrac{3 \pm \sqrt{5}}{2}
                \end{alignat*}
                Observe that $\dfrac{3 - \sqrt{5}}{2} < 2$ and $\dfrac{3 + \sqrt{5}}{2} > 2$. Thus, we reject $x = \dfrac{3 - \sqrt{5}}{2}$ and take $x = \dfrac{3 + \sqrt{5}}{2}$.

                \boxt{$x = \dfrac{3 + \sqrt{5}}{2}$}

                Consider $f(x) = \inv f(x)$. Applying $f$ on both sides of the equation, we have $f^2(x) = f(x)$. Since $x = \dfrac{3 + \sqrt{5}}{2}$ satisfies $f^2(x) = f(x)$, it also satisfies $f(x) = \inv f(x)$.

        \part
            \subpart
                \begin{alignat*}{2}
                    &&g(4) &= 2 + g(2)\\
                    && &= 2 + 2 + g(1)\\
                    && &= 2 + 2 + 1 + g(0)\\
                    && &= 2 + 2 + 1 + 1\\
                    && &= 6\\
                    &&g(7) &= 1 + g(6)\\
                    && &= 1 + 2 + g(3)\\
                    && &= 1 + 2 + 1 + g(2)\\
                    && &= 1 + 2 + 1 + (g(4)-2)\\
                    && &= 1 + 2 + 1 + 6 - 2\\
                    && &= 8\\
                    &&g(12) &= 2 + g(6)\\
                    && &= 2 + (g(7) - 1)\\
                    && &= 2 + 8 - 1\\
                    && &= 9
                \end{alignat*}

                \boxt{$g(4) = 6$, $g(7) = 8$, $g(12) = 9$}

            \subpart
                Consider $g(5)$ and $g(6)$.
                \begin{alignat*}{2}
                    && g(5) &= 1 + g(4)\\
                    && &= 1 + 6\\
                    && &= 7\\
                    && g(6) &= g(7) - 1\\
                    && &= 8 - 1\\
                    && &= 7
                \end{alignat*}
                Since $g(5) = g(6)$, but $5 \neq 6$, $g$ is not one-one. Hence, $\inv g$ does not exist.

                \boxt{$g$ does not have an inverse.}
\end{document}