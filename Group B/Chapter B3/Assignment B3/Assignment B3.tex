\documentclass{echw}

\title{Assignment B3\\Functions}
\author{Eytan Chong}
\date{2024-03-24}

\begin{document}
    \problem{}
        Functions $f$ and $g$ are defined as follows:

        \begin{alignat*}{2}
            f & \colon x \mapsto (x-3)^2 + 6, &&\qquad x \in \R, \, x \leq 2\\
            g & \colon x \mapsto \ln{x-2}, &&\qquad x \in \R, \, x > 3
        \end{alignat*}

        \begin{enumerate}
            \item Show that $\inv f$ exists and define $\inv f$ in a similar form.
            \item Sketch, on the same diagram, the graphs of $f$, $\inv f$ and $f \inv f$.
            \item Find $fg$ and $gf$ if they exist, and find their ranges (where applicable).
        \end{enumerate}

    \solution
        \part
            Note that $f' = 2(x-3) < 0$ for all $x \leq 2$. Thus $f$ is strictly decreasing. Since $f$ is also continuous, $f$ is one-one. Thus, $\inv f$ exists.

            Let $y = f(x) \implies x = \inv f(y)$.

            \begin{alignat*}{2}
                &&y &= f(x)\\
                \implies&&y &= (x-3)^2 + 6\\
                \implies&&(x-3)^2 &= y-6\\
                \implies&&x-3 &= -\sqrt{y-6} \reject{x -3 = \sqrt{y-6} \because x -3 < 0}\\
                \implies&&x &= 3 - \sqrt{y-6}
            \end{alignat*}

            Hence, $\inv f(x) = 3 - \sqrt{x-6}$. Observe that $\dom{\inv f} = \ran f = [f(2), \infty) = [7, \infty)$.

            \boxt{$\inv f \colon x \mapsto 3 - \sqrt{x-6}, \, x \in \R, \, x \geq 7$}

        \part
            \begin{center}
                \begin{tikzpicture}[trim axis left, trim axis right]
                    \begin{axis}[
                        samples = 101,
                        axis y line=middle,
                        axis x line=middle,
                        ytick = {15},
                        xtick = {15},
                        xlabel = {$x$},
                        ylabel = {$y$},
                        ymax=20,
                        ymin=-3,
                        xmin=-3,
                        xmax=20,
                        legend cell align={left},
                        legend pos=outer north east,
                        after end axis/.code={
                            \path (axis cs:0,0) 
                                node [anchor=north east] {$O$};
                            }
                        ]

                        \addplot[plotRed, domain=-3:2] {(x-3)^2 + 6};
            
                        \addlegendentry{$y = f(x)$};

                        \addplot[plotBlue, domain=7:30] {3 - sqrt(x-6)};

                        \addlegendentry{$y = \inv f(x)$};

                        \addplot[plotGreen, domain=7:30] {x};

                        \addlegendentry{$y = f\inv f(x)$};

                        \draw (2, 7) circle[radius=2.5 pt] node[anchor=north] {$(2, 7)$};

                        \draw (7, 2) circle[radius=2.5 pt] node[anchor=east] {$(7, 2)$};

                        \draw (7, 7) circle[radius=2.5 pt] node[anchor=north] {$(7, 7)$};
                    \end{axis}
                \end{tikzpicture}
            \end{center}

        \part
            Note that $\ran g = (0, \infty)$ and $\dom f = (-\infty, 2]$. Hence, $\ran g \nsubseteq \dom f$. Thus, $fg$ does not exist. Note that $\ran f = [7, \infty)$ and $\dom g = (3, \infty)$. Hence, $\ran f \subseteq \dom g$. Thus, $gf$ exists.

            Since $\ln x$ is a strictly increasing function, we have that $g$ is also strictly increasing. Hence, $\ran{gf} = [\ln{7 - 2}, \infty) = [\ln 5, \infty)$.
            
            \boxt{$\ran{gf} = [\ln 5, \infty)$}

    \problem{}
        The function $f$ is defined as follows:

        \begin{equation*}
            f \colon x \mapsto \dfrac1{x^2 - 1}, \qquad x \in \R, \, x \neq -1, \, x \neq 1
        \end{equation*}

        \begin{enumerate}
            \item Sketch the graph of $y = f(x)$.
            \item If the domain of $f$ is further restricted to $x \geq k$, state with a reason the least value of $k$ for which the function $\inv f$ exists.
        \end{enumerate}

         \textbf{In the rest of the question, the domain of $f$ is $x \in \R, \, x \neq -1, \, x \neq 1$, as originally defined.}

        \smallskip

         The function $g$ is defined as follows:

        \begin{equation*}
            g \colon x \mapsto \dfrac1{x-3}, \qquad x \in \R, \, x \neq 2, \, x \neq 3, x \neq 4
        \end{equation*}

        \begin{enumerate}
            \setcounter{enumi}{2}
            \item Find the range of $fg$.
        \end{enumerate}
    
    \solution
        \part
            \begin{center}
                \begin{tikzpicture}[trim axis left, trim axis right]
                    \begin{axis}[
                        samples = 161,
                        axis y line=middle,
                        axis x line=middle,
                        ytick = {-1},
                        xtick = \empty,
                        xlabel = {$x$},
                        ylabel = {$y$},
                        ymin=-3,
                        ymax=3,
                        legend cell align={left},
                        legend pos=outer north east,
                        after end axis/.code={
                            \path (axis cs:0,0) 
                                node [anchor=north east] {$O$};
                            }
                        ]

                        \addplot[plotRed, domain=-4:4, unbounded coords=jump] {1/(x^2 - 1)};
            
                        \addlegendentry{$y = f(x)$};

                        \draw[dotted, thick] (1, 3) -- (1, -3) node[anchor=south west] {$x = 1$};

                        \draw[dotted, thick] (-1, 3) -- (-1, -3) node[anchor=south east] {$x = -1$};
                    \end{axis}
                \end{tikzpicture}
            \end{center}

        \part
            If the domain of $f$ is further restricted to $x \geq 0$, $f$ would pass the horizontal line test, whence $\inv f$ would exist.
            
            \boxt{$\min k = 0$}

        \part
            Observe that $\ran g = \R \setminus \bc{g(2), g(4)} = \R \setminus \bc{-1, 1}$. Hence, $\ran {fg} = \ran{f} = \R \setminus (-1, 0\,]$.

            \boxt{$\ran {fg} = \R \setminus (-1, 0\,]$}

    \problem{}
        The function $f$ is defined by

        \begin{equation*}
            f \colon x \mapsto \dfrac{x}{x^2 - 1}, \qquad x \in \R, \, x \neq -1, \, x \neq 1
        \end{equation*}

        \begin{enumerate}
            \item Explain why $f$ does not have an inverse.
            \item The function $f$ has an inverse if the domain is restricted to $x \leq k$. State the largest value of $k$.
        \end{enumerate}

         The function $g$ is defined by

        \begin{equation*}
            g \colon x \mapsto \ln x - 1, \qquad x \in \R, \, 0 < x < 1
        \end{equation*}

        \begin{enumerate}
            \setcounter{enumi}{2}
            \item Find an expression for $h(x)$ for each of the following cases:
            \begin{enumerate}
                \item $gh(x) = x$
                \item $hg(x) = x^2+1$
            \end{enumerate}
        \end{enumerate}

    \solution
        \part
            Observe that $f\left(\dfrac12\right) = -\dfrac23$ and $f(-2) = -\dfrac23$. Hence, $f\left(\dfrac12\right) = f(-2)$. Since $\dfrac12 \neq -2$, $f$ is not one-one. Thus, $f$ does not have an inverse.

        \part
            \boxt{$\max k = 0$}

        \part
            \subpart
                Note that $gh(x) = x \implies h(x) = \inv g(x)$. Hence, consider $y = g(x) \implies x = h(y)$.

                \begin{alignat*}{2}
                    &&y &= g(x)\\
                    \implies&&y &= \ln x - 1\\
                    \implies&&\ln x &= y+1\\
                    \implies&&x &= e^{y+1}
                \end{alignat*}

                Hence, $h(x) = e^{x+1}$.

                \boxt{$h(x) = e^{x+1}$}

            \subpart
                Let $h = h_2 \circ h_1$ such that $h_1g(x) = x \implies h_1(x) = \inv g(x) \implies h_1(x) = e^{x+1}$.

                \begin{alignat*}{2}
                    &&hg(x) &= x^2+1\\
                    \implies&&h_2h_1g(x) &= x^2+1\\
                    \implies&&h_2(x) &= x^2+1\\
                \end{alignat*}

                Hence, $h(x) = h_2h_1(x) = h_2(e^{x+1}) = \left(e^{x+1}\right)^2 + 1 = e^{2x+2} + 1$

                \boxt{$h(x) = e^{2x+2}+1$}
\end{document}