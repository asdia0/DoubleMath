\documentclass{echw}

\title{Assignment B5A\\Applications of Differentation}
\author{Eytan Chong}
\date{2024-04-11}

\begin{document}
    \problem{}
        \begin{enumerate}
            \item Show, algebraically, that the derivative of the function

            \begin{equation*}
                \ln (1 + x) - \dfrac{2x}{x+2}
            \end{equation*}
    
            \noindent is never negative.
            \item Hence show that $\ln (1+x) \geq \dfrac{2x}{x+2}$ when $x \geq 0$.
        \end{enumerate}
        
    \solution
        Let $f(x) = \ln (1+x) - \dfrac{2x}{x+2} = \ln (1+x) - 2 + \dfrac4{x+2}$.
        \part
            \begin{align*}
                f^\prime(x) &= \dfrac1{1+x} - \dfrac{4}{(x+2)^2}\\
                &= \dfrac{(x+2)^2 - 4(1+x)}{(1+x)(x+2)^2}\\
                &= \dfrac{x^2}{(1+x)(x+2)^2}
            \end{align*}

            Given that $\ln (1+x)$ is defined, it must be that $1 + x > 0$. We also know that $x^2 \geq 0$ and $(x+2)^2 \geq 0$. Hence, $f^\prime(x) \geq 0$ for all $x$ in the domain of $f$ and is thus never negative.

        \part
            Note that $f(0) = \ln (1 + 0) - 2 + \dfrac4{0 + 2} = 0$. Since $f^\prime(x)$ is never negative, $f(x)$ is increasing. Hence, for all $x \geq 0$,
            
            \begin{alignat*}{2}
                &&f(x) &\geq f(0)\\
                \implies&& \ln (1+x) - \dfrac{2x}{x+2} &\geq 0\\
                \implies&&\ln (1+x) &\geq \dfrac{2x}{x+2}
            \end{alignat*}

    \problem{}
        The equation of a curve is $y = ax^2 - 2bx + c$, where $a$, $b$ and $c$ are constants, with $a > 0$.

        \begin{enumerate}
            \item Using differentation, find the coordinates of the turning point on the curve, in terms of $a$, $b$ and $c$. State whether it is a maximum point or a minimum point.
            \item Given that the turning point of the curve lies on the line $y=x$, find an expression for $c$ in terms of $a$ and $b$. Show that in this case, whatever the value of $b$, $c \geq -\dfrac1{4a}$.
            \item Find the numerical values of $a$, $b$ and $c$ when the curve passes through the point $(0, 6)$ and has a turning point at $(2, 2)$.
        \end{enumerate}

        \solution
            \part
                For stationary points, $\der{y}{x} = 0$.

                \begin{alignat*}{2}
                    &&\der{y}{x} &= 0\\
                    \implies&&2ax - 2b &= 0\\
                    \implies&&ax - b &= 0\\
                    \implies&&x &= \dfrac{b}a\\
                    \implies&&y &= a\cdot \dfrac{b}{a}^2 - 2b\cdot\dfrac{b}{a} + c\\
                    && &= \dfrac{b^2}a - 2\cdot \dfrac{b^2}a + c\\
                    && &= -\dfrac{b^2}a + c
                \end{alignat*}

                Since $a > 0$, the graph of $y$ is concave upwards. Thus, there is a maximum point at $\left(\dfrac{b}{a}, -\dfrac{b^2}a + c\right)$.

                \boxt{
                    There is maximum point at $\left(\dfrac{b}{a}, -\dfrac{b^2}a + c\right)$.
                }

            \part
                Since the turning point $\left(\dfrac{b}{a}, -\dfrac{b^2}a + c\right)$  lies on the line $y=x$, 

                \begin{alignat*}{2}
                    &&\dfrac{b}{a} &= -\dfrac{b^2}a + c\\
                    \implies&&c &= \dfrac{b}a + \dfrac{b^2}a\\
                    && &= \dfrac1a (b + b^2)
                \end{alignat*}

                Consider the stationary points of $c$ with respect to $b$. For stationary points, $\der{c}{b} = 0$.

                \begin{alignat*}{2}
                    &&\der{c}{b} &= 0\\
                    \implies&&\dfrac1a (1 + 2b) &= 0\\
                    \implies&& 1 + 2b &= 0\\
                    \implies&& b &= -\dfrac12
                \end{alignat*}

                \begin{table}[h]
                    \centering
                    \begin{tabular}{|c|c|c|c|}
                    \hline
                    $b$ & $\left(-\dfrac12\right)^-$ & $-\dfrac12$ & $\left(-\dfrac12\right)^+$ \\\hline
                    $\der{c}{b}$ & -ve   & 0 & +ve   \\[1ex]\hline
                    \end{tabular}
                \end{table}

                By the First Derivative Test, $c$ achieves a minimum when $b = -\dfrac12$. Observe that when $b = -\dfrac12$, we have $c = \dfrac1a \left(-\dfrac12 + \left(-\dfrac12\right)^2\right) = -\dfrac1{4a}$. Thus, $c \geq -\dfrac1{4a}$ whatever the value of $b$.

            \part
                Since the curve passes through $(0, 6)$, it is obvious to see that $c = 6$. Furthermore, since the curve has a turning point at $(2, 2)$, we know that $\dfrac{b}{a} = 2$ and $-\dfrac{b^2}a + c = 2$.

                \begin{alignat*}{2}
                    &&-\dfrac{b^2}a + c &= 2\\
                    \implies&&-\dfrac{b^2}a + 6 &= 2\\
                    \implies&&-\dfrac{b^2}a &= -4\\
                    \implies&&\dfrac{b^2}a &= 4\\
                    \implies&&\dfrac{b}a \cdot b &= 4\\
                    \implies&&2 \cdot b &= 4\\
                    \implies&&b &= 2\\
                    \implies&&a &= 1
                \end{alignat*}

                \boxt{
                    $a = 1$, $b = 2$, $c = 6$
                }
        
        \problem{}
            The diagram below shows the graph of $y = f(x)$. Sketch the graph of $y = f^\prime(x)$.

            \begin{center}
                \begin{tikzpicture}[trim axis left, trim axis right]
                    \begin{axis}[
                        domain = -5:5,
                        samples = 101,
                        axis y line=middle,
                        axis x line=middle,
                        xtick = {2},
                        ytick = \empty,
                        xlabel = {$x$},
                        ylabel = {$y$},
                        ymin=-5,
                        ymax=5,
                        legend cell align={left},
                        legend pos=outer north east,
                        after end axis/.code={
                            \path (axis cs:0,0) 
                                node [anchor=north east] {$O$};
                            }
                        ]
                        \addplot[plotRed, domain=-5:1] {2/(x-1) + 2};

                        \addlegendentry{$y = f(x)$};

                        \addplot[plotRed, domain=1:3] {6 * (-1/(x-1) + 1)};

                        \addplot[plotRed, domain=3:5] {sqrt(2) * (x-2.293) * e^(0.5 - (x-2.293)^2) + 2};

                        \fill (3, 3) circle[radius=2.5pt] node[anchor=south] {$(3, 3)$};

                        \draw[dotted, thick] (5, 2) -- (-5,2) node[anchor=south west] {$y=2$};

                        \draw[dotted, thick] (1, 5) -- (1, -5) node[anchor=south west, fill=white, opacity = 0.6, text opacity=1] {$x=1$};
                    \end{axis}
                \end{tikzpicture}
            \end{center}

        \solution
            \begin{center}
                \begin{tikzpicture}[trim axis left, trim axis right]
                    \begin{axis}[
                        domain = -5:5,
                        samples = 101,
                        axis y line=middle,
                        axis x line=middle,
                        xtick = {3},
                        ytick = \empty,
                        xlabel = {$x$},
                        ylabel = {$y$},
                        ymax=5,
                        ymin=-5,
                        legend cell align={left},
                        legend pos=outer north east,
                        after end axis/.code={
                            \path (axis cs:0,0) 
                                node [anchor=north east] {$O$};
                            }
                        ]
                        \addplot[plotRed, domain=-5:0.9] {-2/(x-1)^2};

                        \addlegendentry{$y = f^\prime(x)$};

                        \addplot[plotRed, domain=1.2:3, unbounded coords=jump] {6/(x-1)-3};

                        \addplot[plotRed, domain=3:5] {-(x-3)*e^(-(x-3)^2)};

                        \draw[dotted, thick] (1, 5) -- (1, -5) node[anchor=south west] {$x=1$};
                    \end{axis}
                \end{tikzpicture}
            \end{center}

        \problem{}
            The curve $C$ has equation

            \begin{equation*}
                x - y = (x+y)^2
            \end{equation*}

            \noindent It is given that $C$ has only one turning point.

            \begin{enumerate}
                \item Show that $1 + \der{y}{x} = \dfrac{2}{2x + 2y + 1}$.
                \item Hence, or otherwise, show that $\derm{y}{x}2 = -\left(1 + \der{y}{x}\right)^2$.
                \item Hence, state, with a reason, whether the turning point is a maximum or a minimum.
            \end{enumerate}

        \solution
            \part
                Implicitly differentiating the given equation,
                \begin{alignat}{2}
                    &&1 - \der{y}{x} &= 2(x+y)\left(1 + \der{y}{x}\right)\nonumber\\
                    \implies&&1 - \der{y}{x} &= (2x + 2y) + (2x + 2y)\der{y}{x}\nonumber\\
                    \implies&&(2x+2y+1)\der{y}{x} &= 1 - (2x+2y)\nonumber\\
                    \implies&&\der{y}{x} &= \dfrac{1-(2x+2y)}{2x+2y+1}\nonumber\\
                    && &= \dfrac{1-(2x+2y+1)+1}{2x+2y+1}\nonumber\\
                    && &= \dfrac{2}{2x+2y+1} - 1\nonumber\\
                    \implies&&1 + \der{y}{x} &= \dfrac{2}{2x+2y+1}\label{P4-1}
                \end{alignat}

            \part
                Implicitly differentiating Equation~\ref{P4-1},                 
                \begin{alignat*}{2}
                    \derm{y}{x}2 &= -\dfrac{2}{(2x+2y+1)^2} \left(2 + 2\der{y}{x}\right)\\
                    &= -\dfrac{2^2}{(2x+2y+1)^2} \left(1 + \der{y}{x} \right)\\
                    &= -(\dfrac{2}{2x+2y+1})^2\left(1 + \der{y}{x}\right)\\
                    &= -\left(1 + \der{y}{x}\right)^2\left(1 + \der{y}{x}\right)\\
                    &= -\left(1 + \der{y}{x}\right)^3
                \end{alignat*}

            \part
                For turning points, $\der{y}{x} = 0$. Hence $\derm{y}{x}2 = -1(1 + 0)^2 = -1 < 0$. Thus, the turning point is a maximum.

                \boxt{
                    The turning point is a maximum.
                }
\end{document}