\documentclass{echw}

\title{Tutorial B5A\\Applications of Differentiation}
\author{Eytan Chong}
\date{2024-04-04}

\begin{document}
    \problem{}
        The equation of a curve is $y=2x^3+3x^2+6x+4$. Find $\der{y}{x}$ and hence show that $y$ is increasing for all real values of $x$.

    \solution
        \begin{align*}
            \der{y}{x} &= 2\cdot3x^2 + 3 \cdot2x + 6\\
            &= 6x^2+6x+6
        \end{align*}

        \boxt{$\der{y}{x} = 6x^2 + 6x + 6$}

        Observe that $\der{y}{x} = 6x^2 + 6x+6 = 6\left(x + \dfrac12\right)^2 + \dfrac{18}4$. For all $x \in \R$, we have $\left(x + \dfrac12\right)^2 \geq 0$. Hence, $\der{y}{x} > 0$. Thus, $y$ is increasing for all real values of $x$.

    \problem{}
        Find, by differentiation, the $x$-coordinates of all the stationary points on the curve $y = \dfrac{x^3}{(x+1)^2}$ stating, with reasons, the nature of each point.

    \solution
        \begin{alignat*}{2}
            &&y &= \dfrac{x^3}{(x+1)^2}\\
            \implies&&(x+1)^2y&=x^3\\
            \implies&&(x+1)^2\cdot y' + y\cdot2(x+1) &= 3x^2
        \end{alignat*}

        For stationary points, $y' = 0$.

        \begin{alignat*}{2}
            \implies&&y\cdot2(x+1) &= 3x^2\\
            \implies&&\dfrac{x^3}{(x+1)^2}\cdot2(x+1) &=3 3x^2\\
            \implies&&\dfrac{2x^3}{x+1} &= 3x^2\\
            \implies&&2x^3 &= 3x^2(x+1)\\
            \implies&&2x^3 &= 3x^3 + 3x^2\\
            \implies&&x^3+3x^2&=0\\
            \implies&&x^2(x+3)&=0
        \end{alignat*}

        Hence, $x = 0$ or $x = -3$.

        \boxt{The $x$-coordinates of the stationary points are $x =0$ and $x=-3$.}

        \begin{table}[h]
            \centering
            \begin{tabular}{|c|c|c|c|}
            \hline
            $x$          & $0^-$ & 0 & $0^+$ \\\hline
            $\der{y}{x}$ & +ve   & 0 & +ve   \\[1ex]\hline
            \end{tabular}
        \end{table}

        Thus, there is a stationary point of inflexion at $x = 0$.

        \begin{table}[h]
            \centering
            \begin{tabular}{|c|c|c|c|}
            \hline
            $x$          & $(-3)^-$ & $-3$ & $(-3)^+$ \\\hline
            $\der{y}{x}$ & +ve   & 0 & -ve   \\[1ex]\hline
            \end{tabular}
        \end{table}

        Thus, there is a maximum point at $x=-3$.

        \boxt{At $x=0$, there is a stationary point of inflexion.\\At $x=-3$, there is a maximum point.}

    \problem{}
        Differentiate $f(x) = 8\sin \dfrac{x}2 - 4x$ with respect to $x$ and deduce that $f(x) < 0$ for $x>0$.

    \solution
        \begin{align*}
            f'(x) &= 8\cos \dfrac{x}2 \cdot \dfrac12 - \cos x - 4\\
            &= 4\cos\dfrac{x}2 - \cos x- 4
        \end{align*}

        \boxt{$f'(x) = 4\cos\dfrac{x}2 - \cos x- 4$}

        \begin{align*}
            f'(x) &= 4\cos\dfrac{x}2 - \cos x- 4\\
            &= 4\cos \dfrac{x}2 - (2\cos^2 \dfrac{x}2 - 1) - 4\\
            &= -2\left(\cos\dfrac{x}2 - 1\right)^2 - 1
        \end{align*}

        Observe that for all $x \in \R$, $\left(\cos\dfrac{x}2 - 1\right)^2 \geq 0$. Hence, $f'(x) < 0$ for all real values of $x$. Thus, $f(x)$ is strictly decreasing on $\R$.

        Note that $f(0) = 8\sin 0 - \sin 0 - 4 \cdot 0 = 0$. Since $f(x)$ is strictly decreasing, for all $x > 0$, $f(x) < f(0) = 0$.

    \problem{}
        Sketch the graphs of the derivative functions for each of the graphs of the following functions below. In each graph, the point(s) labelled in coordinate form are stationary points.

        \begin{enumerate}
            \item \begin{center}
                \begin{tikzpicture}[trim axis left, trim axis right]
                    \begin{axis}[
                        domain = -0.25:3.75,
                        samples = 101,
                        axis y line=middle,
                        axis x line=middle,
                        xtick = \empty,
                        ytick = \empty,
                        xlabel = {$x$},
                        ylabel = {$y$},
                        legend cell align={left},
                        legend pos=outer north east,
                        after end axis/.code={
                            \path (axis cs:0,0) 
                                node [anchor=north east] {$O$};
                            }
                        ]
                        \addplot[plotRed] {(x-2)^3 + 2};
                        
                        \fill (2, 2) circle[radius=2.5 pt] node[anchor=south] {$(2, 2)$};
                    \end{axis}
                \end{tikzpicture}
            \end{center}
            \item \begin{center}
                \begin{tikzpicture}[trim axis left, trim axis right]
                    \begin{axis}[
                        domain = -5:5,
                        samples = 101,
                        axis y line=middle,
                        axis x line=middle,
                        xtick = \empty,
                        ytick = \empty,
                        xlabel = {$x$},
                        ylabel = {$y$},
                        ymin=-5,
                        ymax=5,
                        legend cell align={left},
                        legend pos=outer north east,
                        after end axis/.code={
                            \path (axis cs:0,0) 
                                node [anchor=north west] {$O$};
                            }
                        ]
                        \addplot[plotRed, unbounded coords=jump] {x + 1/x};
                        
                        \fill (-1, -2) circle[radius=2.5 pt];
                        
                        \node[anchor=east] at (-1, -1.4) {$(-1, -2)$};

                        \fill (1, 2) circle[radius=2.5 pt];
                        
                        \node[anchor=south] at (1.3, 2) {$(1, 2)$};

                        \addplot[dotted, thick] {x};

                        \node[rotate=45] at (4.5, 4) {$y=x$};
                    \end{axis}
                \end{tikzpicture}
            \end{center}
            \item \begin{center}
                \begin{tikzpicture}[trim axis left, trim axis right]
                    \begin{axis}[
                        domain = -5:5,
                        samples = 101,
                        axis y line=middle,
                        axis x line=middle,
                        xtick = \empty,
                        ytick = \empty,
                        xlabel = {$x$},
                        ylabel = {$y$},
                        ymin=-5,
                        ymax=5,
                        legend cell align={left},
                        legend pos=outer north east,
                        after end axis/.code={
                            \path (axis cs:0,0) 
                                node [anchor=north west] {$O$};
                            }
                        ]
                        \addplot[plotRed, domain=-5:-1] {-1/(x+1) + 2};

                        \addplot[plotRed, domain=-1:2] {4.5 * (-1/(x+1) + 1)};

                        \addplot[plotRed, domain=2:5] {sqrt(2) * (x-1.293) * e^(0.5 - (x-1.293)^2) + 2};

                        \fill (2, 3) circle[radius=2.5pt] node[anchor=south] {$(2, 3)$};

                        \draw[dotted, thick] (5, 2) -- (-5,2) node[anchor=north west] {$y=2$};

                        \draw[dotted, thick] (-1, 5) -- (-1, -5) node[anchor=south east] {$x=-1$};
                    \end{axis}
                \end{tikzpicture}
            \end{center}
            \item \begin{center}
                \begin{tikzpicture}[trim axis left, trim axis right]
                    \begin{axis}[
                        domain = -3:2,
                        samples = 101,
                        axis y line=middle,
                        axis x line=middle,
                        xtick = {-1},
                        xticklabels = {$-a$},
                        ytick = \empty,
                        xlabel = {$x$},
                        ylabel = {$y$},
                        ymin=-2,
                        ymax=2,
                        xmin=-4,
                        xmax=4,
                        legend cell align={left},
                        legend pos=outer north east,
                        after end axis/.code={
                            \path (axis cs:0,0) 
                                node [anchor=north west] {$O$};
                            }
                        ]
                        \addplot[plotRed, domain=1:4] {1/(x-1)};

                        \addplot[plotRed, domain=0:1] {sec(pi * (\x +0.5) r)};

                        \addplot[plotRed, domain=-2:0] {-1/x - 1};

                        \addplot[plotRed, domain=-4:-2] {0.824 * (x+1)*e^(-0.5 * (x+1)^2)};

                        \fill (0.5, -1) circle[radius=2.5pt] node[anchor=west] {$(\tfrac{a}2, -a)$};

                        \fill (-2, -0.5) circle[radius=2.5pt] node[anchor=north] {$(-2a, -\tfrac{a}2)$};

                        \draw[dotted, thick] (1, 2) -- (1, -2) node[anchor=south west] {$x=a$};
                    \end{axis}
                \end{tikzpicture}
            \end{center}
        \end{enumerate}

    \solution
        \part
            \begin{center}
                \begin{tikzpicture}[trim axis left, trim axis right]
                    \begin{axis}[
                        domain = -0.25:3.75,
                        samples = 101,
                        axis y line=middle,
                        axis x line=middle,
                        xtick = {2},
                        ytick = \empty,
                        xlabel = {$x$},
                        ylabel = {$y$},
                        legend cell align={left},
                        legend pos=outer north east,
                        after end axis/.code={
                            \path (axis cs:0,0) 
                                node [anchor=north] {$O$};
                            }
                        ]
                        \addplot[plotRed] {3(x-2)^2};
                    \end{axis}
                \end{tikzpicture}
            \end{center}

        \part
            \begin{center}
                \begin{tikzpicture}[trim axis left, trim axis right]
                    \begin{axis}[
                        domain = -3:3,
                        samples = 101,
                        axis y line=middle,
                        axis x line=middle,
                        xtick = {-1, 1},
                        ytick = \empty,
                        xlabel = {$x$},
                        ylabel = {$y$},
                        ymin=-3,
                        ymax=2,
                        legend cell align={left},
                        legend pos=outer north east,
                        after end axis/.code={
                            \path (axis cs:0,0) 
                                node [anchor=north east] {$O$};
                            }
                        ]
                        \addplot[plotRed, unbounded coords=jump] {1 - 1/x^2};

                        \draw[dotted, thick] (-3, 1) -- (3, 1) node[anchor=south east] {$y=1$};
                    \end{axis}
                \end{tikzpicture}
            \end{center}

        \part
            \begin{center}
                \begin{tikzpicture}[trim axis left, trim axis right]
                    \begin{axis}[
                        domain = -5:5,
                        samples = 101,
                        axis y line=middle,
                        axis x line=middle,
                        xtick = {2},
                        ytick = \empty,
                        xlabel = {$x$},
                        ylabel = {$y$},
                        ymax=3,
                        ymin=-1,
                        legend cell align={left},
                        legend pos=outer north east,
                        after end axis/.code={
                            \path (axis cs:0,0) 
                                node [anchor=north east] {$O$};
                            }
                        ]
                        \addplot[plotRed, domain=-5:-1.2, unbounded coords=jump] {-1/(x+1)};

                        \addplot[plotRed, domain=-0.8:2, unbounded coords=jump] {9/(x+1)-3};

                        \addplot[plotRed, domain=2:5] {-(x-2)*e^(-(x-2)^2)};

                        \draw[dotted, thick] (-1, 5) -- (-1, -1) node[anchor=south east] {$x=-1$};
                    \end{axis}
                \end{tikzpicture}
            \end{center}

        \part
            \begin{center}
                \begin{tikzpicture}[trim axis left, trim axis right]
                    \begin{axis}[
                        domain = -3:2,
                        samples = 101,
                        axis y line=middle,
                        axis x line=middle,
                        xtick = {-2, 0.5},
                        xticklabels = {$-2a$, $\tfrac{a}2$},
                        ytick = \empty,
                        xlabel = {$x$},
                        ylabel = {$y$},
                        ymin=-2,
                        ymax=2,
                        xmin=-4,
                        xmax=4,
                        legend cell align={left},
                        legend pos=outer north east,
                        after end axis/.code={
                            \path (axis cs:0,0) 
                                node [anchor=north east] {$O$};
                            }
                        ]
                        \addplot[plotRed, domain=1.2:4] {-1/(x-1)^2};

                        \addplot[plotRed, domain=0.1:0.9] {sec(pi * (\x +0.5) r) * tan(pi * (\x +0.5) r) * pi * x};

                        \addplot[plotRed, domain=-2:-0.1] {1/x^2 -1/4};

                        \addplot[plotRed, domain=-4:-2] {0.25 * (x+2) * e^(-(x+2)^2)};

                        \draw[dotted, thick] (1, 2) -- (1, -2) node[anchor=south west] {$x=a$};
                    \end{axis}
                \end{tikzpicture}
            \end{center}

    \problem{}
        \begin{enumerate}
            \item Given that $y=ax\sqrt{x+2}$ where $a > 0$, find $\der{y}{x}$, expressing your answer as a single algebraic fraction. Hence show that the curve $y = ax\sqrt{x+2}$ has only one turning point, and state its coordinates in exact form.
            \item Sketch the graph of $y = f'(x)$, where $f(x) = ax\sqrt{x+2}$, where $a > 0$.
        \end{enumerate}

    \solution
        \part
            \begin{align*}
                \der{y}{x} &= a \left(x \cdot \dfrac1{2\sqrt{x+2}} + \sqrt{x+2} \right)\\
                &= a \left(\dfrac{x}{2\sqrt{x+2}} + \dfrac{2(x+2)}{2\sqrt{x+2}}\right)\\
                &= \dfrac{a(3x+4)}{2\sqrt{x+2}}
            \end{align*}

            \boxt{$\der{y}{x} = \dfrac{a(3x+4)}{2\sqrt{x+2}}$}

            Consider the stationary points of $y=ax\sqrt{x+2}$. For stationary points, $\der{y}{x} =0$.

            \begin{alignat*}{2}
                &&\der{y}{x} &= 0\\
                \implies&&\dfrac{a(3x+4)}{2\sqrt{x+2}} &= 0\\
                \implies&&a(3x+4) &= 0
            \end{alignat*}

            Since $a > 0$, we have $3x+4=0$, whence $x = -\dfrac43$.

            \begin{table}[h]
                \centering
                \begin{tabular}{|c|c|c|c|}
                \hline
                $x$          & $\left(-\dfrac43\right)^-$ & $-\dfrac43$ & $\left(-\dfrac43\right)^+$ \\\hline
                $\der{y}{x}$ & -ve   & 0 & +ve   \\[1ex]\hline
                \end{tabular}
            \end{table}

            Hence, at $x = -\dfrac43$, there is a turning point (minimum point). Thus, $y = ax\sqrt{x+2}$ has only one turning point.

            Substituting $x = -\dfrac43$ into $y = ax\sqrt{x+2}$, we see that $y = -\dfrac{4a}3 \sqrt{\dfrac23}$. Hence, the coordinate of the turning point is $(-\dfrac43, -\dfrac{4a}3 \sqrt{\dfrac23})$.

            \boxt{$\bp{-\dfrac43, -\dfrac{4a}3 \sqrt{\dfrac23}}$}

        \part
            \begin{center}
                \begin{tikzpicture}[trim axis left, trim axis right]
                    \begin{axis}[
                        domain = -2:10,
                        samples = 180,
                        axis y line=middle,
                        axis x line=middle,
                        ytick = {sqrt(2)},
                        yticklabels = {$\sqrt{2}a$},
                        xtick = {-4/3},
                        xticklabels = {$-\tfrac43$},
                        xlabel = {$x$},
                        ylabel = {$y$},
                        ymin=-2,
                        xmin=-2,
                        legend cell align={left},
                        legend pos=outer north east,
                        after end axis/.code={
                            \path (axis cs:0,0) 
                                node [anchor=north west] {$O$};
                            }
                        ]
                        \addplot[plotRed] {(3*x + 4)/(2 * sqrt(x+2))};
            
                        \addlegendentry{$y = f'(x)$};

                        \draw[dotted, thick] (-2, -2) -- (-2, 5) node[anchor=north east, rotate=90] {$x=-2$};
                    \end{axis}
                \end{tikzpicture}
            \end{center}

    \problem{}
        A particle $P$ moves along the $x$-axis. Initially, $P$ is at the origin $O$. At time $t$ s, the velocity is $v$ ms$^{-1}$ and the acceleration is $a$ ms$^{-2}$. Below is the velocity-time graph of the particle for $0 \leq t \leq 25$.

        \begin{center}
            \begin{tikzpicture}[trim axis left, trim axis right]
                \begin{axis}[
                    domain = -5:5,
                    samples = 101,
                    axis y line=middle,
                    axis x line=middle,
                    xtick = {pi, 2*pi},
                    xticklabels = {15, 25},
                    ytick = \empty,
                    xlabel = {$t$},
                    ylabel = {$v$},
                    ymin=-1.5,
                    ymax=1.5,
                    xmax=7,
                    legend cell align={left},
                    legend pos=outer north east,
                    after end axis/.code={
                        \path (axis cs:0,0) 
                            node [anchor=east] {$O$};
                        }
                    ]
                    \addplot[plotRed, domain=0:pi] {sin(\x r)};

                    \addplot[plotRed, domain=pi:2*pi] {sin(\x r)};

                    \fill (pi/2, 1) circle[radius=2.5pt] node[anchor=south] {$(11, 5)$};

                    \fill (3*pi/2, -1) circle[radius=2.5pt] node[anchor=north] {$(21, -4)$};
                \end{axis}
            \end{tikzpicture}
        \end{center}

        \begin{enumerate}
            \item Describe the motion of the particle for $0 \leq t \leq 25$.
            \item Sketch the acceleration-time graph of the particle $P$.
        \end{enumerate}

    \solution
        \part
            From $t = 0$ to $t = 11$, $P$ speeds up and reaches a top speed of 5 ms$^{-1}$. From $t=11$ to $t=15$, $P$ slows down. At $t=15$, $P$ is instantanteously at rest. From $t=15$ to $t=21$, $P$ speeds up and moves in the opposite direction, reaching a top speed of 4 ms$^{-1}$. From $t=21$ to $t=25$, $P$ slows down. At $t=25$, $P$ is instantanteously at rest.

        \part
            \begin{center}
                \begin{tikzpicture}[trim axis left, trim axis right]
                    \begin{axis}[
                        domain = -5:5,
                        samples = 101,
                        axis y line=middle,
                        axis x line=middle,
                        xtick = {pi/2, pi, 3*pi/2, 2*pi},
                        xticklabels = {11, 15, 21, 25},
                        ytick = \empty,
                        xlabel = {$t$},
                        ylabel = {$v$},
                        ymin=-1.5,
                        ymax=1.5,
                        xmax=7,
                        legend cell align={left},
                        legend pos=outer north east,
                        after end axis/.code={
                            \path (axis cs:0,0) 
                                node [anchor=east] {$O$};
                            }
                        ]
                        \addplot[plotRed, domain=0:pi] {cos(\x r)};

                        \addplot[plotRed, domain=pi:2*pi] {cos(\x r)};

                        \draw[dotted, thick] (pi, -0.3) -- (pi, -1);

                        \draw[dotted, thick] (2*pi, 0) -- (2*pi, 1);
                    \end{axis}
                \end{tikzpicture}
            \end{center}

    \problem{}
        The function $f$ defined by $f(x) = \ln x - 2\left(x-\dfrac12\right)$, where $x \in \R, x > 0$. Find $f'(x)$ and show that the function is decreasing for $x > \dfrac12$. Hence show that for $x > \dfrac12$, $2\left(x - \dfrac12\right) - \ln x > \ln 2$.

    \solution
        \begin{equation*}
            f' = \dfrac1x - 2
        \end{equation*}

        When $x > \dfrac12$, $\dfrac1x < 2 \implies \dfrac1x - 2<0$. Thus, $f'(x) < 0$, whence $f(x)$ is decreasing.

        Note that $f\left(\dfrac12\right) = \ln \dfrac12 - 2\left(\dfrac12 - \dfrac12\right) = -\ln2$. Since $f(x)$ is decreasing for all $x > \dfrac12$, $f(x) < f\left(\dfrac12\right) = -\ln 2 \implies \ln x - 2\left(x - \dfrac12\right) < -\ln2 \implies 2\left(x - \dfrac12\right) - \ln x > \ln 2$.

\end{document}