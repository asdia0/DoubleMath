\documentclass{echw}

\title{Assignment B8\\Applications of Integration I}
\author{Eytan Chong}
\date{2024-06-26}

\begin{document}
    \problem{}
        The diagram shows the region $R$, which is bounded by the axes and the part of the curve $y^2 = 4a(a-x)$ lying in the first quadrant.

        Find, in terms of $a$, the volume, $V_x$, of the solid formed when $R$ is rotated completely about the $x$-axis.

        The volume of the solid formed when $R$ is rotated completely about the $y$-axis is $V_y$. Show that $V_y = \dfrac8{15} V_x$.

        The region $S$, lying in the first quadrant, is bounded by the curve $y^2 = 4a(a-x)$ and the lines $x = a$ and $y = 2a$. Find, in terms of $a$, the volume of the solid formed when $S$ is rotated completely about the $y$-axis.

        \begin{center}
            \begin{tikzpicture}[trim axis left, trim axis right]
                \begin{axis}[
                    domain = 0:2,
                    samples = 101,
                    axis y line=middle,
                    axis x line=middle,
                    xmax=3,
                    ymax=5,
                    xtick = {2},
                    ytick = {4},
                    xticklabels = {$a$},
                    yticklabels = {$2a$},
                    xlabel = {$x$},
                    ylabel = {$y$},
                    legend cell align={left},
                    legend pos=outer north east,
                    after end axis/.code={
                        \path (axis cs:0,0) 
                            node [anchor=north east] {$O$};
                        }
                    ]
                    \addplot[plotRed] {sqrt(4*2*(2-x))};
        
                    \addlegendentry{$y^2 = 4a(a-x)$};

                    \draw[dotted] (2, 0) -- (2, 4);
                    \draw[dotted] (0, 4) -- (2, 4);

                    \node at (0.8, 1.5) {$R$};
                    \node at (1.5, 3) {$S$};
                \end{axis}
            \end{tikzpicture}
        \end{center}

    \solution
        \begin{align*}
            V_x &= \pi \int_0^a y^2 \d x\\
            &= \pi \int_0^a 4a(a-x) \d x\\
            &= 4\pi a \evalint{ax - \dfrac12 x^2}{0}{a}\\
            &= 2\pi a^3
        \end{align*}
        
        \boxt{$V_x = 2\pi a^3$ units$^3$}
        
        \dash

        Note that $x = a-\dfrac{y^2}{4a} \implies x^2 = \bp{a - \dfrac{y^2}{4a}}^2 = a^2 - \dfrac12 y^2 + \dfrac1{16a^2}y^4$. Hence, 
        {\allowdisplaybreaks
        \begin{align*}
            V_y &= \pi \int_0^{2a} x^2 \d y\\
            &= \pi \int_0^{2a} a^2 - \dfrac12 y^2 + \dfrac1{16a^2}y^4 \d y\\
            &= \pi \evalint{a^2y - \dfrac12 \cdot \dfrac13 y^3 + \dfrac1{16a^2}\cdot \dfrac15 y^5}{0}{2a}\\
            &= \pi \bp{2a^3 - \dfrac{8a^3}6 + \dfrac{32a^5}{90a^2}}\\
            &= \dfrac{16}{15} \pi a^3\\
            &= \dfrac8{15} (2\pi a^3)\\
            &= \dfrac8{15} V_x
        \end{align*}}
        \dash
        \begin{align*}
            \volume S &= \text{Volume of Cylinder} - V_y\\
            &= \pi \cdot a^2 \cdot 2a - \dfrac{16}{15} \pi a^3\\
            &= \dfrac{14}{15} \pi a^3
        \end{align*}
        \boxt{The volume required is $\dfrac{14}{15} \pi a^3$ units$^3$.}
        

    \problem{}
        The region bounded by the axes and the curve $y = \cos x$ from $x = 0$ to $x = \dfrac12 \pi$ is divided into two parts, of areas $A_1$ and $A_2$, by the curve $y = \sin x$.

        \begin{enumerate}
            \item Prove that $A_2 = \sqrt2 A_1$.
            \item Find the volume of the solid obtained when the region with area $A_2$ is rotated about the $y$-axis through $2\pi$ radians. Give your answer in exact form.
        \end{enumerate}

        \begin{center}
            \begin{tikzpicture}[trim axis left, trim axis right]
                \begin{axis}[
                    domain = 0:pi/2,
                    samples = 101,
                    axis y line=middle,
                    axis x line=middle,
                    xtick = {pi/2},
                    ytick = \empty,
                    xticklabels = {$\pi/2$},
                    xmax=pi/2 + 0.2,
                    ymax=1.2,
                    xlabel = {$x$},
                    ylabel = {$y$},
                    legend cell align={left},
                    legend pos=outer north east,
                    after end axis/.code={
                        \path (axis cs:0,0) 
                            node [anchor=north east] {$O$};
                        }
                    ]

                    \addplot[plotRed] {cos(\x r)};
        
                    \addlegendentry{$y = \cos x$};

                    \addplot[plotBlue] {sin(\x r)};

                    \addlegendentry{$y = \sin x$};

                    \node at (0.3, 0.65) {$A_1$};

                    \node at (pi/4, 0.3) {$A_2$};
                \end{axis}
            \end{tikzpicture}
        \end{center}

    \solution
        \part
            \begin{center}
                \begin{tikzpicture}[trim axis left, trim axis right]
                    \begin{axis}[
                        domain = 0:pi/2,
                        samples = 101,
                        axis y line=middle,
                        axis x line=middle,
                        xtick = {pi/2, pi/4},
                        ytick = \empty,
                        xticklabels = {$\pi/2$, $\pi/4$},
                        xmax=pi/2 + 0.2,
                        ymax=1.2,
                        xlabel = {$x$},
                        ylabel = {$y$},
                        legend cell align={left},
                        legend pos=outer north east,
                        after end axis/.code={
                            \path (axis cs:0,0) 
                                node [anchor=north east] {$O$};
                            }
                        ]

                        \addplot[plotRed] {cos(\x r)};
            
                        \addlegendentry{$y = \cos x$};

                        \addplot[plotBlue] {sin(\x r)};

                        \addlegendentry{$y = \sin x$};

                        \node at (0.3, 0.65) {$A_1$};

                        \draw[dotted] (pi/4, 0) -- (pi/4, 1/2 * sqrt 2);

                        \node at (pi/4 - 0.2, 0.3) {$A_3$};

                        \node at (pi/4 + 0.2, 0.3) {$A_4$};
                    \end{axis}
                \end{tikzpicture}
            \end{center}

            Let $A_3$ and $A_4$ be the areas as defined on the diagram above. By the symmetry of $y = \sin x$ and $y = \cos x$ about $x = \pi/4$, we have $A_3 = A_4$.
            {\allowdisplaybreaks
            \begin{alignat*}{2}
                && A_3 &= \int_0^{\pi/4} \sin x \d x\\
                && &= \evalint{-\cos x}{0}{\pi/4}\\
                && &= 1 - \dfrac{\sqrt2}2\\
                && A_1 &= \int_0^{\pi/4} \cos x \d x - A_3\\
                && &= \evalint{\sin x}{0}{\pi/4} - \bp{1 - \dfrac{\sqrt2}2}\\
                && &= \dfrac{\sqrt2}{2} - 1 + \dfrac{\sqrt2}2\\
                && &= \sqrt2 - 1\\
                \implies&&A_2 &= 2 A_3\\
                && &= 2\bp{1 - \dfrac{\sqrt2}2}\\
                && &= 2 - \sqrt{2}\\
                && &= \sqrt{2} \bp{\sqrt2 - 1}\\
                && &= \sqrt2 A_1
            \end{alignat*}}

        \part
            Let $V_3$ and $V_4$ be the volumes of the solids obtained when $A_3$ and $A_4$ are rotated about the $y$-axis through $2\pi$ radians, respectively.
            {\allowdisplaybreaks
            \begin{alignat*}{2}
                && V_3 &= 2\pi\int_0^{\pi/4} xy \d x\\
                && &= 2\pi \int_0^{\pi/4} x\sin x \d x\\\\
                && &\begin{array}{r c @{\hspace*{1.0cm}} c}\toprule
                    & D & I \\\cmidrule{1-3}
                    + & x & \sin x \\
                    - & 1 & -\cos x\\
                    + & 0 & -\sin x \\\bottomrule
                \end{array}\\\\
                && &= 2\pi \evalint{-x\cos x + \sin x}{0}{\pi/4}\\
                && &= 2\pi\bs{\bp{-\dfrac\pi4 \cos \dfrac\pi4 + \sin \dfrac\pi4} - \bp{0 + \sin 0}}\\
                && &= 2\pi \bp{-\dfrac\pi4 \cdot \dfrac{\sqrt2}2 + \dfrac{\sqrt2}2}\\
                && &= 2\pi \cdot \dfrac{\sqrt2}2 \bp{1 - \dfrac\pi4}\\
                && &= \sqrt2 \pi \bp{1 -\dfrac\pi4}\\\\
                && V_4 &= 2\pi \int_{\pi/4}^{\pi/2} xy \d x\\
                && &= 2\pi \int_{\pi/4}^{\pi/2} x\cos x \d x\\\\
                && &\begin{array}{r c @{\hspace*{1.0cm}} c}\toprule
                    & D & I \\\cmidrule{1-3}
                    + & x & \cos x \\
                    - & 1 & \sin x\\
                    + & 0 & -\cos x \\\bottomrule
                \end{array}\\\\
                && &= 2\pi \evalint{x\sin x + \cos x}{\pi/4}{\pi/2}\\
                && &= 2\pi \bp{\bs{\dfrac\pi2 \sin \dfrac\pi2 + \cos\dfrac\pi2} - \bs{\dfrac\pi4\sin\dfrac\pi4 + \cos\dfrac\pi4}}\\
                && &= 2\pi \bs{\dfrac\pi2 - \bp{\dfrac\pi4 \cdot \dfrac{\sqrt2}2 + \dfrac{\sqrt2}2}}\\
                && &= 2\pi \bs{\dfrac\pi2 - \dfrac{\sqrt2}2 \bp{1 + \dfrac\pi4}}\\
                && &= \pi^2 - \sqrt2 \pi \bp{1 + \dfrac\pi4}\\
                \implies&&\text{Required volume} &= V_3 + V_4\\
                && &= \sqrt2 \pi \bp{1 - \dfrac\pi4} + \pi^2 - \sqrt2 \pi \bp{1 + \dfrac\pi4}\\
                && &= \pi^2 - 2 \cdot \sqrt2 \pi \cdot \dfrac\pi4\\
                && &= \pi^2 - \dfrac{\sqrt2}2 \pi^2
            \end{alignat*}}
            \boxt{The required volume is $\bp{\pi^2 - \dfrac{\sqrt2}2 \pi^2}$ units$^3$.}

    \problem{}
        A curve has parametric equations
        \[
            x = \cos^2 t, \, y = \sin^3 t, \, 0 \leq t \leq \dfrac12 \pi
        \]
        
        \begin{enumerate}
            \item Sketch the curve.
            \item Show that the area under the curve for $0 \leq t \leq \dfrac12 \pi$ is $2\displaystyle\int_0^{\pi/2} \cos t \sin^4 t \d t$, and find the exact value of the area.
            \item Find the volume of the solid obtained when the region in (b) is rotated about the $y$-axis through $2\pi$ radians.
        \end{enumerate}

    \solution
        \part
            \begin{center}
                \begin{tikzpicture}[trim axis left, trim axis right]
                    \begin{axis}[
                        domain = 0:pi/2,
                        samples = 101,
                        axis y line=middle,
                        axis x line=middle,
                        xtick = {1},
                        ytick = {1},
                        xlabel = {$x$},
                        ylabel = {$y$},
                        xmax=1.1,
                        ymax=1.1,
                        legend cell align={left},
                        legend pos=outer north east,
                        after end axis/.code={
                            \path (axis cs:0,0) 
                                node [anchor=north east] {$O$};
                            }
                        ]
                        \addplot[plotRed] ({cos(\x r)^2}, {sin(\x r)^3});
            
                        \addlegendentry{$x = \cos^2 t, \, y = \sin^3 t$};
                    \end{axis}
                \end{tikzpicture}
            \end{center}

        \part
            Note that $x = 0 \implies t = \dfrac\pi2$ and $x = 1 \implies t = 0$. Hence,
            {\allowdisplaybreaks
            \begin{align*}
                \area &= \int_0^1 y \d x\\
                &= \int_{\pi/2}^0 y \der{x}{t} \d t\\
                &= \int_{\pi/2}^0 \sin^3 t \cdot (-2\cos t \sin t) \d t\\
                &= 2\int_0^{\pi/2} \cos t \sin^4 t \d t\\
                &= B(5/2, 1)\\
                &= \dfrac{\G(5/2) \, \G(1)}{\G(5/2 + 1)}\\
                &= \dfrac{\G(5/2) \cdot 1}{5/2 \cdot \G(5/2)}\\
                &= \dfrac25
            \end{align*}}

            \boxt{The area under the curve is $\dfrac25$ units$^2$.}

        \part
            \begin{align*}
                \volume &= 2\pi \int_0^1 xy \d x\\
                &= 2\pi \int_{\pi/2}^0 \cos^2 t \sin^3 t \cdot (-2\cos t \sin t) \d t\\
                &= 2\pi \cdot 2 \int_0^{\pi/2} \cos^3 t \sin^4 t \d t\\
                &= 2\pi \cdot B(5/2, 2)\\
                &= 2\pi \cdot \dfrac{\G(5/2) \, \G(2)}{\G(5/2 + 2)}\\
                &= 2\pi \cdot \dfrac{\G(5/2) \cdot 1}{7/2 \cdot 5/2 \cdot \G(5/2)}\\
                &= 2\pi \cdot \dfrac27 \cdot \dfrac25 \\
                &= \dfrac8{35}\pi
            \end{align*}

            \boxt{The required volume is $\dfrac8{35} \pi$ units$^3$.}

    \problem{}
        \begin{enumerate}
            \item Given that $f$ is a continuous function, explain, with the aid of a sketch, why the value of
            \[
                \lim_{n \to \infty} \dfrac1n \bs{f\of{\dfrac1n} + f\of{\dfrac2n} + \ldots + f\of{\dfrac{n}{n}}}
            \]
            is $\displaystyle\int_0^1 f(x) \d x$.
            \item Hence, evaluate $\displaystyle \lim_{n \to \infty} \dfrac1n \bp{\dfrac{\sqrt[3]1 + \sqrt[3]2 + \ldots + \sqrt[3]n}{\sqrt[3]n}}$.
        \end{enumerate}

    \solution
        \part
            \begin{center}
                \begin{tikzpicture}[trim axis left, trim axis right]
                    \begin{axis}[
                        domain = 0:1.2,
                        samples = 101,
                        axis y line=middle,
                        axis x line=middle,
                        xtick = {0.1, 0.2, 0.3, 0.4, 0.9, 1},
                        xticklabels = {$\frac1{n}$, $\frac2{n}$, $\frac3{n}$, $\frac4n$, $\frac{n-1}{n}$, $\frac{n}{n}$},
                        ytick = \empty,
                        xlabel = {$x$},
                        ylabel = {$y$},
                        legend cell align={left},
                        legend pos=outer north east,
                        after end axis/.code={
                            \path (axis cs:0,0) 
                                node [anchor=north east] {$O$};
                            }
                        ]
                        \addplot[plotRed] {x + sin(pi * \x r)};
            
                        \addlegendentry{$y = f(x)$};

                        \draw[plotBlue] (0.1, 0) -- (0.1, 0.409);
                        \draw[plotBlue] (0.1, 0.409) -- (0.2, 0.409);

                        \draw[plotBlue] (0.2, 0) -- (0.2, 0.788);
                        \draw[plotBlue] (0.2, 0.788) -- (0.3, 0.788);

                        \draw[plotBlue] (0.3, 0) -- (0.3, 1.11);
                        \draw[plotBlue] (0.3, 1.11) -- (0.4, 1.11);
                        \draw[plotBlue] (0.4, 0) -- (0.4, 1.11);

                        \draw[plotBlue] (0.9, 0) -- (0.9, 1.21);
                        \draw[plotBlue] (0.9, 1.21) -- (1, 1.21);

                        \draw[plotBlue] (1, 0) -- (1, 1.21);
                        \draw[plotBlue] (1, 1) -- (1.1, 1);
                        \draw[plotBlue] (1.1, 0) -- (1.1, 1);

                        \node[plotBlue] at (0.65, 0.5) {$\ldots$};
                    \end{axis}
                \end{tikzpicture}
            \end{center}

            The area of the rectangles in the above figure is given by
            \[
                \dfrac1n \bs{f\of{\dfrac1n} + f\of{\dfrac2n} + \ldots + f\of{\dfrac{n}{n}}}
            \]
            This gives an approximation of the signed area under the curve from $x = \dfrac1n$ to $x = \dfrac{n}{n} = 1$. As $n \to \infty$, the widths of the rectangles become smaller and the approximation becomes exact. Hence, 
            \[
                \lim_{n \to \infty} \dfrac1n \bs{f\of{\dfrac1n} + f\of{\dfrac2n} + \ldots + f\of{\dfrac{n}{n}}} = \int_0^1 f(x) \d x
            \]

        \part
            {\allowdisplaybreaks
            \begin{align*}
                \lim_{n \to \infty} \dfrac1n \bp{\dfrac{\sqrt[3]1 + \sqrt[3]2 + \ldots + \sqrt[3]n}{\sqrt[3]n}} &= \lim_{n \to \infty} \dfrac1n \bs{\sqrt[3]{\dfrac1n} + \sqrt[3]{\dfrac2n} + \ldots + \sqrt[3]{\dfrac{n}{n}}}\\
                &= \int_0^1 \sqrt[3]{x} \d x\\
                &= \evalint{\dfrac1{1/3 + 1} x^{1/3 + 1}}01\\
                &= \dfrac34
            \end{align*}}

            \boxt{$\displaystyle\lim_{n \to \infty} \dfrac1n \bp{\dfrac{\sqrt[3]1 + \sqrt[3]2 + \ldots + \sqrt[3]n}{\sqrt[3]n}} = \dfrac34$}

    \problem{}
        The function $f$ satisfies $f'(x) > 0$ for $a \leq x \leq b$, and $g$ is the inverse of $f$. By making a suitable change of variable, prove that
        \[
            \int_a^b f(x) \d x = b\b - a\a - \int_\a^\b g(y) \d y
        \]
        where $\a = f(a)$ and $\b = f(b)$. Interpret this formula geometrically by means of a sketch where $\a$ and $a$ are positive. Verify this result for the case where $f(x) = e^{2x}$, $a = 0$, $b = 1$.

        Prove similarly and interpret geometrically the formula
        \[
            2\pi \int_a^b xf(x) \d x = \pi(b^2\b - a^2\a) - \pi\int_\a^\b \bs{g(y)}^2 \d y
        \]

    \solution
        \begin{alignat*}{2}
            && \int_\a^\b g(y) \d y &= \int_a^b \inv f\of{f(x)} f'(x) \d x \usub{y &= f(x)\\\implies \d y &= f'(x) \d x}\\
            && &= \int_a^b xf'(x) \d x\\\\
            && &\begin{array}{r c @{\hspace*{1.0cm}} c}\toprule
                & D & I \\\cmidrule{1-3}
                + & x & f'(x) \\
                - & 1 & f(x) \\\bottomrule
            \end{array}\\\\
            && &= \evalint{xf(x)}{a}{b} - \int_a^b f(x) \d x \\
            && &= b\b - a\a - \int_a^b f(x) \d x\\
            \implies&&\int_a^b f(x) \d x &= b\b - a\a - \int_\a^\b g(y) \d y
        \end{alignat*}

        \dash

        \begin{center}
            \begin{tikzpicture}[trim axis left, trim axis right]
                \begin{axis}[
                    domain = 0:2.2,
                    samples = 101,
                    axis y line=middle,
                    axis x line=middle,
                    xtick = {0.5, 2},
                    ytick = {1.375, 4},
                    xticklabels = {$a$, $b$},
                    yticklabels = {$\a$, $\b$},
                    xlabel = {$x$},
                    ylabel = {$y$},
                    legend cell align={left},
                    legend pos=outer north east,
                    after end axis/.code={
                        \path (axis cs:0,0) 
                            node [anchor=north east] {$O$};
                        }
                    ]
                    \addplot[plotRed] {x^3 - 3*x^2 + 4*x};
        
                    \addlegendentry{$y = f(x)$};

                    \draw[dotted] (0.5, 0) -- (0.5, 1.375);
                    \draw[dotted] (0, 1.375) -- (0.5, 1.375);
                    \draw[dotted] (2, 0) -- (2, 4);
                    \draw[dotted] (0, 4) -- (2, 4);

                    \node at (0.3, 0.5) {$A$};
                    \node at (0.12, 0.9) {$C$};
                    \node at (1.3, 1) {$B$};
                    \node at (0.7, 3) {$D$};
                \end{axis}
            \end{tikzpicture}
        \end{center}

        Consider the above diagram. We clearly have $\area (A + C) = a\a$, $\area (A + B + C + D) = b\b$, $\area B = \displaystyle\int_a^b f(x) \d x$ and $\area D = \displaystyle\int_\a^\b g(y) \d y$. Thus,
        \begin{align*}
            \int_a^b f(x) \d x &= \area B \\
            &= \area (A + B + C + D) - \area (A + C) - \area D \\
            &= b\b - a\a - \int_\a^\b g(y) \d y
        \end{align*}

        \dash

        \textbf{Standard Way.}
        \[
            \int_0^1 e^{2x} \d x = \evalint{\dfrac12 e^{2x}}{0}{1} = \dfrac12 e^2 - \dfrac12
        \]

        \textbf{Via Formula.} Let $f(x) = e^{2x}$. Then $g(x) = \dfrac12 \ln x$. Hence, $\a = g(0) = 1$ and $\b = g(1) = e^2$. Invoking the above formula,
        \begin{align*}
            \int_0^1 e^{2x} \d x &= 1 \cdot e^2 - 0 \cdot 1 - \int_1^{e^2} \dfrac12 \ln x \d x\\
            &= e^2 - \dfrac12 \evalint{x \ln x - x}{1}{e^2}\\
            &= e^2 - \dfrac12\bs{\bp{e^2 \ln e^2 - e^2} - \bp{\ln 1 - 1}}\\
            &= e^2 - \dfrac12 \bp{e^2 + 1}\\
            &= \dfrac12 e^2 - \dfrac12
        \end{align*}
        Hence, the formula holds for the above case.

        \dash
        {\allowdisplaybreaks
        \begin{alignat*}{2}
            && \int_\a^\b \bs{g(y)}^2 \d y &= \int_\a^\b \bs{\inv f\of{f(x)}}^2 f'(x) \d x \usub{y &= f(x)\\\implies \d y &= f'(x) \d x}\\
            && &= \int_a^b x^2 f'(x) \d x\\\\
            && &\begin{array}{r c @{\hspace*{1.0cm}} c}\toprule
                & D & I \\\cmidrule{1-3}
                + & x^2 & f'(x) \\
                - & 2x & f(x) \\\bottomrule
            \end{array}\\\\
            && &= \evalint{x^2f(x)}{a}{b} - 2\int_a^b xf(x) \d x\\
            && &= b^2\b - a^2\a - 2\int_a^b x f(x) \d x\\
            \implies&&2\int_a^b x f(x) \d x &= b^2\b - a^2\a - \int_\a^\b \bs{g(y)}^2 \d y\\
            \implies&&2\pi\int_a^b x f(x) \d x &= \pi\bp{b^2\b - a^2\a} - \pi \int_\a^\b \bs{g(y)}^2 \d y
        \end{alignat*}}
        \dash
        
        \begin{center}
            \begin{tikzpicture}[trim axis left, trim axis right]
                \begin{axis}[
                    domain = 0:2.2,
                    samples = 101,
                    axis y line=middle,
                    axis x line=middle,
                    xtick = {0.5, 2},
                    ytick = {1.375, 4},
                    xticklabels = {$a$, $b$},
                    yticklabels = {$\a$, $\b$},
                    xlabel = {$x$},
                    ylabel = {$y$},
                    legend cell align={left},
                    legend pos=outer north east,
                    after end axis/.code={
                        \path (axis cs:0,0) 
                            node [anchor=north east] {$O$};
                        }
                    ]
                    \addplot[plotRed] {x^3 - 3*x^2 + 4*x};
        
                    \addlegendentry{$y = f(x)$};

                    \draw[dotted] (0.5, 0) -- (0.5, 1.375);
                    \draw[dotted] (0, 1.375) -- (0.5, 1.375);
                    \draw[dotted] (2, 0) -- (2, 4);
                    \draw[dotted] (0, 4) -- (2, 4);

                    \node at (0.3, 0.5) {$A$};
                    \node at (0.12, 0.9) {$C$};
                    \node at (1.3, 1) {$B$};
                    \node at (0.7, 3) {$D$};
                \end{axis}
            \end{tikzpicture}
        \end{center}
        Let $\volume R$ represent the volume of the solid obtained when the region $R$ is rotated completely about the $y$-axis.

        We clearly have $\volume (A + B + C + D) = \pi b^2 \b$, $\volume(A + C) = \pi a^2 \a$, $\volume B = 2\pi \displaystyle\int_a^b xf(x) \d x$ (using the shell method), and $\volume D = \pi \displaystyle\int_\a^\b \bs{g(y)}^2 \d y$ (using the disc method). Thus,
        \begin{align*}
            2\pi \displaystyle\int_a^b xf(x) \d x &= \volume B \\
            &= \volume (A + B + C + D) - \volume (A + C) - \volume D\\
            &= \pi b^2 \b - \pi a^2 \a - \pi \displaystyle\int_\a^\b \bs{g(y)}^2 \d y\\
            &= \pi\bp{b^2\b - a^2\a} - \pi \int_\a^\b \bs{g(y)}^2 \d y
        \end{align*}
\end{document}