\clearpage
\section{Assignment A2}

\begin{problem}
    By considering the graphs of $y = \cos x$ and $y = -\frac14 x$, or otherwise, show that the equation $x + 4\cos x = 0$ has one negative root and two positive roots.

    Use linear interpolation, once only, on the interval $[-1.5, 1]$ to find an approximation to the negative root of the equation $x + 4\cos x = 0$ correct to 2 decimal places.

    \begin{center}
        \begin{tikzpicture}[trim axis left, trim axis right]
            \begin{axis}[
                axis y line=middle,
                axis x line=middle,
                xtick = {3.595},
                xticklabels = {$\a$},
                ytick = \empty,
                xlabel = {$x$},
                ylabel = {$y$},
                xmin=2.3,
                legend cell align={left},
                legend pos=outer north east,
                after end axis/.code={
                    \path (axis cs:2.3,0) 
                        node [anchor=east] {$O$};
                    }
                ]

                \draw[thick,black!70,TwoMarks=0.1] (2.4,0) -- (2.6,0);
                
                \addplot[plotRed, domain=2.5:4] {x + 4*cos(\x r)};
    
                \addlegendentry{$y = x+4\cos x$};
            \end{axis}
        \end{tikzpicture}
    \end{center}

    The diagram shows part of the graph of $y = x + 4\cos x$ near the larger positive root, $\a$, of the equation $x + 4\cos x = 0$. Explain why, when using the Newton-Raphson method to find $\a$, an initial approximation which is smaller than $\a$ may not be satisfactory.

    Use the Newton-Raphson method to find $\a$ correct to 2 significant figures. You should demonstrate that your answer has the required accuracy.
\end{problem}
\begin{solution}
    \begin{center}
        \begin{tikzpicture}[trim axis left, trim axis right, scale=0.86]
            \begin{axis}[
                domain = -2:5,
                samples = 101,
                axis y line=middle,
                axis x line=middle,
                xtick = \empty,
                ytick = \empty,
                xlabel = {$x$},
                ylabel = {$y$},
                ymax=1.2,
                ymin=-1.2,
                legend cell align={left},
                legend pos=outer north east,
                after end axis/.code={
                    \path (axis cs:0,0) 
                        node [anchor=north east] {$O$};
                    }
                ]
                \addplot[plotRed, name path=f2] {cos(\x r)};
    
                \addlegendentry{$y=\cos x$};

                \addplot[plotBlue, name path=f1] {-0.25 * x};
    
                \addlegendentry{$y = -\tfrac14 x$};
    
                \fill [name intersections={of=f1 and f2,by={E1, E2, E3}}] (E1) circle[radius=2.5pt];

                \fill (E2) circle[radius=2.5pt];

                \fill (E3) circle[radius=2.5pt];
            \end{axis}
        \end{tikzpicture}
    \end{center}

    Note that $x + 4\cos x = 0 \implies \cos x = -\frac14 x$. Plotting the graphs of $y = \cos x$ and $y = -\frac14 x$, we see that there is one negative root and two positive roots. Hence, the equation $x + 4\cos x = 0$ has one negative root and two positive roots.

    Let $f(x) = x+4\cos x$. Let $\b$ be the negative root of the equation $f(x) = 0$. Using linear interpolation on the interval $[-1.5. -1]$, \[\b = \frac{-1.5f(-1) - (-1)f(-1.5)}{f(-1)-f(1.5)} = -1.24 \todp{2}.\]

    There is a minimum at $x = m$ such that $m$ is between the two positive roots. Hence, when using the Newton-Raphson method, an initial approximation which is smaller than $m$ would result in subsequent approximations being further away from the desired root $\a$. Hence, an initial approximation that is smaller than $\a$ may not be satisfactory.

    We know from the above graph that $\a \in (\pi, 3\pi/2)$. We hence pick $3\pi/2$ as our initial approximation. Using the Newton-Raphson method $x_{n+1} = x_n - \frac{f(x_n)}{f'(x_n)}$ with $x_1 = 3\pi/2$, we have
    \begin{table}[H]
        \centering
        \begin{tabular}{|c|c|}
        \hline
        $r$ & $x_r$ \\ \hline
        1 & $\frac32 \pi$ \\ \hline
        2 & 3.7699 \\ \hline
        3 & 3.6106 \\ \hline
        4 & 3.5955 \\ \hline
        5 & 3.5953 \\ \hline
        \end{tabular}
    \end{table}
    
    \noindent Since $f(3.55)f(3.65) = (-0.1)(0.2) < 0$, we have $\a \in (3.55, 3.65)$. Hence, $\a = 3.6 \tosf{2}$.
\end{solution}

\begin{problem}
    Find the coordinates of the stationary points on the graph $y = x^3 + x^2$. Sketch the graph and hence write down the set of values of the constant $k$ for which the equation $x^3 + x^2 = k$ has three distinct real roots.

    The positive root of the equation $x^3 + x^2 = 0.1$ is denoted by $\a$.

    \begin{enumerate}
        \item Find a first approximation to $\a$ by linear interpolation on the interval $0 \leq x \leq 1$.
        \item With the aid of a suitable figure, indicate why, in this case, linear interpolation does not give a good approximation to $\a$.
        \item Find an alternative first approximation to $\a$ by using the fact that if $x$ is small then $x^3$ is negligible when compared to $x^2$.
    \end{enumerate}
\end{problem}
\begin{solution}
    For stationary points, $y' = 0$. \[y' = 0 \implies 3x^2 + 2x = 0 \implies x(3x+2) = 0.\] Hence, $x = 0$ or $x = -2.3$. When $x = 0$, $y = 0$. When $x = -2/3$, $y = 4/{27}$. Thus, the coordinates of the stationary points of $y = x^3 + x^2$ are $(0, 0)$ and $(-2/3, 4/27)$.

    \begin{center}
        \begin{tikzpicture}[trim axis left, trim axis right, scale=0.86]
            \begin{axis}[
                domain = -1.1:0.4,
                samples = 101,
                axis y line=middle,
                axis x line=middle,
                xtick = {-1},
                ytick = \empty,
                xlabel = {$x$},
                ylabel = {$y$},
                legend cell align={left},
                legend pos=outer north east,
                after end axis/.code={
                    \path (axis cs:0,0) 
                        node [anchor=north east] {$O$};
                    }
                ]
                \addplot[plotRed] {x^3 + x^2};
    
                \addlegendentry{$y=x^3 + x^2$};

                \fill (-2/3, 4/27) circle[radius=2.5pt] node[anchor=south] {$\bp{-2/3, 4/27}$};
            \end{axis}
        \end{tikzpicture}
    \end{center}

    Therefore, $k \in (0, 4/27)$. The solution set of $k$ is thus $\bc{k \in \RR : 0 < k < 4/27}$.

    \begin{ppart}
        Let $f(x) = x^2 + x^2 - 0.1$. Using linear interpolation on the interval $[0, 1]$, \[\a = \frac{-f(0)}{f(1)-f(0)} = \frac1{20}.\]
    \end{ppart}
    \clearpage
    \begin{ppart}
        \begin{center}
            \begin{tikzpicture}[trim axis left, trim axis right, scale=0.86]
                \begin{axis}[
                    domain = 0:1,
                    samples = 101,
                    axis y line=middle,
                    axis x line=middle,
                    xtick = {0.279},
                    xticklabels={$\a$},
                    ytick = {-0.1},
                    xlabel = {$x$},
                    ylabel = {$y$},
                    ymin=-0.2,
                    legend cell align={left},
                    legend pos=outer north east,
                    after end axis/.code={
                        \path (axis cs:0,0) 
                            node [anchor=south east] {$O$};
                        }
                    ]
                    \addplot[plotRed] {x^3 + x^2 -0.1};
        
                    \addlegendentry{$y=x^3 + x^2-0.1$};

                    \fill (-2/3, 4/27) circle[radius=2.5pt] node[anchor=south] {$\bp{-\frac23, \frac4{27}}$};

                    \draw[plotBlue] (0, -0.1) -- (1, 1.9);
                \end{axis}
            \end{tikzpicture}
        \end{center}

        On the interval $[0, 1]$, the gradient of $y = x^3 + x^2 - 0.1$ changes considerably. Hence, linear interpolation gives an approximation much less than the actual value.
    \end{ppart}
    \begin{ppart}
        For small $x$, $x^3$ is negligible when compared to $x^2$. Consider $g(x) = x^2-0.1$. Then the positive root of $g(x)=0$ is approximately $\a$. Hence, an alternative approximation to $\a$ is $\sqrt{0.1} = 0.316 \tosf{3}$.
    \end{ppart}
\end{solution}

\begin{problem}
    The equation $2\cos x - x =0$ has a root $\a$ in the interval $[1, 1.2]$. Iterations of the form $x_{n+1} = F(x_n)$ are based on each of the following rearrangements of the equation:

    \begin{enumerate}
        \item $x = 2\cos x$
        \item $x = \cos x + \frac12 x$
        \item $x = \frac23 (\cos x + x)$
    \end{enumerate}

        Determine which iteration will converge to $\a$ and illustrate your answer by a `staircase' or `cobweb' diagram. Use the most appropriate iteration with $x_1 = 1$, to find $\a$ to 4 significant figures. You should demonstrate that your answer has the required accuracy.
\end{problem}
\begin{solution}
    \begin{ppart}
        Consider $f(x) = 2\cos x$. Then $f'(x) = -2\sin x$. Observe that $\sin x$ is increasing on $[1, 1.2]$. Since $\sin 1 > \frac12$, $\abs{f'(x)} > 1$ for all $x \in [1, 1.2]$. Thus, fixed-point iteration fails and will not converge to $\a$.

        \begin{center}
            \begin{tikzpicture}[trim axis left, trim axis right, scale=0.86]
                \begin{axis}[
                    domain = 0:pi/2,
                    samples = 101,
                    axis y line=middle,
                    axis x line=middle,
                    xtick = {1.2},
                    ytick = \empty,
                    xlabel = {$x$},
                    ylabel = {$y$},
                    ymax=2.2,
                    xmax=1.7,
                    legend cell align={left},
                    legend pos=outer north east,
                    after end axis/.code={
                        \path (axis cs:0,0) 
                            node [anchor=east] {$O$};
                        }
                    ]
                    \addplot[plotRed] {2 *cos(\x r)};

                    \addlegendentry{$y=2\cos x$};

                    \addplot[plotBlue] {x};
                    
                    \addlegendentry{$y=x$};
        
                    \draw[dotted, thick] (1.2, 0) -- (1.2, 0.725);

                    \begin{scope}[decoration={
                        markings,
                        mark=at position 0.5 with {\arrow{>}}}
                        ] 

                        \draw[postaction={decorate}] (1.2, 0.725) -- (0.725, 0.725);

                        \draw[postaction={decorate}] (0.725, 0.725) -- (0.725, 1.497);

                        \draw[postaction={decorate}] (0.725, 1.497) -- (1.497, 1.497);

                        \draw[postaction={decorate}] (1.497, 1.497) -- (1.497, 0.147);
                    \end{scope}
                \end{axis}
            \end{tikzpicture}
        \end{center}
    \end{ppart}
    \begin{ppart}
        Consider $f(x) = \cos x + \frac12 x$. Then $f'(x) = -\sin x + \frac12 -\bp{\sin x - \frac12}$. Since $0 \leq \sin x \leq 1$ for $x \in \bs{0, \frac{\pi}2}$, and $[1, 1.2] \subset \bs{0, \frac{\pi}2}$, we know $-\frac12 \leq \sin x - \frac12 \leq \frac12$ for $x \in [1, 1.2]$. Thus, $0 \leq \abs{\sin x - \frac12} \leq \frac12$ for $x \in [1, 1.2]$. Hence, fixed-point iteration will work and converge to $\a$.

        \begin{center}
            \begin{tikzpicture}[trim axis left, trim axis right, scale=0.86]
                \begin{axis}[
                    domain = 0:2,
                    samples = 101,
                    axis y line=middle,
                    axis x line=middle,
                    xtick = {1.7},
                    xticklabels = {1.2},
                    ytick = \empty,
                    xlabel = {$x$},
                    ylabel = {$y$},
                    ymax=1.4,
                    xmax=2,
                    legend cell align={left},
                    legend pos=outer north east,
                    after end axis/.code={
                        \path (axis cs:0,0) 
                            node [anchor=east] {$O$};
                        }
                    ]
                    \addplot[plotRed] {cos(\x r) + 0.5 * x};

                    \addlegendentry{$y=\cos x + \tfrac12 x$};

                    \addplot[plotBlue] {x};

                    \addlegendentry{$y=x$};

                    \draw[dotted, thick] (1.7, 0) -- (1.7, 0.721);

                    \begin{scope}[decoration={
                        markings,
                        mark=at position 0.5 with {\arrow{>}}}
                        ] 

                        \draw[postaction={decorate}] (1.7, 0.721) -- (0.721, 0.721);

                        \draw[postaction={decorate}] (0.721, 0.721) -- (0.721, 1.112);

                        \draw[postaction={decorate}] (0.721, 1.112) -- (1.112, 1.112);

                        \draw[postaction={decorate}] (1.112, 1.112) -- (1.112, 0.999);
                    \end{scope}
                \end{axis}
            \end{tikzpicture}
        \end{center}
    \end{ppart}
    \begin{ppart}
        Consider $f(x) = \frac23 (\cos x + x)$. Then $f'(x) = \frac23(-\sin x + 1)$. For fixed-point iteration to converge to $\a$, we need $\abs{f'(x)} < 1$ for $x$ near $\a$. It thus suffices to show that $\abs{-\sin x + 1} < \frac32$ for all $x \in [1, 1.2]$. Observe that $1 - \sin x$ is strictly decreasing and positive for $x \in \bs{0, \frac{\pi}2}$. Since $1 - \sin 1 < \frac32$, and $[1, 1.2] \subset \bs{0, \frac{\pi}2}$, we have that $\abs{-\sin x + 1} < \frac32$ for all $x \in [1, 1.2]$. Thus, $\abs{f'(x)} < 1$ for $x$ near $\a$. Hence, fixed-point iteration will work and converge to $\a$.

        \begin{center}
            \begin{tikzpicture}[trim axis left, trim axis right, scale=0.86]
                \begin{axis}[
                    domain = 0:3.3,
                    samples = 101,
                    axis y line=middle,
                    axis x line=middle,
                    xtick = {3},
                    xticklabels = {1.2},
                    ytick = \empty,
                    xlabel = {$x$},
                    ylabel = {$y$},
                    ymax=1.6,
                    legend cell align={left},
                    legend pos=outer north east,
                    after end axis/.code={
                        \path (axis cs:0,0) 
                            node [anchor=east] {$O$};
                        }
                    ]
                    \addplot[plotRed] {2/3 * (cos(\x r) + x)};

                    \addlegendentry{$y=\tfrac23 (\cos x + x)$};

                    \addplot[plotBlue] {x};

                    \addlegendentry{$y=x$};

                    \draw[dotted, thick] (3, 0) -- (3, 1.34);

                    \begin{scope}[decoration={
                        markings,
                        mark=at position 0.5 with {\arrow{>}}}
                        ] 

                        \draw[postaction={decorate}] (3, 1.34) -- (1.34,1.34);

                        \draw[postaction={decorate}] (1.34,1.34) -- (1.34, 1.046);

                        \draw[postaction={decorate}] (1.34, 1.046) -- (1.046, 1.046);
                    \end{scope}
                \end{axis}
            \end{tikzpicture}
        \end{center}

        For $x \in [1, 1.2]$, $\abs{\frac23 (-\sin x + 1)} < \abs{-\sin x + \frac12} < 1$. Thus, $x_{n+1} = \frac23 (\cos x_n + x_n)$ is the most suitable iteration as it will converge to $\a$ the quickest. Using $F(x_{n+1}) = \frac23 (\cos x_n + x_n)$ with $x_1 = 1$,
        \begin{table}[H]
            \centering
            \begin{tabular}{|c|c|}
            \hline
            $r$ & $x_r$ \\ \hline
            1 & 1 \\ \hline
            2 & 1.02687 \\ \hline
            3 & 1.02958 \\ \hline
            4 & 1.02984 \\ \hline
            5 & 1.02986 \\ \hline
            \end{tabular}
        \end{table}
        Since $F(1.0295) > 1.0295$ and $F(1.0305) < 1.0305$, we have $\a \in (1.0295, 1.0305)$. Hence, $\a = 1.030 \tosf{4}$.
    \end{ppart}
\end{solution}