\section{JC1 WA2 - H2 Further Mathematics 9649}

\begin{problem}
    Referred to the origin $O$, points $A$ and $B$ have position vectors $\vec a$ and $\vec b$ respectively where $\vec a$ and $\vec b$ are non-zero and non-parallel vectors. The point $C$ is such that $\oa{OC} = m \oa{OA}$ where $m$ is a constant. The point $D$ lies on $AB$ produced such that $B$ divides $AD$ in the ratio $1 : 2$.

    \begin{enumerate}
        \item Express the area of triangle $ADC$ in the form $k \abs{\vec a \crossp \vec b}$, where $k$ is an expression in terms of $m$. Show your working clearly.
        \item If $\oa{AC}$ is a unit vector, give a geometrical interpretation of the value of $\abs{\vec b \crossp \oa{AC}}$ and find the possible values of $m$ in terms of $\abs{\vec a}$.
    \end{enumerate}
\end{problem}
\begin{solution}
    \begin{ppart}
        \[\oa{OC} = m\oa{OA} = m \vec a \implies \oa{AC} = \oa{OC} - \oa{OA} = m \vec a - \vec a = (m-1) \vec a.\] By the Ratio Theorem, \[\oa{OB} = \frac{\oa{OD} + 2 \cdot \oa{OA}}{1 + 2}.\] Hence, \[\oa{OD} = 3\oa{OB} - 2\oa{OA} = 3\vec b - 2 \vec a \implies \oa{AD} = \oa{OD} - \oa{OA} = 3\vec b - 3\vec a.\] Thus,
        \begin{gather*}
            \area \triangle ADC = \frac12 \abs{\oa{AC} \crossp \oa{AD}} = \frac12 \abs{(m-1)\vec a \crossp (3\vec b - 3\vec a)} = \frac32 \abs{m-1} \abs{\vec a \crossp (\vec b - \vec a)}\\
            = \frac32 \abs{m-1} \abs{\vec a \crossp \vec b - \vec a \crossp \vec a} = \frac32 \abs{m-1} \abs{\vec a \crossp \vec b}
        \end{gather*}
        whence $k = \frac32 \abs{m-1}$.
    \end{ppart}
    \begin{ppart}
        Since $\oa{AC}$ is parallel to $\vec a$, if $\oa{AC}$ is a unit vector, then $\oa{AC} = \hat{\vec a}$. Hence, $\abs{\vec b \crossp \oa{AC}} = \abs{\vec b \crossp \hat{\vec a}}$ is the shortest distance from $B$ to the line $OA$.

        Since $\oa{AC}$ is a unit vector, we have \[\abs{\oa{AC}} = \abs{(m-1) \vec a} = 1 \implies \abs{m-1} = \frac1{\abs{\vec a}} \implies m = 1 \pm \frac1{\abs{\vec a}}.\]
    \end{ppart}
\end{solution}

\begin{problem}
    Marine biologist experts calculated that when the concentration of chemical $X$ in a sea inlet reaches 6 milligrams per litre (mg/l), the level of pollution endangers marine life. A factory wishes to release waste containing chemical $X$ into the inlet. It claimed that the discharge will not endanger the marine life, and they provided the local authority with the following information:

    \begin{itemize}
        \item There is no presence of chemical $X$ in the sea inlet at present.
        \item The plain is to discharge chemical $X$ on a weekly basis into the sea inlet. The discharge, which will be done at the beginning of each week, will result in an increase in concentration of 2.3 mg/l of chemical $X$ in the inlet.
        \item The tidal streams will remove 7\% of chemical $X$ from the inlet at the end of every day.
    \end{itemize}

    \begin{enumerate}
        \item Form a recurrence relation for the concentration level of chemical $X$, $u_n$, at the beginning of week $n$. Hence, find the concentration at the beginning of week $n$.
        \item Should the local authority allow the factory to go ahead with the discharge if they are concerned with the marine life at the sea inlet? Justify your answer.
    \end{enumerate}
\end{problem}
\begin{solution}
    \begin{ppart}
        We have \[u_n = 0.93^7 u_{n-1} + 2.3, \quad u_0 = 0.\]

        Let $k$ be the constant such that $u_n + k = 0.93^7 (u_{n-1} + k)$. Then $k = \frac{2.3}{0.93^7 - 1}$. Hence, 
        \begin{gather*}
            u_n - \frac{2.3}{1 - 0.93^7} = 0.93^7 \bp{u_{n-1} - \frac{2.3}{1 - 0.93^7}} = 0.93^{7n} \bp{u_0 - \frac{2.3}{1 - 0.93^7}} = - \frac{2.3 \cdot 0.93^{7n}}{1 - 0.93^7}\\
            \implies u_n = \frac{2.3}{1 - 0.93^7} - \frac{2.3 \cdot 0.93^{7n}}{1 - 0.93^7}.
        \end{gather*}
    \end{ppart}
    \begin{ppart}
        \[\lim_{n \to \infty} u_n = \lim_{n \to \infty} \bp{\frac{2.3}{1 - 0.93^7} - \frac{2.3 \cdot 0.93^{7n}}{1 - 0.93^7}} = \frac{2.3}{1 - 0.93^7} = 5.77 \tosf{3}.\] Since $5.77 < 6$, if the local authority's only concern is marine life, they should allow the factory to go ahead with the discharge.
    \end{ppart}
\end{solution}

\begin{problem}
    Referred to the origin $O$, the position vector of the point $A$ is $3 \vec i - 2 \vec j - 6 \vec k$ and the Cartesian equation of the line $l_1$ is $x-1 = 2-y = 2z+6$. 
        
    \begin{enumerate}
        \item Find the position vector of the foot of perpendicular from $A$ to $l_1$.
    \end{enumerate}

    Line $l_2$ has the vector equation $\vec r = \cveciiix{-1}6{-1} + \m \cveciiix{-6}{6}{-3}$, where $\m \in \RR$.

    \begin{enumerate}
        \setcounter{enumi}{1}
        \item Find the shortest distance between $l_1$ and $l_2$.
        \item Given that $l_2$ is the reflection of $l_1$ about the line $l_3$, find the vector equation of the line $l_3$.
    \end{enumerate}
\end{problem}
\begin{solution}
    \begin{ppart}
        Note that $l_1$ has vector equation
        \[
            l_1 : \vec r = \cveciii12{-3} + \l \cveciii{2}{-2}{1}, \, \l \in \RR
        \]
        Let $F$ be the foot of perpendicular from $A$ to $l_1$. Since $F$ is on $l_1$, $\oa{OF} = \cveciiix12{-3} + \l \cveciiix{2}{-2}{1}$ for some $\l \in \RR$.
        Thus, \[\oa{AF} = \oa{OF} - \oa{OA} = \cveciii12{-3} + \l \cveciii{2}{-2}{1} - \cveciii3{-2}{-6} = \cveciii{-2}43 + \l \cveciii{2}{-2}{1}.\] Note also that $\oa{AF}$ is perpendicular to $l_1$. Hence, \[\bs{\cveciii{-2}43 + \l \cveciii{2}{-2}{1}} \cdot \cveciii2{-2}1 = 0 \implies -9 + 9\l = 0 \implies \l = 1.\] Thus, \[\oa{OF} = \cveciii12{-3} + \cveciii{2}{-2}{1} = \cveciii30{-2}.\]
    \end{ppart}
    \begin{ppart}
        Note that $\cveciiix{-6}6{-3}\parallel \cveciiix2{-2}1$. Hence, $l_2$ is parallel to $l_1$. Hence, the shortest distance between $l_1$ and $l_2$ is the perpendicular distance from a point on $l_1$ to $l_2$, which is
        \begin{gather*}
            \frac{\abs{\bs{\cveciiix12{-3} - \cveciiix{-1}6{-1}} \crossp \cveciiix2{-2}1}}{\abs{\cveciiix2{-2}1}} = \frac1{\sqrt{9}} \abs{\cveciii2{-4}{-2} \crossp \cveciii{2}{-2}1}\\
            = \frac23 \abs{\cveciii{-1}21 \crossp \cveciii2{-2}1} = \frac23 \abs{\cveciii43{-2}} = \frac23 \sqrt{29} \units.
        \end{gather*}
    \end{ppart}
    \begin{ppart}
        Observe that $l_3$ passes through the midpoint of $\cveciiix12{-3}$ and $\cveciiix{-1}6{-1}$, which evaluates to \[\frac12 \bs{\cveciii12{-3} + \cveciii{-1}6{-1}} = \cveciii04{-2}.\] $l_3$ is also parallel to both $l_1$ and $l_2$. Hence, \[l_3 : \vec r = \cveciii04{-2} + \n \cveciii2{-2}1, \, \n \in \RR.\]
    \end{ppart}
\end{solution}

\begin{problem}
    A first order recurrence relation is given as \[u_{n+1} \bs{u_n + \bp{\frac12}^n} + u_n \bs{\bp{\frac12}^{n+1} - 10} = 10\bp{\frac12}^n - \bp{\frac12}^{2n+1} - 16\] where $u_1$ = 1.

    \begin{enumerate}
        \item Using the substitution $u_n = \frac{v_n}{v_{n-1}} - \bp{\frac12}^n$ where $v_{n-1} \neq 0$, show that the recurrence relation can be expressed as a second order recurrence relation of the form $v_{n+1} + av_n + 16v_{n-1} = 0$, where $a$ is a constant to be found.
        \item By first solving the second order recurrence relation in (a), find an expression for $u_n$ in terms of $n$.
        \item Describe what happens to the value of $u_n$ for large values of $n$.
    \end{enumerate}
\end{problem}
\clearpage
\begin{solution}
    \begin{ppart}
        Substituting in $v_n$ for $u_n$ into the LHS of the recurrence relation, we get
        \begin{align*}
            &u_{n+1} \bs{u_n + \bp{\frac12}^n} + u_n \bs{\bp{\frac12}^{n+1} - 10}\\
            &\hspace{2em}= \bs{\frac{v_{n+1}}{v_n} - \bp{\frac12}^{n+1}} \bs{\frac{v_n}{v_{n-1}}} + \bs{\frac{v_n}{v_{n-1}} - \bp{\frac12}^n} \bs{\bp{\frac12}^{n+1} - 10}\\
            &\hspace{2em}= \frac{v_{n+1}}{v_{n-1}} - 10 \bp{\frac{v_n}{v_{n-1}}} - \bp{\frac12}^{2n+1} + 10\bp{\frac12}^n.
        \end{align*}
        Cancelling terms from the RHS, we get \[\frac{v_{n+1}}{v_{n-1}} -10 \bp{\frac{v_n}{v_{n-1}}} = -16 \implies v_{n+1} - 10v_n + 16v_{n-1} = 0.\] Hence, $a = -10$.
    \end{ppart}
    \begin{ppart}
        Consider the characteristic equation of $v_n$. \[x^2 - 10x + 16 = (x-2)(x-8) = 0.\] Hence, 2 and 8 are the roots of the characteristic equation. Thus, \[v_n = A \cdot 2^n + B \cdot 8^n.\]

        Consider $u_1$. \[u_1 = \frac{v_1}{v_0} - \frac12 = 1 \implies  \frac{2A + 8B}{A + B} = \frac32 \implies \frac{4A + 16B}{A + B} = 3 \implies A = -13B.\]
        
        Now observe that
        \begin{gather*}
            \frac{v_n}{v_{n-1}} = \frac{A \cdot 2^n + B \cdot 8^n}{A \cdot 2^{n-1} + B \cdot 8^{n-1}} = \frac{-13 \cdot 2^n + 8^n}{-13 \cdot 2^{n-1} + 8^{n-1}} = 8\bp{\frac{-13 \cdot 2^n + 8^n}{-52 \cdot 2^n + 8^n}} \\
            = 8\bp{1 + \frac{39 \cdot 2^n}{-52 \cdot 2^n +  8^n}} = 8 - \frac{312 \cdot 2^n}{52 \cdot 2^n - 8^n}.
        \end{gather*}
        Thus, \[u_n = \frac{v_n}{v_{n-1}} - \bp{\frac12}^n = 8 - \frac{312 \cdot 2^n}{52 \cdot 2^n - 8^n} - \bp{\frac12}^n.\]
    \end{ppart}
    \begin{ppart}
        \[\lim_{n \to \infty} u_n = \lim_{n \to \infty} \bs{8 - \frac{312 \cdot 2^n}{52 \cdot 2^n - 8^n} - \bp{\frac12}^n} = 8.\] Thus, $u_n$ converges to 8 for large values of $n$.
    \end{ppart}
\end{solution}