\documentclass{echw}

\title{Assignment A8\\Vectors II}
\author{Eytan Chong}
\date{2024-05-17}

\begin{document}
    \problem{}
        Find the position vector of the foot of the perpendicular from the point with position vector $\vec c$ to the line with equation $\vec r = \vec a + \lambda \vec b$, $\lambda \in \mathbb{R}$. Leave your answers in terms of $\vec a$, $\vec b$ and $\vec c$.

    \solution
        Let the foot of the perpendicular be $F$. We have that $\overrightarrow{OF} = \vec a + \lambda \vec b$ for some real $\lambda$, and $\overrightarrow{CF} \cdot \vec b = 0$.
        
        \begin{alignat*}{2}
            &&\overrightarrow{CF} \cdot \vec b &= 0\\
            \implies&&\left(\overrightarrow{OF} - \overrightarrow{OC}\right) \cdot \vec b &= 0\\
            \implies&&(\vec a + \lambda \vec b - \vec c) \cdot \vec b &= 0\\
            \implies&&\lambda \vec b \cdot \vec b + (\vec a - \vec c) \cdot \vec b&= 0\\
            \implies&&\lambda \abs{\vec b}^2 &= (\vec c - \vec a) \cdot \vec b\\
            \implies&&\lambda &= \dfrac{(\vec c - \vec a) \cdot \vec b}{\abs{\vec b}^2}\\
            \implies&&\overrightarrow{OF} &= \vec a + \dfrac{(\vec c - \vec a) \cdot \vec b}{\abs{\vec b}^2} \, \vec b
        \end{alignat*}

        \boxt{
            $\overrightarrow{OF} = \vec a + \dfrac{(\vec c - \vec a) \cdot \vec b}{\abs{\vec b}^2} \, \vec b$
        }

    \problem{}
        The point $O$ is the origin, and points $A$, $B$, $C$ have position vectors given by $\overrightarrow{OA} = 6\vec i$, $\overrightarrow{OB} = 3 \vec j$, $\overrightarrow{OC} = 4\vec k$. The point $P$ is on line $AB$ between $A$ and $B$, and is such that $AP = 2PB$. The point $Q$ has position vector given by $\overrightarrow{OQ} = q\vec i$, where $q$ is a scalar.

        \begin{enumerate}
            \item Express, in terms of $\vec i$, $\vec j$, $\vec k$, the vector $\overrightarrow{CP}$.
            \item Show that the line $BQ$ has equation $\vec r = 3\vec j + t(q \vec i - 3\vec j)$, where $t$ is a parameter. Give an equation of the line $CP$ in a similar form.
            \item Find the value of $q$ for which the lines $CP$ and $BQ$ are perpendicular.
            \item Find the sine of the acute angle between the lines $CP$ and $BQ$ in terms of $q$.
        \end{enumerate}

    \solution
        We have that $\overrightarrow{OA} = \cveciii600$, $\overrightarrow{OB} = \cveciii030$ and $\overrightarrow{OC} = \cveciii004$.

        \part
            By the Ratio Theorem,
            \begin{alignat*}{2}
                &&\overrightarrow{OP} &= \dfrac{2\,\overrightarrow{OB} + \overrightarrow{OA}}{1 + 2}\\
                && &= \dfrac13 \left(2\cveciii030 + \cveciii600\right)\\
                && &= \cveciii220\\
                \implies&&\overrightarrow{CP} &= \overrightarrow{OP} - \overrightarrow{OC}\\
                && &= \cveciii220 - \cveciii004\\
                && &= \cveciii22{-4}
            \end{alignat*}

            \boxt{
                $\overrightarrow{CP} = 2\vec i + 2\vec j - 4\vec k$
            }

        \part
            Note that $\overrightarrow{BQ} = \overrightarrow{OQ} - \overrightarrow{OB} = \cveciii{q}00 - \cveciii030 = \cveciii{q}{-3}0$. Thus, $BQ$ is given by

            \begin{align*}
                \vec r &= \cveciii030 + t\cveciii{q}{-3}0, \, t \in \mathbb{R}\\
                \implies\vec r&= 3\vec j + t(q\vec i - 3\vec j), \, t \in \mathbb{R}
            \end{align*}

            Note that $\overrightarrow{CP} = \cveciii22{-4} = 2\cveciii11{-2}$. Hence, $CP$ is given by

            \begin{equation*}
                \vec r = \cveciii004 + u\cveciii11{-2}, \, u \in \mathbb{R}
            \end{equation*}

            \boxt{
                $CP: \vec r = 4\vec k + u(\vec i + \vec j - 2\vec k), \, u \in \mathbb{R}$
            }

        \part
            Since $CP$ is perpendicular to $BQ$, we have $\overrightarrow{CP} \cdot \overrightarrow{BQ} = 0$.

            \begin{alignat*}{2}
                &&\overrightarrow{CP} \cdot \overrightarrow{BQ} &= 0\\
                \implies&&2\cveciii11{-2} \cdot \cveciii{q}{-3}0 &= 0\\
                \implies&&q - 3 + 0 &= 0\\
                \implies&&q &= 3
            \end{alignat*}

            \boxt{
                $q = 3$
            }

        \part
            Let $\theta$ be the acute angle between $CP$ and $BQ$.

            {\allowdisplaybreaks
            \begin{alignat*}{2}
                &&\abs{\cveciii11{-2} \times \cveciii{q}{-3}0} &= \abs{\cveciii11{-2}} \abs{\cveciii{q}{-3}0} \sin \theta\\
                \implies&&\abs{\cveciii{-6}{2q}{3-q}} &=\abs{\cveciii11{-2}} \abs{\cveciii{q}{-3}0} \sin \theta\\
                \implies&&\sqrt{(-6)^2 + (2q)^2 + (3-q)^2} &= \sqrt{1^2 + 1^2 + (-2)^2} \cdot \sqrt{q^2 + (-3)^2 + 0^2} \cdot \sin \theta\\
                \implies&&\sqrt{36 + 4q^2 + 9 -6q + q^2} &= \sqrt{6} \cdot \sqrt{q^2 + 9} \cdot \sin \theta\\
                \implies&&\sqrt{5q^2 - 6q + 45} &= \sqrt{6(q^2 + 9)} \sin \theta\\
                \implies&&\sqrt{5q^2 - 6q + 45} &= \sqrt{6q^2 + 54} \sin \theta\\
                \implies&&\sin\theta &= \dfrac{\sqrt{5q^2 - 6q + 45}}{\sqrt{6q^2 + 54}}\\
                && &= \sqrt{\dfrac{5q^2 - 6q + 45}{6q^2 + 54}}
            \end{alignat*}
            }

            \boxt{
                $\sin\theta = \sqrt{\dfrac{5q^2 - 6q + 45}{6q^2 + 54}}$
            }

    \problem{}
        Line $l_1$ passes through the point $A$ with position vector $3\vec i - 2\vec k$ and is parallel to $-2\vec i + 4\vec j - \vec j$. Line $l_2$ has Cartesian equation given by $\dfrac{x-1}2 = y = z + 3$.

        \begin{enumerate}
            \item Show that the two lines intersect and find the coordinates of their point of intersection.
            \item Find the acute angle between the two lines $l_1$ and $l_2$. Hence, or otherwise, find the shortest distance from point $A$ to line $l_2$.
            \item Find the position vector of the foot $N$ of the perpendicular from $A$ to the line $l_2$. The point $B$ lies on the line $AN$ produced and is such that $N$ is the mid-point of $AB$. Find the position vector of $B$.
        \end{enumerate}

    \solution
        We have that
        \begin{equation*}
            l_1 : \vec r = \cveciii30{-2} + \lambda\cveciii{-2}4{-1}, \, \lambda \in \mathbb{R}
        \end{equation*}
        \noindent and
        \begin{equation*}
            l_2 : \vec r = \cveciii10{-3} + \mu\cveciii211, \, \mu \in \mathbb{R}
        \end{equation*}

        \part
            Consider $l_1 = l_2$.
            \begin{alignat*}{2}
                &&l_1 &= l_2\\
                \implies&&\cveciii30{-2} + \lambda\cveciii{-2}4{-1} &= \cveciii10{-3} + \mu\cveciii211\\
                \implies&&\mu\cveciii211 - \lambda\cveciii{-2}4{-1} &= \cveciii201
            \end{alignat*}
            This gives the following system:
            \begin{equation*}
                \systeme{2\mu + 2\lambda = 2, \mu - 4\lambda = 0, \mu + \lambda = 1}
            \end{equation*}
            \noindent which has the unique solution $\mu = \dfrac45$ and $\lambda = \dfrac15$. Thus, the intersection point $P$ has position vector $\cveciii30{-2} + \dfrac15\cveciii{-2}4{-1} = \dfrac15\cveciii{13}{4}{-11}$ and thus has coordinates $\coordiii{\dfrac{13}5}{\dfrac45}{-\dfrac{11}5}$.

            \boxt{
                $\coordiii{\dfrac{13}5}{\dfrac45}{-\dfrac{11}5}$
            }

        \part
            Let $\theta$ be the acute angle between $l_1$ and $l_2$.
            \begin{alignat*}{2}
                &&\cos\theta &= \dfrac{\abs{\cveciii{-2}4{-1} \cdot \cveciii211}}{\abs{\cveciii{-2}4{-1}} \abs{\cveciii211}}\\
                && &= \dfrac{\abs{-4 + 4 - 1}}{\sqrt{21}\cdot\sqrt{6}}\\
                && &= \dfrac1{\sqrt{126}}\\
                \implies&&\theta &= \arccos \dfrac1{\sqrt{126}}\\
                && &= 84.9^{\circ} \todp{1}
            \end{alignat*}
            
            \boxt{
                $\theta = 84.9^{\circ}$
            }

            Note that
            \begin{alignat*}{2}
                AP &= \sqrt{\left(\dfrac{17}5 - 3\right)^2 + \left(-\dfrac45 - 0\right)^2 + \left(-\dfrac95 - (-2)\right)^2}\\
                &= \sqrt{\dfrac{21}{25}}\\
                &= \dfrac{\sqrt{21}}5
            \end{alignat*}

            Since $\sin\theta = \dfrac{AN}{AP}$, we have that $AN = AP\sin\theta$.
            \begin{align*}
                AN &= \dfrac{\sqrt{21}}5 \sin \arccos \dfrac1{\sqrt{126}}\\
                &= \dfrac{\sqrt{21}}5 \cdot \dfrac{\sqrt{\left(\sqrt{126}\right)^2 - 1}}{\sqrt{126}}\\
                &= \dfrac{\sqrt{21}}5 \cdot \dfrac{\sqrt{125}}{\sqrt{126}}\\
                &= \dfrac{\sqrt{21}}5 \cdot \dfrac{5\sqrt{5}}{\sqrt{6} \cdot \sqrt{21}}\\
                &= \dfrac{\sqrt{5}}{\sqrt6}\\
                &= \sqrt{\dfrac56}
            \end{align*}

            \boxt{
                The shortest distance between $A$ and $l_2$ is $\sqrt{\dfrac56}$ units.
            }

        \part
            Since $N$ is on $l_2$, we have that $\overrightarrow{ON}= \cveciii10{03} + \mu\cveciii211$ for some real $\mu$.
            \begin{alignat*}{2}
                &&\overrightarrow{AN} \cdot \cveciii211 &= 0\\
                \implies&&\left(\overrightarrow{ON} - \overrightarrow{OA}\right) \cdot \cvecii211 &= 0\\
                \implies&&\left(\cveciii10{-3} + \mu\cveciii211 - \cveciii30{-2}\right) \cdot \cveciii211 &= 0\\
                \implies&&\cveciii{-2+2\mu}{\mu}{-1+\mu} \cdot \cveciii211 &= 0\\
                \implies&&2(-2 + 2\mu) + \mu + (-1 + \mu) &= 0\\
                \implies&&\mu &= \dfrac56\\
                \implies&&\overrightarrow{ON} &= \cveciii10{-3} + \dfrac56 \cveciii211\\
                && &= \dfrac16 \cveciii{16}5{-13}
            \end{alignat*}

            \boxt{
                $\overrightarrow{ON} = \dfrac16 \cveciii{16}5{-13}$
            }

            By the Ratio Theorem,
            \begin{alignat*}{2}
                &&\overrightarrow{ON} &= \dfrac{\overrightarrow{OA} + \overrightarrow{OB}}2\\
                \implies&&\overrightarrow{OB} &= 2\overrightarrow{ON} - \overrightarrow{OA}\\
                && &= \dfrac26\cveciii{16}5{-13} - \cveciii30{-2}\\
                && &= \dfrac13\cveciii75{-7}
            \end{alignat*}

            \boxt{
                $\overrightarrow{OB} = \dfrac13\cveciii75{-7}$
            }

\end{document}