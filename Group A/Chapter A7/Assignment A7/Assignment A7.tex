\documentclass{echw}

\title{Assignment A7\\Vectors I}
\author{Eytan Chong}
\date{2024-05-07}

\begin{document}
    \problem{}
        The points $A$ and $B$ have position vectors relative to the origin $O$, denoted by $\vec a$ and $\vec b$ respectively, where $\vec a$ and $\vec b$ are non-parallel vectors. The point $P$ lies on $AB$ such that $AP : PB = \l : 1$. The point $Q$ lies on $OP$ extended such that $OP = 2PQ$ and $\oa{BQ} = \oa{OA} + \m \oa{OB}$. Find the values of the real constants $\l$ and $\m$.

    \solution
        By the Ratio Theorem,
        \begin{alignat*}{2}
            &&\oa{OP} &= \dfrac{\vec a + \l \vec b}{1 + \l}\\
            \implies&&\oa{OQ} &= \oa{OP} + \oa{PQ}\\
            && &= \oa{OP} + \dfrac12\oa{OP}\\
            && &= \dfrac32 \cdot \dfrac{\vec a + \l \vec b}{1 + \l}\\
            &&\oa{BQ} &= \oa{OA} + \m \oa{OB}\\
            \implies&&\oa{OB} + \oa{BQ} &= \oa{OA} + (1 + \m)\oa{OB}\\
            \implies&&\oa{OQ} &= \oa{OA} + (1 + \m)\oa{OB}\\
            \implies&&\dfrac32 \cdot \dfrac{\vec a + \l \vec b}{1 + \l} &= \vec a + (1 + \m)\vec b
        \end{alignat*}

        Since $\vec a$ and $\vec b$ are non-parallel, we have the following system:
        \[
            \systeme{\dfrac32 \cdot \dfrac1{1 + \l} = 1, \dfrac32 \cdot \dfrac\l{1 + \l} = 1 + \m
            }
        \]
         which has the unique solution $\l = \dfrac12$ and $\m = -\dfrac12$.

        \boxt{$\l = \dfrac12$, $\m = -\dfrac12$}

    \problem{}
        Given that $\vec a = \vec i + \vec j$, $\vec b = 4 \vec i - 2 \vec j + 6 \vec k$ and $\vec p = \l \vec a + (1 - \l) \vec b$ where $\l \in \R$, find the possible value(s) of $\l$ for which the angle between $\vec p$ and $\vec k$ is $45\deg$.

    \solution
        \begin{alignat*}{2}
            &&\vec p &= \l \vec a + (1 - \l)\vec b\\
            && &= \l \cveciii110 + (1 - \l)\cveciii4{-2}6\\
            && &= \cveciii{4 - 3 \l}{-2 + 3\l}{6 - 6\l}\\
            \implies&&\abs{\vec p}^2 &= (4 - 3\l)^2 + (-2 + 3\l)^2 + (6 - 6\l)^2\\
            && &= 54\l^2 - 108\l + 56
        \end{alignat*}

        Since the angle between $\vec p$ and $\vec k$ is $45\deg$,
        \begin{alignat*}{2}
            &&\cos 45\deg &= \dfrac{\vec p \dotp \vec k}{\abs{\vec p} \abs{\vec k}}\\
            \implies&&\dfrac1{\sqrt2} &= \dfrac{(4-3\l) \cdot 0 + (-2 + 3\l) \cdot 0 + (6 - 6\l) \cdot 1}{\abs{\vec p} \cdot 1}\\
            \implies&&\dfrac{\abs{\vec p}}{\sqrt 2} &= 6 - 6\l\\
            \implies&&\dfrac{\abs{\vec p}^2}{2} &= (6 - 6\l)^2\\
            \implies&&\dfrac{54\l^2 - 108\l + 56}2 &= 36\l^2 - 72\l + 36\\
            \implies&&9\l^2 - 18\l + 8 &= 0\\
            \implies&&(3\l - 2)(3\l - 4) &= 0
        \end{alignat*}

        Hence, $\l = \dfrac23, \dfrac43$. However, we must reject $\l = \dfrac43$ since $6 - 6\l = \dfrac{\abs{\vec p}}{\sqrt{2}} > 0 \implies \l < 1$.

        \boxt{$\l = \dfrac23$}

    \problem{}
        \begin{enumerate}
            \item $\vec a$ and $\vec b$ are non-zero vectors such that $\vec a = (\vec a \dotp \vec b) \vec b$. State the relation between the directions of $\vec a$ and $\vec b$, and find $\abs{\vec b}$.
            \item $\vec a$ is a non-zero vector such that $\vec a = \sqrt3$ and $\vec b$ is a unit vector. Given that $\vec a$ and $\vec b$ are non-parallel and the angle between them is $\dfrac56 \pi$, find the exact value of the length of projection of $\vec a$ on $\vec b$. By considering $(2\vec a + \vec b)\dotp(2 \vec a + \vec b)$, or otherwise, find the exact value of $\abs{2\vec a + \vec b}$.
        \end{enumerate}

    \solution
        \part
            $\vec a$ and $\vec b$ either have the same or opposite direction.

            Let $\vec b = \l \vec a$ for some $\l \in \R$.
            \begin{alignat*}{2}
                &&\vec a &= (\vec a \dotp \vec b) \vec b\\
                \implies&&\vec a &= (\vec a \dotp \l \vec a) \l \vec a\\
                \implies&&\vec a &= \l^2 \abs{\vec a}^2 \vec a\\
                \implies&&\l^2\abs{\vec a}^2 &= 1\\
                \implies&&\l\abs{\vec a} &= \pm 1\\
                \implies&&\l &= \pm \dfrac1{\abs{\vec a}}\\
                \implies&&\vec b &= \pm \dfrac{\vec a}{\abs{\vec a}}\\
                \implies&&\vec b &= \pm \hat{\vec a}\\
                \implies&&\abs{\vec b} &= 1
            \end{alignat*}

            \boxt{$\abs{\vec b} = 1$}

        \part
            {\allowdisplaybreaks
            \begin{alignat*}{2}
                &&\abs{\vec a \dotp \vec b} &= \abs{\vec a} \abs{\vec b} \cos \dfrac56\pi\\
                && &= \sqrt3 \cdot 1 \cdot \bp{-\dfrac{\sqrt3}2}\\
                && &= -\dfrac32\\
                \implies&&\text{Length of projection of $\vec a$ on $\vec b$} &= \abs{\vec a \dotp \hat{\vec b}}\\
                && &= \abs{\vec a \dotp \vec b}\\
                && &= \abs{-\dfrac32}\\
                && &= \dfrac32
            \end{alignat*}}

            \boxt{$\text{Length of projection of $\vec a$ on $\vec b$} = \dfrac32$}

            \begin{alignat*}{2}
                &&\abs{2\vec a + \vec b}^2 &= (2\vec a + \vec b) \dotp (2\vec a + \vec b)\\
                && &= 2\vec a \dotp 2\vec a + 2\vec a \dotp \vec b + \vec b \dotp 2\vec a + \vec b \dotp \vec b\\
                && &= 4\vec a \dotp \vec a + 4 \vec a \dotp \vec b + \vec b \dotp \vec b\\
                && &= 4 \cdot 3 + 4 \cdot \bp{-\dfrac32} + 1^2\\
                && &= 7\\
                \implies&&\abs{2\vec a + \vec b} &= \sqrt7
            \end{alignat*}

            \boxt{$\abs{2\vec a + \vec b} = \sqrt7$}

    \problem{}
        The points $A$, $B$, $C$, $D$ have position vectors $\vec a$, $\vec b$, $\vec c$, $\vec d$ given by $\vec a = \vec i + 2\vec j + 3\vec k$, $\vec b = \vec i + 2\vec j + 2\vec k$, $\vec c = 3\vec i + 2 \vec j + \vec k$, $\vec d = 4\vec i - \vec j - \vec k$, respectively. The point $P$ lies on $AB$ produced such that $AP = 2AB$, and the point $Q$ is the mid-point of $AC$.

        \begin{enumerate}
            \item Show that $PQ$ is perpendicular to $AQ$.
            \item Find the area of the triangle $APQ$.
            \item Find a vector perpendicular to the plane $ABC$.
            \item Find the cosine of the angle between $\oa{AD}$ and $\oa{BD}$.
        \end{enumerate}

    \solution
        We recentre the vectors such that $\vec a$ is the origin. This gives $\vec a' = \cveciii000$, $\vec b' = \cveciii00{-1}$, $\vec c' = \cveciii20{-2}$, $\vec d = \cveciii3{-3}{-4}$. Hence, $\oa{OP}'$ is clearly $\cveciii00{-2}$, while $\oa{OQ}' = \dfrac12 \vec c' = \cveciii10{-1}$

        \part
            \begin{align*}
                \oa{PQ} \dotp \oa{AQ} &= \oa{PQ}' \dotp \oa{OQ}'\\
                &= \bp{\oa{OQ}' - \oa{OP}'} \dotp (\oa{OQ}')\\
                &= \cveciii101 \dotp \cveciii10{-1}\\
                &= 1 + 0 - 1\\
                &= 0
            \end{align*}

            Since $\oa{PQ} \dotp \oa{AQ} = 0$, the lines $PQ$ and $AQ$ must be perpendicular.

        \part
            \begin{align*}
                \area \triangle APQ &= \dfrac12 \abs{\oa{AP} \crossp \oa{AQ}'}\\
                &= \dfrac12 \abs{\oa{OP}' \crossp \oa{OQ}'}\\
                &= \dfrac12 \abs{\cveciii00{-2} \crossp \cveciii10{-1}}\\
                &= \dfrac12 \abs{\cveciii0{-2}0}\\
                &= 1
            \end{align*}

            \boxt{$\area \triangle APQ = 1$}

        \part
            \begin{align*}
                \vec b' \crossp \vec c' &= \cveciii00{-1} \crossp \cveciii20{-2}\\
                &= \cveciii0{-2}0
            \end{align*}

            \boxt{$\cveciii0{-2}0$ is perpendicular to the plane $ABC$.}

        \part
            Let the angle between $\oa{AD}$ and $\oa{BD}$ be $\t$. Note that $\oa{BD} = \vec d' - \vec b' = \cveciii3{-3}{-4} - \cveciii00{-1} = \cveciii3{-3}{-3} = -3\cveciii1{-1}{-1}$.

            \begin{align*}
                \cos\t &= \dfrac{\oa{AD} \dotp \oa{BD}}{\abs{\oa{AD}} \abs{\oa{BD}}}\\
                &= \dfrac1{\sqrt{3^2 + (-3)^2 + (-3)^2} \cdot 3\sqrt{1^2 + (-1)^2 + (-1)^2}} \cveciii3{-3}{-4} \dotp 3\cveciii1{-1}{-1}\\
                &= \dfrac1{3\sqrt{102}} \cdot 3\Big(3 \cdot 1 + (-3) \cdot (-1) + (-4) \cdot (-1)\Big)\\
                &= \dfrac{10}{\sqrt{102}}
            \end{align*}

            \boxt{$\cos \t = \dfrac{10}{\sqrt{102}}$}
\end{document}