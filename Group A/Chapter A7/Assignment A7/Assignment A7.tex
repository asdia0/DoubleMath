\documentclass{echw}

\title{Assignment A7\\Vectors I}
\author{Eytan Chong}
\date{2024-05-07}

\begin{document}
    \problem{}
        The points $A$ and $B$ have position vectors relative to the origin $O$, denoted by $\vec a$ and $\vec b$ respectively, where $\vec a$ and $\vec b$ are non-parallel vectors. The point $P$ lies on $AB$ such that $AP : PB = \lambda : 1$. The point $Q$ lies on $OP$ extended such that $OP = 2PQ$ and $\overrightarrow{BQ} = \overrightarrow{OA} + \mu \overrightarrow{OB}$. Find the values of the real constants $\lambda$ and $\mu$.

    \solution
        By the Ratio Theorem,
        \begin{alignat*}{2}
            &&\overrightarrow{OP} &= \dfrac{\vec a + \lambda \vec b}{1 + \lambda}\\
            \implies&&\overrightarrow{OQ} &= \overrightarrow{OP} + \overrightarrow{PQ}\\
            && &= \overrightarrow{OP} + \dfrac12\overrightarrow{OP}\\
            && &= \dfrac32 \cdot \dfrac{\vec a + \lambda \vec b}{1 + \lambda}\\
            &&\overrightarrow{BQ} &= \overrightarrow{OA} + \mu \overrightarrow{OB}\\
            \implies&&\overrightarrow{OB} + \overrightarrow{BQ} &= \overrightarrow{OA} + (1 + \mu)\overrightarrow{OB}\\
            \implies&&\overrightarrow{OQ} &= \overrightarrow{OA} + (1 + \mu)\overrightarrow{OB}\\
            \implies&&\dfrac32 \cdot \dfrac{\vec a + \lambda \vec b}{1 + \lambda} &= \vec a + (1 + \mu)\vec b
        \end{alignat*}

        Since $\vec a$ and $\vec b$ are non-parallel, we have the following system:
        \begin{equation*}
            \begin{cases}
                \dfrac32 \cdot \dfrac1{1 + \lambda} &= 1\\
                \dfrac32 \cdot \dfrac\lambda{1 + \lambda} &= 1 + \mu
            \end{cases}
        \end{equation*}
        \noindent which has the unique solution $\lambda = \dfrac12$ and $\mu = -\dfrac12$.

        \boxt{
            $\lambda = \dfrac12$, $\mu = -\dfrac12$
        }

    \problem{}
        Given that $\vec a = \vec i + \vec j$, $\vec b = 4 \vec i - 2 \vec j + 6 \vec k$ and $\vec p = \lambda \vec a + (1 - \lambda) \vec b$ where $\lambda \in \mathbb{R}$, find the possible value(s) of $\lambda$ for which the angle between $\vec p$ and $\vec k$ is $45^{\circ}$.

    \solution
        \begin{alignat*}{2}
            &&\vec p &= \lambda \vec a + (1 - \lambda)\vec b\\
            && &= \lambda \cveciii110 + (1 - \lambda)\cveciii4{-2}6\\
            && &= \cveciii{4 - 3 \lambda}{-2 + 3\lambda}{6 - 6\lambda}\\
            \implies&&\abs{\vec p}^2 &= (4 - 3\lambda)^2 + (-2 + 3\lambda)^2 + (6 - 6\lambda)^2\\
            && &= 54\lambda^2 - 108\lambda + 56
        \end{alignat*}

        Since the angle between $\vec p$ and $\vec k$ is $45^{\circ}$,
        \begin{alignat*}{2}
            &&\cos 45^{\circ} &= \dfrac{\vec p \cdot \vec k}{\abs{\vec p} \abs{\vec k}}\\
            \implies&&\dfrac1{\sqrt2} &= \dfrac{(4-3\lambda) \cdot 0 + (-2 + 3\lambda) \cdot 0 + (6 - 6\lambda) \cdot 1}{\abs{\vec p} \cdot 1}\\
            \implies&&\dfrac{\abs{\vec p}}{\sqrt 2} &= 6 - 6\lambda\\
            \implies&&\dfrac{\abs{\vec p}^2}{2} &= (6 - 6\lambda)^2\\
            \implies&&\dfrac{54\lambda^2 - 108\lambda + 56}2 &= 36\lambda^2 - 72\lambda + 36\\
            \implies&&9\lambda^2 - 18\lambda + 8 &= 0\\
            \implies&&(3\lambda - 2)(3\lambda - 4) &= 0
        \end{alignat*}

        Hence, $\lambda = \dfrac23, \dfrac43$. However, we must reject $\lambda = \dfrac43$ since $6 - 6\lambda = \dfrac{\abs{\vec p}}{\sqrt{2}} > 0 \implies \lambda < 1$.

        \boxt{
            $\lambda = \dfrac23$
        }

    \problem{}
        \begin{enumerate}
            \item $\vec a$ and $\vec b$ are non-zero vectors such that $\vec a = (\vec a \cdot \vec b) \vec b$. State the relation between the directions of $\vec a$ and $\vec b$, and find $\abs{\vec b}$.
            \item $\vec a$ is a non-zero vector such that $\vec a = \sqrt3$ and $\vec b$ is a unit vector. Given that $\vec a$ and $\vec b$ are non-parallel and the angle between them is $\dfrac56 \pi$, find the exact value of the length of projection of $\vec a$ on $\vec b$. By considering $(2\vec a + \vec b)\cdot(2 \vec a + \vec b)$, or otherwise, find the exact value of $\abs{2\vec a + \vec b}$.
        \end{enumerate}

    \solution
        \part
            $\vec a$ and $\vec b$ either have the same or opposite direction.

            Let $\vec b = \lambda \vec a$ for some $\lambda \in \mathbb{R}$.
            \begin{alignat*}{2}
                &&\vec a &= (\vec a \cdot \vec b) \vec b\\
                \implies&&\vec a &= (\vec a \cdot \lambda \vec a) \lambda \vec a\\
                \implies&&\vec a &= \lambda^2 \abs{\vec a}^2 \vec a\\
                \implies&&\lambda^2\abs{\vec a}^2 &= 1\\
                \implies&&\lambda\abs{\vec a} &= \pm 1\\
                \implies&&\lambda &= \pm \dfrac1{\abs{\vec a}}\\
                \implies&&\vec b &= \pm \dfrac{\vec a}{\abs{\vec a}}\\
                \implies&&\vec b &= \pm \hat{\vec a}\\
                \implies&&\abs{\vec b} &= 1
            \end{alignat*}

            \boxt{
                $\abs{\vec b} = 1$
            }

        \part
            {\allowdisplaybreaks
            \begin{alignat*}{2}
                &&\abs{\vec a \cdot \vec b} &= \abs{\vec a} \abs{\vec b} \cos \dfrac56\pi\\
                && &= \sqrt3 \cdot 1 \cdot \left(-\dfrac{\sqrt3}2\right)\\
                && &= -\dfrac32\\
                \implies&&\text{length of projection of $\vec a$ on $\vec b$} &= \abs{\vec a \cdot \hat{\vec b}}\\
                && &= \abs{\vec a \cdot \vec b}\\
                && &= \abs{-\dfrac32}\\
                && &= \dfrac32
            \end{alignat*}
            }

            \boxt{
                $\text{length of projection of $\vec a$ on $\vec b$} = \dfrac32$
            }

            \begin{alignat*}{2}
                &&\abs{2\vec a + \vec b}^2 &= (2\vec a + \vec b) \cdot (2\vec a + \vec b)\\
                && &= 2\vec a \cdot 2\vec a + 2\vec a \cdot \vec b + \vec b \cdot 2\vec a + \vec b \cdot \vec b\\
                && &= 4\vec a \cdot \vec a + 4 \vec a \cdot \vec b + \vec b \cdot \vec b\\
                && &= 4 \cdot 3 + 4 \cdot \left(-\dfrac32\right) + 1^2\\
                && &= 7\\
                \implies&&\abs{2\vec a + \vec b} &= \sqrt7
            \end{alignat*}

            \boxt{
                $\abs{2\vec a + \vec b} = \sqrt7$
            }

    \problem{}
        The points $A$, $B$, $C$, $D$ have position vectors $\vec a$, $\vec b$, $\vec c$, $\vec d$ given by $\vec a = \vec i + 2\vec j + 3\vec k$, $\vec b = \vec i + 2\vec j + 2\vec k$, $\vec c = 3\vec i + 2 \vec j + \vec k$, $\vec d = 4\vec i - \vec j - \vec k$, respectively. The point $P$ lies on $AB$ produced such that $AP = 2AB$, and the point $Q$ is the mid-point of $AC$.

        \begin{enumerate}
            \item Show that $PQ$ is perpendicular to $AQ$.
            \item Find the area of the triangle $APQ$.
            \item Find a vector perpendicular to the plane $ABC$.
            \item Find the cosine of the angle between $\overrightarrow{AD}$ and $\overrightarrow{BD}$.
        \end{enumerate}

    \solution
        We recenter the vectors such that $\vec a$ is the origin. This gives $\vec a' = \cveciii000$, $\vec b' = \cveciii00{-1}$, $\vec c' = \cveciii20{-2}$, $\vec d = \cveciii3{-3}{-4}$. Hence, $\overrightarrow{OP}'$ is clearly $\cveciii00{-2}$, while $\overrightarrow{OQ}' = \dfrac12 \vec c' = \cveciii10{-1}$

        \part
            \begin{align*}
                \overrightarrow{PQ} \cdot \overrightarrow{AQ} &= \overrightarrow{PQ}' \cdot \overrightarrow{OQ}'\\
                &= \left(\overrightarrow{OQ}' - \overrightarrow{OP}'\right) \cdot (\overrightarrow{OQ}')\\
                &= \cveciii101 \cdot \cveciii10{-1}\\
                &= 1 + 0 - 1\\
                &= 0
            \end{align*}

            Since $\overrightarrow{PQ} \cdot \overrightarrow{AQ} = 0$, the lines $PQ$ and $AQ$ must be perpendicular.

        \part
            \begin{align*}
                \area \triangle APQ &= \dfrac12 \abs{\overrightarrow{AP} \times \overrightarrow{AQ}'}\\
                &= \dfrac12 \abs{\overrightarrow{OP}' \times \overrightarrow{OQ}'}\\
                &= \dfrac12 \abs{\cveciii00{-2} \times \cveciii10{-1}}\\
                &= \dfrac12 \abs{\cveciii0{-2}0}\\
                &= 1
            \end{align*}

            \boxt{
                $\area \triangle APQ = 1$
            }

        \part
            \begin{align*}
                \vec b' \times \vec c' &= \cveciii00{-1} \times \cveciii20{-2}\\
                &= \cveciii0{-2}0
            \end{align*}

            \boxt{
                $\cveciii0{-2}0$ is perpendicular to the plane $ABC$.
            }

        \part
            Let the angle between $\overrightarrow{AD}$ and $\overrightarrow{BD}$ be $\theta$. Note that $\overrightarrow{BD} = \vec d' - \vec b' = \cveciii3{-3}{-4} - \cveciii00{-1} = \cveciii3{-3}{-3} = -3\cveciii1{-1}{-1}$.

            \begin{align*}
                \cos\theta &= \dfrac{\overrightarrow{AD} \cdot \overrightarrow{BD}}{\abs{\overrightarrow{AD}} \abs{\overrightarrow{BD}}}\\
                &= \dfrac1{\sqrt{3^2 + (-3)^2 + (-3)^2} \cdot 3\sqrt{1^2 + (-1)^2 + (-1)^2}} \cveciii3{-3}{-4} \cdot 3\cveciii1{-1}{-1}\\
                &= \dfrac1{3\sqrt{102}} \cdot 3\Big(3 \cdot 1 + (-3) \cdot (-1) + (-4) \cdot (-1)\Big)\\
                &= \dfrac{10}{\sqrt{102}}
            \end{align*}

            \boxt{
                $\cos \theta = \dfrac{10}{\sqrt{102}}$
            }
\end{document}