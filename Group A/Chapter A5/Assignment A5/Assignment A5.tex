\documentclass{jhwhw}

\title{Assignment A5\\Recurrence Relations}
\author{Eytan Chong}
\date{2024-04-18}

\begin{document}
    \problem{}
        In an auction at a charity gala dinner, a group of wealthy businessmen are competing with each other to be the highest bidder. Each time one of them makes a bid amount, another counter-bids by 50\% more, less a service charge of ten dollars (e.g. If A bids \$1000, then B will bid \$1490). Let $u_n$ be the amount at the $n$th bid and $u_1$ be the initial amount.

        \begin{enumerate}
            \item Write down a recurrence relation that describes the bidding process.
            \item Show that $u_n = \$(1.5^{n-1} (u_1 - 20) + 20)$.
            \item The target amount to be raised is \$1 234 567 and the bidding stops when the bid amount meets or crosses this target amount. Given that $u_1 = 111$,
            \begin{enumerate}
                \item state the least number of bids required to meet this amount.
                \item find the winning bid amount, correct to the nearest thousand dollars.
            \end{enumerate}
        \end{enumerate}

    \solution
        \part
            \boxt{
                $u_{n+1} = 1.5u_n - 10$
            }

        \part
            Let $k$ be the constant such that $u_{n+1} + k = 1.5(u_n + k) \implies 0.5k = -10 \implies k = -20$. Hence, $u_{n+1} - 20 = 1.5(u_n - 20)$.
            \begin{alignat*}{2}
                &&u_{n+1} - 20 &= 1.5(u_n - 20)\\
                \implies&&u_n - 20 &= 1.5^{n-1}(u_1 - 20)\\
                \implies&&u_n &= 1.5^{n-1}(u_1 - 20) + 20
            \end{alignat*}

        \part
            \subpart
                Let $m$ be the least integer such that $u_m \geq 1234567$.
                \begin{alignat*}{2}
                    &&u_m &\geq 1234567\\
                    \implies&&1.5^{m-1}(111 - 20) + 20 &\geq 1234567\\
                    \implies&&m &\geq 1 + \log_{1.5}\dfrac{1234567-20}{111-20}\\
                    && &= 24.5 \tosf{3}
                \end{alignat*}

                Hence, $m = 25$.

                \boxt{
                    It takes at least 25 bids to meet this amount.
                }

            \subpart
                \begin{align*}
                    u_{25} &= 1.5^{25-1}(111 - 20)\\
                    &= 1532000 \text{ (to nearest thousand)}
                \end{align*}

                \boxt{
                    The winning bid is \$1 532 000.
                }

    \problem{}
        Solve these recurrence relations together with the initial conditions.

        \begin{enumerate}
            \item $u_{n+2} = -u_n + 2u_{n+1}$, for $n \geq 0$, $u_0 = 5$, $u_1 = -1$.
            \item $4u_n = 4u_{n-1} + u_{n-2}$, for $n \geq 2$, $u_0 = a$, $u_1 = b$, $a, b \in \mathbb{R}$.
        \end{enumerate}

    \solution
        \part
            Consider the characteristic equation of $u_n$.
            \begin{alignat*}{2}
                &&x^2 - 2x + 1 &= 0\\
                \implies&&(x-1)^2 &= 0
            \end{alignat*}

            Hence, the only root of the characteristic equation is 1. Thus,
            \begin{align*}
                u_n &= (A + Bn)\cdot1^n\\
                &= A + Bn
            \end{align*}

            Since $u_0 = 5$,
            \begin{alignat*}{2}
                &&5 &= A + B \cdot 0\\
                \implies&&A &= 5
            \end{alignat*}

            Since $u_1 = -1$,
            \begin{alignat*}{2}
                &&-1 &= A + B \cdot 1\\
                \implies&&B &= -1 - A\\
                && &= -6
            \end{alignat*} 

            Thus,
            \boxt{
                $u_n = 5 - 6n$
            }

        \part
            \begin{alignat*}{2}
                &&4u_n &= 4u_{n-1} + u_{n-2}\\
                \implies&&u_n &= u_{n-1} + \dfrac14 u_{n-2}
            \end{alignat*}

            Consider the characteristic equation of $u_n$.
            \begin{alignat*}{2}
                &&x^2 - x - \dfrac14 &= 0\\
                \implies&&\left(x - \dfrac12\right)^2 - \dfrac12 &= 0\\
                \implies&&x &= \dfrac12 \pm \sqrt{\dfrac12}\\
                && &= \dfrac{1 \pm \sqrt2}{2}
            \end{alignat*}

            Hence, the roots of the characteristic equation are $x = \dfrac{1 + \sqrt2}{2}$ and $x = \dfrac{1 - \sqrt2}2$. Thus,

            \begin{equation*}
                u_n = A \left(\dfrac{1 + \sqrt2}{2}\right)^n + B\left(\dfrac{1 - \sqrt2}{2}\right)^n
            \end{equation*}

            Since $u_0 = a$,
            \begin{alignat*}{2}
                &&a &= A \left(\dfrac{1 + \sqrt2}{2}\right)^0 + B\left(\dfrac{1 - \sqrt2}{2}\right)^0\\
                && &= A + B\\
                \implies&&B &= a - A
            \end{alignat*}

            Since $u_1 = b$,
            \begin{alignat*}{2}
                &&b &= A \left(\dfrac{1 + \sqrt2}{2}\right)^1 + B\left(\dfrac{1 - \sqrt2}{2}\right)^1\\
                && &= \dfrac12 \left(A + B + \sqrt2(A - B)\right)\\
                && &= \dfrac12 \left(a + \sqrt2(A - (a-A))\right)\\
                && &= \dfrac12 \left(a + \sqrt2(2A - a)\right)\\
                \implies&&A &= \dfrac12 \left(\dfrac1{\sqrt2} (2b -a) + a\right)\\
                && &= \dfrac{\sqrt2 - 1}{2\sqrt2}a + \dfrac{1}{\sqrt{2}}b\\
                \implies&& B &= a - \left(\dfrac{\sqrt2 - 1}{2\sqrt2}a + \dfrac{1}{\sqrt{2}}b\right)\\
                && &= \dfrac{\sqrt2 + 1}{2\sqrt2}a - \dfrac{1}{\sqrt{2}}b
            \end{alignat*}

            Thus,
            \boxt{
                $u_n = \left(\dfrac{\sqrt2 - 1}{2\sqrt2}a + \dfrac{1}{\sqrt{2}}b\right)\left(\dfrac{1 + \sqrt2}{2}\right)^n + \left(\dfrac{\sqrt2 + 1}{2\sqrt2}a - \dfrac{1}{\sqrt{2}}b\right)\left(\dfrac{1 - \sqrt2}{2}\right)^n$
            }

    \problem{}
        A passcode is generated using the digits 1 to 5, with repetitions allowed. The passcodes are classified into two types. A Type $A$ passcode has an even number of the digit 1, while a Type $B$ passcode has an odd number of the digit 1. For example, a Type $A$ passcode is 1231, and a Type $B$ passcode is 1541213. Let $a_n$ and $b_n$ denote the number of $n$-digit Type $A$ and Type $B$ passcodes respectively.

        \begin{enumerate}
            \item State the values of $a_1$ and $a_2$.
            \item By considering the relationship between $a_n$ and $b_n$, show that
            \begin{equation*}
                a_n = xa_{n-1} + y^{n-1}, \qquad n \geq 2
            \end{equation*}
            where $x$ and $y$ are constants to be determined.
            \item Using the substitution $c_n = za_n + y^n$, where $z$ is a constant to be determined, find a first order linear recurrence relation for $c_n$. Hence find the general term formula for $a_n$.
        \end{enumerate}

    \solution
        \part
            \boxt{
                $a_1 = 4$, $a_2 = 17$
            }

        \part
            Let $P$ be a $n$-digit passcode with Type $T$, where $T \in \{A, B\}$. We also let Type $T^\prime$ be such that $T^\prime \in \{A, B\} \setminus T$.
            
            By concatenating a digit from 1 to 5 to $P$, 5 $(n+1)$-digit passcodes can be created. Let $P^\prime$ denote a newly passcode that is created via this process. If the digit 1 is concatenated, then $P^\prime$ is of Type $T^\prime$. If the digit 1 is not concatenated, then $P^\prime$ is of Type $T$. There are 4 choices for such a case. This hence gives the recurrence relations
            \begin{align*}
                a_n &= 4a_{n-1} + b_{n-1}\\
                b_n &= 4b_{n-1} + a_{n-1}
            \end{align*}

            Note that $a_n + b_n = 5(a_{n+1} + b_{n+1})$. Thus, $a_n + b_n = 5^{n-1}(a_1 + b_1) = 5^{n-1}(4 + 1) = 5^n$. Hence,
            \begin{align*}
                a_n &= 4a_{n-1} + b_{n-1}\\
                &= 3a_{n-1} + a_{n-1} + b_{n-1}\\
                &= 3a_{n-1} + 5^{n-1}
            \end{align*}
            \noindent whence $x = 3$ and $y = 5$.
        
        \part
            \begin{align*}
                c_n &= za_n + y^n\\
                &= za_n + 5^n\\
                &= z\left(3a_{n-1} + 5^{n-1}\right) + 5^n\\
                &= 3za_{n-1} + z5^{n-1} + 5\cdot5^{n-1}\\
                &= 3\left(za_{n-1} + 5^{n-1}\right) + z5^{n-1} + 5\cdot5^{n-1} - 3 \cdot 5^{n-1}\\
                &= 3c_{n-1} + (2 + z) 5^{n-1}\\
            \end{align*}

            Let $z = -2$. Then,
            \begin{align*}
                c_n &= 3c_{n-1}\\
                &= 3^{n-1} c_1\\
                &= 3^{n-1} \left(-2 a_1 + 5^1\right)\\
                &= -3 \cdot 3^{n-1}\\
            \end{align*}

            Note that $a_n = \dfrac1z \left(c_n - y^n\right)$. Thus,
            \begin{align*}
                a_n &= -\dfrac12\left(-3 \cdot 3^{n-1} - 5^n\right)\\
                &= \dfrac12 \left(3^n - 5^n\right)
            \end{align*}

            \boxt{
                $a_n = \dfrac12 \left(3^n - 5^n\right)$
            }

\end{document}