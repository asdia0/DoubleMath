\documentclass{jhwhw}

\title{Assignment A3\\Sequences and Series I}
\author{Eytan Chong}
\date{2024-04-01}

\begin{document}
    \problem{}
        A university student has a goal of saving at least \$1 000 000 (in Singapore dollars). He begins working at the start of the year 2019. In order to achieve his goal, he saves 40\% of his annual salary at the end of each year. If his annual salary in the year 2019 is \$40800 and it increases by 5\% (of his previous year's salary) every year, find

        \begin{enumerate}
            \item his annual savings in 2027 (to the nearest dollar),
            \item his total savings at the end of $n$ years.
        \end{enumerate}

        \noindent What is the minimum number of complete years for which he has to work in order to achieve his goal?

    \solution
        Let \$$u_n$ be his annual salary in the $n$th year after 2019, with $n \in \mathbb{N}$. Then $u_{n+1} = 1.05\cdot u_n$, with $u_0 = 40800$. Hence, $u_n = 40800 \cdot 1.05^n$.

        \part
            In 2027, $n = 8$. Hence,
            
            \begin{align*}
                \text{Annual savings in 2027} &= 0.40 \cdot u_8 \\
                &= 0.40 \cdot 40800 \cdot 1.05^8\\
                &= 24112 \text{ (to nearest integer)}
            \end{align*}

            \boxt{
                His annual savings in 2027 will be \$24112.
            }

        \part
            \begin{align*}
                \sum\limits_{k=0}^{n-1} 0.40 \cdot u_k &= \sum\limits_{k=0}^{n-1} 0.40 \cdot 40800 \cdot 1.05^k \\
                &= 16320 \sum\limits_{k=0}^{n-1} 1.05^k \\
                &= 16320 \sum\limits_{k=1}^{n} 1.05^{k-1} \\
                &= 16320 \cdot 1.05^{-1} \cdot \sum\limits_{k=1}^{n} 1.05^k \\
                &= 16320 \cdot 1.05^{-1} \cdot \dfrac{1.05 \cdot (1.05^k - 1)}{1.05-1} \\
                &= 326400 (1.05^k - 1)
            \end{align*}

            \boxt{
                His total savings at the end of $n$ years is \$$326400 (1.05^k - 1)$.
            }

            \begin{alignat*}{2}
                && 326400 (1.05^n - 1) &\geq 1 000 000\\
                && 1.05^n - 1 &\geq \dfrac{1 000 000}{326400}\\
                && 1.05^n &\geq \dfrac{1 000 000}{326400}+1\\
                && n &\geq \log_{1.05}(\dfrac{1 000 000}{326400}+1)\\
                && &= 28.7 \tosf{3}
            \end{alignat*}

            Since $n \in \mathbb{N}$, the minimum value of $n$ is 29. 

            \boxt{
                He has to work for a minimum of 29 complete years.
            }


    \problem{}
        \begin{enumerate}
            \item A rope of length $200\pi$ cm is cut into pieces to form as many circles as possible, whose radii follow an arithmetic progression with common difference $0.25$ cm. Given that the smallest circle has an area of $\pi$ cm$^2$, find the area of the largest circle in terms of $\pi$.
            \item The sum of the first $n$ terms of a sequence is given by $S_n = \alpha^{-n}-1$, where $\alpha$ is a non-zero constant, $\alpha \neq 1$.
            \begin{enumerate}
                \item Show that the sequence is a geometric progression and state its common ratio in terms of $\alpha$.
                \item Find the set of values of $\alpha$ for which the sum to infinity of the sequence exists.
                \item Find the value of the sum to infinity.
            \end{enumerate}
        \end{enumerate}

    \solution
        \part
            Let the sequence $r_n$ be the radius of the $n$th smallest circle, in centimeters. Hence, $r_n = \dfrac14 + r_{n-1}$. Since the smallest circle has area $\pi$ cm$^2$, $r_1 = 1$. Thus, $r_n = 1 + \dfrac14 (n - 1)$.

            Consider the $n$th partial sum of the circumferences.

            \begin{align*}
                \sum_{k=1}^n 2\pi r_k &= 2\pi \sum_{k=1}^n \left(1 + \dfrac14 (n-1)\right)\\
                &= 2\pi \left(n + \dfrac14 \cdot \dfrac{n(n+1)}2 - \dfrac14 n\right)\\
                &= 2\pi \left(\dfrac34 n + \dfrac18 n(n+1)\right)\\
                &= \dfrac14 \pi \left(6n + n(n+1)\right)\\
                &= \dfrac14 \pi (n^2 + 7n)\\
            \end{align*}

            Since the rope has length $200\pi$ cm, we have the inequality

            \begin{alignat*}{2}
                && \sum_{k=1}^n 2\pi r_k &\leq 200 \pi\\
                && \dfrac14 \pi (n^2 + 7n) &\leq 200 \pi\\
                && n^2 + 7n &\leq 800 \\
                && n^2 + 7n - 800 &\leq 0 \\
                && (n+32)(n-25) &\leq 0 \\
            \end{alignat*}

            Hence, $n \leq 25$. Since the rope is cut to form as many circles as possible, $n = 25$.

            Observe that $r_25 = 1 + \dfrac14 (25 - 1) = 7$. Hence, the largest circle has area $\pi \cdot 7^2 = 49\pi$ cm$^2$.

            \boxt{
                The largest circle has area $49\pi$ cm$^2$.
            }

        \part
            Let $S_n = \sum\limits_{k=1}^n u_k$.

            \begin{align*}
                u_{n+1} &= S_{n+1} - S_n\\
                &= \alpha^{-(n+1)}-1-(\alpha^{-n}-1)\\
                &= \alpha^{-n} \cdot \alpha^{-1} - \alpha^{-n}\\
                &= \alpha^{-n}(\alpha^{-1} - 1)
            \end{align*}

            \subpart
                \textbf{Test for Geometric Progression}

                \begin{align*}
                    \dfrac{u_{n+1}}{u_n} &= \dfrac{\alpha^{-(n+1)}(\alpha^{-1} - 1)}{\alpha^{-n}(\alpha^{-1} - 1)}\\
                    &= \dfrac{\alpha^{-(n+1)}}{\alpha^{-n}}\\
                    &= \alpha^{-1}
                \end{align*}

                Since $\alpha^{-1}$ is a constant, $u_n$ is in geometric progession with common ratio $\alpha^{-1}$.

                \boxt{
                    The common ratio of the sequence is $\alpha^{-1}$.
                }

            \subpart
                Consider $L = \lim\limits_{n \to \infty} S_n = \lim\limits_{n \to \infty} (\alpha^{-n} - 1)$. For $L$ to exist, we need $\lim\limits_{n \to \infty} \alpha^{-n}$ to exist. Hence, $\abs{\alpha^{-1}} < 1$, whence $\abs{\alpha} > 1$. Thus, $\alpha < -1$ or $\alpha > 1$.

                \boxt{
                    $\{ x \in \mathbb{R} \colon x < -1 \lor x > 1 \}$
                }

            \subpart
            Since $\abs{\alpha^{-1}} < 1$, we know $\lim\limits_{n \to infty} \alpha^{-n} = 0$. Hence,

            \begin{align*}
                \lim\limits_{n \to \infty} S_n &= \lim\limits_{n \to \infty} (\alpha^{-n} - 1)\\
                &= -1
            \end{align*}

            \boxt{
                The sum to infinity of the sequence is -1.
            }

    \problem{}
        A sequence $u_1, u_2, u_3, \ldots$ is such that $u_{n+1} = 2u_n + An$, where $A$ is a constant and $n \geq 1$.

        \begin{enumerate}
            \item Given that $u_1 = 5$ and $u_2 = 15$, find $A$ and $u_3$.
        \end{enumerate}

        \noindent It is known that the $n$th term of this sequence is given by

        \begin{equation*}
            u_n = a(2^n) + bn +c
        \end{equation*}

        \noindent where $a$, $b$ and $c$ are constants.

        \begin{enumerate}
            \setcounter{enumi}{1}
            \item Find $a$, $b$ and $c$.
        \end{enumerate}

    \solution
        \part
            \begin{align*}
                u_2 &= 2u_1 + A\cdot1\\
                &= 2\cdot 5 + A\\
                &= 10 + A\\
                &= 15
            \end{align*}

            Hence, $A = 5$.

            \begin{align*}
                u_3 &= 2u_2 + A\cdot 2\\
                &= 2 \cdot 15 + 2\cdot5\\
                &= 40
            \end{align*}

            \boxt{
                $A = 15$, $u_3 = 40$
            }

        \part
            Since $u_1=5$, $u_2 = 15$ and $u_3 = 40$, we have the following system of equations.

            \begin{numcases}{}
                2a + b + c &= 5\label{P3-1}\\
                4a + 2b + c &= 15\label{P3-2}\\
                8a + 3b + c &= 40
            \end{numcases}

            Subtracting Equation~\ref{P3-2} from twice of Equation~\ref{P3-1} yields $c=-5$. This reduces our system of equations to 

            \begin{numcases}{}
                4a+2b &= 20\label{P3-3}\\
                8a+3b &= 45\label{P3-4}
            \end{numcases}

            Subtracting Equation~\ref{P3-4} from twice of Equation~\ref{P3-3} yields $b=-5$, whence $a = \dfrac{15}2$.

            \boxt{
                $a = \dfrac{15}2$, $b=-5$, $c=-5$
            }

    \problem{}
        The graphs of $y = \dfrac13 (2^x)$ and $y=x$ intersect at $x = \alpha$ and $x=\beta$ where $\alpha < \beta$. A sequence of real numbers $x_1, x_2, x_3, \ldots$ satisfies the recurrence relation

        \begin{equation*}
            x_{n+1} = \dfrac13 (2^{x_n}), \qquad n \geq 1
        \end{equation*}

        \begin{enumerate}
            \item Prove algebraically that, if the sequence converges, then it converges to either $\alpha$ or $\beta$.
            \item By using the graphs of $y=\dfrac13(2^x)$ and $y=x$, prove that
            \begin{itemize}
                \item if $\alpha < x_n < \beta$, then $\alpha < x_{n+1} < x_n$
                \item if $x_n < \alpha$, then $x_n < x_{n+1} < \alpha$
                \item if $x_n > \beta$, then $x_n < x_{n+1}$
            \end{itemize}

            Describe the behaviour of the sequence for the three cases.
        \end{enumerate}

    \solution
        \part
            Let $L = \lim\limits_{n \to \infty} x_n$.

            \begin{alignat*}{2}
                && x_{n+1} &= \dfrac13 (2^{x_n})\\
                \implies&& \lim_{n \to \infty} x_{n+1} &= \lim_{n \to \infty} \dfrac13 (2^{x_n})\\
                \implies&& L &= \dfrac13 (2^L)\\
            \end{alignat*}

            Since $y=x$ and $y=\dfrac13 (2^x)$ intersect only at $x=\alpha$ and $x = \beta$, then $\alpha$ and $\beta$ are the only roots of $x = \dfrac13 (2^x)$. Since $L$ is also a root of $x = \dfrac13 (2^x)$, $L$ must be either $\alpha$ or $\beta$.

        \part
            \begin{center}
                \begin{tikzpicture}[trim axis left, trim axis right]
                    \begin{axis}[
                        domain = -1:4,
                        samples = 101,
                        axis y line=middle,
                        axis x line=middle,
                        xtick = {0.458, 3.313},
                        xticklabels = {$\alpha$, $\beta$},
                        ytick = \empty,
                        xlabel = {$x$},
                        ylabel = {$y$},
                        legend cell align={left},
                        legend pos=outer north east,
                        after end axis/.code={
                            \path (axis cs:0,0) 
                                node [anchor=north east] {$O$};
                            }
                        ]
                        \addplot[plotRed] {1/3*2^x};
            
                        \addlegendentry{$y = \tfrac13 (2^x)$};
            
                        \addplot[plotBlue] {x};
            
                        \addlegendentry{$y=x$};

                        \draw[dotted, thick] (-1.5, 0.458) -- (4, 0.458);

                        \draw[dotted, thick] (0.458, 0) -- (0. 458, 0.458);

                        \draw[dotted, thick] (3.313, 0) -- (3.313, 3.313);

                        \draw[dotted, thick] (1.89, 1.89) -- (1.89, 0);

                        \draw[dotted, thick] (-0.7, -0.7) -- (-0.7, 0.205);

                        \draw[dotted, thick] (3.5, 0) -- (3.5, 3.771);

                        \fill (1.89, 1.89) circle[radius=2.5 pt] node[anchor=south east] {$x_n$};

                        \fill (1.89, 1.24) circle[radius=2.5 pt] node[anchor=north west] {$x_{n+1}$};

                        \fill (-0.7, -0.7) circle[radius=2.5 pt] node[anchor=west] {$x_{n}$};

                        \fill (-0.7, 0.205) circle[radius=2.5 pt] node[anchor=south] {$x_{n+1}$};

                        \fill (3.5, 3.5) circle[radius=2.5 pt] node[anchor=north west] {$x_{n}$};

                        \fill (3.5, 3.771) circle[radius=2.5 pt] node[anchor=south east] {$x_{n+1}$};
                    \end{axis}
                \end{tikzpicture}
            \end{center}

            \boxt{
                \begin{center}
                    If $\alpha < x_n < \beta$, then $x_n$ is decreasing and converges to $\alpha$.\\
                    If $x_n < \alpha$, then $x_n$ is increasing and converges to $\alpha$.\\
                    If $x_n > \beta$, then $x_n$ is increasing and diverges.
                \end{center}
            }
\end{document}