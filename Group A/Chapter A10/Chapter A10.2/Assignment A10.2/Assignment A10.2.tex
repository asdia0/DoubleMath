\documentclass{echw}

\title{Assignment A10.2\\Complex Numbers}
\author{Eytan Chong}
\date{2024-07-18}

\begin{document}
    \problem{}
        On an Argand diagram, mark and label clearly the points $P$ and $Q$ representing the complex numbers $p$ and $q$ respectively, where \[p = \cos\dfrac\pi4 + i\sin\dfrac\pi4, \qquad q = 2\cos\dfrac\pi4 + 2i\sin\dfrac\pi4.\]

        Find the moduli and arguments of the complex numbers $a$, $b$, $c$, $d$ and $e$, where $a = p^4$, $b = q^2$, $c = -ip$, $d = \dfrac1q$, $e = p + p\cconj$.

        On your Argand diagram, mark and label the points $A$, $B$, $C$, $D$ and $E$ representing these complex numbers.

        Find the area of triangle $COQ$.

        Find the modulus and argument of $p^{13/3}q^{45/2}$.

    \solution
        \begin{center}
            \begin{tikzpicture}[trim axis left, trim axis right, scale=1.5]
                \begin{axis}[
                    domain = 0:10,
                    samples = 101,
                    axis y line=middle,
                    axis x line=middle,
                    xtick = \empty,
                    ytick = \empty,
                    xmax=2,
                    xmin=-2,
                    ymin=-1.2,
                    ymax=5,
                    xlabel = {$\Re$},
                    ylabel = {$\Im$},
                    legend cell align={left},
                    legend pos=outer north east,
                    after end axis/.code={
                        \path (axis cs:0,0) 
                            node [anchor=north east] {$O$};
                        }
                    ]
        
                    \coordinate (R) at (10,0);
                    \coordinate[label=above:$P$] (P) at (0.707, 0.707);
                    \coordinate[label=above:$Q$] (Q) at (1.414, 1.414);
                    \coordinate[label=above:$A$] (A) at (-1, 0);
                    \coordinate[label=right:$B$] (B) at (0, 4);
                    \coordinate[label=below:$C$] (C) at (0.707, -0.707);
                    \coordinate[label=below:$D$] (D) at (0.35355, -0.35355);
                    \coordinate[label=above:$E$] (E) at (1.414, 0);
                    \coordinate (O) at (0, 0);
            
                    \draw (O) -- (Q);
                    \draw (O) -- (C);
            
                    \fill (A) circle[radius=2.5pt];
                    \fill (B) circle[radius=2.5pt];
                    \fill (C) circle[radius=2.5pt];
                    \fill (D) circle[radius=2.5pt];
                    \fill (E) circle[radius=2.5pt];
                    \fill (P) circle[radius=2.5pt];
                    \fill (Q) circle[radius=2.5pt];

                    \draw pic [draw, angle radius=7mm, ""] {angle = E--O--P};
                    \draw pic [draw, angle radius=9mm, ""] {angle = C--O--E};

                    \node[anchor=south] at (0.7, -0.1) {$\pi/4$};
                    \node[anchor=north] at (0.8, 0.1) {$\pi/4$};
                \end{axis}
            \end{tikzpicture}
        \end{center}

        Note that $p = e^{i\pi/4}$ and $q = 2e^{i\pi/4}$.
        \begin{align*}
            a &= p^4 = \bp{e^{i\pi/4}}^4 = e^{i\pi}\\
            b &= q^2 = \bp{2e^{i\pi/4}}^2 = 4e^{i\pi/2}\\
            c &= -ip = e^{-i\pi/2} e^{i\pi/4} = e^{-i\pi/4}\\
            d &= \frac1q = \frac12 e^{-i\pi/4}\\
            e &= p + p\cconj = 2 \Re p = 2\cos{\frac\pi4} = \sqrt2
        \end{align*}

        \[
        \begin{array}{r c @{\hspace*{1.0cm}} c}\toprule
            & \text{modulus} & \text{argument} \\\cmidrule{1-3}
            a & 1 & \pi\\
            b & 4 & \pi/2\\
            c & 1 & -\pi/4\\
            d & 1/2 & -\pi/4\\
            e & \sqrt2 & 0\\\bottomrule
        \end{array}
        \]

        Since $\angle COQ = \dfrac\pi2$, we have $\area \triangle COQ = \dfrac12 \cdot 2 \cdot 1 = 1$ units$^2$.

        \boxt{$\area \triangle COQ = 1$ units$^2$}

        \begin{align*}
            p^{13/3}q^{45/2} &= \bp{e^{i\pi/4}}^{13/3} \bp{2e^{i\pi/4}}^{45/2}\\
            &= e^{i\pi 13/12} \cdot 2^{45/2} e^{i\pi 45/8}\\
            &= 2^{45/2} e^{i\pi 161/24}\\
            &= 2^{45/2} e^{i\pi 17/24}
        \end{align*}
        \boxt{$\abs{p^{13/3}q^{45/2}} = e^{45/2}$, $\arg{p^{13/3}q^{45/2}} = \dfrac{17}{24}\pi$}

    \problem{}
        The complex number $q$ is given by $q = \dfrac{e^{i2\t}}{1 - e^{i2\t}}$, where $0 < \t < 2\pi$. In either order,
        \begin{enumerate}
            \item find the real part of $q$,
            \item show that the imaginary part of $q$ is $\dfrac12 \cot\t$.
        \end{enumerate}

    \solution
        \begin{align*}
            q &= \dfrac{e^{i2\t}}{1 - e^{i2\t}}\\
            &= \dfrac{e^{i\t}}{e^{-i\t} - e^{i\t}}\\
            &= -\dfrac{e^{i\t}}{e^{i\t} - e^{-i\t}}\\
            &= -\dfrac{e^{i\t} / 2i}{(e^{i\t} - e^{-i\t})/2i}\\
            &= -\dfrac{\cos \t + i\sin \t}{2i} \cdot \dfrac{1}{\sin \t}\\
            &= -\dfrac{-i(\cos \t + i\sin \t)}{2} \cdot \dfrac{1}{\sin \t}\\
            &= \dfrac{- \sin \t + i\cos \t}{2} \cdot \dfrac{1}{\sin \t}\\
            &= \dfrac{-1 + i\cot \t}{2}\\
            &= -\dfrac12 + i \dfrac12 \cot \t
        \end{align*}
        \boxt{$\Re q = -\dfrac12$, $\Im q = \dfrac12 \cot \t$}
    

    \problem{}
        The complex numbers $z$ and $w$ are such that $z = 4\bp{\cos \dfrac34\pi + i\sin \dfrac34 \pi}$ and $w = 1 - i\sqrt3$. $z\cconj$ denotes the conjugate of $z$.

        \begin{enumerate}
            \item Find the modulus $r$ and the argument $\t$ of $\dfrac{w^2}{z\cconj}$, where $r > 0$ and $-\pi < \t < \pi$.
            \item Given that $\bp{\dfrac{w^2}{z\cconj}}^n$ is purely imaginary, find the set of values that $n$ can take.
        \end{enumerate}

    \solution
        \part
            Note that $z = 4e^{i3\pi/4}$ and $w = 2\bp{\dfrac12 - i\dfrac{\sqrt3}2} = 2\bs{\cos{-\dfrac\pi3} + i\sin{-\dfrac\pi3}} = 2e^{-i\pi/3}$.
            \begin{align*}
                \dfrac{w^2}{z\cconj} &= \dfrac{\bp{2e^{-i\pi/3}}^2}{4e^{-i3\pi/4}}\\
                &= \dfrac{4e^{-i2\pi/3}}{4e^{-i3\pi/4}}\\
                &= \dfrac{e^{-i2\pi/3}}{e^{-i3\pi/4}}\\
                &= e^{i\pi/12}
            \end{align*}
            \boxt{$r = 1$, $\t = \dfrac\pi{12}$}

        \part
            Note that $\bp{\dfrac{w^2}{z\cconj}}^n = \bp{e^{i\pi/12}}^n = e^{in\pi/12}$. Since $\bp{\dfrac{w^2}{z\cconj}}^n$ is purely imaginary, we have $\arg \bp{\dfrac{w^2}{z\cconj}}^n = \dfrac\pi2 + \pi k$, where $k \in \Z$. Thus, $\dfrac{n\pi}{12} = \dfrac\pi2 + \pi k$, whence $n =  6 + 12k$.
            \boxt{$n \in \{k \in \Z : 6 + 12 k\}$}

    \problem{}
        The complex number $w$ has modulus $\sqrt2$ and argument $\dfrac14\pi$ and the complex number $z$ has modulus $\sqrt2$ and argument $\dfrac56 \pi$.

        \begin{enumerate}
            \item By first expressing $w$ and $z$ in the form $x + iy$, find the exact real and imaginary parts of $w + z$.
            \item On the same Argand diagram, sketch the points $P$, $Q$, $R$ representing the complex numbers $z$, $w$, and $z + w$ respectively. State the geometrical shape of the quadrilateral $OPRQ$.
            \item Referring the Argand diagram in part (b), find $\arg{w + z}$ and show that $\tan \dfrac{11}{24}\pi = \dfrac{a + \sqrt2}{\sqrt6 + b}$, where $a$ and $b$ are constants to be determined.
        \end{enumerate}

    \solution
        \part
            \[w = \sqrt2 e^{i\pi/4} = \sqrt2 \bp{\cos \dfrac\pi4 + i\sin\dfrac\pi4} = \sqrt2 \bp{\dfrac1{\sqrt2} + i \dfrac1{\sqrt2}} = 1 + i\]
            \[z = \sqrt2 e^{i5\pi/6} = \sqrt2 \bp{\cos \dfrac56 \pi + i\sin \dfrac56\pi} = \sqrt2\bp{-\dfrac{\sqrt3}2 + i\dfrac12} = -\dfrac{\sqrt3}{\sqrt2} + i\dfrac1{\sqrt2}\]
            \[\implies w + z = (1 + i) + ( -\dfrac{\sqrt3}{\sqrt2} + i\dfrac1{\sqrt2}) = \bp{1 - \dfrac{\sqrt3}{\sqrt2}} + i\bp{1 + \dfrac1{\sqrt2}}\]

            \boxt{$w + z = \bp{1 - \dfrac{\sqrt3}{\sqrt2}} + i\bp{1 + \dfrac1{\sqrt2}}$}

        \part
            \begin{center}
                \begin{tikzpicture}[trim axis left, trim axis right]
                    \begin{axis}[
                        domain = 0:10,
                        samples = 101,
                        axis y line=middle,
                        axis x line=middle,
                        xtick = \empty,
                        ytick = \empty,
                        xmax=1.5,
                        xmin=-2.7,
                        ymin=0,
                        ymax=3,
                        xlabel = {$\Re$},
                        ylabel = {$\Im$},
                        legend cell align={left},
                        legend pos=outer north east,
                        after end axis/.code={
                            \path (axis cs:0,0) 
                                node [anchor=north] {$O$};
                            }
                        ]
            
                        \coordinate (R1) at (10,0);
                        \coordinate (L) at (-10, 0);
                        \coordinate[label=left:$P$] (P) at (-2.4495, 0.70711);
                        \coordinate[label=right:$Q$] (Q) at (1, 1);
                        \coordinate[label=above:$R$] (R) at (-1.4495, 1.70711);
                        \coordinate (O) at (0, 0);
                
                        \draw (O) -- (P);
                        \draw (O) -- (Q);
                        \draw (P) -- (R);
                        \draw (R) -- (Q);
                        \draw (O) -- (R);
                
                        \fill (P) circle[radius=2.5pt];
                        \fill (Q) circle[radius=2.5pt];
                        \fill (R) circle[radius=2.5pt];

                        \draw pic [draw, angle radius=8mm, ""] {angle = R1--O--Q};
                        \draw pic [draw, angle radius=12mm, ""] {angle = P--O--L};

                        \node at (0.6, 0.2) {$\frac\pi4$};
                        \node at (-1, 0.14) {$\frac\pi6$};
                    \end{axis}
                \end{tikzpicture}
            \end{center}
            \boxt{$OPRQ$ is a parallelogram.}

        \part
            Note that $\angle POQ = \pi - \dfrac\pi6 - \dfrac\pi4 = \dfrac7{12} \pi$. Since $\abs{z} = \abs{w}$, we have $OP = OQ$, whence $\angle ROQ = \dfrac12 \cdot \dfrac7{12}\pi = \dfrac7{24}\pi$. Hence, $\arg{w + z} = \dfrac\pi4 + \dfrac7{24}\pi = \dfrac{13}{24}\pi$.

            \boxt{$\arg{w + z} = \dfrac{13}{24}\pi$}

            Thus, \[\tan{\dfrac{13}{24}\pi} = \dfrac{1 + 1/\sqrt2}{1 - \sqrt3/\sqrt2} = \dfrac{\sqrt2 + 1}{\sqrt2 - \sqrt3} = \dfrac{2 + \sqrt2}{2 - \sqrt6}\]
            However, $\tan{\dfrac{13}{24}\pi} = -\tan{\pi - \dfrac{13}{24}} = -\tan{\dfrac{11}{24}\pi}$. Hence, \[\tan{\dfrac{11}{24}\pi} = -\dfrac{2 + \sqrt2}{2 - \sqrt6} = \dfrac{2 + \sqrt2}{\sqrt6 - 2}\]
            \boxt{$a = 2$, $b = -2$}

    \problem{}
        The complex number $z$ is given by $z = 2\bp{\cos\b + i\sin\b}$ where $0 < \b < \dfrac\pi2$.

        \begin{enumerate}
            \item Show that $\dfrac{z}{4 - z^2} = (k\csc\b)i$, where $k$ is positive real constant to be determined.
            \item State the argument of $\dfrac{z}{4 - z^2}$, giving your reasons clearly.
            \item Given the complex number $w = -\sqrt3 + i$, find the three smallest positive integer values of $n$ such that $\bp{\dfrac{z}{4 - z^2}}(w\cconj)^n$ is a real number.
        \end{enumerate}

    \solution
        \part
            Observe that $z = 2\bp{\cos\b + i\sin\b} = 2e^{i\b}$. Hence,
            \begin{align*}
                \dfrac{z}{4 - z^2} &= \dfrac{2e^{i\b}}{4 - \bp{2e^{i\b}}^2}\\
                &= \dfrac{2e^{i\b}}{4 - 4e^{i2\b}}\\
                &= \dfrac12 \cdot \dfrac{e^{i\b}}{1 - e^{i2\b}}\\
                &= \dfrac12 \cdot \dfrac{1}{e^{-i\b} - e^{i\b}}\\
                &= -\dfrac12 \cdot \dfrac{1}{e^{i\b} - e^{-i\b}}\\
                &= -\dfrac12 \cdot \dfrac{1/2i}{(e^{i\b} - e^{-i\b})/2i}\\
                &= -\dfrac12 \cdot \dfrac1{2i} \cdot \dfrac1{\sin \b}\\
                &= -\dfrac12 \cdot -\dfrac{i}{2} \cdot \csc\b\\
                &= \bp{\dfrac14 \csc\b} i
            \end{align*}
            \boxt{$k = \dfrac14$}

        \part
            Since $0 < \b < \dfrac\pi2$, we know that $\csc \b > 0$. Hence, $\Im{\dfrac{z}{4 - z^2}} > 0$. Furthermore, $\Re{\dfrac{z}{4 - z^2}} = 0$. Thus, $\arg{\dfrac{z}{4 - z^2}} = \dfrac\pi2$.
            \boxt{$\arg{\dfrac{z}{4 - z^2}} = \dfrac\pi2$}

        \part
            Note that $w = -\sqrt3 + i = 2\bp{-\dfrac{\sqrt3}2 + \dfrac12 i} = 2\bs{\cos \dfrac56\pi + i\sin \dfrac56 \pi} = 2e^{-i5\pi/6}$. Hence,
            \begin{align*}
                \bp{\dfrac{z}{4 - z^2}}(w\cconj)^n &= \bp{\dfrac14 \csc\b} i \cdot \bp{2e^{-i5\pi/6}}^n\\
                &= \dfrac14 \csc\b 2^n \cdot e^{i\pi/2} \cdot e^{-i5n\pi/6}\\
                &= \dfrac14 \csc\b 2^n \cdot e^{i\pi(1/2 - 5n/6)}
            \end{align*}
            Hence, $\arg{\bp{\dfrac{z}{4 - z^2}}(w\cconj)^n} = \pi\bp{\dfrac12 - \dfrac56 n}$. However, for $\bp{\dfrac{z}{4 - z^2}}(w\cconj)^n$ to be a real number, we required $\arg{\bp{\dfrac{z}{4 - z^2}}(w\cconj)^n} = \pi k$, where $k \in \Z$. Hence,
            \begin{alignat*}{2}
                && \pi\bp{\dfrac12 - \dfrac56 n} &= \pi k\\
                \implies&& \dfrac12 - \dfrac56 n &= k\\
                \implies&& 3 - 5n &= 6k\\
                \implies&& 3 - 5n &\equiv 0 \pmod{6}\\
                \implies&& 5n &\equiv 3 \pmod{6}\\
                \implies&& -1 \cdot n &\equiv 3 \pmod{6}\\
                \implies&& n &\equiv 3 \pmod{6}
            \end{alignat*}
            Hence, the three smallest possible values of $n$ are $3$, $9$ and $15$.
            \boxt{$3$, $9$, $15$}

\end{document}