\documentclass{echw}

\title{Tutorial A10A\\Complex Numbers}
\author{Eytan Chong}
\date{2024-07-04}

\begin{document}
    \problem{}
        Given that $z = 3 - 2i$ and $w = 1 + 4i$, express in the form $a + bi$, where $a, b \in \R$:
        \begin{enumerate}
            \item $z + 2w$
            \item $zw$
            \item $z/w$
            \item $(w-w\cconj)^3$
            \item $z^4$
        \end{enumerate}

    \solution
        \part
            \begin{align*}
                z + 2w &= (3-2i) + 2(1 + 4i)\\
                &= 3 - 2i + 2 + 8i\\
                &= 5 + 6i
            \end{align*}

            \boxt{$z + 2w = 5 + 6i$}

        \part
            \begin{align*}
                zw &= (3-2i)(1+4i)\\
                &= 3 + 12i - 2i + 8\\
                &= 11 + 10i
            \end{align*}
            
            \boxt{$zw = 11 + 10i$}

        \part
            \begin{align*}
                \dfrac{z}{w} &= \dfrac{3-2i}{1+4i}\\
                &= \dfrac{(3-2i)(1-4i)}{(1+4i)(1-4i)}\\
                &= \dfrac{3 - 12i - 2i - 8}{1^2 + 4^2}\\
                &= \dfrac{-5 - 14i}{17}\\
                &= -\dfrac5{17} - \dfrac{14}{17}i
            \end{align*}

            \boxt{$\dfrac{z}{w} = -\dfrac5{17} - \dfrac{14}{17}i$}

        \part
            \begin{align*}
                (w - w\cconj)^3 &= (2 \Im{w} \, i)^3\\
                &= (8i)^3\\
                &= -512i
            \end{align*}
            
            \boxt{$(w - w\cconj)^3 = -512i$}

        \part
            \begin{align*}
                z^4 &= (3-2i)^4\\
                &= 3^4 + 4 \cdot 3^3 (-2i) + 6 \cdot 3^2 (-2i)^2 + 4 \cdot 3 (-2i)^3 + (-2i)^4\\
                &= 81 - 216i - 216 + 96i + 16\\
                &= -119 - 120i
            \end{align*}
            
            \boxt{$z^4 = -119 - 120i$}

    \problem{}
        Is the following true or false in general?
        \begin{enumerate}
            \item $\Im{zw} = \Im{z} \Im{w}$
            \item $\Re{zw} = \Re{z} \Re{w}$
        \end{enumerate}

    \solution
        Let $z = a + bi$ and $w = c + di$. Then $zw = (a+bi)(c+di) = (ac-bd) + (ad+bc)i$.

        \part
            \[
                \Im{zw} = ad+bc \neq bd = \Im{z}\Im{w}
            \]

            \boxt{The statement $\Im{zw} = \Im{z} \Im{w}$ is false in general.}

        \part
            \[
                \Re{zw} = ac-bd \neq ac = \Re{z}\Re{w}
            \]

            \boxt{The statement $\Re{zw} = \Re{z} \Re{w}$ is false in general.}

    \problem{}
        \begin{enumerate}
            \item Find the complex number $z$ such that $\dfrac{z-2}{z} = 1 + i$.
            \item Given that $u = 2 + i$ and $v = -2 + 4i$, find in the form $a + bi$, where $a, b \in \R$, the complex number $z$ such that $\dfrac1z = \dfrac1u + \dfrac1v$.
        \end{enumerate}

    \solution
        \part
            \begin{alignat*}{2}
                &&\dfrac{z-2}{z} &= 1 + i\\
                \implies&&z-2 &= z + iz\\
                \implies&&iz &= -2\\
                \implies&&z &= -\dfrac{2}{i}\\
                && &= 2i
            \end{alignat*}

            \boxt{$z = 2i$}

        \part
            \begin{alignat*}{2}
                &&\dfrac1z &= \dfrac1{2 + i} + \dfrac1{-2+4i}\\
                && &= \dfrac{2-i}{2^2 + 1^2} + \dfrac{-2-4i}{2^2 + 4^2}\\
                && &= \dfrac{8-4i}{20} + \dfrac{-2-4i}{20}\\
                && &= \dfrac{6 - 8i}{20}\\
                && &= \dfrac{3 - 4i}{10}\\
                \implies&&z &= \dfrac{10}{3-4i}\\
                && &= 10 \cdot \dfrac{3+4i}{3^2 + 4^2}\\
                && &= \dfrac65 + \dfrac85 i
            \end{alignat*}

            \boxt{$z = \dfrac65 + \dfrac85 i$}

    \problem{}
        The complex numbers $z$ and $w$ are $1 + ai$ and $b - 2i$ respectively, where $a$ and $b$ are real and $a$ is negative. Given that $zw\cconj = 8i$, find the exact values of $a$ and $b$.

    \solution
        \begin{alignat*}{2}
            &&zw\cconj &= 8i\\
            \implies&& (1 + ai)(b+2i) &= 8i\\
            \implies&& b + 2i + abi - 2a &= 8i\\
            \implies&& (b-2a) + (-6 + ab)i &= 0
        \end{alignat*}

        Comparing real parts, we have $b - 2a = 0$, whence $b = 2a$. Comparing imaginary parts, we have $-6 + ab = 0$, whence $2a^2 = 6 \implies a = -\sqrt{3} \implies b = -2\sqrt3$.

        \boxt{$a = -\sqrt3$, $b = -2\sqrt3$}

    \problem{}
        Find, in the form $x + iy$, the two complex numbers $z$ satisfying both of the equations
        \[
            \dfrac{z}{z\cconj} = \dfrac35 + \dfrac45 i \quad \text{and} \quad zz\cconj = 5.
        \]

    \solution
        Multiplying both equations together, we have $z^2 = 3 + 4i$. Let $z = x + iy$, with $x, y \in \R$. We thus have $z^2 = x^2 - y^2 + 2ixy = 3 + 4i$. Comparing real and imaginary parts, we obtain the following system:
        \[
            \begin{cases}
                \begin{aligned}
                    x^2 - y^2 &= 3\\
                    2xy &= 4
                \end{aligned}
            \end{cases}
        \]
        Squaring the second equation yields $x^2y^2 = 4$. From the first equation, we have $x^2 = 3 + y^2$. Thus, $y^2(3 + y^2) = 4 \implies y^2 = 1 \implies y = \pm 1 \implies x = \pm 2$. Hence, $z = 2 + i$ or $z = -2-i$.

        \boxt{$z = 2+i \lor -2-i$}

    \problem{}
        \begin{enumerate}
            \item Given that $iw + 3z = 2 + 4i$ and $w + (1 - i)z = 2 - i$, find $z$ and $w$ in the form of $x + iy$, where $x$ and $y$ are real numbers.
            \item Determine the value of $k$ such that $z = \dfrac{1-ki}{\sqrt3 + i}$ is purely imaginary, where $k \in \R$.
        \end{enumerate}

    \solution
        \part
            Let $w = a + bi$ and $z = c + di$.
            From the first equation, we have
            \begin{alignat*}{2}
                &&iw + 3z &= 2 + 4i\\
                \implies&&i(a+bi) + 3(c+di) &= 2+4i\\
                \implies&&ai -b + 3c + 3di &= 2 + 4i\\
                \implies&&(-b + 3c) + (a + 3d)i &= 2 + 4i
            \end{alignat*}
            From the second equation, we have
            \begin{alignat*}{2}
                &&w + (1-i)z &= 2 - i\\
                \implies&&a + bi + (1-i)(c + di) &= 2-i\\
                \implies&&a + bi + c + di - ci + d &= 2 - i\\
                \implies&&(a + c + d) + (b -c + d)i &= 2 - i
            \end{alignat*}
            Comparing real and imaginary parts from the two resultant equations, we have the following system:
            \[
                \systeme{-b+3c=2,a+3d=4,a+c+d=2,b-c+d=-1}
            \]
            which has the unique solution $a = 1$, $b = -2$, $c = 0$ and $d = 1$. Hence, $w = 1 - 2i$ and $z = i$.

            \boxt{$w = 1 - 2i$, $z = i$}

        \part
            \begin{align*}
                z &= \dfrac{1 - ki}{\sqrt3 + i}\\
                &= \dfrac{(1-ki)(\sqrt3 - i)}{{\sqrt3}^2 + 1^2} \\
                &= \dfrac14(\sqrt3 - i - k\sqrt3 i - k)\\
                &= \dfrac14 \bs{(\sqrt3 - k) - (1 + k\sqrt3)i}
            \end{align*}
            Since $z$ is purely imaginary, $\Re{z} = 0$. Hence, $\sqrt3 - k = 0 \implies k = \sqrt3$.

            \boxt{$k = \sqrt3$}

    \problem{}
        \begin{enumerate}
            \item The complex number $x + iy$ is such that $(x + iy)^2 = i$. Find the possible values of the real numbers $x$ and $y$, giving your answers in exact form.
            \item Hence, find the possible values of the complex number $w$ such that $w^2 = -i$.
        \end{enumerate}

    \solution
        \part
            Note that $(x+iy)^2 = x^2 - y^2 + 2xyi = i$. Comparing real and imaginary parts, we have
            \[
                \begin{cases}
                    \begin{aligned}
                        x^2 - y^2 &= 0\\
                        2xy &= 1
                    \end{aligned}
                \end{cases}
            \]
            Note that the second equation implies that both $x$ and $y$ have the same sign. Hence, from the first equation, we have $x = y$. Thus, $x^2 = 1 \implies x = y = \pm \dfrac1{\sqrt2}$.

            \boxt{$x = y = \pm \dfrac1{\sqrt2}$}

        \part
            \begin{alignat*}{2}
                && w^2 &= -i\\
                \implies&&-w^2 &= i\\
                \implies&&(wi)^2 &= i\\
                \implies&&wi &= \pm \dfrac1{\sqrt2} (1 + i)\\
                \implies&&w &= \pm \dfrac1i \cdot \dfrac1{\sqrt2} (1 + i)\\
                && &= \mp i \cdot \dfrac1{\sqrt2} (1 + i)\\
                && &= \mp \dfrac1{\sqrt2} (i - 1)\\
                && &= \pm \dfrac1{\sqrt2} (1 - i)
            \end{alignat*}

            \boxt{$w = \dfrac1{\sqrt2} (1 - i) \lor -\dfrac1{\sqrt2} (1 - i)$}

    \problem{}
        \begin{enumerate}
            \item The roots of the equation $z^2 = -8i$ are $z_1$ and $z_2$. Find $z_1$ and $z_2$ in Cartesian form $x + iy$, showing your working.
            \item Hence, or otherwise, find in Cartesian form the roots $w_1$ and $w_2$ of the equation $w^2 + 4w + (4 + 2i) = 0$.
        \end{enumerate}

    \solution
        \part
            Let $z = x + iy$ where $x, y \in \R$. Then $(x+iy)^2 = x^2 - y^2 + 2xyi = -8i$. Comparing real and imaginary parts, we have the following system:
            \[
                \begin{cases}
                    \begin{aligned}
                        x^2 - y^2 &= 0\\
                        2xy &= 8
                    \end{aligned}
                \end{cases}
            \]
            From the second equation, we know that $x$ and $y$ have opposite signs. Hence, from the first equation, we have that $x = -y$. Thus, $x^2 = 4 \implies x = \pm 2 \implies y = \mp 2$. Thus, $z = \pm2 (1-i)$, whence $z_1 = 2 - 2i$ and $z_2 = -2 + 2i$.

            \boxt{$z_1 = 2 - 2i$, $z_2 = -2 + 2i$}

        \part
            \begin{alignat*}{2}
                &&w^2 + 4w + (4 + 2i) &= 0\\
                \implies&&(w + 2)^2 &= -2i\\
                \implies&&(2w + 4)^2 &= -8i\\
                \implies&&2w + 4 &= \pm2 (1 - i)\\
                \implies&&w + 2 &= \pm (1 - i)\\
                \implies&&w &= 2 \pm (1 - i)
            \end{alignat*}

            \boxt{$w_1 = 3 - i$, $w_2 = -1-i$}

    \problem{}
        One of the roots of the equations $2x^3 - 9x^2 + 2x + 30 = 0$ is $3 + i$. Find the other roots of the equation.

    \solution
        Let $P(x) = 2x^3 - 9x^2 + 2x + 30$. Since $P(x)$ is a polynomial with real coefficients, and $P(3+i) = 0$, we have that $(3+i)\cconj = 3 - i$ is a root of $P(x)$. Let $a$ be the real root of $P(x)$. We hence have
        \[
            P(x) = 2x^3 - 9x^2 + 2x + 30 = 2(x - a)\bs{x - (3+i)}\bs{x - (3-i)}
        \]
        Comparing constants,
        \begin{alignat*}{2}
            &&2\cdot -a \cdot -(3+i) \cdot -(3-i) &= 30\\
            \implies&&2 \cdot -a \cdot (3^2 + 1^2) &= 30\\
            \implies&&a &= -\dfrac{30}{2 \cdot 10}\\
            && &= -\dfrac32
        \end{alignat*}
        
        \boxt{The other roots are $3 -i$ and $-\dfrac32$.}

    \problem{}
        Obtain a cubic equation having 2 and $\dfrac54 - \dfrac{\sqrt7}4 i$ as two of its roots, in the form $az^3 + bz^2 + cz + d = 0$, where $a$, $b$, $c$ and $d$ are real integral coefficients to be determined.

    \solution
        Let $P(z) = az^3 + bz^2 + cz + d$. Since $P(z)$ is a polynomial with real coefficients, and $P\of{\dfrac54 - \dfrac{\sqrt7}4 i} = 0$, we have that $\bp{\dfrac54 - \dfrac{\sqrt7}4 i}\cconj = \dfrac54 + \dfrac{\sqrt7}4 i$ is also a root of $P(z)$. Hence,
        \begin{align*}
            P(z) &= k(z-2)\bs{z - \bp{\dfrac54 - \dfrac{\sqrt7}4 i}}\bs{z - \bp{\dfrac54 + \dfrac{\sqrt7}4 i}}\\
            &= k(z-2)\bs{\bp{z - \dfrac54} + \dfrac{\sqrt7}4 i}\bs{\bp{z - \dfrac54} - \dfrac{\sqrt7}4 i}\\
            &= k(z-2)\bs{\bp{z - \dfrac54}^2 + \bp{\dfrac{\sqrt7}4}^2}\\
            &= k(z-2)\bp{z^2 - \dfrac52z + \dfrac{25}{16} + \dfrac7{16}}\\
            &= k(z-2)\bp{z^2 - \dfrac52z + 2}\\
            &= 2k(z-2)(2z^2-5z+4)\\
            &= 2k(2z^3 - 5z^2 + 4z - 4z^2 + 10z - 8)\\
            &= 2k(2z^3 - 9z^2 + 14z - 8)
        \end{align*}
        Taking $k = \dfrac12$, we have $P(z) = 2z^3 - 9z^2 + 14z - 8$, whence $a = 2$, $b = -9$, $c = 14$ and $d = -8$.

        \boxt{$P(z) = 2z^3 - 9z^2 + 14z - 8$}

    \problem{}
        \begin{enumerate}
            \item Verify that $-1 + 5i$ is a root of the equation $w^2 + (-1-8i)w + (-17+7i) = 0$. Hence, or otherwise, find the second root of the equation in Cartesian form, $p + iq$, showing your working.
            \item The equation $z^3 - 5z^2 + 16z + k = 0$, where $k$ is a real constant, has a root $z = 1 + ai$, where $a$ is a positive real constant. Find the values of $a$ and $k$, showing your working.
        \end{enumerate}

    \solution
        \part
            Let $P(w) = w^2 + (-1-8i)w + (-17+7i)$. Consider $P(-1 + 5i)$.
            \begin{align*}
                P(-1+5i) &= (-1+5i)^2 + (-1-8i)(-1+5i) + (-17+7i)\\
                &= (1 - 10i - 25) + (1 - 5i + 8i + 40) + (-17 + 7i)\\
                &= (1 - 25 + 1 + 40 - 17) + (-10 - 5 + 8 + 7)i\\
                &= 0
            \end{align*}
            Hence, $-1+5i$ is a root of $w^2 + (-1-8i)w + (-17+7i) = 0$.

            We have that $p + iq$ is also a root of the equation.
            \begin{align*}
                P(w) &= \bs{w - (-1+5i)}\bs{w - (p + iq)}\\
                &= (w + 1 - 5i)(w - p - iq)\\
                &= w^2 - pw - qi w + w - p - iq - 5iw + 5ip -5q\\
                &= w^2 + (-p - qi + 1 - 5i)w + (-p - iq + 5ip - 5q)\\
                &= w^2 + \bs{(1-p) - (5+q)i}w + \bs{-(p + 5q) + (5p - q)i}
            \end{align*}
            Comparing the imaginary and real parts of the coefficients of $w$, we have $1-p = -1$ and $q+5 = 8$, whence $p = 2$ and $q = 3$.
            
            \boxt{The second root of the equation is $2 + 3i$.}

        \part
            Let $P(z) = z^3 - 5z^2 + 16z + k$. Then $P(1 + ai) = 0$.
            \begin{alignat*}{2}
                &&P(1 + ai) &= 0\\
                \implies&&(1 + ai)^3 - 5(1 + ai)^2 + 16(1 + ai) + k &= 0\\
                \implies&&\bs{1 + 3ai + 3(ai)^2 + (ai)^3} - 5(1 + 2ai - a^2) + (16 + 16ai) + k &= 0\\
                \implies&&1 + 3ai - 3a^2 - a^3i - 5 - 10ai + 5a^2 + 16 + 16ai + k &= 0\\
                \implies&&(12 + k + 2a^2) + (9-a^2)ai &= 0
            \end{alignat*}
            Comparing real and imaginary parts, we have $9-a^2 = 0 \implies a = 3$ and $12 + k + 2a^2 = 0 \implies k = -30$.
            \boxt{$a = 3$, $k = -30$}
        
        
\end{document}