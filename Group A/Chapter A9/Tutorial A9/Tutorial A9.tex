\documentclass{echw}

\title{Tutorial A9\\Vectors III}
\author{Eytan Chong}
\date{2024-05-20}

\begin{document}
    \problem{}
        A student claims that a unique plane can always be defined based on the given information. True or False? (Whenever a line is mentioned, assume the vector equation is known.)

        \begin{table}[h]
            \centering
            \begin{tabularx}{\textwidth}{r X l}
            & \textbf{Statement} & \textbf{T/F}\\\hline
            (a) & Any 2 vectors parallel to the plane and a point lying on the plane. & False \\
            (b) & Any 3 distinct points lying on the plane. & False \\
            (c) & A vector perpendicular to the plane and a point lying on the plane. & True\\
            (d) & A line $l$ perpendicular to the plane and a particular point on $l$ lying on the plane. & True\\
            (e) & A line $l$ lying on the plane. & False\\
            (f) & A line $l$ and a point not on $l$, both lying on the plane. & True\\
            (g) & A pair of distinct, intersecting lines, both lying on the plane. & True\\
            (h) & A pair of distinct, parallel lines, both lying on the plane. & True\\
            (i) & A pair of skew lines both parallel to the plane. & False\\
            (j) & 2 intersecting lines both parallel to the plane. & False
            \end{tabularx}
        \end{table}

    \problem{}
        Find the equations of the following planes in parametric, scalar product and Cartesian form:

        \begin{enumerate}
            \item The plane passes through the point with position vector $7\vec i + 2 \vec j -3\vec j$ and is parallel to $\vec i + 3\vec j$ and $4\vec j - 2\vec k$.
            \item The plane passes through the points $A(2, 0, 1)$, $B(1, -1, 2)$ and $C(1, 3, 1)$.
            \item The plane passes through the point with position vector $7\vec i$ and is parallel to the plane $\vec r = (2 - p + q)\vec i + (p + 3q)\vec j + (-2-3q)\vec k$, $p, q \in \R$.
            \item The plane contains the line $l : \vec r = (-2\vec i + 5\vec j -3\vec k) + \l(2\vec i + \vec j + 2\vec k)$, $\l \in \R$ and is perpendicular to the plane $\pi : \vec r \dotp (7\vec i + 4\vec j + 5\vec k) = 2$.
        \end{enumerate}

    \solution
        \part
            \textbf{Parametric.}
            Note that $\cveciii04{-2} = 2\cveciii02{-1}$. Hence, the plane has parametric form
            \[
                \vec r = \cveciii72{-3} + \l\cveciii130 + \m\cveciii02{-1}, \, \l, \m \in \R
            \]
             \textbf{Scalar Product.}
            Note that $\vec n = \cveciii130 \crossp \cveciii02{-1} = \cveciii{-3}12 \implies d = \cveciii72{-3} \dotp \vec n = \cveciii72{-3} \dotp \cveciii{-3}12 = -25$. Thus, the plane has scalar product form
            \[
                \vec r \dotp \cveciii{-3}12 = -2
            \]
             \textbf{Cartesian.} Let $\vec r = \cveciii{x}{y}{z}$. From the scalar product form, we have
            \[
                -3x + y + 2z = -25
            \]
            \[
                \begin{array}{r @{\hspace*{1.0cm}} l}\toprule
                    \textbf{Form} & \textbf{Equation} \\\cmidrule{1-2}
                    \textbf{Parametric} & \vec r = \cveciii72{-3} + \l\cveciii130 + \m\cveciii02{-1}, \, \l, \m \in \R \\
                    \textbf{Scalar Product} & \vec r \dotp \cveciii{-3}12 = -25 \\
                    \textbf{Cartesian} & -3x + y + 2z = -25 \\\bottomrule
                \end{array}
            \]

        \part
            \textbf{Parametric.} Since the plane passes through the points $A$, $B$ and $C$, it is parallel to both $\oa{AB} = -\cveciii11{-1}$ and $\oa{AC}= \cveciii{-1}30$. Hence, the plane has parametric form
            \[
                \vec r = \cveciii201 + \l\cveciii11{-1} + \m\cveciii{-1}30, \, \l, \m \in \R
            \]
             \textbf{Scalar Product.} Note that $\vec n = \cveciii11{-1} \crossp \cveciii{-1}30 = \cveciii314 \implies d = \cveciii201 \dotp \vec n = \cveciii201 \dotp \cveciii314 = 10$. Thus, the plane has scalar product form
            \[
                \vec r \dotp \cveciii314 = 10
            \]
             \textbf{Cartesian.} Let $\vec r = \cveciii{x}{y}{z}$. From the scalar product form, we have
            \[
                3x + y + 4z = 10
            \]
            \[
                \begin{array}{r @{\hspace*{1.0cm}} l}\toprule
                    \textbf{Form} & \textbf{Equation} \\\cmidrule{1-2}
                    \textbf{Parametric} & \vec r = \cveciii201 + \l\cveciii11{-1} + \m\cveciii{-1}30, \, \l, \m \in \R \\
                    \textbf{Scalar Product} & \vec r \dotp \cveciii314 = 10 \\
                    \textbf{Cartesian} & 3x + y + 4z = 10 \\\bottomrule
                \end{array}
            \]

        \part
            \textbf{Parametric.} Note that the plane is parallel to $\vec r = \cveciii{2 - p + q}{p + 3q}{-2 - 3q} = \cveciii20{-1} + p\cveciii{-1}10 + q\cveciii13{-3}$ and passes through $\cveciii700$. Hence, the plane has parametric form
            \[
                \vec r = \cveciii700 + \l\cveciii{-1}10 + \m\cveciii13{-3}, \, \l, \m \in \R
            \]
             \textbf{Scalar Product.} Note that $\vec n = \cveciii{-1}10 \crossp \cveciii13{-3} = \cveciii{-3}{-3}{-4} = -\cveciii334 \implies d = \cveciii700 \dotp \vec n = \cveciii700 \dotp -\cveciii334 = -21$. Thus, the plane has scalar product form $\vec r \dotp -\cveciii334 = -21$, which simplifies to
            \[
                \vec r \dotp \cveciii334 = 21
            \]
             \textbf{Cartesian.} Let $\vec r = \cveciii{x}{y}{z}$. From the scalar product form, we have
            \[
                3x + 3y + 4z = 21
            \]
            \[
                \begin{array}{r @{\hspace*{1.0cm}} l}\toprule
                    \textbf{Form} & \textbf{Equation} \\\cmidrule{1-2}
                    \textbf{Parametric} & \vec r = \cveciii700 + \l\cveciii{-1}10 + \m\cveciii13{-3}, \, \l, \m \in \R \\
                    \textbf{Scalar Product} & \vec r \dotp \cveciii334 = 21 \\
                    \textbf{Cartesian} & 3x + 3y + 4z = 21 \\\bottomrule
                \end{array}
            \]

        \part
            \textbf{Parametric.} Since the plane contains the line with equation $\vec r = \cveciii{-2}5{-3} + \l\cveciii212, \, \l \in \R$, the plane passes through the point with position vector $\cveciii{-2}5{-3}$ and is parallel to the vector $\cveciii212$. Furthermore, since the plane is perpendicular to the plane with normal $\cveciii745$, it must be parallel to said vector. Thus, the plane has the following parametric form:
            \[
                \vec r = \cveciii{-2}5{-3} + \l\cveciii212 + \m\cveciii745, \, \l, \m \in \R
            \]
             \textbf{Scalar Product.} Note that $\vec n = \cveciii212 \crossp \cveciii745 = \cveciii{-3}41 \implies d = \cveciii{-2}5{-3} \dotp \vec n = \cveciii{-2}5{-3} \dotp \cveciii{-3}41 = 23$. Thus, the plane has scalar product form
            \[
                \vec r \dotp \cveciii{-3}41 = 23
            \]
             \textbf{Cartesian.} Let $\vec r = \cveciii{x}{y}{z}$. From the scalar product form, we have
            \[
                -3x + 4y + z = 23
            \]
            \[
                \begin{array}{r @{\hspace*{1.0cm}} l}\toprule
                    \textbf{Form} & \textbf{Equation} \\\cmidrule{1-2}
                    \textbf{Parametric} & \vec r = \cveciii{-2}5{-3} + \l\cveciii212 + \m\cveciii745, \, \l, \m \in \R \\
                    \textbf{Scalar Product} & \vec r \dotp \cveciii{-3}41 = 23 \\
                    \textbf{Cartesian} & -3x + 4y + z = 23 \\\bottomrule
                \end{array}
            \]


    \problem{}
        The line $l$ passes through the points $A$ and $B$ with coordinates $(1, 2, 4)$ and $(-2, 3, 1)$ respectively. The plane $p$ has equation $3x - y + 2z = 17$. Find

        \begin{enumerate}
            \item the coordinates of the point of intersection of $l$ and $p$,
            \item the acute angle between $l$ and $p$,
            \item the perpendicular distance from $A$ to $p$, and
            \item the position vector of the foot of the perpendicular from $B$ to $p$.
        \end{enumerate}

         The line $m$ passes through the point $C$ with position vector $6\vec i + \vec j$ and is parallel to $2\vec j + \vec k$.

        \begin{enumerate}
            \setcounter{enumi}{4}
            \item Determine whether $m$ lies in $p$.
        \end{enumerate}

    \solution
        Note that $\oa{OA} = \cveciii124$ and $\oa{OB} = \cveciii{-2}31$, whence $\oa{AB} = -\cveciii3{-1}3$. Thus, the line $l$ has vector equation
        \[
            \vec r = \cveciii124 + \l\cveciii3{-1}3, \, \l \in \R
        \]

        Also note that the equation of the plane $p$ can be written as
        \[
            \vec r \dotp \cveciii3{-1}2 = 17
        \]

        \part
            Let the point of intersection of $l$ and $p$ be $P$. Consider $l = p$.
            \begin{alignat*}{2}
                &&l &= p\\
                \implies&&\bs{\cveciii124 + \l\cveciii3{-1}3} \dotp \cveciii3{-1}2 &= 17\\
                \implies&&\cveciii124 \dotp \cveciii3{-1}2 + \l\cveciii3{-1}3 \dotp \cveciii3{-1}2 &= 17\\
                \implies&&9 + 16\l &= 17\\
                \implies&&\l &= \dfrac12
            \end{alignat*}

            Thus, $\oa{OP} = \cveciii124 + \dfrac12\cveciii3{-1}3 = \cveciii{5/2}{3/2}{11/2}$, whence $P\bp{\dfrac52, \dfrac32, \dfrac{11}2}$.

            \boxt{$\bp{\dfrac52, \dfrac32, \dfrac{11}2}$}
            
        \part
            Let $\t$ be the acute angle between $l$ and $p$.
            \begin{alignat*}{2}
                &&\sin\t &= \dfrac{\abs{\cveciii3{-1}3 \dotp \cveciii3{-1}2}}{\abs{\cveciii3{-1}3} \abs{\cveciii3{-1}2}}\\
                && &= \dfrac{16}{\sqrt{266}}\\
                \implies&&\t &= 78.8 \deg \todp{1}
            \end{alignat*}

            \boxt{$\t = 78.8 \deg$}

        \part
            Note that $\oa{AP} = -\dfrac12 \cveciii3{-1}3$.
            \begin{align*}
                \text{Perpendicular distance} &= \abs{\oa{AP} \dotp \hat{\vec n}}\\
                &= \abs{-\dfrac12\cveciii3{-1}3 \dotp \cveciii3{-1}2} \Bigg/ \abs{\cveciii3{-1}2}\\
                &= \dfrac8{\sqrt{14}}
            \end{align*}

            \boxt{The perpendicular distance from $A$ to $p$ is $\dfrac8{\sqrt{14}}$ units.}
            
        \part
            Let $F$ be the foot of the perpendicular from $B$ to $p$. Since $F$ is on $p$, we have $\oa{OF} \dotp \cveciii3{-1}2 = 17$. Furthermore, since $BF$ is perpendicular to $p$, we have $\oa{BF} = \l \vec n = \l \cveciii3{-1}2$ for some $\l \in \R$. We hence have $\oa{OF} = \oa{OB} + \oa{BF} = \cveciii{-2}31 + \l \cveciii3{-1}2$.

            \begin{alignat*}{2}
                &&\oa{OF} \dotp \cveciii3{-1}2 &= 17\\
                \implies&&\bs{\cveciii{-2}31 + \l \cveciii3{-1}2} \dotp \cveciii3{-1}2 &= 17\\
                \implies&&\cveciii{-2}31 \dotp \cveciii3{-1}2 + \l\cveciii3{-1}2 \dotp \cveciii3{-1}2 &= 17\\
                \implies&&-7 + 14\l &= 17\\
                \implies&&\l &= \dfrac{12}7
            \end{alignat*}

            Thus, $\oa{OF} = \cveciii{-2}31 + \dfrac{12}7 \cveciii3{-1}2 = \dfrac17\cveciii{22}9{31}$.

            \boxt{$\oa{OF} = \dfrac17\cveciii{22}9{31}$}
        
        \part
            Note that $m$ has the vector equation
            \[
                \vec r = \cveciii610 + \l \cveciii021, \, \l \in \R
            \]

            Consider $m \dotp \vec n$.
            \begin{align*}
                m \dotp \vec n &= \bs{\cveciii610 + \l \cveciii021} \dotp \cveciii3{-1}2\\
                &= \cveciii610 \dotp \cveciii3{-1}2 + \l\cveciii021 \dotp \cveciii3{-1}2\\
                &= 17 + 0\l\\
                &= 17
            \end{align*}

            Since $m \dotp \vec n = 17$ for all $\l \in \R$, $m$ lies in $p$.

            \boxt{$m$ lies in $p$.}

    \problem{}
        A plane contains distinct points $P$, $Q$, $R$ and $S$, of which no 3 points are collinear. What can be said about the relationship between the vectors $\oa{PQ}$, $\oa{PR}$ and $\oa{PS}$?

    \solution
        Each of the three vectors can be expressed as a unique linear combination of the other two.

    \problem{}
        \begin{enumerate}
            \item Interpret geometrically the vector equation $\vec r = \vec a + t\vec b$ where $\vec a$ and $\vec b$ are constant vectors and $t$ is a parameter.
            \item Interpret geometrically the vector equation $\vec r \dotp \vec n = d$, where $\vec n$ is a constant unit vector and $d$ is a constant scalar, stating what $d$ represents.
            \item Given that $\vec b \dotp \vec n \neq 0$, solve the equations $\vec r = \vec a + t\vec b$ and $\vec r \dotp \vec n = d$ to find $\vec r$ in terms of $\vec a$, $\vec b$, $\vec n$ and $d$. Interpret the solution geometrically.
        \end{enumerate}

    \solution
        \part
            The vector equation $\vec r = \vec a + t\vec b$ represents a line with direction vector $\vec b$ that passes through the point with position vector $\vec a$.

        \part
            The vector equation $\vec r \dotp \vec n = d$ represents a plane perpendicular to $\vec n$ that has a perpendicular distance of $d$ units from the origin. Here, a negative value of $d$ corresponds to a plane $d$ units from the origin in the opposite direction of $\vec n$.

        \part
            \begin{alignat*}{2}
                &&\vec r \dotp \vec n &= d\\
                \implies&&\bp{\vec a + t\vec b} \dotp \vec n &= d\\
                \implies&&\vec a \dotp \vec n + t\vec b \dotp \vec n &= d\\
                \implies&&t &= \dfrac{d - \vec a \dotp \vec n}{\vec b \dotp \vec n}\\
                \implies&&\vec r &= \vec a + \dfrac{d - \vec a \dotp \vec n}{\vec b \dotp \vec n} \, \vec b
            \end{alignat*}
            $\vec a + \dfrac{d - \vec a \dotp \vec n}{\vec b \dotp \vec n} \, \vec b$ is the position vector of the point of intersection of the line and plane.

    \problem{}
        The planes $p_1$ and $p_2$ have equations $\vec r \dotp \cveciii2{-2}1 = 1$ and $\vec r \dotp \cveciii{-6}32 = -1$ respectively, and meet in the line $l$.

        \begin{enumerate}
            \item Find the acute angle between $p_1$ and $p_2$.
            \item Find a vector equation for $l$.
            \item The point $A(4, 3, c)$ is equidistant from the planes $p_1$ and $p_2$. Calculate the two possible values of $c$.
        \end{enumerate}

    \solution
        \part
            Let $\t$ the acute angle between $p_1$ and $p_2$.
            \begin{alignat*}{2}
                &&\cos\t &= \dfrac{\abs{\cveciii2{-2}1 \dotp \cveciii{-6}32}}{\abs{\cveciii2{-2}1} \abs{\cveciii{-6}32}}\\
                && &= \dfrac{16}{21}\\
                \implies&&\t &= 40.4 \deg \todp{1}
            \end{alignat*}

            \boxt{$\t = 40.4$}

        \part
            Observe that $p_1$ has the Cartesian equation $2x-2y+z = 1$ and $p_2$ has the Cartesian equation $-6x + 3y + 2z = -1$. Consider $p_1 = p_2$. Solving both Cartesian equations simultaneously gives the solution
            \[
                \systeme*{
                    x = -\frac16 + \frac76t,
                    y = -\frac23 + \frac53 t,
                    z = t
                }
            \]
             for all $t \in \R$. The line $l$ thus has vector equation $\vec r = \cveciii{x}{y}{z} = -\dfrac16\cveciii140 + t\cveciii7{10}6, \, t \in \R$.

            \boxt{$\vec r = -\dfrac16\cveciii140 + t\cveciii7{10}6, \, t \in \R$}

        \part
            Let $Q$ be the point with position vector $-\dfrac16 \cveciii140$, whence $\oa{AQ} = -\dfrac16 \cveciii{25}{22}{6c}$. Since $Q$ lies on $l$, it lies on both $p_1$ and $p_2$. Since $A$ is equidistant to $p_1$ and $p_2$, the perpendicular distances from $A$ to $p_1$ and $p_2$ are equal.
            \begin{alignat*}{2}
                &&\abs{\oa{AQ} \dotp \cveciii2{-2}1} \Bigg/ \abs{\cveciii2{-2}1} &= \abs{\oa{AQ} \dotp \cveciii{-6}32} \Bigg/ \abs{\cveciii{-6}32}\\
                \implies&&\dfrac13 \abs{-\dfrac16 \cveciii{25}{22}{6c} \dotp \cveciii2{-2}1} &= \dfrac17 \abs{-\dfrac16 \cveciii{25}{22}{6c} \dotp \cveciii{-6}32}\\
                \implies&&\dfrac13 \, \abs{1 + c} &= \dfrac17 \, \abs{-14 + 2c}\\
                \implies&&\abs{7 + 7c} &= \abs{-42 + 6c}\\
            \end{alignat*}

            \case{1}{} $(7+7c)(-42+6c) > 0 \implies 7+7c = -42+6c \implies c = -49$

            \case{2}{} $(7+7c)(-42+6c) < 0 \implies 7+7c = -(-42+6c) \implies c = -\dfrac{35}{13}$

            \boxt{$c = -49 \lor c = -\dfrac{35}{13}$}

    \problem{}
        A plane $\Pi$ has equation $\vec r \dotp (2\vec i + 3\vec j) = -6$.
        
        \begin{enumerate}
            \item Find, in vector form, an equation for the line passing through the point $P$ with position vector $2\vec i + \vec j + 4\vec k$ and normal to the plane $\Pi$.
            \item Find the position vector of the foot $Q$ of the perpendicular from $P$ to the plane $\Pi$ and hence find the position vector of the image of $P$ after the reflection in the plane $\Pi$.
            \item Find the sine of the acute angle between $OQ$ and the plane $\Pi$.
        \end{enumerate}

         The plane $\Pi'$ has equation $\vec r \dotp (\vec i + \vec j + \vec k) = 5$.
        
        \begin{enumerate}
            \setcounter{enumi}{3}
            \item Find the position vector of the point $A$ where the planes $\Pi$, $\Pi'$ and the plane with equation $\vec r \dotp \vec i = 0$ meet.
            \item Hence, or otherwise, find also the vector equation of the line of intersection of planes $\Pi$ and $\Pi'$.
        \end{enumerate}

    \solution
        \part
            Let $l$ be the required line. Since $l$ is normal to $\Pi$, it is parallel to the normal vector of $\Pi$, $\cveciii230$. Thus, $l$ has vector equation
            
            \boxt{$l: \vec r = \cveciii214 + \l\cveciii230, \, \l \in \R$}

        \part
            Since $Q$ is on $\Pi$, $\oa{OQ} \dotp \cveciii230 = -6$. Furthermore, observe that $Q$ is also on the line $l$. Thus, $\oa{OQ} = \cveciii214 + \l\cveciii230$ for some $\l \in \R$.
            {\allowdisplaybreaks
            \begin{alignat*}{2}
                &&\oa{OQ} \dotp \cveciii230 &= -6\\
                \implies&&\bs{\cveciii214 + \l\cveciii230} \dotp \cveciii230 &= -6\\
                \implies&&\cveciii214 \dotp \cveciii230 + \l\cveciii230 \dotp \cveciii230 &= -6\\
                \implies&&7 + 13\l &= -6\\
                \implies&&\l &= -1
            \end{alignat*}}

            Thus, $\oa{OQ} = \cveciii214 - \cveciii230 = \cveciii0{-2}4$.

            \boxt{$\oa{OQ} = \cveciii0{-2}4$}

            Let the reflection of $P$ in $\Pi$ be $P'$. We have that $\oa{PQ} = \oa{QP'}$.
            \begin{alignat*}{2}
                &&\oa{PQ} &= \oa{QP'}\\
                \implies&&\oa{OQ} - \oa{OP} &= \oa{OP'}- \oa{OQ}\\
                \implies&&\oa{OP'} &= 2\oa{OQ} - \oa{OP}\\
                && &=2\cveciii0{-2}4 - \cveciii214\\
                && &=\cveciii{-2}{-5}4
            \end{alignat*}

            \boxt{$\oa{OP'} =\cveciii{-2}{-5}4$}

        \part
            Let $\t$ be the acute angle between $OQ$ and $\Pi$.
            \begin{align*}
                \sin\t &= \dfrac{\abs{\cveciii0{-2}4 \dotp \cveciii230}}{\abs{\cveciii0{-2}4} \abs{\cveciii230}}\\
                &= \dfrac{3}{\sqrt{65}}
            \end{align*}

            \boxt{$\sin\t = \dfrac3{\sqrt{65}}$}
            
        \part
            Let $\oa{OA} = \cveciii{x}{y}{z}$. We thus have the following system:
            \[
                \begin{cases}
                    \cveciii{x}{y}{z} \dotp \cveciii230 = -6 \implies 2x+3y=-6\\
                    \cveciii{x}{y}{z} \dotp \cveciii111 = 5 \implies x+y+z = 5\\
                    \cveciii{x}{y}{z} \dotp \cveciii100 = 0 \implies x = 0
                \end{cases}
            \]
            Solving, we obtain $x = 0$, $y = -2$ and $z = 7$.

            \boxt{$\oa{OA} = \cveciii0{-2}7$}

        \part
            Let the line of intersection of $\Pi$ and $\Pi'$ be $l'$. Observe that $A$ is on $\Pi$ and $\Pi'$ and thus lies on $l'$. Hence,
            \[
                l': \vec r = \cveciii0{-2}7 + \l\vec b, \, \l \in \R
            \]
            Since $l'$ lies on both $\Pi$ and $\Pi'$, $\vec b$ is perpendicular to the normals of both planes, i.e. $\cveciii230$ and $\cveciii111$. Thus, $\vec b = \cveciii230 \crossp \cveciii111 = \cveciii3{-2}{-1}$.

            \boxt{$l': \vec r = \cveciii0{-2}7 + \l\cveciii3{-2}{-1}, \, \l \in \R$}

    \problem{}
        \begin{center}
            \begin{tikzpicture}
                \coordinate[label=left:$A$] (A) at (0, 0);
                \coordinate[label=right:$B$] (B) at (6, 0);
                \coordinate[label=right:$C$] (C) at (7.5, 1.5);
                \coordinate[label=above right:$D$] (D) at (1.5, 1.5);
                \coordinate[label=left:$E$] (E) at (0, 4);
                \coordinate[label=right:$F$] (F) at (6, 4);
                \coordinate[label=below right:$G$] (G) at (7.5, 5.5);
                \coordinate[label=below right:$H$] (H) at (1.5, 5.5);
                \coordinate[label=above:$L$] (L) at (0.75, 6);
                \coordinate[label=above:$M$] (M) at (6.75, 6);

                \draw (A) -- (B);
                \draw (B) -- (C);
                \draw (B) -- (F);
                \draw (F) -- (G);
                \draw (C) -- (G);
                \draw (A) -- (E);
                \draw (E) -- (F);
                \draw[dotted] (A) -- (D);
                \draw[dotted] (D) -- (C);
                \draw[dotted] (D) -- (H);
                \draw[dotted] (E) -- (H);
                \draw[dotted] (H) -- (G);
                \draw (E) -- (L);
                \draw[dotted] (L) -- (H);
                \draw (F) -- (M);
                \draw (M) -- (G);
                \draw (L) -- (M);

                \draw pic [draw, angle radius=12mm, "$\t$"] {angle = H--E--L};

                \draw[very thick, ->] (A) -- (1, 0) node[anchor=north] {$\vec i$};
                \draw[very thick, ->] (A) -- (0.707, 0.707) node[anchor=west] {$\vec j$};
                \draw[very thick, ->] (A) -- (0, 1) node[anchor=east] {$\vec k$};

                \node[anchor=north] at ($(A)!0.5!(B)$) {3 m};
                \node[anchor=west] at ($(B)!0.5!(C)$) {2 m};
                \node[anchor=west] at ($(C)!0.5!(G)$) {2 m};
            \end{tikzpicture}
        \end{center}

         The diagram shows a garden shed with horizontal base $ABCD$, where $AB = 3$ m and $BC = 2$ m. There are two vertical rectangular walls $ABFE$ and $DCGH$, where $AE = BF = CG = DH = 2$ m. The roof consists of two rectangular planes $EFML$ and $HGML$, which are inclined at an angle $\t$ to the horizontal such that $\tan \t = \frac34$.

        The point $A$ is taken as the origin and the vectors $\vec i$, $\vec j$ and $\vec k$, each of length 1 m, are taken along $AB$, $AD$ and $AE$ respectively.

        \begin{enumerate}
            \item Verify that the plane with equation $\vec r \dotp (22\vec i + 33\vec j - 12\vec k) = 66$ passes through $B$, $D$ and $M$.
            \item Find the perpendicular distance, in metres, from $A$ to the plane $BDM$.
            \item Find a vector equation of the straight line $EM$.
            \item Show that the perpendicular distance from $C$ to the straight line $EM$ is 2.91 m, correct to 3 significant figures.
        \end{enumerate}

    \solution
        \part
            We have $\oa{AB} = \cveciii300$, $\oa{BF} = \oa{AE} = \cveciii002$ and $\oa{FG} = \oa{AD} = \cveciii020$. Let $T$ be the midpoint of $FG$. We have $\oa{FT} = \cveciii010$ and $\dfrac{TM}{FT} = \tan\t = \dfrac34$, whence $\oa{TM} = \dfrac34 \cveciii001 = \cveciii00{3/4}$.
            \begin{align*}
                \oa{AM} &= \oa{AB} + \oa{BF} + \oa{FT} + \oa{TM}\\
                &= \cveciii300 + \cveciii002 + \cveciii010 + \cveciii00{3/4}\\
                &= \dfrac14\cveciii{12}4{11}
            \end{align*}
            Consider $\oa{AB} \dotp \cveciii{22}{33}{-12}$, $\oa{AD} \dotp \cveciii{22}{33}{-12}$ and $\oa{AM} \dotp \cveciii{22}{33}{-12}$.
            \begin{alignat*}{2}
                \oa{AB} \dotp \cveciii{22}{33}{-12} &= \cveciii300 \dotp \cveciii{22}{33}{-12} &= 66\\
                \oa{AD} \dotp \cveciii{22}{33}{-12} &= \cveciii020 \dotp \cveciii{22}{33}{-12} &= 66\\
                \oa{AM} \dotp \cveciii{22}{33}{-12} &= \dfrac14\cveciii{12}4{11} \dotp \cveciii{22}{33}{-12} &= 66
            \end{alignat*}            
            Since $\oa{AB}$, $\oa{AD}$ and $\oa{AM}$ satisfy the equation $\vec r \dotp \cveciii{22}{33}{-12} = 66$, they all lie on the plane with said equation.

        \part
            \begin{align*}
                \text{Perpendicular distance} &= \abs{\oa{AB} \dotp \hat{\vec n}}\\
                &= \abs{\cveciii300 \dotp \cveciii{22}{33}{-12}} \Bigg/ \abs{\cveciii{22}{33}{-12}}\\
                &= \dfrac{66}{\sqrt{1717}}
            \end{align*}

            \boxt{The perpendicular distance from $A$ to the plane $BDM$ is $\dfrac{66}{\sqrt{1717}}$ units.}

        \part
            Observe that $\oa{EM} = \oa{AM} - \oa{AE} = \dfrac14 \cveciii{12}43$. Hence, the line $EM$ has vector equation

            \boxt{$\vec r = \cveciii002 + \l \cveciii{12}43, \, \l \in \R$}

        \part
            Note that $\oa{EC} = \oa{AC} - \oa{AE} = \cveciii32{-2}$.
            \begin{align*}
                \text{Perpendicular distance} &= \abs{\oa{EC} \crossp \cveciii{12}43} \Bigg/ \abs{\cveciii{12}43}\\
                &= \dfrac1{13} \abs{\cveciii32{-2} \crossp \cveciii{12}43}\\
                &= \dfrac1{13} \abs{\cveciii{14}{-33}{-12}}\\
                &= \dfrac{\sqrt{1429}}{13}\\
                &= 2.91 \tosf{3}
            \end{align*}

    \problem{}
        The planes $\pi_1$ and $\pi_2$ have equations
        \[
            x + y - z = 0 \text{ and } 2x-4y+z+12=0
        \]
         respectively. The point $P$ has coordinates $(3, 8, 2)$ and $O$ is the origin.

        \begin{enumerate}
            \item Verify that the vector $\vec i + \vec j + 2\vec k$ is parallel to both $\pi_1$ and $\pi_2$.
            \item Find the equation of the plane which passes through $P$ and is perpendicular to both $\pi_1$ and $\pi_2$.
            \item Verify that $(0, 4, 4)$ is a point common to both $\pi_1$ and $\pi_2$, and hence or otherwise, find the equation of the line of intersection of $\pi_1$ and $\pi_2$, giving your answer in the form $\vec r = \vec a + \l \vec b$, $\l \in \R$.
            \item Find the coordinates of the point in which the line $OP$ meets $\pi_2$.
            \item Find the length of projection of $OP$ on $\pi_1$.
        \end{enumerate}

    \solution
        Note that $\pi_1$ and $\pi_2$ have the vector equations
        \[
            \vec r \dotp \cveciii11{-1} = 0 \text{ and } \vec r \dotp \cveciii2{-4}1 = -12
        \]
        respectively.

        \part
            Observe that $\cveciii112 \dotp \cveciii11{-1} = \cveciii112 \dotp \cveciii2{-4}1 = 0$. Thus, the vector $\cveciii112$ is perpendicular to the normal vectors of both $\pi_1$ and $\pi_2$ and is hence parallel to them.

        \part
            Let the required plane be $\pi_3$. Since $\pi_3$ is perpendicular to both $\pi_1$ and $\pi_2$, its normal vector is parallel to both planes. Thus, $\vec n = \cveciii112 \implies d = \cveciii382 \dotp \vec n = \cveciii382 \dotp \cveciii112 = 15$. $\pi_3$ hence has the vector equation
            
            \boxt{$\vec r \dotp \cveciii112 = 15$}

        \part
            Since $\cveciii044 \dotp \cveciii11{-1} = 0$ and $\cveciii044 \dotp \cveciii2{-4}1 = -12$, $(0, 4, 4)$ satisfies the vector equation of both $\pi_1$ and $\pi_2$ and thus lies on both planes.

            Let $l$ be the line of intersection of $\pi_1$ and $\pi_2$. Since $(0, 4, 4)$ is a point common to both planes, $l$ passes through it. Furthermore, since $l$ lies on both $\pi_1$ and $\pi_2$, it is perpendicular to the normal vector of both planes and hence has direction vector $\cveciii11{-1} \crossp \cveciii2{-4}1 = -3\cveciii112$. Thus, $l$ can be expressed as
            
            \boxt{$l : \vec r = \cveciii044 + \l\cveciii112, \, \l \in \R$}

        \part
            Note that the line $OP$, denoted $l_{OP}$ has equation
            \[
                l_{OP} : \vec r = \l \cveciii382, \, \l \in \R
            \]

            Consider $l_{OP} = \pi_2$.
            \begin{alignat*}{2}
                &&l_{OP} &= \pi_2\\
                \implies&&\m\cveciii382 \dotp \cveciii2{-4}1 &= -12\\
                \implies&&-24\m &= -12\\
                \implies&&\m &= \dfrac12
            \end{alignat*}

            Hence, $OP$ meets $\pi_2$ at $\bp{\dfrac32, \dfrac82, \dfrac22} = \bp{\dfrac32, 4, 1}$.

            \boxt{$\bp{\dfrac32, 4, 1}$}

        \part
            \begin{align*}
                \text{Length of projection} &= \oa{OP} \crossp \cveciii11{-1} \Bigg/ \abs{\cveciii11{-1}}\\
                &= \dfrac1{\sqrt{3}} \abs{\cveciii382 \crossp \cveciii11{-1}}\\
                &= \dfrac1{\sqrt3} \abs{\cveciii{-10}5{-5}}\\
                &= \dfrac1{\sqrt3} \cdot 5\sqrt{6}\\
                &= 5\sqrt{2}
            \end{align*}

            \boxt{The length of projection of $OP$ on $\pi_1$ is $5\sqrt2$ units.}

    \problem{}
        The line $l_1$ passes through the point $A$, whose position vector is $3\vec i -5\vec j -4\vec k$, and is parallel to the vector $3\vec i + 4\vec j + 2\vec k$. The line $l_2$ passes through the point $B$, whose position vector is $2\vec i + 3\vec j + 5\vec k$, and is parallel to the vector $\vec i - \vec j - 4\vec k$. The point $P$ on $l_1$ and $Q$ on $l_2$ are such that $PQ$ is perpendicular to both $l_1$ and $l_2$. The plane $\Pi$ contains $PQ$ and $l_1$.

        \begin{enumerate}
            \item Find a vector parallel to $PQ$.
            \item Find the equation of $\Pi$ in the forms $\vec r = \vec a + \l \vec b + \m \vec c$, $\l, \m \in \R$ and $\vec r \dotp \vec n = D$.
            \item Find the perpendicular distance from $B$ to $\Pi$.
            \item Find the acute angle between $\Pi$ and $l_2$.
            \item Find the position vectors of $P$ and $Q$.
        \end{enumerate}

    \solution
        \part
            Note that $l_1$ and $l_2$ have vector equations
            \[
                \vec r = \cveciii3{-5}{-4} + \l\cveciii342, \, \l \in \R \text{ and } \vec r = \cveciii235 + \m\cveciii1{-1}{-4}, \, \m \in \R
            \]
            respectively. Since $PQ$ is perpendicular to both $l_1$ and $l_2$, it is parallel to $\cveciii342 \crossp \cveciii1{-1}{-4} = \cveciii{-14}{14}{-7} = -7\cveciii2{-2}1$.

            \boxt{$PQ$ is parallel to $\cveciii2{-2}1$.}

        \part
            Since $\Pi$ contains $PQ$ and $l_1$, it is parallel to $\cveciii2{-2}1$ and $\cveciii342$. Also note that $\Pi$ contains $\cveciii3{-5}{-4}$. Thus, 
            
            \boxt{$\Pi : \vec r = \cveciii3{-5}{-4} + \l \cveciii2{-2}1 + \m \cveciii342, \, \l, \m \in \R$}

            Note that $\cveciii2{-2}1 \crossp \cveciii342 = \cveciii{-8}{-1}{14} = -\cveciii81{-14}$. We hence take $\vec n = \cveciii81{-14}$, whence $d = \cveciii3{-5}{-4} \dotp \vec n = \cveciii3{-5}{-4} \dotp \cveciii81{-14} = 75$.
            
            \boxt{$\Pi: \vec r \dotp \cveciii81{-14} = 75$}
 
        \part
            Note that $\oa{AB} = \cveciii{-1}89$. Hence,
            \begin{align*}
                \text{Perpendicular distance} &= \dfrac{\abs{\cveciii{-1}89 \dotp \cveciii81{-14}}}{\abs{\cveciii81{-14}}}\\
                &= \dfrac{126}{\sqrt{261}}
            \end{align*}

            \boxt{The perpendicular distance from $B$ to $\Pi$ is $\dfrac{126}{\sqrt{261}}$ units.}

        \part
            Let $\t$ be the acute angle between $\Pi$ and $l_2$.
            \begin{alignat*}{2}
                &&\sin\t &= \dfrac{\abs{\cveciii1{-1}{-4} \dotp \cveciii81{-14}}}{\abs{\cveciii1{-1}{-4}} \abs{\cveciii81{-14}}}\\
                && &= \dfrac{7}{\sqrt{58}}\\
                \implies&&\t &= 66.8 \deg \todp{1}
            \end{alignat*}

            \boxt{$\t = 66.8 \deg$}

        \part
            Since $P$ is on $l_1$, we have $\oa{OP} = \cveciii3{-5}{-4} + \l\cveciii342$ for some $\l \in \R$. Similarly, since $Q$ is on $l_2$, we have $\oa{OQ} = \cveciii235 + \m\cveciii1{-1}4$ for some $\m \in \R$. Thus,
            \begin{align*}
                \oa{PQ} &= \oa{OQ} - \oa{OP}\\
                &= \bs{\cveciii235 + \m\cveciii1{-1}4} - \bs{\cveciii3{-5}{-4} + \l\cveciii342}\\
                &= \cveciii{-1}89 - \l\cveciii342 + \m\cveciii1{-1}{-4}
            \end{align*}
            Recall that $PQ$ is parallel to $\cveciii2{-2}1$. Hence, $\oa{PQ}$ can be expressed as $\nu \cveciii2{-2}1$ for some $\nu \in \R$. 
            \begin{alignat*}{2}
                &&\cveciii{-1}89 - \l\cveciii342 + \m\cveciii1{-1}{-4} &= \nu \cveciii2{-2}1\\
                \implies&&\l\cveciii342 - \m\cveciii1{-1}{-4} + \nu \cveciii2{-2}1 &= \cveciii{-1}89\\
                \implies&&\l\cveciii342 + \m\cveciii{-1}14 + \nu \cveciii2{-2}1 &= \cveciii{-1}89 
            \end{alignat*}
            This gives the following system:
            \[
                \systeme[\l\m\n]{
                    3\l - \m + 2\n = -1,
                    4\l + \m -2\n = 8,
                    2\l + 4\m + \n = 9
                }
            \]
            which has the unique solution $\l = 1$, $\m = 2$ and $\n = -1$. Thus,
            \begin{alignat*}{2}
                \oa{OP} &= \cveciii3{-5}{-4} + \cveciii342 &= \cveciii6{-1}{-2}\\
                \oa{OQ} &= \cveciii235 + 2\cveciii1{-1}{-4} &= \cveciii41{-3}
            \end{alignat*}

            \boxt{$\oa{OP} = \cveciii6{-1}{-2}$, $\oa{OQ} = \cveciii41{-3}$}

    \problem{}
        The equations of three planes $p_1$, $p_2$ and $p_3$ are
        \begin{align*}
            2x-5y+3z &= 3\\
            3x+2y-5z&=-5\\
            5x+\l y + 17z &= \m
        \end{align*}
         respectively, where $\l$ and $\m$ are constants. The planes $p_1$ and $p_2$ intersect in a line $l$.

        \begin{enumerate}
            \item Find a vector equation of $l$.
            \item Given that all three planes meet in the line $l$, find $\l$ and $\m$.
            \item Given instead that the three planes have no point in common, what can be said about the values of $\l$ and $\m$?
            \item Find the Cartesian equation of the plane which contains $l$ and the point $(1, -1, 3)$.
        \end{enumerate}

    \solution
        \part
            Consider $p_1 = p_2$.
            \[
                \systeme{
                    2x-5y+3z= 3,
                    3x+2y-5z=-5
                }
            \]
            The above system has solution
            \[
                \systeme*{
                    x = -1 + t,
                    y = -1 + t,
                    z = t
                }
            \]
            for all $t \in \R$. Thus, the line $l$ has vector equation
            \begin{align*}
                \vec r &= \cveciii{x}{y}{z}\\
                &= \cveciii{-1+t}{-1+t}{t}\\
                &= \cveciii{-1}{-1}0 + t\cveciii111, \, t \in \R
            \end{align*}
            \boxt{$l : \vec r = \cveciii{-1}{-1}0 + t\cveciii111, \, t \in \R$}

        \part
            Since all three planes meet in the line $l$, $l$ must satisfy the equation of $p_3$. Substituting the above solution to the given equation, we have
            \begin{alignat*}{2}
                &&5(-1 + t) + \l(-1 + t) + 17t &= \m\\
                \implies&&(22+\l)t - (5+\l+\m) &= 0
            \end{alignat*}
            Comparing the coefficients of $t$ and the constant terms, we have the following system:
            \[
                \systeme[\l\m]{
                    22 + \l = 0,
                    5 + \l + \m = 0
                }
            \]
            which has the unique solution $\l = -22$ and $\m = 17$.

            \boxt{$\l = -22$, $\m = 17$}

        \part
            If the three planes have no point in common, we have
            \[
                (22+\l)t - (5+\l+\m) \neq 0
            \]
            for all $t \in \R$. To satisfy this relation, we need $22 + \l = 0$ and $5 + \l + \m \neq 0$, whence $\l = -22$ and $\m \neq 17$.

            \boxt{$\l = -22$, $\m \neq 17$}

        \part
            Note that $\cveciii{-1}{-1}0$ lies on $l$ and is thus contained on the required plane. Observe that $\cveciii{-1}{-1}0 - \cveciii1{-1}3 = \cveciii{-2}0{-3}$. Thus, the required plane is parallel to $\cveciii111$ and $\cveciii{-2}0{-3}$ and hence has vector equation
            \[
                \vec r = \cveciii{-1}{-1}0 + \l\cveciii111 + \m\cveciii{-2}0{-3}, \, \l, \m \in \R
            \]
            Observe that $\vec n = \cveciii111 \crossp \cveciii{-2}03 = \cveciii{-3}12$, whence $d = \cveciii{-1}{-1}0 \dotp \vec n = \cveciii{-1}{-1}0 \dotp \cveciii{-3}12 = 2$. Thus, the required plane has the equation
            \[
                \vec r \dotp \cveciii{-3}12 = 2
            \]
            Let $\vec r = \cveciii{x}{y}{z}$. It follows that the plane has Cartesian equation
            \boxt{$-3x + y + 2 = 2$}


    \problem{}
        The planes $p_1$ and $p_2$, which meet in line $l$, have equations $x - 2y + 2z = 0$ and $2x - 2y + z = 0$ respectively.

        \begin{enumerate}
            \item Find an equation of $l$ in Cartesian form.
        \end{enumerate}

        The plane $p_3$ has equation $(x-2y+2z) + c(2x-2y+z) = d$.

        \begin{enumerate}
            \setcounter{enumi}{1}
            \item Given that $d = 0$, show that all 3 planes meet in the line $l$ for any constant $c$.
            \item Given instead that the 3 planes have no point in common, what can be said about the value of $d$?
        \end{enumerate}

    \solution
        \part
            Consider $p_1 = p_2$. This gives the system
            \[
                \systeme{
                    x-2y+2z=0,
                    2x-2y+z=0
                }
            \]
            which has solution $x = t$, $y = \dfrac32 t$ and $z = $. Thus, $l$ has Cartesian equation
            \boxt{$x = \dfrac23 y = z$}

        \part
            When $d = 0$, $p_3$ has equation
            \[
                (x-2y+2z) + c(2x-2y+z) = 0
            \]
            Observe that the line $l$ satisfies the equations $x-2y+2z = 0$ and $2x-2y+z = 0$. Hence, $l$ also satisfies the equation that gives $p_3$ for all $c$. Thus, $p_3$ contains $l$, implying that all 3 planes meet in the line $l$.

        \part
            If the 3 planes have no point in common, then $l$ does not have any point in common with $p_3$. That is, all points on $l$ satisfy the relation
            \[
                (x-2y+2z) + c(2x-2y+z) \neq d
            \]
            Since $x - 2y + 2z = 0$ and $2x - 2y + z = 0$ for all points on $l$, the LHS simplifies to 0. Thus, to satisfy the above relation, we require $d \neq 0$.

            \boxt{$d \neq 0$}

    \problem{}
        \begin{center}
            \begin{tikzpicture}
                \coordinate[label=left:$O$] (O) at (0, 0);
                \coordinate[label=below:$A$] (A) at (-3, -1);
                \coordinate[label=below:$B$] (B) at (1, -1);
                \coordinate[label=above right:$C$] (C) at (3, 1);
                \coordinate[label=above right:$D$] (D) at (-1, 1);
                \coordinate[label=above:$V$] (V) at (0, 3);
                \coordinate[label=right:$P$] (P) at (6, -1);
                \coordinate (S1) at (-6, -1);
                \coordinate (S2) at (-3, 2);
                \coordinate (S3) at (-3, 6);
                \coordinate (S4) at (-6, 3);

                \draw (A) -- (B);
                \draw (C) -- (D);
                \draw[dotted] (A) -- (D);
                \draw (B) -- (C);
                \draw[dotted] (O) -- (V);
                \draw (A) -- (V);
                \draw (B) -- (V);
                \draw (C) -- (V);
                \draw[dotted] (D) -- (V);

                \fill (P) circle[radius=2.5pt];
                \draw[dotted] (B) -- (P);
                
                \draw[dotted] (A) -- (S1);
                \draw (S1) -- (S2);
                \draw (S2) -- (S3);
                \draw (S3) -- (S4);
                \draw (S4) -- (S1);

                \node at (-4.5, 2.3) {Screen};

                \draw[very thick, ->] (O) -- (0.5, 0) node[anchor=north] {$\vec i$};
                \draw[very thick, ->] (O) -- (0.354, 0.354) node[anchor=south] {$\vec j$};
                \draw[very thick, ->] (O) -- (0, 0.5) node[anchor=east] {$\vec k$};
            \end{tikzpicture}
        \end{center}

         A right opaque pyramid with square base $ABCD$ and vertex $V$ is placed at ground level for a shadow display, as shown in the diagram. $O$ is the centre of the square base $ABCD$, and the perpendicular unit vectors $\vec i$, $\vec j$ and $\vec k$ are in the directions of $\oa{AB}$, $\oa{AD}$ and $\oa{OV}$ respectively. The length of $AB$ is 8 units and the length of $OV$ is $2h$ units.

        A point light source for this shadow display is placed at the point $P(20, -4, 0)$ and a screen of height 35 units is placed with its base on the ground such that the screen lies on a plane with vector equation $\vec r \dotp \cveciii100 = \a$, where $\a < -4$.

        \begin{enumerate}
            \item Find a vector equation of the line depicting the path of the light ray from $P$ to $V$ in terms of $h$.
            \item Find an inequality between $\a$ and $h$ so that the shadow of the pyramid cast on the screen will not exceed the height of the screen.
        \end{enumerate}

         The point light source is now replaced by a parallel light source whose light rays are perpendicular to the screen. It is also given that $h = 10$.
        
        \begin{enumerate}
            \setcounter{enumi}{2}
            \item Find the exact length of the shadow cast by the edge $VB$ on the screen.
        \end{enumerate}

         A mirror is placed on the plane $VBC$ to create a special effect during the display.

        \begin{enumerate}
            \setcounter{enumi}{3}
            \item Find a vector equation of the plane $VBC$ and hence find the angle of inclination made by the mirror with the ground.
        \end{enumerate}

    \solution
        \part
            Note that $\oa{OV} = \cveciii00{2h}$ and $\oa{OP} = \cveciii{20}{-4}0$, whence $\oa{PV} = \cveciii{-20}{4}{2h} = 2\cveciii{-10}2h$. Thus, the line from $P$ to $V$, denoted $l_{PV}$, has vector equation
            
            \boxt{$l_{PV} : \vec r = \cveciii{20}{-4}0 + \l\cveciii{-10}2h, \, \l \in \R$}

        \part
            Let the point of intersection between $l_{PV}$ and the screen be $I$.
            \begin{alignat*}{2}
                &&\bs{\cveciii{20}{-4}0 + \l\cveciii{-10}2h} \dotp \cveciii100 &= \a\\
                \implies&&\cveciii{20}{-4}0 \dotp \cveciii100 + \l\cveciii{-10}2h \dotp \cveciii100 &= \a\\
                \implies&&20-10\l &= \a\\
                \implies&&\l &= \dfrac{20-\a}{10}
            \end{alignat*}

            Hence, $\oa{OI} = \cveciii{20}{-4}0 + \dfrac{20-\a}{10}\cveciii{-10}2h$. To prevent the shadow from exceeding the screen, we require the $\vec k$ component of $\oa{OI}$ to be less than the height of the screen, i.e. 35 units. This gives the inequality $\dfrac{20-\a}{10} \cdot h \leq 35$, whence
            
            \boxt{$h \leq \dfrac{350}{20 - \a}$}

        \part
            Since the light rays emitted by the light source are now perpendicular to the screen, the image of some point with coordinates $(a, b, c)$ on the screen is given by $(\a, b, c)$. Thus, the image of $B(4, -4, 0)$ and $V(0, 0, 20)$ on the screen have coordinates $(\a, -4, 0)$ and $(\a, 0, 20)$. The length of the shadow cast by $VB$ is thus given by
            \[
                \sqrt{(\a - \a)^2 + (-4-0)^2 + (0-20)^2} = 4\sqrt{26}
            \]

            \boxt{The shadow cast by the edge $VB$ on the screen is $4\sqrt{26}$ units long.}
            
        \part
            Note that $\oa{BV} = 4\cveciii{-1}15$ and $\oa{BC} = 8\cveciii010$. Hence, the plane $VBC$ is parallel to $\cveciii{-1}15$ and $\cveciii010$. Note that $\cveciii{-1}15 \crossp \cveciii010 = -\cveciii501$. Thus, $\vec n = \cveciii501$, whence $d = \cveciii00{20} \dotp \vec n = \cveciii00{20} \dotp \cveciii501 = 20$. Thus, the plane $VBC$ has vector equation
            
            \boxt{$\vec r \dotp \cveciii501 = 20$}

            Observe that the ground is given by the vector equation
            \[
                \vec r \dotp \cveciii001 = 0
            \]
            Let $\t$ be the angle of inclination made by the mirror with the ground.
            \begin{alignat*}{2}
                &&\cos\t &= \dfrac{\cveciii501 \dotp \cveciii001}{\abs{\cveciii501}\abs{\cveciii001}}\\
                && &= \dfrac1{\sqrt{26}}\\
                \implies&&\t &= 78.7 \deg \todp{1}
            \end{alignat*}

            \boxt{$\t = 78.7 \deg$}
\end{document}
