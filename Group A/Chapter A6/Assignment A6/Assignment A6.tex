\documentclass{echw}

\title{Assignment A6\\Polar Coordinates}
\author{Eytan Chong}
\date{2024-04-26}

\begin{document}
    \problem{}
        The planet Mercury travels around the sun in an elliptical orbit given approximately by
        \begin{equation*}
            r = \dfrac{3.442 \times 10^7}{1 - 0.206\cos\theta}
        \end{equation*}
        \noindent where $r$ is measured in miles and the sun is at the pole.

        Sketch the orbit and find the distance from Mercury to the sun at the aphelion (greatest distance from the sun) and at the perihelion (shortest distance from the sun).

    \solution
        \begin{center}
            \begin{tikzpicture}[trim axis left, trim axis right]
                \begin{axis}[
                    domain = 0:2*pi,
                    samples = 100,
                    axis y line=middle,
                    axis x line=middle,
                    xtick = \empty,
                    ytick = \empty,
                    xlabel = {$\theta=0$},
                    ylabel = {$\theta = \frac\pi2$},
                    legend cell align={left},
                    legend pos=outer north east,
                    after end axis/.code={
                        \path (axis cs:0,0) 
                            node [anchor=north east] {$O$};
                        }
                    ]
                    \addplot[color=plotRed,data cs=polarrad] {1/(1 - 0.206*cos(\x r))};
        
                    \addlegendentry{$r = \frac{3.442 \times 10^7}{1 - 0.206\cos\theta}$};
                \end{axis}
            \end{tikzpicture}
        \end{center}

        At the apehlion, $r$ is maximised. Hence, the denominator of $r$ is minimised. Hence, $\cos \theta$ is maximised. Since the maximimum value of $\cos \theta$ is 1,
        \begin{align*}
            r &= \dfrac{3.442 \times 10^7}{1 - 0.206\cdot1}\\
            &= 4.34 \times 10^7 \tosf{3}
        \end{align*}

        Hence, the distance from Mercury to the sun at the aphelion is $4.34 \times 10^7$ miles.

        \boxt{
            $4.34 \times 10^7$ miles
        }

        At the perihelion, $r$ is minimised. Hence, the denominator of $r$ is maximised. Hence, $\cos \theta$ is mimimised. Since the minimum value of $\cos \theta$ is $-1$,
        \begin{align*}
            r &= \dfrac{3.442 \times 10^7}{1 - 0.206\cdot-1}\\
            &= 2.85 \times 10^7 \tosf{3}
        \end{align*}

        Hence, the distance from Mercury to the sun at the perihelion is $2.85 \times 10^7$ miles.

        \boxt{
            $2.85 \times 10^7$ miles
        }

    \problem{}
        A variable point $P$ has polar coordinates $\coord{r}{\theta}$, and fixed points $A$ and $B$ have polar coordinates $\coord{1}{0}$ and $\coord{1}{\pi}$ respectively. Given that $P$ moves so that the product $PA\cdot PB = 2$, show that
        \begin{equation*}
            r^2 = \cos2\theta + \sqrt{3 + \cos^22\theta}
        \end{equation*}

        \begin{enumerate}
            \item Given that $r \geq 0$ and $0 \leq \theta \leq 2\pi$, find the maximum and minimum values of $r$, and the values of $\theta$ at which they occur.
            \item Verify that the path taken by $P$ is symmetry about the lines $\theta = 0$ and $\theta = \dfrac{\pi}{2}$, giving your reasons.
        \end{enumerate}

    \solution
        Note that $A$ and $B$ have Cartesian coordinates $\coord{1}{0}$ and $\coord{-1}{0}$ respectively. Let $P\coord{x}{y}$. Using the Distance Formula,
        \begin{align*}
            PA^2 &= (x-1)^2 + y^2\\
            PB^2 &= (x+1)^2 + y^2
        \end{align*}

        Since $PA \cdot PB = 2$,
        \begin{alignat*}{2}
            &&PA^2 \cdot PB^2 &= 4\\
            \implies&&\left((x-1)^2 + y^2\right)\left((x+1)^2 + y^2\right) &= 4\\
            \implies&&(x-1)^2(x+1)^2 + y^2\left((x-1)^2 + (x+1)^2\right) + y^4 &= 4\\
            \implies&&\left(x^2 - 1\right)^2 + y^2\left(2x^2 + 2\right) + y^4 &= 4\\
            \implies&&x^4 - 2x^2 + 1 + 2x^2y^2 + 2y^2 + y^4 &= 4\\
            \implies&&\left(x^4 + 2x^2y^2 + y^4\right) - 2\left(x^2 - y^2\right) &= 3\\
            \implies&&\left(x^2 + y^2\right)^2 - 2\left(x^2 - y^2\right) &= 3
        \end{alignat*}

        Note that
        \begin{align*}
            x^2 - y^2 &= (r\cos\theta)^2 - (r\sin\theta)^2\\
            &= r^2 \left(cos^2\theta - \sin^2\theta\right)\\
            &= r^2 \cos2\theta
        \end{align*}

        Hence,
        \begin{alignat*}{2}
            &&\left(r^2\right)^2 - 2r^2\cos2\theta &= 3\\
            \implies&&r^2 &= \dfrac{-\left(-2\cos2\theta\right) \pm \sqrt{(-2\cos2\theta)^2 - 4\cdot1\cdot-3}}{2\cdot1}\\
            && &= \cos2\theta \pm \sqrt{\cos^22\theta + 3}
        \end{alignat*}

        Since $\sqrt{\cos^22\theta + 3} > \cos2\theta$ and $r^2 \geq 0$, we reject the negative case. Thus,
        \begin{equation*}
            r^2 = \cos2\theta + \sqrt{3 + \cos^22\theta}
        \end{equation*}

        \part
            The maximum value of $r$ is achieved when $A$, $B$ and $P$ are collinear, whence $\theta = 0, \pi$.
            \begin{alignat*}{2}
                &&r^2 &= \cos(2 \cdot 0) + \sqrt{3 + \cos^2 (2\cdot0)}\\
                && &= 3\\
                \implies&&r &= \sqrt{3}
            \end{alignat*}

            \boxt{
                The maximum value of $r$ is $\sqrt{3}$ and occurs when $\theta = 0, \pi$.
            }

            The minimum value of $r$ is achieved when $P$ is equidistant from $A$ and $B$, whence $\theta = \dfrac12 \pi, \dfrac32 \pi$.
            \begin{alignat*}{2}
                &&r^2 &= \cos\left(2 \cdot \dfrac12 \pi\right) + \sqrt{3 + \cos^2 \left(2\cdot\dfrac12 \pi\right)}\\
                && &= 1\\
                \implies&&r &= 1
            \end{alignat*}

            \boxt{
                The minimum value of $r$ is 1 and occurs when $\theta = \dfrac12 \pi, \dfrac32 \pi$.
            }

        \part
            \begin{equation*}
                \left((x-1)^2 + y^2\right)\left((x+1)^2 + y^2\right) = 4
            \end{equation*}

            Consider the transformation $y \mapsto -y$.
            \begin{alignat*}{2}
                && \left((x-1)^2 + (-y)^2\right)\left((x+1)^2 + (-y)^2\right) &= 4\\
                \implies&&\left((x-1)^2 + y^2\right)\left((x+1)^2 + y^2\right) &= 4
            \end{alignat*}

            Since the path taken by $P$ is invariant under the transformation $y \mapsto -y$, it is symmetric about the $x$-axis, or the line $\theta = 0$.

            Consider the transformation $x \mapsto -x$.
            \begin{alignat*}{2}
                && \left((-x-1)^2 + y^2\right)\left((-x+1)^2 + y^2\right) &= 4\\
                \implies&&\left((x-1)^2 + y^2\right)\left((x+1)^2 + y^2\right) &= 4
            \end{alignat*}
            
            Since the path taken by $P$ is invariant under the transformation $x \mapsto -x$, it is symmetric about the $y$-axis, or the line $\theta = \dfrac{\pi}2$.

    \problem{}
        \begin{enumerate}
            \item Explain why the curve with equation $x^3 + 2xy^2 - a^2y = 0$ where $a$ is a positive constant lies entirely in the region $\abs{x} \leq 2^{-\frac34}a$.
            \item Show that the polar equation of this curve is $r^2 = \dfrac{a^2\tan\theta}{2-\cos^2\theta}$.
            \item Sketch the curve.
        \end{enumerate}

    \solution
        \part
            \begin{alignat*}{2}
                &&x^3 + 2xy^2 - a^2y &= 0\\
                \implies&&2y^2 - a^2y + x^3 &= 0
            \end{alignat*}

            Consider the discriminant $\Delta = \left(-a^2\right)^2 - 4\cdot 2x \cdot x^3 = a^4 - 8x^4$. For points on the curve, $\Delta \geq 0$.
            \begin{alignat*}{2}
                &&\Delta &\geq 0\\
                \implies&&a^4 - 8x^4 &\geq 0\\
                \implies&&x^4 &\leq 2^{-3} a^4\\
                \implies&&\abs{x} &\leq 2^{-\frac34} a 
            \end{alignat*}

        \part
            \begin{alignat*}{2}
                &&x^3 + 2xy^2 - a^2y &= 0\\
                \implies&&x^3 + 2xy^2 &= a^2y\\
                \implies&&2x^3 + 2xy^2 - x^3 &= a^2y\\
                \implies&&2x\left(x^2 + y^2\right) - x^3 &= a^2 y\\
                \implies&&2xr^2 - x^3 &= a^2y\\
                \implies&&2r^2 - x^2 &= a^2\dfrac{y}{x}\\
                \implies&&2r^2 - (r\cos\theta)^2 &= a^2\tan\theta\\
                \implies&&2r^2 - r^2\cos^2\theta &= a^2\tan\theta\\
                \implies&&r^2\left(2 - \cos^2\theta\right) &= a^2\tan\theta\\
                \implies&&r^2 &= \dfrac{a^2\tan\theta}{2-\cos^2\theta}
            \end{alignat*}

        \part
            \begin{center}
                \begin{tikzpicture}[trim axis left, trim axis right]
                    \begin{axis}[
                        domain = 0:2*pi,
                        samples = 100,
                        axis y line=middle,
                        axis x line=middle,
                        xtick = \empty,
                        ytick = \empty,
                        xmin=-2,
                        xmax=2,
                        ymin=-2,
                        ymax=2,
                        xlabel = {$\theta=0$},
                        ylabel = {$\theta = \frac\pi2$},
                        legend cell align={left},
                        legend pos=outer north east,
                        after end axis/.code={
                            \path (axis cs:0,0) 
                                node [anchor=north east] {$O$};
                            }
                        ]
                        \addplot[color=plotRed,data cs=polarrad, unbounded coords = jump] {sqrt((tan(\x r))/(2 - cos(\x r)^2))};
            
                        \addlegendentry{$r^2 = \frac{a^2\tan\theta}{2-\cos^2\theta}$};
                    \end{axis}
                \end{tikzpicture}
            \end{center}

    \problem{}
        The curve $C$ has polar equation $r = 1 - \sin 3\theta$, where $0 \leq \theta \leq 2\pi$.

        \begin{enumerate}
            \item Sketch the curve $C$, showing the tangents at the pole and the intersections with the axes.
            \item Find the gradient of the curve at the point where $\theta = \dfrac{\pi}3$, giving your answer in the form $a + b\sqrt3$, where $a$ and $b$ are constants to be determined.
        \end{enumerate}

    \solution
        \part
            \begin{center}
                \begin{tikzpicture}[trim axis left, trim axis right]
                    \begin{axis}[
                        domain = 0:2*pi,
                        samples = 100,
                        axis y line=middle,
                        axis x line=middle,
                        xtick = {-1, 1},
                        xticklabels = {$\coord{1}{\pi}$, $\coord{1}{0}$},
                        ytick = {2},
                        yticklabels = {$\coord{2}{\frac\pi2}$},
                        ymax=2.3,
                        xlabel = {$\theta=0$},
                        ylabel = {$\theta = \frac\pi2$},
                        legend cell align={left},
                        legend pos=outer north east,
                        after end axis/.code={
                            \path (axis cs:0,0) 
                                node [anchor=north east] {$O$};
                            }
                        ]
                        \addplot[color=plotRed,data cs=polarrad] {1 - sin(3 * \x r)};
            
                        \addlegendentry{$r = 1 - \sin3\theta$};
                        
                        \node[anchor=west] at (0, -1) {$\theta = \frac32 \pi$};

                        \addplot[dotted, thick, domain=0:sqrt(3)] {1/sqrt(3) * x};

                        \addplot[dotted, thick, domain=-sqrt(3):0] {-1/sqrt(3) * x};

                        \node[anchor=south west, rotate=-25] at (-sqrt 3, 1) {$\theta = \frac56 \pi$};

                        \node[anchor=south east, rotate=25] at (sqrt 3, 1) {$\theta = \frac16 \pi$};
                    \end{axis}
                \end{tikzpicture}
            \end{center}

            Consider $\theta = 0 \implies r = 1 - \sin (3 \cdot 0) = 1$. Thus, $C$ intersects the horizontal axis at $\coord{1}{0}$. By symmetry, $C$ also intersects the horizontal axis at $\coord{1}{\pi}$.
            
            Consider $\theta = \dfrac\pi2 \implies r = 1 - \sin\left(3 \cdot \dfrac\pi2\right) = 2$. Thus, $C$ intersects the vertical axis at $\coord{2}{\dfrac\pi2}$.

            Consider $\theta = \dfrac32 \pi \implies r = 1 - \sin\left(3 \cdot \dfrac32 pi\right) = 0$. Thus, $C$ intersects the pole.

            For tangents at the pole, $r = 0 \implies \sin 3\theta = 1 \implies 3\theta = \dfrac12 \pi, \dfrac52\pi, \dfrac92 \pi$. Hence, $\theta = \dfrac16 \pi, \dfrac56 \pi, \dfrac32 \pi$.

        \part
            Note that
            \begin{align*}
                \left.\der{r}{\theta}\right|_{\theta = \frac\pi3} &= \left.\left(-3\cos3\theta\right)\right|_{\theta = \frac\pi3}\\
                &= -3\cos\left(3 \cdot \dfrac\pi3\right)\\
                &= 3
            \end{align*}

            Further observe that when $\theta = \dfrac\pi3$, $r = 1 - \sin\left(3 \cdot \dfrac\pi3\right) = 1$.

            \begin{align*}
                \left.\der{y}{x}\right|_{\theta = \frac\pi3} &= \left.\left(\dfrac{\der{r}{\theta}\sin\theta + r\cos\theta}{\der{r}{\theta}\cos\theta - r\sin\theta}\right)\right|_{\theta = \frac\pi3}\\
                &= \dfrac{3 \cdot \sin\frac\pi3 + 1 \cdot \cos\frac\pi3}{3 \cdot \cos\frac\pi3 - 1 \cdot \sin\frac\pi3}\\
                &= \dfrac{3 \cdot \frac{\sqrt3}{2} + \frac12}{3 \cdot \frac12 - \frac{\sqrt3}{2}}\\
                &= \dfrac{3 \sqrt{3} + 1}{3 - \sqrt{3}}\\
                &= \dfrac{(3 \sqrt{3} + 1)(3 + \sqrt3)}{(3 - \sqrt{3})(3 + \sqrt3)}\\
                &= \dfrac{9\sqrt3 + 3 + 9 + \sqrt3}{9 - 3}\\
                &= \dfrac{12 + 10\sqrt3}{6}\\
                &= 2 + \dfrac53\sqrt{3}
            \end{align*}

            \boxt{
                When $\theta = \dfrac\pi3$, the gradient of the curve is $\left(2 + \dfrac53 \sqrt3\right)$.
            }
\end{document}