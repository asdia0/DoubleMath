\documentclass{echw}

\title{Tutorial A4\\Series and Sequences II}
\author{Eytan Chong}
\date{2024-03-28}

\begin{document}
    \problem{}
        True or False? Explain your answers briefly.

        \begin{enumerate}
            \item $\displaystyle\sum\limits_{r=1}^n (2r + 3) = \sum\limits_{k=1}^n (2k + 3)$
            \item $\displaystyle\sum\limits_{r = 1}^n \bp{\dfrac1r + 5} = \sum\limits_{r=1}^n \dfrac1r + 5$
            \item $\displaystyle\sum\limits_{r=1}^n \dfrac1r = 1 \Big/ {\sum\limits_{r=1}^n r}$
            \item $\displaystyle\sum\limits_{r=1}^n c = \sum\limits_{r=0}^{n-1} (c+1)$
        \end{enumerate}

    \solution
        \part
            Since both sums differ only by dummy variables, they are equal.

            \boxt{True}

        \part
            Summation is distributive. Since $\sum\limits_{r = 1}^n 5$ is not equal to $5$ in general, the equality does not hold.

            \boxt{False}

        \part
            In general, $\displaystyle\sum \dfrac{a}{b} \neq {\displaystyle\sum a} \Big/ {\displaystyle\sum b}$.

            \boxt{False}

        \part
            Since $c$ is a constant with respect to $r$, $\displaystyle\sum\limits_{r=1}^n c = nc \neq n(c+1) = \sum\limits_{r=0}^{n-1} (c+1)$.

            \boxt{False}

    \problem{}
        Write the following series in $\Sigma$ notation twice, with $r = 1$ as the lower limit in the first and $r =0$ as the lower limit in the second.

        \begin{enumerate}
            \item $-2 + 1 + 4 + \ldots + 40$
            \item $a^2 + a^4 + a^6 + \ldots + a^{50}$
            \item $\dfrac13 + \dfrac15 + \dfrac17 + \ldots + n$th term
            \item $1 - \dfrac12 + \dfrac14 - \dfrac18 + \ldots$ to $n$ terms
            \item $\dfrac1{2\cdot4} + \dfrac1{3\cdot5} + \dfrac1{4 \cdot 6} + \ldots + \dfrac1{28 \cdot 30}$
        \end{enumerate}

    \solution
        \part
            Observe that $-2 + 1 + 4 + \ldots + 40$ is in arithmetic progression with a common difference of 3.
            \begin{align*}
                -2 + 1 + 4 + \ldots + 40 &= \sum_{r = 1}^{15} (3r-5)\\
                &= \sum_{r = 0}^{14} (3(r+1)-5)\\
                &= \sum_{r = 0}^{14} (3r-2)
            \end{align*}

            \boxt{$\displaystyle-2 + 1 + 4 + \ldots + 40 = \sum\limits_{r = 1}^{15} (3r-5) = \sum\limits_{r = 0}^{14} (3r-2)$}

        \part
            Observe that $a^2 + a^4 + a^6 + \ldots + a^{50}$ is in geometric progression with a common ratio of $a^2$.
            \begin{align*}
                a^2 + a^4 + a^6 + \ldots + a^{50} &= \sum_{r=1}^{25} a^{2r}\\
                &= \sum_{r=0}^{24} a^{2(r+1)}\\
                &= \sum_{r=0}^{24} a^{2r + 2}
            \end{align*}

            \boxt{$\displaystyle a^2 + a^4 + a^6 + \ldots + a^{50} = \sum\limits_{r=1}^{25} a^{2r} = \sum\limits_{r=0}^{24} a^{2r + 2}$}

        \part
            \begin{align*}
                \dfrac13 + \dfrac15 + \dfrac17 + \ldots + n\text{th term} &= \sum_{r=1}^n \dfrac1{2r+1}\\
                &= \sum_{r=0}^{n-1} \dfrac1{2(r+1)+1}\\
                &= \sum_{r=0}^{n-1} \dfrac1{2r+3}
            \end{align*}

            \boxt{$\displaystyle \dfrac13 + \dfrac15 + \dfrac17 + \ldots + n\text{th term} = \sum\limits_{r=1}^n \dfrac1{2r+1} = \sum\limits_{r=0}^{n-1} \dfrac1{2r+3}$}

        \part
            \begin{align*}
                1 - \dfrac12 + \dfrac14 - \dfrac18 + \ldots \text{ to $n$ terms} &= \sum_{r = 1}^n \left( -\dfrac12\right)^{r-1}\\
                &= \sum_{r = 0}^{n-1} \left( -\dfrac12\right)^{(r+1)-1}\\
                &= \sum_{r = 0}^{n-1} \left( -\dfrac12\right)^{r}
            \end{align*}

            \boxt{$\displaystyle1 - \dfrac12 + \dfrac14 - \dfrac18 + \ldots \text{ to $n$ terms} = \sum\limits_{r = 1}^n \left( -\dfrac12\right)^{r-1} = \sum\limits_{r = 0}^{n-1} \left( -\dfrac12\right)^{r}$}

        \part
            \begin{align*}
                \dfrac1{2\cdot4} + \dfrac1{3\cdot5} + \dfrac1{4 \cdot 6} + \ldots + \dfrac1{28 \cdot 30} &= \sum_{r=1}^{27} \dfrac1{(r+1)(r+3)}\\
                &= \sum_{r=0}^{26} \dfrac1{((r+1)+1)((r+1)+3)}\\
                &= \sum_{r=0}^{26} \dfrac1{(r+2)(r+4)}
            \end{align*}

            \boxt{$\displaystyle\dfrac1{2\cdot4} + \dfrac1{3\cdot5} + \dfrac1{4 \cdot 6} + \ldots + \dfrac1{28 \cdot 30} = \sum\limits_{r=1}^{27} \dfrac1{(r+1)(r+3)} = \sum\limits_{r=0}^{26} \dfrac1{(r+2)(r+4)}$}

    \problem{}
        Without using the GC, evaluate the following sums.

        \begin{enumerate}
            \item $\displaystyle\sum\limits_{r=1}^{50} (2r-7)$
            \item $\displaystyle\sum\limits_{r=1}^a (1-a-r)$
            \item $\displaystyle\sum\limits_{r=2}^n \left(\ln r + 3^r\right)$
            \item $\displaystyle\sum\limits_{r=1}^\infty \left(\dfrac{2^r-1}{3^r}\right)$
        \end{enumerate}

    \solution
        \part
            \begin{align*}
                \sum\limits_{r=1}^{50} (2r-7) &= 2\sum\limits_{r=1}^{50} r - 7 \sum\limits_{r=1}^{50} 1\\
                &= 2 \cdot \dfrac{50 \cdot 51}2 - 7 \cdot 50\\
                &= 2200
            \end{align*}

            \boxt{$\displaystyle\sum\limits_{r=1}^{50} (2r-7) = 2200$}

        \part
            \begin{align*}
                \sum\limits_{r=1}^a (1-a-r) &= (1-a)\sum\limits_{r=1}^a 1 - \sum\limits_{r=1}^a r\\
                &= (1-a)\cdot a - \dfrac{a(a+1)}2\\
                &= \dfrac{a}{2} \cdot 2(1-a) - \dfrac{a}2 \cdot (a+1)\\
                &= \dfrac{a}{2} (2 - 2a - (a + 1))\\
                &= \dfrac{a}2 (1-3a)
            \end{align*}

            \boxt{$\displaystyle\sum\limits_{r=1}^a (1-a-r) = \dfrac{a}2 (1-3a)$}

        \part
            \begin{align*}
                \sum\limits_{r=2}^n \left(\ln r + 3^r\right) &= \sum\limits_{r=2}^n \ln r + \sum\limits_{r=2}^n 3^r\\
                &= \ln n! + \sum\limits_{r=1}^{n-1} 3^{r+1}\\
                &= \ln n! + 3\sum\limits_{r=1}^{n-1} 3^r\\
                &= \ln n! + 3 \cdot \dfrac{3(3^{n-1} - 1)}{3-1}\\
                &= \ln n! +\dfrac92 \left(3^{n-1} - 1\right)
            \end{align*}

            \boxt{$\displaystyle\sum\limits_{r=2}^n \left(\ln r + 3^r\right) = \ln n! +\dfrac92 \left(3^{n-1} - 1\right)$}

        \part
            \begin{align*}
                \sum\limits_{r=1}^\infty \left(\dfrac{2^r-1}{3^r}\right) &= \sum\limits_{r=1}^\infty \dfrac{2^r}{3^r} - \sum\limits_{r=1}^\infty \dfrac1{3^r}\\
                &= \sum\limits_{r=1}^\infty \left(\dfrac23\right)^r - \sum\limits_{r=1}^\infty \left(\dfrac1{3}\right)^r\\
                &= \dfrac{\tfrac23}{1-\tfrac23} - \dfrac{\tfrac13}{1- \tfrac13}\\
                &= \dfrac32
            \end{align*}

            \boxt{$\displaystyle\sum\limits_{r=1}^\infty \left(\dfrac{2^r-1}{3^r}\right) = \dfrac32$}

    \problem{}
        The $n$th term of a series is $2^{n-2} + 3n$. Find the sum of the first $N$ terms.

    \solution
        \begin{align*}
            \sum\limits_{r = 1}^N \left(2^{r-2} + 3r\right) &= \sum\limits_{r = 1}^N 2^{r-2} + 3\sum\limits_{r = 1}^N r\\
            &= \dfrac{2^{1-2}\left(2^N - 1\right)}{2-1} + \dfrac{3N(N+1)}2\\
            &= \dfrac{2^N - 1}{2} + \dfrac{3N(N+1)}2\\
            &= \dfrac12 \left(2^N + 3N^2 + 3N - 1\right)
        \end{align*}

        \boxt{The sum of the first $N$ terms is $\dfrac12 \left(2^N + 3N^2 + 3N - 1\right)$}

    \problem{}
        The $r$th term, $u_r$, of a series is given by $u_r = \left(\dfrac13\right)^{3r-2} + \left(\dfrac13\right)^{3r-1}$. Express $\sum\limits_{r=1}^n u_r$ in the form $A\left(1 - \dfrac{B}{27^n}\right)$, where $A$ and $B$ are constants. Deduce the sum to infinity of the series.

    \solution
        \begin{align*}
            \sum\limits_{r=1}^n u_r &= \sum\limits_{r=1}^n \left(\left(\dfrac13\right)^{3r-2} + \left(\dfrac13\right)^{3r-1}\right)\\
            &= \sum\limits_{r=1}^n \left(\dfrac13\right)^{3r-2} + \sum\limits_{r=1}^n \left(\dfrac13\right)^{3r-1}\\
            &= 9\sum\limits_{r=1}^n \left(\dfrac13\right)^{3r} + 3\sum\limits_{r=1}^n \left(\dfrac13\right)^{3r}\\
            &= 12\sum\limits_{r=1}^n \left(\dfrac13\right)^{3r}\\
            &= 12\sum\limits_{r=1}^n \left(\dfrac1{27}\right)^{r}\\
            &= 12 \cdot \dfrac{\tfrac1{27}(1-(\tfrac1{27})^n)}{1-\tfrac1{27}}\\
            &= 12 \cdot \dfrac{1-(\tfrac1{27})^n}{27 - 1}\\
            &= \dfrac{6}{13} \left(1-\dfrac1{27^n}\right)
        \end{align*}

        \boxt{$\displaystyle\sum\limits_{r=1}^n u_r = \dfrac{6}{13} \left(1-\dfrac1{27^n}\right)$}

        \begin{align*}
            \sum\limits_{r=1}^\infty u_r &= \lim_{n \to \infty} \sum\limits_{r=1}^n u_r\\
            &= \lim_{n \to \infty} \dfrac{6}{13} \left(1-\dfrac1{27^n}\right)\\
            &= \dfrac{6}{13} (1-0)\\
            &= \dfrac6{13}
        \end{align*}

        \boxt{$\displaystyle\sum\limits_{r=1}^\infty u_r = \dfrac6{13}$}

    \problem{}
        The $r$th term, $u_r$, of a series is given by $u_r = \ln \dfrac{r}{r+1}$. Find $\sum\limits_{r=1}^n u_r$ in terms of $n$. Comment on whether the series converges.

    \solution
        \begin{align*}
            \sum\limits_{r=1}^n u_r &= \sum\limits_{r=1}^n \ln \dfrac{r}{r+1}\\
            &= \sum\limits_{r=1}^n (\ln r - \ln{r+1})\\
            &= \sum\limits_{r=1}^n \ln r - \sum\limits_{r=1}^n\ln{r+1}\\
            &= \sum\limits_{r=1}^n \ln r - \sum\limits_{r=2}^{n+1}\ln r\\
            &= \ln 1 + \sum\limits_{r=2}^{n+1} \ln r - \ln{n+1} - \sum\limits_{r=2}^{n+1}\ln r\\
            &= \ln 1 - \ln{n+1}\\
            &= \ln \dfrac1{n+1}
        \end{align*}

        \boxt{$\displaystyle\sum\limits_{r=1}^n u_r = \ln \dfrac1{n+1}$}

        Observe that as $n \to \infty$, $\ln \dfrac1{n+1} \to \ln 0$. Hence, $\displaystyle\sum\limits_{r=1}^n u_r$ diverges to infinity. Thus, $u_n$ does not converge.

    \problem{}
        Given that $\sum\limits_{r=1}^n r^2 = \dfrac{n}6 (n+1)(2n+1)$, without using the GC, find the following sums.

        \begin{enumerate}
            \item $\displaystyle\sum\limits_{r=0}^n (r(r+4) + n)$
            \item $\displaystyle\sum\limits_{r=n+1}^{2n} (2r-1)^2$
            \item $\displaystyle\sum\limits_{r=-15}^{20} r(r-2)$
        \end{enumerate}

    \solution
        \part
            \begin{align*}
                \sum\limits_{r=0}^n (r(r+4) + n) &= \sum\limits_{r=0}^n \left(r^2 + 4r + n\right) \\
                &= \sum\limits_{r=0}^n r^2 + 4\sum\limits_{r=0}^n r +n \sum\limits_{r=0}^n 1 \\
                &= \sum\limits_{r=1}^n r^2 + 4\sum\limits_{r=1}^n r +n \sum\limits_{r=0}^n 1 \\
                &= \dfrac{n}6 (n+1)(2n+1) + 4 \cdot \dfrac{n(n+1)}2 + n(n+1) \\
                &= \dfrac{n}6 (n+1)(2n+1) + 2n(n+1) + n(n+1) \\
                &= (n+1)\left( \dfrac{n}6 (2n+1) + 2n + n\right)\\
                &= \dfrac{n}6 (n+1)\left(2n+1 + 12 + 6\right)\\
                &= \dfrac{n}6 (n+1)(2n+19)
            \end{align*}

            \boxt{$\displaystyle\sum\limits_{r=0}^n (r(r+4) + n) = \dfrac{n}6 (n+1)(2n+19)$}

        \part
            \begin{align*}
                \sum\limits_{r=n+1}^{2n} (2r-1)^2 &= \sum\limits_{r=1}^{n} (2(r+n)-1)^2\\\
                &= \sum\limits_{r=1}^{n} (2r + 2n-1)^2\\
                &= \sum\limits_{r=1}^{n} \left(4r^2 + 4(2n-1)r + (2n-1)^2\right)\\
                &= 4\sum\limits_{r=1}^{n} r^2 + 4(2n-1)\sum\limits_{r=1}^{n} r + (2n-1)^2\sum\limits_{r=1}^{n}1\\
                &= 4 \cdot\dfrac{n}6 (n+1)(2n+1) + 4(2n-1)\dfrac{n(n+1)}2 + n(2n-1)^2\\
                &= \dfrac{2}3 \cdot n(n+1)(2n+1) + 2n(2n-1)(n+1) + n(2n-1)^2\\
                &= \dfrac13 n \left(2(n+1)(2n+1) + 6(2n-1)(n+1) + 3(2n-1)^2\right)\\
                &= \dfrac13 n \left(4n^2 +4n + 2n + 2 + 12n^2 -6n + 12n -6 + 12n^2 -12n +3\right)\\
                &= \dfrac13 n \left(28n^2 -1 \right)
            \end{align*}

            \boxt{$\displaystyle\sum\limits_{r=n+1}^{2n} (2r-1)^2 = \dfrac13 n \left(28n^2 -1 \right)$}

        \part
            \begin{align*}
                \sum\limits_{r=-15}^{20} r(r-2) &= \sum\limits_{r=1}^{36} (r-16)((r-16)-2)\\
                &= \sum\limits_{r=1}^{36} (r-16)(r-18)\\
                &= \sum\limits_{r=1}^{36} \left(r^2 - 34r + 288\right)\\
                &= \sum\limits_{r=1}^{36} r^2 - 34\sum\limits_{r=1}^{36}r + 288 \sum\limits_{r=1}^{36} 1\\
                &= \dfrac{36}6 \cdot (36+1)(2\cdot36 + 1) - 34 \cdot \dfrac{36 \cdot 37}2 + 288 \cdot 36\\
                &= 3930
            \end{align*}

            \boxt{$\displaystyle\sum\limits_{r=-15}^{20} r(r-2) = 3930$}

    \problem{}
        Let $S = \displaystyle\sum\limits_{r=0}^\infty \dfrac{(x-2)^r}{3^r}$ where $x \neq 2$. Find the range of values of $x$ such that the series $S$ converges. Given that $x=1$, find

        \begin{enumerate}
            \item the value of $S$
            \item $S_n$, in terms of $n$, where $S_n = \sum\limits_{r=0}^{n-1} \dfrac{(x-2)^r}{3^r}$
            \item the least value of $n$ for which $\abs{S_n - S}$ is less than 0.001\% of $S$
        \end{enumerate}

    \solution
        \begin{align*}
            S &= \sum\limits_{r=0}^\infty \dfrac{(x-2)^r}{3^r}\\
            &= \sum\limits_{r=0}^\infty \left(\dfrac{x-2}3 \right)^r
        \end{align*}

        For $S$ to converge, we must have $\abs{\dfrac{x-2}3} < 1$.

        \case{1}{} $\dfrac{x-2}3 < 1 \implies x-2 < 3 \implies x < 5$

        \case{2}{} $-\dfrac{x-2}3 < 1 \implies \dfrac{x-2}3 > -1 \implies x-2 > -3 \implies x > -1$
        
        Putting both inequalities together, we see that $-1 < x < 5$.

        \boxt{For $S$ to converge, we must have $-1 < x < 5$, $x \neq 2$.}

        \part
            When $x=1$, 
            \begin{align*}
                S &= \sum\limits_{r=0}^\infty \left(-\dfrac{1}3 \right)^r \\
                &= \dfrac{1}{1-(-\tfrac13)}\\
                &= \dfrac34
            \end{align*}

            \boxt{$S = \dfrac34$}

        \part
            \begin{align*}
                S_n &= \sum\limits_{r=0}^{n-1} \dfrac{(-1)^r}{3^r}\\
                &= \sum\limits_{r=0}^{n-1} \left(-\dfrac{1}3 \right)^r\\
                &= \sum\limits_{r=1}^{n} \left(-\dfrac{1}3 \right)^{r-1}\\
                &= -3 \sum\limits_{r=1}^{n} \left(-\dfrac{1}3 \right)^{r}\\
                &= -3 \cdot \dfrac{-\tfrac13 (1 - (-\tfrac13)^{n})}{1-(-\tfrac13)}\\
                &= \dfrac34 \left(1 - \left(-\dfrac13\right)^n\right)
            \end{align*}

            \boxt{$S_n = \dfrac34 \left(1 - \left(-\dfrac13\right)^n\right)$}

        \part
            \begin{alignat*}{2}
                &&\abs{S_n - S} &< 0.001\% S\\
                \implies&&\abs{\dfrac{S_n - S}S} &< \dfrac{0.001}{100}\\
                \implies&&\abs{\dfrac{\tfrac34(1 - (-\tfrac13)^n)}{\tfrac34} - 1} &< \dfrac1{100000}\\
                \implies&&\abs{-\left(-\dfrac13\right)^n} &< \dfrac1{100000}\\
                \implies&&\left(\dfrac13\right)^n &< \dfrac1{10000}\\
                \implies&&n &> \log_{\tfrac13}\dfrac1{100000}\\
                && &= 10.5
            \end{alignat*}

            Since $n \in \N$, the least value of $n$ is 11.

            \boxt{The least value of $n$ is 11.}

    \problem{}
        Given that $\displaystyle\sum\limits_{r=1}^n r^2 = \dfrac{n}6 (n+1)(2n+1)$,

        \begin{enumerate}
            \item write down $\displaystyle\sum\limits_{r=1}^{2k} r^2$ in terms of $k$
            \item find $2^2 + 4^2 + 6^2 + \ldots + (2k)^2$.
        \end{enumerate}

        Hence, show that $\displaystyle\sum\limits_{r=1}^k (2r-1)^2 = \dfrac{k}3(2k+1)(2k-1)$.

    \solution
        \part
            \begin{align*}
                \sum\limits_{r=1}^{2k} r^2 &= \dfrac{2k}6 (2k + 1)(2(2k) + 1)\\
                &= \dfrac{k}3 (2k+1)(4k+1)
            \end{align*}

            \boxt{$\displaystyle\sum\limits_{r=1}^{2k} r^2 = \dfrac{k}3 (2k+1)(4k+1)$}

        \part
            \begin{align*}
                2^2 + 4^2 + 6^2 + \ldots + (2k)^2 &= \sum_{r=1}^k (2r)^2\\
                &= \sum_{r=1}^k 4r^2\\
                &= 4 \sum_{r=1}^k r^2\\
                &= 4 \cdot \dfrac{k}6 (k+1)(2k+1)\\
                &=\dfrac{2k}3 (k+1)(2k+1)
            \end{align*}

            \boxt{$2^2 + 4^2 + 6^2 + \ldots + (2k)^2 = \dfrac{2k}3 (k+1)(2k+1)$}

            \begin{align*}
                \sum\limits_{r=1}^k (2r-1)^2 &= \sum_{r=1}^{2k} r^2 - \sum_{r=1}^k (2r)^2\\
                &= \dfrac{k}3 (2k+1)(4k+1) - \dfrac{2k}3 (k+1)(2k+1)\\
                &= \dfrac{k}3 (2k+1) ((4k+1) - 2(k+1))\\
                &= \dfrac{k}3 (2k+1) (2k-1)\\
            \end{align*}

    \problem{}
        Given that $u_n = e^{nx} - e^{(n+1)x}$, find $\displaystyle\sum\limits_{n=1}^N u_n$ in terms of $N$ and $x$. Hence, determine the set of values of $x$ for which the infinite series $u_1 + u_2 + u_3 + \ldots$ is convergent and give the sum to infinity for cases where this exists.

    \solution
        \begin{align*}
            \sum\limits_{n=1}^N u_n &= \sum\limits_{n=1}^N \left(e^{nx} - e^{(n+1)x}\right)\\
            &= \sum_{n=1}^N e^{nx} - \sum_{n=1}^N e^{(n+1)x}\\
            &= \sum_{n=1}^N e^{nx} - \sum_{n=2}^{N+1} e^{nx}\\
            &= \left(e^x + \sum_{n=2}^N e^{nx}\right) - \left(\sum_{n=2}^{N} e^{nx} + e^{(N+1)x}\right)\\
            &= e^x - e^{(N+1)x}\\
            &= e^x(1 - e^{Nx})
        \end{align*}

        \boxt{$\displaystyle\sum\limits_{n=1}^N u_n = e^x(1 - e^{Nx})$}

        For the infinite series $u_1 + u_2 + u_3 + \ldots$ to converge, we require $e^x$ to converge. Hence, $x \leq 0$. Equivalently, $x \in \R^-_0$.

        \boxt{$\R^-_0$}

        \case{1}{$x=0$}
        \begin{align*}
            \lim_{N \to \infty} \sum\limits_{n=1}^N u_n &= \lim_{N \to \infty} e^x(1 - e^{Nx}) \\
            &= \lim_{N \to \infty} e^0(1 - e^{N\cdot0}) \\
            &= \lim_{N \to \infty} 1(1 - 1) \\
            &= 0
        \end{align*}

        \boxt{When $x = 0$, the sum to infinity is $0$.}

        \case{2}{$x < 0$}
        \begin{align*}
            \lim_{N \to \infty} \sum\limits_{n=1}^N u_n &= \lim_{N \to \infty} e^x(1 - e^{Nx}) \\
            &= e^x(1 - 0)\\
            &= e^x
        \end{align*}

        \boxt{When $x < 0$, the sum to infinity is $e^x$.}

    \problem{}
        Given that $r$ is a positive integer and $f(r) = \dfrac1{r^2}$, express $f(r) - f(r+1)$ as a single fraction. Hence, prove that $\displaystyle\sum\limits_{r=1}^{4n} \left(\dfrac{2r+1}{r^2(r+1)^2}\right) = 1 - \dfrac1{(4n+1)^2}$. Give a reason why the series is convergent and state the sum to infinity. Find $\displaystyle\sum\limits_{r=2}^{4n} \left(\dfrac{2r-1}{r^2(r-1)^2}\right)$.

    \solution
        \begin{align*}
            f(r) - f(r+1) &= \dfrac1{r^2} - \dfrac1{(r+1)^2}\\
            &= \dfrac{(r+1)^2 - r^2}{r^2(r+1)^2}\\
            &= \dfrac{2r+1}{r^2(r+1)^2}
        \end{align*}

        \boxt{$f(r) - f(r+1) = \dfrac{2r+1}{r^2(r+1)^2}$}

        \begin{align*}
            \sum\limits_{r=1}^{4n} \left(\dfrac{2r+1}{r^2(r+1)^2}\right) &= \sum\limits_{r=1}^{4n} (f(r) - f(r+1))\\
            &= \sum\limits_{r=1}^{4n} f(r) - \sum\limits_{r=1}^{4n} f(r+1)\\
            &= \sum\limits_{r=1}^{4n} f(r) - \sum\limits_{r=2}^{4n+1} f(r)\\
            &= \left(f(1) + \sum\limits_{r=2}^{4n} f(r)\right) - \left(\sum\limits_{r=2}^{4n} f(r) + f(4n+1)\right)\\
            &= f(1) - f(4n+1)\\
            &= 1 - \dfrac{1}{(4n+1)^2}
        \end{align*}

        As $r \to \infty$, $\displaystyle\sum\limits_{r=1}^{4n} \left(\dfrac{2r+1}{r^2(r+1)^2}\right) = 1 - \dfrac{1}{(4n+1)^2} \to 1$. Hence, the series is convergent and converges to 1.

        \boxt{The sum to infinity is 1.}

        \begin{align*}
            \sum\limits_{r=2}^{4n} \left(\dfrac{2r-1}{r^2(r-1)^2}\right) &= \sum\limits_{r=1}^{4n-1} \left(\dfrac{2(r+1)-1}{(r+1)^2 \, r^2}\right)\\
            &= \sum\limits_{r=1}^{4n-1} \left(\dfrac{2r+1}{r^2(r+1)^2}\right)\\
            &= \sum\limits_{r=1}^{4n-1} (f(r) - f(r+1))\\
            &= \sum\limits_{r=1}^{4n} (f(r) - f(r+1)) - (f(4n) - f(4n+1))\\
            &= 1 - f(4n+1) - (f(4n) - f(4n+1))\\
            &= 1 - f(4n)\\
            &= 1 - \dfrac1{(4n)^2}\\
            &= 1 - \dfrac1{16n^2}
        \end{align*}

        \boxt{$\displaystyle\sum\limits_{r=2}^{4n} \left(\dfrac{2r-1}{r^2(r-1)^2}\right) = 1 - \dfrac1{16n^2}$}

    \problem{}
        \begin{enumerate}
            \item Express $\dfrac1{(2x+1)(2x+3)(2x+5)}$ in partial fractions.
            \item Hence, show that $\displaystyle\sum\limits_{r=1}^n \dfrac1{(2r+1)(2r+3)(2r+5)} = \dfrac1{60} - \dfrac1{4(2n+3)(2n+5)}$.
            \item Deduce the sum of $\dfrac1{1\cdot 3\cdot5} + \dfrac1{3 \cdot 5 \cdot 7} + \dfrac1{3 \cdot 5 \cdot 7 \cdot 9} + \ldots + \dfrac1{41 \cdot 43 \cdot 45}$.
        \end{enumerate}

    \solution
        \part
            Let $u = 2x+3$. Then $\dfrac1{(2x+1)(2x+3)(2x+5)} = \dfrac{1}{(u-2)u(u+2)}$. 
            \begin{alignat*}{2}
                &&\dfrac{1}{(u-2)u(u+2)} &= \dfrac{A}{u-2} + \dfrac{B}{u} + \dfrac{C}{u+2}\\
                \implies &&1 &= A(u)(u+2) + B(u-2)(u+2) + C(u-2)u\\
                && &= Au^2 + 2Au + Bu^2 - 4B + Cu^2 - 2Cu\\
                && &= (A+B+C)u^2 + (A-C)u - 4B
            \end{alignat*}

            Comparing the coefficients of $u^2$, $u$ and constant terms, we have the following system of equations.
            \[
                \systeme[ABC]{
                    A + B + C  = 0,
                    A - C = 0,
                    -4B = 1
                }
            \]

            Solving, we obtain $A = \dfrac18$, $B = -\dfrac14$ and $C = \dfrac18$. Hence,
            \begin{alignat*}{2}
                &&\dfrac{1}{(u-2)u(u+2)} &= \dfrac{1}{8(u-2)} - \dfrac{1}{4u} + \dfrac{1}{8(u+2)}\\
                \implies&&\dfrac1{(2x+1)(2x+3)(2x+5)} &= \dfrac{1}{8(2x+1)} - \dfrac{1}{4(2x+3)} + \dfrac{1}{8(2x+5)}
            \end{alignat*}

            \boxt{$\dfrac1{(2x+1)(2x+3)(2x+5)} = \dfrac{1}{8(2x+1)} - \dfrac{1}{4(2x+3)} + \dfrac{1}{8(2x+5)}$}

        \part
            \begin{align}
                &\sum\limits_{r=1}^n \dfrac1{(2r+1)(2r+3)(2r+5)}\nonumber\\ &= \sum\limits_{r=1}^n \left(\dfrac{1}{8(2r+1)} - \dfrac{1}{4(2r+3)} + \dfrac{1}{8(2r+5)}\right)\nonumber\\
                &= \dfrac18 \sum\limits_{r=1}^n \dfrac{1}{2r+1} - \dfrac14 \sum\limits_{r=1}^n\dfrac{1}{2r+3} + \dfrac18 \sum\limits_{r=1}^n\dfrac{1}{2r+5}\nonumber\\
                &= \left(\dfrac18 \sum\limits_{r=1}^n \dfrac{1}{2r+1} - \dfrac18 \sum\limits_{r=1}^n\dfrac{1}{2r+3}\right)+ \left(\dfrac18 \sum\limits_{r=1}^n\dfrac{1}{2r+5} - \dfrac18 \sum\limits_{r=1}^n\dfrac{1}{2r+3}\right)\nonumber\\
                &= \dfrac18 \left(\left(\sum\limits_{r=1}^n \dfrac{1}{2r+1} -\sum\limits_{r=1}^n\dfrac{1}{2r+3}\right)+ \left(\sum\limits_{r=1}^n\dfrac{1}{2r+5} - \sum\limits_{r=1}^n\dfrac{1}{2r+3}\right)\right)\label{P12-1}
            \end{align}

            Consider $\displaystyle\sum\limits_{r=1}^n \dfrac{1}{2r+1} - \sum\limits_{r=1}^n\dfrac{1}{2r+3}$.
            \begin{align}
                \sum\limits_{r=1}^n \dfrac{1}{2r+1} - \sum\limits_{r=1}^n\dfrac{1}{2r+3} &= \sum\limits_{r=1}^n \dfrac{1}{2r+1} - \sum\limits_{r=2}^{n+1}\dfrac{1}{2r+1}\nonumber\\
                &= \left(\dfrac13 + \sum\limits_{r=2}^n\dfrac{1}{2r+1}\right) - \left(\sum\limits_{r=2}^n\dfrac{1}{2r+1} + \dfrac{1}{2(n+1) + 1}\right)\nonumber\\
                &= \dfrac13 - \dfrac1{2n+3}\label{P12-2}
            \end{align}

            Consider $\displaystyle\sum\limits_{r=1}^n\dfrac{1}{2r+5} - \sum\limits_{r=1}^n\dfrac{1}{2r+3}$.
            \begin{align}
                \sum\limits_{r=1}^n\dfrac{1}{2r+5} - \sum\limits_{r=1}^n\dfrac{1}{2r+3} &= \sum\limits_{r=1}^n\dfrac{1}{2r+5} - \sum\limits_{r=0}^{n-1}\dfrac{1}{2r+5}\nonumber\\
                &= \left(\sum\limits_{r=0}^n \dfrac{1}{2r+5} - \dfrac15\right) - \left(\sum\limits_{r=0}^n \dfrac{1}{2r+5} - \dfrac1{2n+5} \right)\nonumber\\
                &= \dfrac1{2n+5} - \dfrac15\label{P12-3}
            \end{align}

            Substituting Equations~\ref{P12-2} and~\ref{P12-3} into Equation~\ref{P12-1}, we have
            \begin{align*}
                \sum\limits_{r=1}^n \dfrac1{(2r+1)(2r+3)(2r+5)} &= \dfrac18 \left(\dfrac13 - \dfrac1{2n+3} + \dfrac1{2n+5} - \dfrac15\right)\\
                &= \dfrac18 \left(\dfrac2{15} - \dfrac1{2n+3} + \dfrac1{2n+5}\right)\\
                &= \dfrac18 \left(\dfrac2{15} - \dfrac{2}{(2n+3)(2n+5)}\right)\\
                &= \dfrac1{60} - \dfrac1{4(2n+3)(2n+5)}
            \end{align*}

        \part
            \begin{align*}
                & \dfrac1{1\cdot 3\cdot5} + \dfrac1{3 \cdot 5 \cdot 7} + \dfrac1{3 \cdot 5 \cdot 7 \cdot 9} + \ldots + \dfrac1{41 \cdot 43 \cdot 45} \\
                &= \sum_{r = 0}^{20} \dfrac1{(2r+1)(2r+3)(2r+5)}\\
                &= \dfrac1{1 \cdot 3 \cdot 5} + \sum_{r = 1}^{20} \dfrac1{(2r+1)(2r+3)(2r+5)}\\
                &= \dfrac1{15} + \dfrac1{60} - \dfrac1{4(2\cdot20+3)(2\cdot20+5)}\\
                &= \dfrac{161}{1935}
            \end{align*}

            \boxt{$\dfrac1{1\cdot 3\cdot5} + \dfrac1{3 \cdot 5 \cdot 7} + \dfrac1{3 \cdot 5 \cdot 7 \cdot 9} + \ldots + \dfrac1{41 \cdot 43 \cdot 45} = \dfrac{161}{1935}$}
\end{document}